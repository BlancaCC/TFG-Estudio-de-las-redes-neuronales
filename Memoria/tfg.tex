% !TEX program = pdflatex
% !TEX encoding = UTF-8 Unicode

% TFG basado en la plantilla de la clase `scrbook` del paquete
% KOMA-script para la elaboración de un TFG siguiendo las
% directrices del la comisión del Grado en Matemáticas de
% la Universidad de Granada.

% Autor de la plantilla: Francisco Torralbo Torralbo
% miércoles, 29 de abril de 2020

% Autor de la memoria: Blanca Cano Camarero

\documentclass{scrbook}

\KOMAoptions{%
  fontsize=10pt,        % Tamaño de fuente
  paper=a4,             % Tamaño del papel
  headings=normal,      % Tamaño de letra para los títulos: small, normal, big
  % parskip=half,         % Espacio entre párrafos: full (una línea) o half (media línea)
  headsepline=false,    % Una linea separa la cabecera del texto
  cleardoublepage=empty,% No imprime cabecera ni pie en páginas en blanco
  chapterprefix=false,  % No antepone el texto "capítulo" antes del número
  appendixprefix=false,	% No antepone el texto "Apéndice" antes de la letra
  listof=totoc,		    	% Añade a la tabla de contenidos la lista de tablas y figuras
  index=totoc,			    % Añade a la talba de contenidos una entrada para el índice
  bibliography=totoc,	  % Añade a la tabla de contenidos una entrada para bibliografía
  BCOR=5mm,           % Reserva margen interior para la encuadernación.
                        % El valor dependerá el tipo de encuadernado y del grosor del libro.
  DIV=10,             % Cálcula el diseño de página según ciertos
                        % parámetros. Al aumentar el número aumentamos el ancho de texto y disminuimos el ancho del margen. Una opción de 14 producirá márgenes estrechos y texto ancho.
}


% INFORMACIÓN PARA LA VERSIÓN IMPRESA
% Si el documento ha de ser impreso en papel de tamaño a4 pero el tamaño del documento (elegido en \KOMAoptions con la ocpión paper) no es a4 descomentar la línea que carga el paquete `crop` más abajo. El paquete crop se encargará de centrar el documento en un a4 e imprimir unas guías de corte. El procedimiento completo para imprenta sería el siguiente:
% 0. Determinar, según el tipo de encuadernación del documento, el ancho reservado para el proceso de encuadernación (preguntar en la imprenta), es decir, la anchura del área del papel que se pierde durante el proceso de encuadernación. Fijar la varibale BCOR de \KOMAoptions a dicho valor.
% 1. Descomentar la siguiente línea e imprimir una única página con las guías de corte
% 2. Cambiar la opción `cross` por `cam` (o `off`) en el paquete crop y volver a compilar. Imprimir el documento (las guías de corte impresas no inferfieren con el texto).
% 3. Usar la página con las guías impresas en el punto 1 para cortar todas las páginas.

% \usepackage[a4, odd, center, pdflatex, cross]{crop} % Permite imprimir el documento en un a4 (si el tamaño es más pequeño) mostrando unas guías de corte. Útil para imprenta.

% VERSIÓN ELECTRÓNICA PARA TABLETA
% Las opciones siguientes seleccionan un tamaño de impresión similar a una tableta de 9 pulgadas con márgenes estrechos. Útil para producir una versión en pdf para ser leída en una tableta en lugar de impresa.
% Para que la portada quede centrada correctamente hay que editar el
% archivo `portada.tex` y eliminar el entorno `addmargin`

% \KOMAoptions{fontsize=10pt, paper=19.7104cm:14.7828cm, twoside=false, BCOR=0cm, DIV=14}

% ---------------------------------------------------------------------
%	PAQUETES
% ---------------------------------------------------------------------

% CODIFICACIÓN E IDIOMA
% ---------------------------------------------------------------------
\usepackage[utf8]{inputenc} 			    % Codificación de caracteres

% Selección del idioma: cargamos por defecto inglés y español (aunque este último es el idioma por defecto para el documento). Cuando queramos cambiar de idioma escribiremos:
% \selectlanguage{english} o \selectlanguage{spanish}

\usepackage[english, spanish, es-nodecimaldot, es-noindentfirst, es-tabla]{babel}

% Opciones cargadas para el paquete babel:
  % es-nodecimaldot: No cambia el punto decimal por una coma en modo matemático.
  % es-noindentfirst: No sangra los párrafos tras los títulos.
  % es-tabla: cambia el título del entorno `table` de "Cuadro" a "Tabla"

% Otras opciones del paquete spanish-babel:
  \unaccentedoperators % Desactiva los acentos en los operadores matemáticso (p.e. \lim, \max, ...). Eliminar esta opción si queremos que vayan acentuados

% MATEMÁTICAS
% ---------------------------------------------------------------------
\usepackage{amsmath, amsthm, amssymb} % Paquetes matemáticas
\usepackage{mathtools}                % Añade mejoras a amsmath
\mathtoolsset{showonlyrefs=true}      % sólo se numeran las ecuaciones que se usan
\usepackage[mathscr]{eucal} 					% Proporciona el comando \mathscr para
                                      % fuentes de tipo manuscrito en modo matemático sin sobreescribir el comando \mathcal

% TIPOGRAFÍA
% ---------------------------------------------------------------------
% El paquete microtype mejora la tipografía del documento.
\usepackage[activate={true,nocompatibility},final,tracking=true,kerning=true,spacing=true,factor=1100,stretch=10,shrink=10]{microtype}

% Las tipografías elegidas para el documento son las siguientes
% Normal font: 			URW Palladio typeface.
% Sans-serif font: 	Iwona
% Monospace font: 	Inconsolata
% Consultar http://www.tug.dk/FontCatalogue/ para seleccionar otra tipografía.
% Es conveniente elegir aquellas que tienen soporte matemático.
\usepackage[T1]{fontenc}
\usepackage[sc, osf]{mathpazo} \linespread{1.05}
\usepackage[scaled=.95,type1]{cabin} % sans serif in style of Gill Sans
\usepackage{inconsolata}
% \renewcommand{\sfdefault}{iwona}


% Selecciona el tipo de fuente para los títulos (capítulo, sección, subsección) del documento.
\setkomafont{disposition}{\sffamily\bfseries}

% Cambia el ancho de la cita. Al inicio de un capítulo podemos usar el comando \dictum[autor]{cita} para añadir una cita famosa de un autor.
\renewcommand{\dictumwidth}{0.45\textwidth}

\recalctypearea % Necesario tras definir la tipografía a usar.

% TABLAS, GRÁFICOS Y LISTADOS DE CÓDIGO
% ---------------------------------------------------------------------
\usepackage{booktabs}
% \renewcommand{\arraystretch}{1.5} % Aumenta el espacio vertical entre las filas de un entorno tabular

\usepackage{xcolor, graphicx}
% Carpeta donde buscar los archivos de imagen por defecto
\graphicspath{{img/}}

% IMAGEN DE LA PORTADA
% Existen varias opciones para la imagen de fondo de la portada del TFG. Todas ellas tienen en logotipo de la universidad de Granada en la cabecera. Las opciones son las siguientes:
% 1. portada-ugr y portada-ugr-color: diseño con marca de agua basada en el logo de la UGR (en escala de grises y color).
% 2. portada-ugr-sencilla y portada-ugr-sencilla-color: portada únicamente con el logotipo de la UGR en la cabecera.
\usepackage{eso-pic}
\newcommand\BackgroundPic{%
	\put(0,0){%
		\parbox[b][\paperheight]{\paperwidth}{%
			\vfill
			\centering
      % Indicar la imagen de fondo en el siguiente comando
			\includegraphics[width=\paperwidth,height=\paperheight,%
			keepaspectratio]{portada/portada-ugr-sencilla-color}%
			\vfill
}}}

\usepackage{listings} % Para la inclusión de trozos de código

% CABECERAS
% ---------------------------------------------------------------------
% Si queremos modificar las cabeceras del documento podemos usar el paquete
% `scrlayer-scrpage` de KOMA-Script. Consultar la documentación al respecto.
% \usepackage[automark]{scrlayer-scrpage}

% VARIOS
% ---------------------------------------------------------------------

%\usepackage{showkeys}	% Muestra las etiquetas del documento. Útil para revisar las referencias cruzadas.

% ÍNDICE
% Para generar el índice hay que compilar el documento con MakeIndex. Generalmente los editores se encargan de ello automáticamente.
% ----------------------------------------------------------------------
% \index{} para añadir un elemento
% \index{main!sub} para añadir un elementos "sub" bajo la categoría "main".
% \index{termino|textbf} para dar formato al número de página (negrita).
% \index{termino|see{termino relacionado}} para crear una referencia cruzada

% Ejemplo: \index{espacio homogéneo}, \index{superficie!mínima}, \index{esfera|see{espacio homogéneo}}
\usepackage{makeidx}
%\usepackage{showidx} % Muestra en el margen del documento las entradas añadidas al índice. Útil para revisar el documento. Si está activo el índice no se genera
\makeindex

% ---------------------------------------------------------------------
% COMANDOS Y ENTORNOS
% ---------------------------------------------------------------------
% Cargamos un archivo externo donde hemos incluido todos los comandos
% propios que vamos a usar en el documento.
% DEFINICIÓN DE COMANDOS Y ENTORNOS

% CONJUNTOS DE NÚMEROS

  \newcommand{\N}{\mathbb{N}}     % Naturales
  \newcommand{\R}{\mathbb{R}}     % Reales
  \newcommand{\Z}{\mathbb{Z}}     % Enteros
  \newcommand{\Q}{\mathbb{Q}}     % Racionales
  \newcommand{\C}{\mathbb{C}}     % Complejos

  %%%%%%%%% Mis comandos %%%%%%%%%
% Para escribir código y pseudo código  
\usepackage{minted}

\usepackage{algorithmic}
% Para la definición de redes neuronales de una sola capa 
\newcommand{\Hu}{\mathcal{H}(X,Y)}  % Espacio de las redes neuronales

% Notas en el margen
\usepackage{sidenotes}
  \newcommand{\afines}{\mathcal{A}(\R^d)}
  \newcommand{\pmc}{\mathcal{H}_G(\R^d,\R)}%{\sum ^r (G)}  % Red neurona una capa una salida
  \newcommand{\pmcg}{ \sum \prod^d (G)} % Generalización red neuronal
  \newcommand{\fC}{\mathcal{C}(\R^d)} %conjunto de funciones continuas en R^r -> R
  \newcommand{\fM}{\mathcal{M}(\R^d)} % Conjunto funiones medibles
  \newcommand{\rrnn}{ \mathcal{H}(\R^d,\R)} % Red neuronal  sin subíndice
  \newcommand{\rrnng}{ \sum \prod^d (\psi)} % Red neuronal  generalizado
  \newcommand{\dist}{\rho_{\mu}}     % Distancia de una medida
  \newcommand{\dlp}{\rho_{p}} % Distancia de los espacios Lp
  % Múltiples salidas 
  \newcommand{\fCC}{\mathcal{C}(\R^d ,\R^s)}
  \newcommand{\fMM}{\mathcal{M}(\R^d , \R^s)}
  \newcommand{\rrnnmc}{ \mathcal{H}(\R^d,\R^s)} 
  \newcommand{\rrnnsmn}{ \mathcal{H}_n(\R^d,\R^s)} % Red neuronal salida múltiple con n neuronas
  \newcommand{\rrnngmc}{ \sum \prod^{d,s} (\psi)} 
  %%%%%%%%% Mis comandos %%%%%%%%%5
\usepackage{sidenotes} % Notas en el margen
\newcommand{\margenimagen}{
  \newgeometry{
      left=2.5cm, % Margen izquierdo
    right=5cm, % Margen derecho
    bottom=2.5cm % Margen inferior}
  }
}
\usepackage{caption}
\usepackage{subcaption}

% TEOREMAS Y ENTORNOS ASOCIADOS

  % \newtheorem{theore<m}{Theorem}[chapter]
  \newtheorem*{teorema*}{Teorema}
  \newtheorem{teorema}{Teorema}[chapter]
  \newtheorem{proposicion}{Proposición}[chapter]
  \newtheorem{lema}{Lema}[chapter]
  \newtheorem{corolario}{Corolario}[chapter]

    \theoremstyle{definition}
  \newtheorem{definicion}{Definición}[chapter]
  \newtheorem{ejemplo}{Ejemplo}[chapter]

    \theoremstyle{remark}
  \newtheorem{observacion}{Observación}[chapter]


\DeclareMathOperator{\sign}{signo}
\usepackage[inline]{enumitem}
\usepackage{mathtools}
\usepackage[spanish,onelanguage,linesnumbered,ruled,vlined]{algorithm2e}
\usepackage{listingsutf8}
\lstset{language=Python,
        literate=
          {ó}{{\'o}}1
          {í}{{\'i}}1
          {á}{{\'a}}1
          {ú}{{\'u}}1
          {é}{{\'e}}1
          {ñ}{{\v{n}}}1
}
\usepackage{tocloft}
\setlength{\cftfignumwidth}{2.55em}
\DeclareMathOperator*{\argmin}{arg\,min}

\SetKwRepeat{Struct}{struct \{}{\}}%

% Para las notas del margen 
%Nota los colores seleccionados han sido creados con una paleta inclusiva
% https://palett.es/6a94a8-013e3b-7eb645-31d331-26f27d
\definecolor{darkRed}{rgb}{ 0.149, 0.99, 0.49}%{1,0.1,0.1}
\definecolor{dark_green}{rgb}{0, 0.24, 0.23} %{0.2, 0.7, 0.2}
\definecolor{blue}{rgb}{0.41, 0.58, 0.659} % sobreeescribimos el azul
\newcommand{\smallMarginSize}{1.8cm}
\newcommand{\bigMarginSize}{3cm}
\newcommand{\maginLetterSize}{\footnotesize} %{\scriptsize}%

% Para los iconos 
\usepackage{fontawesome}
% Alias aclaraciones 
% dark_green
\newcommand{\iconoAclaraciones}{\faQuestionCircleO $\quad$} %\faQuestionCircleO
% blue
\newcommand{\iconoProfundizar}{\faSearch  $\quad$}
\newcommand{\iconoClave}{\faLightbulbO  $\quad$} % \faLightbulbO %\faKey

% Contenido original 
\usepackage{lipsum}
\usepackage[%
linewidth=5pt,
outerlinecolor=red,
outerlinewidth=5pt,
innerlinewidth=0pt,
outerlinecolor=red,
roundcorner=5pt
%middlelinecolor= yellow,
middlelinewidth=0.4pt,
%roundcorner=1pt,
topline = false,
rightline = false,
leftline = false,
bottomline = false,
rightmargin=0pt,
skipabove=0pt,
skipbelow=0pt,
leftmargin=-1cm,
backgroundcolor=black!7,
%innerleftmargin=1cm,
%innerrightmargin=0pt,
%innertopmargin=0pt,
%innerbottommargin=0pt,
%frametitlebackgroundcolor=yellow,
]{mdframed}

\newenvironment{aportacionOriginal}
  {\mdfsetup{
    frametitle={\colorbox{black!7}{ \textcolor{white}{\Large Aportación original}}},
    %frametitleaboveskip=-\ht\strutbox,
    %frametitlealignment=\center
    }
  \begin{mdframed}%
  }
  {\end{mdframed}}

  % ancho imágenes en tabla
  \newcommand{\coeficienteAncho}{.3}




% --------------------------------------------------------------------
% INFORMACIÓN DEL TFG Y EL AUTOR
% --------------------------------------------------------------------
\usepackage{xspace} % Para problemas de espaciado al definir comandos

\newcommand{\miTitulo}{Optimización de redes neuronales\xspace}
\newcommand{\miNombre}{Blanca Cano Camarero\xspace}
\newcommand{\miGrado}{Doble Grado en Ingeniería Informática y Matemáticas}
\newcommand{\miFacultad}{Escuela Técnica Superior de Ingenierías Informática y de Telecomunicación \\ Facultad de Ciencias}
\newcommand{\miUniversidad}{Universidad de Granada}
% Añadir tantos tutores como sea necesario separando cada uno de ellos
% mediante el comando `\\\medskip` y una línea en blanco
\newcommand{\miTutor}{
  Juan Julián Merelo Guervós\\ \emph{Arquitectura y tecnología de computadores}
  \\\medskip

  Francisco Javier Meri de la Maza\\ \emph{Análisis matemático}
}
\newcommand{\miCurso}{2021-2022\xspace}

% HYPERREFERENCES
% --------------------------------------------------------------------
\usepackage{xurl}
\usepackage[pagebackref]{hyperref}
\input{paquetes/hyperref}

\begin{document}

% --------------------------------------------------------------------
% FRONTMATTER
% -------------------------------------------------------------------
%\frontmatter % Desactiva la numeración de capítulos y usa numeración romana para las páginas

% \pagestyle{plain} % No imprime cabeceras

\begin{titlepage}
 
 
\newlength{\centeroffset}
\setlength{\centeroffset}{-0.5\oddsidemargin}
\addtolength{\centeroffset}{0.5\evensidemargin}
\thispagestyle{empty}

\noindent\hspace*{\centeroffset}\begin{minipage}{\textwidth}

\centering
\includegraphics[width=0.9\textwidth]{imagenes/logo_ugr.jpg}\\[1.4cm]

\textsc{ \Large TRABAJO FIN DE GRADO\\[0.2cm]}
\textsc{ INGENIERÍA EN ...}\\[1cm]
% Upper part of the page
% 
% Title
{\Huge\bfseries Titulo del Proyecto\\
}
\noindent\rule[-1ex]{\textwidth}{3pt}\\[3.5ex]
{\large\bfseries Subtitulo del Proyecto}
\end{minipage}

\vspace{2.5cm}
\noindent\hspace*{\centeroffset}\begin{minipage}{\textwidth}
\centering

\textbf{Autor}\\ {Nombre Apellido1 Apellido2 (alumno)}\\[2.5ex]
\textbf{Directores}\\
{Nombre Apellido1 Apellido2 (tutor1)\\
Nombre Apellido1 Apellido2 (tutor2)}\\[2cm]
\includegraphics[width=0.3\textwidth]{imagenes/etsiit_logo.png}\\[0.1cm]
\textsc{Escuela Técnica Superior de Ingenierías Informática y de Telecomunicación}\\
\textsc{---}\\
Granada, mes de 201
\end{minipage}
%\addtolength{\textwidth}{\centeroffset}
%\vspace{\stretch{2}}
\end{titlepage}



\include{preliminares/titulo}
%\include{preliminares/declaracion-originalidad}
%% !TeX root = ../libro.tex
% !TeX encoding = utf8
%
%*******************************************************
% Resumen
%*******************************************************

% \manualmark
% \markboth{\textsc{Introducción}}{\textsc{Introducción}}

\chapter*{Resumen}\label{ch:resumen}
%\addcontentsline{toc}{chapter}{Resumen}

Existe en la actualidad un desequilibrio entre resultados empíricos 
y teóricos de redes neuronales llegando incluso a contradicción
 (como se comenta en la introducción del capítulo 
 \ref{chapter:Introduction-neuronal-networks}), será por tanto
nuestro primer objetivo construir una teoría sólida
que de cabida a 
 optimizaciones de fundamento teórico; 
una revisión y
 purga de cualquier artificio existente sobre 
 redes neuronales carente de fundamento matemático. 

Como resultado de ello se ha propuesto e implementado 
un nuevo modelo de red neuronal así como sus 
métodos de aprendizaje y evaluación. 
Además se ha dado un criterio de selección de 
funciones de activación y un algoritmo de 
inicialización de pesos que mejora los ya existentes. 
Todos los resultados han conducido a la creación de 
la biblioteca \textit{OptimizedNeuralNetwork.jl}, 
que contiene la implementación de nuestros modelos y
 métodos optimizados. 


La estructura de la memoria es la siguiente: 


\paragraph{PALABRAS CLAVE:}
\begin{itemize*}[label=,itemsep=1em,itemjoin=\hspace{1em}]
  \item redes neuronales
  \item optimización 
  \item funciones de activación 
  \item inicialización de pesos
  \item Biblioteca de aprendizaje automático
\end{itemize*}

\endinput

%% !TeX root = ../libro.tex
% !TeX encoding = utf8
%
%*******************************************************
% Summary
%*******************************************************

\selectlanguage{english}
\chapter*{Summary}\label{ch:summary}
%\addcontentsline{toc}{chapter}{Summar}

Nowadays experimental research in Neural Networks are more advanced than theoretical
results. 
From this we want to establish a solid mathematical theory from which to optimize the current neural network models. 


As a result of our study we have proposed a new neural 
network model and adapted and optimized already used 
evaluation and learning methods to it. 
Moreover, we have discovered some theorems that prove the 
equivalence among some activation functions and propose a new
 algorithm to initialize weights of neural networks. By the 
first result, we obtain a criteria to choose the most 
suitable activation function to maintain accuracy and reduce computational cost.
 By the second one, we might accelerate 
learning convergence methods.

In addition, the models, methods and algorithms have been 
implemented in a Julia library calls \textit{OptimizedNeuralNetwork.jl}

All the theory development, designs, decisions and result are 
written in this memory which have the following structure: 
\begin{itemize}
 \item \textbf{Chapter \ref{ch00:methodology}: Description of the methodology followed.} We have organized our project according to an agile philosophy  based on personas methodology, users stories, milestones and tests. The method has conducted and linked mathematical and technical results and implementations, giving them coherence and validation methods. 

 \item \textbf{Chapter \ref{chapter:Introduction-neuronal-networks}: Description of the learning problem.} We defined the characteristic and type of machine learning problems. We would focus on supervised learning ones. 

 \item \textbf{Chapter \ref{ch03:teoria-aproximar}:  Approximation theory.} In order to establish a solid theory we would start our work trying to solve machine learning problems by traditional approximation methods.  The main result we obtained is the Stone Weierstrass’s theorem. As a conclusion of this chapter we would know the virtues and faults of traditional methods and understand the necessity of new methods and structures such as neural networks. 

 \item \textbf{Chapter \ref{chapter4:redes-neuronales-aproximador-universal}: Neuronal networks are universal approximators.}  In this chapter we introduce our neural network model and compare it with the conventional ones. In order to show it is well defined we will prove the universal convergence of our model to any measurable function. In addition, we will give some results about how our model solves classification and regression problems meanwhile the number of its neurons rises. Finally, we will argue if all of those math results can actually resolve real problems, the idea behind the debate is the computability representation of real numbers. 

 \item \textbf{Chapter  \ref{chapter:construir-redes-neuronales}: The design and implementation of neural networks.} We would describe carefully the design and implementation of our model of neural network, thanks to that we will obtain some mathematical results about bias and classification function that would be useful to compare our model with the conventional ones and justify 
our selection. Moreover, we would explain, justify and design  learning and evaluation methods to our models. These methods are optimized versions of Forward Propagation and Backpropagation. 

\item \textbf{Chapter \ref{funciones-activacion-democraticas-mas-demoscraticas}: Democratization of activation functions.} We would explain at this chapter if there are better activation functions. In the direction of that we will prove two original results that show that there are families of activation functions that with the same conditions will resolve problems with the same accuracy. As a result, if we compare the computational cost of the members of those families and choose the faster one, we would obtain a method to optimize evaluation and learning of neural networks without loss accuracy. We have used the Wilcoxon signed-rank test as a statistical hypothesis test so as to give a rigorous study of our criteria. 

\item \textbf{Chapter \ref{section:inicializar_pesos}: Weight initializing algorithm.} Since the Backpropagation and other iterative  methods are sensible to the initial value of a neural network, we will show an original method to initialize its weights from training data. This process not only will produce a better initial step but also has lower computational cost than Backpropagation.  To test the potential of this method we will use the Wilcoxon signed-rank test again and also, from the experiment requirement our OptimizedNeuralNetwork.jl library will be born. In this chapter we will also explain all the decisions done during the design and implementation of the library in other to be as efficient as we could.	

\item \textbf{Chapter \ref{ch08:genetic-selection}: Use of genetic algorithm in the selection of activation function.} In this chapter we will explain a future work. Given a fixed number of neurons, the selection of its activation function may be crucial to reduce the train and test error.  However, adding more free params to the search space increases its complexity and at same time the cost of finding a solution.  However, the result obtained at chapter 6 and a property of our neural model will reduce the space complexity.
\end{itemize} 

\paragraph{KEYWORDS:}
\begin{itemize*}[label=,itemsep=1em,itemjoin=\hspace{1em}]
  \item neural networks
  \item optimization
  \item activation functions
  \item weights initializing
  \item machine learning library

\end{itemize*}

% Al finalizar el resumen en inglés, volvemos a seleccionar el idioma español para el documento
\selectlanguage{spanish}
\endinput

%\include{preliminares/dedicatoria}                % Opcional
\include{preliminares/tablacontenidos}
            % Opcional

% \pagestyle{scrheadings} % A partir de ahora sí imprime cabeceras
% TODO
%%%%%%%%%%%%%%%%%%%%%%%%%%%%%%%%%%%%%%%%%%%%%%%%%%%%%%%%%%%%%%%%%%%%%%%%%%%%%%%
%
% Introducción sobre algoritmos de cálculo de los pesos de una red neuronal 
%
%%%%%%%%%%%%%%%%%%%%%%%%%%%%%%%%%%%%%%%%%%%%%%%%%%%%%%%%%%%%%%%%%%%%%%%%%%%%%%

\chapter{Explicitación del cálculo de una red neuronal}

Se han concretado en los capítulos anteriores que con con redes neuronales 
podemos aproximar funciones medibles, la cuestión ahora reside en ¿cómo se consigue tales 

Ya expusimos en la sección \ref{algoritmo-forward-propagation} cómo evaluar
una red neuronal, la cuestión entonces radica en ¿cuál es la red neuronal
buscada? es decir a nivel práctico sería ¿cómo puedo calcular matrices de pesos 
adecuadas?

Explicaremos en este capítulo el método tradicional y ampliamente usado de 
\textit{backpropagation} \ref{algoritmo-de-backpropagation} y otros más novedosos como
\textcolor{red}{TODO : añadir otros métodos de actualización de pesos. La idea sería que 
los buscados tengan un coste menor para así optimizar la velocidad de aprendizaje.}


%

% --------------------------------------------------------------------
% MAINMATTER
% --------------------------------------------------------------------
%\mainmatter % activa la numeración de capítulos, resetea la numeración de las páginas y usa números arábigos
\setpartpreamble[c][0.75\linewidth]{
	%\bigskip % Deja un espacio vertical en la parte superiọ-r
  
}

%Metodología 
%%%%%%%%%%%%%%%%%%%%%%%%%%%%%%%%%%
%% Comentarios previos 
%%%%%%%%%%%%%%%%%%%%%%%%%%%%%%%%%%
% Convenio colores notas
%%%%%%%%%%%%%%%%%%%%%%%%%%%%%%%%%%

\section*{Comentario previo}

Se pretende con este documento presentar un  trabajo de fin de grado de Ingeniería Informática y Matemáticas y que cualquier miembro del tribunal; 
independientemente de la vertiente a la que pertenezca sea capaz de comprenderlo en su totalidad.  
Para ello se ha acompañado la exposición con notas en los márgenes aclaratorias, que siguen el siguiente código de color y icono: 

\begin{itemize}
    \item  \iconoAclaraciones \textcolor{dark_green}{ Color 1}: Comentarios para 
    aclarar conceptos matemáticos o informáticos ofertar la idea intuitiva que 
    se esconde, donde no se presuponen conocimientos avanzados en 
    la materia. 
    \item  \iconoProfundizar \textcolor{blue}{  Color 2}: Comentarios para una reflexión más profunda o que indique nuevas áreas que explorar. 
    \item  \iconoClave  \textcolor{darkRed}{  Color 3}: Concepto clave y destacable que tendrá un papel fundamental a posteriori.  
\end{itemize}

%Nota los colores seleccionados han sido creados con una paleta inclusiva
% https://palett.es/6a94a8-013e3b-7eb645-31d331-26f27d

Además a lo largo del trabajo se han realizado aportaciones propias y resultados novedosos, estos aparecerán destacados de la siguiente forma: 

\begin{aportacionOriginal}
    Este recuadro representa cómo se indicará de ahora en adelante alguna aportación original a la materia. 
\end{aportacionOriginal}

%%%%%%%%%%%%%%%%%%%%%%%%%%%%%%%%%%%%%%%%%%%%%%%%%%%%%%%%%%%%%%%%%%%%%%%%%%%%%%
%
% Introducción sobre algoritmos de cálculo de los pesos de una red neuronal 
%
%%%%%%%%%%%%%%%%%%%%%%%%%%%%%%%%%%%%%%%%%%%%%%%%%%%%%%%%%%%%%%%%%%%%%%%%%%%%%%

\chapter{Explicitación del cálculo de una red neuronal}

Se han concretado en los capítulos anteriores que con con redes neuronales 
podemos aproximar funciones medibles, la cuestión ahora reside en ¿cómo se consigue tales 

Ya expusimos en la sección \ref{algoritmo-forward-propagation} cómo evaluar
una red neuronal, la cuestión entonces radica en ¿cuál es la red neuronal
buscada? es decir a nivel práctico sería ¿cómo puedo calcular matrices de pesos 
adecuadas?

Explicaremos en este capítulo el método tradicional y ampliamente usado de 
\textit{backpropagation} \ref{algoritmo-de-backpropagation} y otros más novedosos como
\textcolor{red}{TODO : añadir otros métodos de actualización de pesos. La idea sería que 
los buscados tengan un coste menor para así optimizar la velocidad de aprendizaje.}


%%%%%%%%%%%%%%%%%%%%%%%%%%%%%%%%%%%%%%%%%%%%%%%%%%%%%%%%%%%%%%%%
% Justificación de porqué se ha seleccionado Julia como lenguaje de programación 
%%%%%%%%%%%%%%%%%%%%%%%%%%%%%%%%%%%%%%%%%%%%%%%%%%%%%%%%%%%%%%%%

\section{Herramientas utilizadas}

\subsection{GitHub}
Como servicio externo hemos usado \href{https://github.com}{GitHub}, ya que permite implementar de manera eficaz
 todo el desarrollo ágil: 
desde la planificación, comunicación, test e incluso difusión y acceso a los resultados. 

\subsection{Lenguaje de programación Julia}
Hemos seleccionado como lenguaje de programación \href{https://julialang.org}{Julia}, 
esto se debe a que nos ofrece \textit{benchmarks} muy competitivos\footnote{Véase los resultado expuestos en 
\url{https://julialang.org/benchmarks/}, web consultada por última vez el 22 de mayo de 2022.} 
, al nivel de C. Así como la disponibilidad  de bibliotecas usadas en ciencia de datos de  
la de lenguajes como \textit{R} y \textit{Python}. 

\part{Teoría subyacente }%\label{part:conceptos-previos}

% Filosofía a seguir   
%%%%%%%%%%%%%%%%%%%%%%%%%%%%%%%%%%%%%%%%%%%%%%%%%%%%%%%%%%%%%%%%%%%%%%%%%%%
%% Introducción para comenzar la teoría hablando de la filosofía que se va a seguir 
%%
%%%%%%%%%%%%%%%%%%%%%%%%%%%%%%%%%%%%%%%%%%%%%%%%%%%%%%%%%%%%%%%%%%%%%

\part{Teoría subyacente}

No es usual en un manual que trate sobre redes neuronales encontrarse en su interior con un 
capítulo sobre teoría de la aproximación, pero tampoco es nuestra intención
hacer de este documento una recopilación de todo lo usual, sino todo lo contrario.

Existe en la actualidad un desequilibrio entre resultados empíricos y teóricos de redes neuronales llegando incluso a contradicción (como se comenta en la introducción del capítulo \ref{chapter:Introduction-neuronal-networks}), será por tanto
nuestro primer objetivo conseguir una revisión y purga de cualquier artificio existente sobre redes neuronales carente de fundamento matemático. 

El fin de esto no es más que construir una teoría sólida que de cabida a 
optimizaciones de fundamento teórico. La estructura de la memoria es la siguiente: 


%%% Descripción de los capítulos antigua 
Para ello nuestro \textit{modus operandi} será el siguiente: 
Se describirá el conjunto y características de problemas que pretendemos abarcar  en el capítulo \ref{chapter:Introduction-neuronal-networks}. 
Se comentará las limitaciones e inconvenientes que presenta un enfoque clásico 
basado en teoría de la aproximación en el capítulo \ref{ch03:teoria-aproximar}.
A continuación en el capítulo \ref{chapter4:redes-neuronales-aproximador-universal}, 
presentarán las redes neuronales como un modelo eficiente.
 
Al final del mismo capítulo se introduce la definición que hemos determinado por conveniente de red neuronal y que es producto de los capítulos 
\ref{ch03:teoria-aproximar} y \ref{chapter4:redes-neuronales-aproximador-universal}.
 
Tras todo el fundamento teórico en \ref{chapter:construir-redes-neuronales} se explicitará el diseño de la red neuronal modelizada así como los algoritmo de evaluación y aprendizaje.

En los capítulos \ref{funciones-activacion-democraticas-mas-demoscraticas} y \ref{section:inicializar_pesos} se explican además otros resultados para optimizar el coste computacional.

% Redes neuronales Definición de la clase de redes neuronales 
% !TeX root = ../../tfg.tex
% !TeX encoding = utf8

\chapter{Introducción redes neuronales}
\section{Objetivos}  

A lo largo de estos capítulos definiremos y explicaremos la teoría y
construcción comprendida en las redes neuronales.  Basaremos su
construcción en el perceptrón y el perceptrón multicapa. 

TODO: Añadir referencias a los capítulos cuando estén. 
% !TeX root = ../../tfg.tex
% !TeX encoding = utf8
%
%*******************************************************
% Qué es el aprendizaje automático
%*******************************************************

\section{Concepto de aprendizaje}\label{ch:Aprendizaje}

El término de Aprendizaje Automático 
\cite{hisour} 
fue acuñado en 1959 por Arthur Samuel 
para hacer referencia a los sistemas informáticos que 
pueden \textit{aprender} por sí mismos, es decir, mejorar su 
eficacia y rendimiento de forma autónoma a partir de los datos, 
sin que en esas mejoras intervenga un programador.

\begin{marginfigure}
    \includegraphics[width=\marginparwidth]{introduccion_redes_neuronales/aprendizaje_introduccion/arthur_samuel.jpg}
    \caption{Arthur L. Samuel (1901-1990) }
    \cite{samuel-wikipedia}
    \small
     se graduó en Ingeniería Electrónica en el MIT. 
     Trabajó en los laboratorios Bell, la Universidad de Illinois,
      en IBM y en la Universidad de Stanford (1966). 
      Fue un pionero de los videojuegos y la inteligencia artificial. 
      Popularizó el término \textit{Machine Learning} en 1959. 
      Implementó una IA para jugar a las damas, el primer caso  
      de éxito en aprendizaje automático. Contribuyó notablemente 
      en el desarrollo de TeX.
\end{marginfigure}

Fue en 1997 cuando Tom Mitchell propuso una definición 
formal de aprendizaje 
\cite{tom-michell-machine-learning}, 
aproximada a la dada en el libro \textit{Learning from data}
\cite{learning-from-data-1-2}, que expondremos en seguida.

El aprendizaje es un proceso por el cual se estima una dependencia desconocida 
(input-output) o la estructura de un sistema a partir de un número finito de 
observaciones. Se compone de tres elementos principales: 

\begin{itemize}
    \item Un generador o función de distribución de la cual se extraen 
    vectores aleatorios 
    $x \in I \subset \mathbb R^ n$ 
    que dependen de una función de densidad desconocida \footnote{De hecho, encontrar esta función resolvería el problema de aprendizaje}.
    
    \item Un sistema que produce un vector de salida $y$ por cada entrada del vector $x$ a partir del valor fijo $p(y|x)$, que es desconocido también. 
    
    \item Una \textit{learning machine} dependiente de parámetros $w$, que en el caso más general no es  más que un conjunto de funciones abstractas cuyos elementos son de la forma $f(x,w)$.
\end{itemize}

\begin{marginfigure}
    \includegraphics[width=\marginparwidth]{introduccion_redes_neuronales/aprendizaje_introduccion/tom_mitchell.jpg}
   \caption{Tom M. Mitchell}
   \small
    Tom M. Mitchell
     \cite{mitchell-wikipedia} 
    (1951) estudió Ingeniería Electrónica en el MIT, 
    se doctoró en la Universidad de Stanford y ejerce en la Universidad de Carnegie Mellon. 
    Es conocido por su libro 
    de texto Machine Learning. Ha recibido numerosas condecoraciones y 
   participa en asociaciones que promueven la ciencia y la ingeniería.
\end{marginfigure}



Así pues, el objetivo del aprendizaje automático es encontrar una función que se aproxime a la función de densidad desconocida.

Por lo general la teoría se fundamenta en minimizar el error de estimadores, como el error cuadrático medio, ya que este se trata de un UMVUE  (estimador insesgado de mínima varianza). 

$$ECM = \frac{1}{n} \sum_{i=0} ^n (f(x_i) - y_i)^2,$$

donde $f(x_i)$ representa la predicción y $y_i$ la etiqueta de entrenamiento de $x_i$, es decir, su valor real. 

% Componentes del aprendizaje   
\subsection{Componentes del aprendizaje}  
A nivel práctico y en nuestro caso, los elementos que consideraremos para el aprendizaje y los cuales 
serán susceptibles de contribuir a la optimización buscada son
(capítulo 1 \cite{MostafaLearningFromData}): 


\begin{itemize}
    \item Una entrada $x$. 
    \item Una función objetivo ideal y desconocida
     $f: \mathcal X \longrightarrow \mathcal{Y}$. 
     Donde  $\mathcal X$ es el espacio de entrada al que pertenecen las entradas $x$ y $\mathcal{Y}$ el espacio de salida. 
    \item Un conjunto de datos de entrenamiento $\mathcal D.$
    \item Un algoritmo de aprendizaje que consiste en seleccionar una fórmula $g: \mathcal X \longrightarrow \mathcal{Y}$ que aproxime $f$. La fórmula $g$ pertenece 
    a un conjunto $\mathcal H$ de hipótesis. 
\end{itemize}

En nuestro caso $\mathcal{H}$ será el conjunto de todas las posibles redes neuronales y $g$ la red neuronal seleccionada. 


% Tipos de aprendizaje 
\subsection{Tipos de aprendizaje}  

(Información proveniente del capítulo 1 \cite{MostafaLearningFromData})

El aprendizaje a partir de los datos consiste en el uso de un 
un conjunto de observaciones con el fin de descubrir un patrón, modelo o ley del proceso subyacente. 

En función de ciertas características del conjunto de datos se 
tienen distintos tipos de aprendizaje.  

\subsubsection{Aprendizaje supervisado}
Cuando el conjunto de datos de entrenamiento contiene de manera explícita lo que es una salida correcta respecto a una entrada estamos frente a un caso de \textbf{aprendizaje no supervisado}.   

Un ejemplo sería el reconocimiento de dígitos  manuscritos donde cada imagen de un dígito tiene asociado cuál es. 


\subsubsection{Aprendizaje por refuerzo}  
Se trata de un problema de aprendizaje por refuerzo 
cuando conjunto de datos de entrenamiento no explicita la salida, en su lugar contiene posibles salidas junto a la bondad de éstas. 

Pongamos por ejemplo que se quiere enseñar a un sistema a jugar
a las damas; de todo el espacio de jugadas posibles, se conocerían tan solo el resultado de algunas situaciones, por ejemplo en la que uno de los jugadores ha ganado.  

\subsubsection{Aprendizaje no supervisado}  

En este tipo de aprendizaje, los datos de entrenamiento tampoco contienen ninguna información de la salida.
 Tan solo se tienen los datos de entrada. El \textbf{aprendizaje no supervisado} 
 consiste en la tarea de encontrar patrones y estructuras en los datos de entrada, 
 así como de crear una una abstracción de los datos.  


Las redes neuronales son partícipes en los tres tipos de aprendizaje 
recién mencionados
\cite{8612259}, \cite{DBLP:journals/corr/BakerGNR16}, \cite{10.5555/2955491.2955578}. Sin embargo centraremos nuestro estudio en el caso 
de aprendizaje supervisado. 

Además, podríamos clasificar también el tipo de problemas en función del tipo de salida requerida (capítulo 4 pag 179
\cite{BishopPaterRecognition}), en regresión, clasificación o probabilistas. 
Dada una salida cualquiera, la adaptación de esta a un dominio concreto se consigue por medio de funciones 
no lineales, las cuales denotamos como \textbf{funciones de activación}. 


% Teoría de la aproximación 
% !TeX root = ../../tfg.tex
% !TeX encoding = utf8
%%%%
% OBJETIVOS SOBRE EL CAPÍTULO DE TEORÍA DE LA APROXIMACIÓN 
%%%%%%%%

\chapter{Teoría de la aproximación}
 \label{ch03:teoria-aproximar}

 Puesto que queremos fundamentar desde el origen las redes neuronales,
vamos a tratar de abordar el problema de aprendizaje  \ref{sec:Aprendizaje} desde resultados clásicos de teoría de la aproximación con el fin de analizar su carencias y virtudes (ver conclusiones \ref{ch03:conclusiones-teoria-aproximacion}). Para ello nuestro objetivo en esta sección será desarrollar la teoría necesaria para demostrar y analizar el teorema de Stone Weierstrass. 







% !TeX root = ../../tfg.tex
% !TeX encoding = utf8
%
%*******************************************************
% Polinomios de Bernstein
%*******************************************************

\section{Polinomios de Bernstein}\label{ch:Bernstein}  

%% Resumen capítulo 
En esta sección introduciremos los polinomios de Bernstein,   
que vistos como una serie nos asegurarán una convergencia uniformemente a cualquier 
función continua en un compacto y serán esenciales para nuestra prueba del teorema de Stone-Weierstrass, 
las pruebas se basan en el manual \cite{the-elements-of-real-analysis}.  

Comenzaremos recordando el Teorema del Binomio de Newton. 

%% Teorema Binomio de Newton

\begin{teorema}[Binomio de Newton]
    Cualquier potencia de un binomio $x+y$ con $x,y \in \R$,  puede ser expandido en una suma de la forma
    \[
        (x+y)^n = \sum_{k=0}^n \binom{n}{k} x^{k}y^{n-k}. 
    \]
\end{teorema}  
%%% Idea intuitiva y desigualdad 
Tomando ahora para esta igualdad $x \in \R, y= 1-x$ se tiene que 

\begin{equation}\label{eq:uno_igual_binomio}
    1 = (x+ (1-x))^n = \sum_{k=0}^n \binom{n}{k} x^{k} (1-x)^{n-k}.
\end{equation}

Dada cualquier función $f$ definida en $x$ podríamos multiplicar la ecuación 
\eqref{eq:uno_igual_binomio} por $f(x)$ resultando 

\begin{equation}\label{eq:f_igual_binomio}
    f(x) = \sum_{k=0}^n f(x) \binom{n}{k} x^{k} (1-x)^{n-k}.
\end{equation} 

Y tomando como dominio $I=[0,1]$ de $f: I \longrightarrow \R$,
 nos encontramos
frente a una ecuación muy sugerente para introducir $B_n(x)$, el \textit{Polinomio n-ésimo  de Bernstein }. 
El cual pretende  aproximar la función $f$ a través de los puntos $\frac{k}{n}$ con $n \in \N$ fijo
y $k \in \{0,...,n \}.$

\begin{definicion}[Polinomios de Bernstein] \label{def:Bernstein}
    Dada cierta función $f: [0,1] \rightarrow \R$, se define el n-ésimo polinomio de Bernstein para $f$ como 

    $$B_n(x) = B_n(x;f)=\sum_ {k=0}^{n} \left( f \left( \frac{k}{n} \right) \binom{n}{k} x^k (1-x)^{n-k} \right).$$

\end{definicion}

Faltaría por ver si efectivamente nuestro polinomio construido
 \textit{aproxima lo suficientemente bien} a la $f$ originaria. 
Basándonos en la igualdad \eqref{eq:f_igual_binomio} y 
la diferencia entre $f(x)$ y $B_n(x)$ se concluye que

%\begin{equation}
%    f(x)-B_n(x) = \sum_{k=0}^n \left(f(x) - f \left( \frac{k}{n} \right)\right)
%    \binom{n}{k} x^{k} (1-x)^{n-k}
%\end{equation} 

%Tomando valor absoluto resulta 
\begin{equation} \label{eq:berstein_diferencia}
    |f(x)-B_n(x)| \leq \sum_{k=0}^n \left|f(x) - f \left( \frac{k}{n} \right)\right|
    \binom{n}{k} x^{k} (1-x)^{n-k}
\end{equation} 

Ante tal igualdad la intuición ya nos
hace pensar que sea convergente al aumentar el tamaño de la partición.
 En efecto, en nuestro sucesivo teorema, nos cercioraremos que \ref{def:Bernstein}
 es uniformemente convergente a $f$ en un compacto. 

Para la demostración será necesario el siguiente resultado técnico: 
% Lema ecuación que se usa para una de las demostraciones del lema de Bernstein
\begin{lema}
    Se desea probar la siguiente igualdad 

    \begin{equation} \label{eq:binomio_segunda_suma}
        \frac{1}{n} x (1-x)= \sum_{k=0}^{n}  \left( x-\frac{k}{n} \right)^2  \binom{n}{k} x^{k} (1-x)^{n-k}.
     \end{equation}
\end{lema}
% Nota margen sobre el lema
\marginpar{
    \textcolor{dark_green}{    
        \textbf{
            Resultado técnico. 
        }
    }
    La demostración del lema \ref{eq:binomio_segunda_suma}
    se ha incluido para que todo resultado esté fundamentado, solo se utiliza para establecer una cota necesaria en una desigualdad del teorema \ref{teo:aproximacion_bernstein}.  
     
}
\begin{proof}

    %%% propiedades coeficientes binomial 

    Tengamos ahora presente las siguientes igualdades 
    \begin{equation} \label{eq:binomio_menos_uno}
        \binom{n-1}{k-1} = \frac{(n-1)!}{(k-1)! (n-1-(k-1))!} = \frac{k}{n} \binom{n}{k}
    \end{equation}
    \begin{equation} \label{eq:binomio_menos_dos}
        \binom{n-2}{k-2} = \frac{(n-2)!}{(k-2)! (n-2-(k-2))!} = \frac{k(k-1)}{n(n-1)} \binom{n}{k}
    \end{equation}

    Partiendo de la igualdad \eqref{eq:uno_igual_binomio}:
    \begin{equation}
        1 = (x+ (1-x))^n = \sum_{k=0}^n \binom{n}{k} x^{k} (1-x)^{n-k}.
    \end{equation}

    Reemplazamos la $n$ por $n-1$ y la $k$ por $j$ y tenemos 
    \begin{equation}
        1 = \sum_{j=0}^{n-1} \binom{n-1}{j} x^{j} (1-x)^{(n-1)-j}.
    \end{equation}
    Multiplicamos por $x$ y aplicamos la igualdad \eqref{eq:binomio_menos_uno} resultando 

    \begin{equation}
        x = \sum_{j=0}^{n-1} \frac{j+1}{n} \binom{n}{j+1} x^{j+1} (1-x)^{n-(j+1)}.
    \end{equation}

    Renombramos $k= j+1$, por lo que resulta
    \begin{equation}
        x = \sum_{k=1}^{n} \frac{k}{n} \binom{n}{k} x^{k} (1-x)^{n-k}.
    \end{equation}

    Como el término con $k=0$ es nulo podemos añadirlo a la sumatoria
    
    \begin{equation} \label{eq:desarrollo_binomio_uno}
        x = \sum_{k=0}^{n} \frac{k}{n} \binom{n}{k} x^{k} (1-x)^{n-k}.
    \end{equation}
    %------------------ caso 2, no te confundas --------------------
    Haremos ahora un razonamiento similar sustituyendo $n$ por $n-2$.
    Partiendo de \eqref{eq:uno_igual_binomio} se tiene que 
    \begin{equation}
        1 = (x+ (1-x))^n = \sum_{k=0}^n \binom{n}{k} x^{k} (1-x)^{n-k}.
    \end{equation}
    Reemplazamos la $n$ por $n-2$ y la $k$ por $j$ y tenemos 
    \begin{equation}
        1 = \sum_{j=0}^{n-2} \binom{n-2}{j} x^{j} (1-x)^{(n-2)-j}.
    \end{equation}
    Multiplicamos por $x^2$ y aplicamos la igualdad \eqref{eq:binomio_menos_dos} resultando 
    \begin{equation}
        x^2 = \sum_{j=0}^{n-2} \frac{(j+2)(j+1)}{n(n-1)} \binom{n}{j+2} x^{j+2} (1-x)^{n-(j+2)}.
    \end{equation}
    Renombramos $k= j+2$, por lo que resulta
    \begin{equation}
        x^2 = \sum_{k=2}^{n} \frac{k(k-1)}{n(n-1)} \binom{n}{k} x^{k} (1-x)^{n-k}.
    \end{equation}
    Como con los términos $k=0$ y $k=1$ se anula, podemos añadir dichos índices sin modificar la suma 
    \begin{equation}
        x^2 = \sum_{k=0}^{n} \frac{k(k-1)}{n(n-1)} \binom{n}{k} x^{k} (1-x)^{n-k}.
    \end{equation}
    Podemos reescribir la ecuación resultando \label{eq:desarrollo_binomio_dos}: 
    \begin{equation}
      (n^2 - n)  x^2 = \sum_{k=0}^{n} (k^2 - k) \binom{n}{k} x^{k} (1-x)^{n-k}.
    \end{equation}
    
    
%--------------- fin de las igualdades del binomio de Newton 

Sumando las dos expresiones que hemos obtenido
 \eqref{eq:desarrollo_binomio_uno} y \eqref{eq:desarrollo_binomio_dos}
 resultando 
 \begin{equation} 
    (n^2 - n)  x^2 + nx= \sum_{k=0}^{n} ((k^2 - k)+k) \binom{n}{k} x^{k} (1-x)^{n-k}.
  \end{equation}
  Dividimos todo entre $n$. 
  \begin{equation} \label{eq:binomio_tras_suma}
    (1 - \frac{1}{n})  x^2 + \frac{1}{n}x= \sum_{k=0}^{n} \left( \frac{k}{n} \right)^2 \binom{n}{k} x^{k} (1-x)^{n-k}.
  \end{equation}
  A continuación sumamos a la igualdad \eqref{eq:binomio_tras_suma} la ecuación \eqref{eq:uno_igual_binomio} multiplicada por $x^2$ y la ecuación \eqref{eq:desarrollo_binomio_uno}
  multiplicada por $-2x$ resultando: 
  \begin{equation} 
    \left(1 - \frac{1}{n} + 1 -2 \right)  x^2 + \frac{1}{n}x= \sum_{k=0}^{n} \left( \left( \frac{k}{n} \right)^2 + x^2 -2x \right) \binom{n}{k} x^{k} (1-x)^{n-k}.
  \end{equation}
  Sacando factor común en el miembro de la izquierda y escribiendo como un cuadrado el factor de la derecha,  resulta la igualdad \ref{eq:binomio_segunda_suma} que buscábamos
  \begin{equation} 
     \frac{1}{n} x (1-x)= \sum_{k=0}^{n}  \left( x-\frac{k}{n} \right)^2  \binom{n}{k} x^{k} (1-x)^{n-k}.
  \end{equation}

    
\end{proof}

\begin{teorema}[Teorema de aproximación de Bernstein]\label{teo:aproximacion_bernstein}

    Sea $f$ una función continua de un intervalo $I = [0,1]$ con imágenes en los reales. 
    La secuencia de polinomio de Bernstein
    \ref{def:Bernstein} converge uniformemente a $f$ en $I.$
    
\end{teorema}
Recordaremos antes la definición de convergencia uniforme: 

\begin{definicion}[Convergencia uniforme para funciones reales]

    Dado $E$ un conjunto y $\{f_n\}_{n \in \N}$ una sucesión de funciones de $E$
    a los reales, se dice 
    que dicha sucesión converge uniformemente si para cualquier $\varepsilon > 0$ existe un número natural $m$ tal que 
    para todo $x   \in E$ y cualquier natural $n$ que cumpla $n \geq m$ se tiene que 

    \begin{equation*}
        |f_n(x) - f(x) | < \varepsilon
    \end{equation*}
    
\end{definicion}

% Demostración de la convergencia de los polinomios de Bernstein
Comencemos pues con la demostración del teorema \ref{teo:aproximacion_bernstein}.
\begin{proof}
    
    Para cualquier $\varepsilon > 0$ queremos probar que existe un $m_\varepsilon  \in \N$ tal que para 
    todo $x \in I$ e $n \geq m_\varepsilon$  se tiene que 
    $|f(x) - B_n(x)| < \varepsilon$.
    
     Para ello por estar $f$ definida en un intervalo, 
    se tienen dos consecuencias claves: 
    \begin{enumerate}
        \item Está acotada, supongamos por $M \in \R$, esto es $|f(x)| \leq M$. \label{consecuencia:M}
        \item En virtud del teorema de Heine-Cantor $f$ es uniformemente continua, es decir, por estar $f$ definida en un compacto,  para cualquier $\varepsilon >0$ existirá un $\delta_\varepsilon$
        tal que para cualesquiera $x,y \in I$ que cumplan $|x-y| < \delta_\varepsilon$ entonces $|f(x)-f(y)| < \varepsilon$. \label{consecuencia:delta}
    \end{enumerate}
    %% Cota 2M para  cada sumando
    En virtud de la consecuencia \refeq{consecuencia:M}. 
    Dado $N \in  \N$ fijo pero arbitrario, para cualquier $k \in \{1, ..., N\}$ se tiene que
    $\frac{k}{N} \in I$ y tomando $x \in I$ podemos acotar por la desigualdad triangular

    $$\left|f(x)- f\left( \frac{k}{N} \right) \right| \leq |f(x)| + \left|f \left( \frac{k}{N}\right) \right|\leq 2M.$$  
    Por lo que 
    \begin{equation*}
        |f(x)-B_n(x)| \sum_{k=0}^n \left|f(x) - f \left( \frac{k}{n} \right)\right| \leq
     \binom{n}{k} x^{k} (1-x)^{n-k}.
    \end{equation*}

    Puesto que tenemos que $f$ acotada por $M$ y es uniformemente continua, para valores de $k$ tales que $\frac{k}{n}$ 
    esté próxima a $x$, tal término de la sumatoria será pequeño por la continuidad de $f$ en $x$. Por otro lado
    si está lo suficientemente alejado, tan solo podremos acotar tal término por $2M$.  

    Para $\varepsilon > 0$ y para $\delta_\varepsilon$ de la definición de continuidad uniforme de $f$ 
    podemos encontrar un 
    \begin{equation} \label{eq:cota-de-la-n}
        n \geq \sup \left\{ (\delta_\varepsilon)^{-4}, \frac{M^2}{ \varepsilon^2}\right\}.
    \end{equation}
    Separaremos pues nuestra sumatoria en los siguientes dos conjuntos. 
    % Conjunto de los que son menores
    \begin{equation}
        \mathcal{A}_{n x} = \left\{ k \text{ tales que } k \in \{0,..., n\} \text{ y  } \left|x - \frac{k}{n}\right| < n^{\frac{-1}{4}} \leq \delta_\varepsilon \right\},
    \end{equation}
    % Conjunto de los que son mayores
    \begin{equation}
        \mathcal{B}_{n x} = \{0,..., n\} - \mathcal{A}_{n x}. 
    \end{equation}

     Para los elementos de $\mathcal{A}_{n x}$ se obtiene la siguiente estimación: 

     \begin{equation*}
        \begin{split}
        \sum_{k \in \mathcal A } \left|f(x) - f \left( \frac{k}{n} \right)\right|
     \binom{n}{k} x^{k} (1-x)^{n-k}
     \leq 
     \sum_{k \in \mathcal{A}_{n x} } \varepsilon \binom{n}{k} x^{k} (1-x)^{n-k} 
     =  \\
      = \varepsilon \sum_{k \in \mathcal{A}_{n x} }  \binom{n}{k} x^{k} (1-x)^{n-k} 
     \leq 
     \varepsilon \sum_{k = 0} ^ n  \binom{n}{k} x^{k} (1-x)^{n-k} = 
     \varepsilon
        \end{split}
    \end{equation*}

    Para el resto de sumandos para los que $|x - \frac{k}{n}| \geq  n^{\frac{-1}{4}}$ se tiene que 
    $(x - \frac{k}{n})^2 \geq  n^{\frac{-1}{2}}$ cota. Procederemos a acotar tales términos, para ello partimos de la ecuación (\refeq{eq:berstein_diferencia}) y  la acotación de $f$. 

    \begin{equation*}
        |f(x)-B_n(x)| \sum_{k=0}^n \left|f(x) - f \left( \frac{k}{n} \right)\right|  
     \leq \sum_{k=0}^n 2M 
    \binom{n}{k} x^{k} (1-x)^{n-k}
    \end{equation*}
    
    \begin{align*}
        & \sum_{k \in \mathcal{B}_{n x}} 2M 
    \binom{n}{k} x^{k} (1-x)^{n-k}
    \\ 
    & 
    = 2M \sum_{k \in \mathcal{B}_{n x}} 
    % multiplicamos y dividimos por diferencias 
    \frac{
        \left(x- \frac{k}{n}\right)^2
    }{
        \left(x- \frac{k}{n}\right)^2
    }
    \binom{n}{k} x^{k} (1-x)^{n-k}
    \\
    &
    % cota de la fracción introducida 
    % y más sumandos
    \leq 2M 
    \sqrt{n}
    \sum_{k \in \mathcal{B}_{n x}} 
    \left(x- \frac{k}{n}\right)^2
    \binom{n}{k} x^{k} (1-x)^{n-k}
    \\
    & % cota del lema
    \leq
    2M \sqrt{n}
    \left(
        \frac{1}{n} x(1-x)
    \right)
    \leq 
    \frac{M}{2 \sqrt{n}},
    \end{align*}

    Donde para la penúltima desigualdad se ha usado 
    la desigualdad (\refeq{eq:binomio_segunda_suma}) del lema previo y para la última que en el intervalo $I$ se satisface que 
    \begin{equation*}
        x (1-x) \leq \frac{1}{4}.
    \end{equation*}

    Además recordemos que se puede tomar un  $n$ que satisfazca (\refeq{eq:cota-de-la-n}) y entonces se concluye que para los valores de
    $\mathcal{B}_{n x}$ se puede acotar la desigualdad por $\varepsilon$. 
    
    Por tanto para un $n$ convenientemente seleccionado, se ha acotado la desigualdad para los  indices $\mathcal{A}_{n x}$ y $\mathcal{B}_{n x}$, es decir todos, por lo que concluimos que 
    \begin{equation*}
        |f(x) - B_n(x)| \leq 2 \varepsilon,
    \end{equation*}
    independiente del valor de $x$, por lo que se prueba que la secuencia de polinomios de Bernstein converge uniformemente a $f$ en $I$.
\end{proof}
% !TeX root = ../../tfg.tex
% !TeX encoding = utf8
%
%*******************************************************
% Teorema de Aproximación Weierstrass 
%*******************************************************
% Nota margen sobre Idea intuitiva homeomorfismo
\marginpar{\maginLetterSize
\iconoAclaraciones \textcolor{dark_green}{     
\textbf{¿Qué es un homeomorfismo?}}.

    A nivel intuitivo, un homeomorfismo sería 
    la transformación de un objeto en otro,
    con la condición de que tal cambio consista
    en una deformación \textit{sin roturas}. 
    Es decir, si éste  fuera de plastilina, 
    existiría un homeomorfismo entre la figura 
    original y 
    aquellas que pudiéramos construir 
     sin separar totalmente la masilla
    o crear y eliminar agujeros. 
    Ejemplo de objetos homeomorfos serías:
    un pelota y un vaso, 
    una taza y un rosco.
    El vaso y la taza no serían homeomorfos, 
    ya que el número de agujeros que tienen es distinto. 
}

Realizando un repaso global habiendo acabado el teorema,
 se pueden extraer que, junto a un ingenioso 
 manejo de operaciones y acotaciones; la clave del resultado reside  en las consideraciones
en \refeq{consecuencia:M} y \refeq{consecuencia:delta} y estas a su vez en la 
compacidad de $I$.

Por su parte, la selección del dominio de $I = [0,1]$ viene determinada ya que 
 los nodos $\{ \frac{k}{N} \colon k\in \{0,..., N\}\}$ sobre los que se construye el \textit{N-ésimo polinomio de Bernstein}  deben pertenecer a $I$.

Sin embargo, tal dificultad es fácilmente salvable con un homeomorfismo. 



Como resultado de relajar el dominio donde se define $f$, pidiéndole tan solo
compacidad nace el siguiente corolario.  

\begin{corolario}[Teorema de aproximación de Weierstass] \label{teo:Teorema-Weierstrass}
    Sea $f$ una función continua definida en un intervalo cerrado y acotado y con valores reales. Se tiene que $f$ puede ser aproximada uniformemente con polinomios. 
\end{corolario}  

\begin{proof}
    Si $f$ se encuentra definida en $[a,b]$ con $a<b$ y bastará considerar la función
    \begin{equation*}
        g(t) = f( (b-a)t + a) \text{ con } t \in [0,1].
    \end{equation*}
 Esta función está definida en $[0,1]$, tiene la misma imagen que $f$ y 
    mantiene la continuidad ya que se ha construido a través del homeomorfismo 
    $H:[0,1] \longrightarrow [a,b]$, con $H(t) = (b-a)t + a$. 

    En virtud del teorema de convergencia \ref{teo:aproximacion_bernstein}
    $g$ podrá ser aproximada uniformemente por una sucesión de polinomios de Bernstein $\{S_n\}_{n \in \N}$. Gracias a ella se puede
    construir la sucesión $\{S_n \circ H^{-1}\}_{n \in \N}$, que aproxima uniformemente a $f$. 
\end{proof}
% !TeX root = ../../tfg.tex
% !TeX encoding = utf8
%
%*******************************************************
% Teorema de Stone Weiertrass 
%*******************************************************

\section{Teorema de Stone-Weierstrass }\label{ch:TeoremaStoneWeiertrass}

\begin{teorema}[Teorema de Stone-Weierstrass] 

    Sea $K$ un subconjunto compacto de $\R^p$ y sea $\mathcal{A}$ una colección de 
    funciones continuas de $K$ a $\R$ cumpliendo $\mathcal{A}$ y 
    que separa puntos en $K$, es decir cumpliendo las siguientes propiedades: 

    \begin{enumerate}
        \item La función constantemente uno, definida como $e(x)=1$, para cualquier $x\in K$ pertenece a $\mathcal{A}$.
        \item $\mathcal{A}$ es cerrado para sumas y producto para escalares reales. Si $f,g$ pertenece a  $\mathcal{A}$, entonces $\alpha f + \beta g$ pertenece a $\mathcal{A}$ . 
        \item $\mathcal{A}$ es cerrado para productos. Para $f,g \in \mathcal A$, se tiene que $fg$ pertenece a $\mathcal{A}$. 
        \item Separación de $K$, es decir, si $x \neq y$ pertenecen a $K$, entonces existe una función $f$ en $\mathcal{A}$  de tal manera que $f(x) \neq f(y)$. 
    \end{enumerate}
    
    Se tiene que toda función continua de $K$ a $\R$ puede ser aproximada en $K$ por funciones de $\mathcal A$. 

\end{teorema}  

\begin{proof}
    Comenzaremos probando que para cualquier  $f \in \mathcal{A}$, 
    la función $|f|$ se puede aproximar todo lo que queramos por elementos de $\mathcal{A}$, 
    esto 
    Para cualquier $\varepsilon > 0$ y $f \in \mathcal{A}$, 
    existe $g \in \mathcal{A}$  tal que 
    \begin{equation*}
        |g(x) - |f(x)|| < \varepsilon \quad \forall x \in K.
    \end{equation*}

     Por el teorema de Heine, para $f \in \mathcal A$ está acotada por tomar imagen en un compacto, es decir $|f(x)| \leq M$ para $x \in K.$  

    Consideremos ahora la función valor absoluto, $\phi(t)=|t|$ definida en el dominio $I = [-M, M].$
    Por el teorema de aproximación de Weierstrass 
    \ref{teo:Teorema-Weierstrass}
    para cualquier $\varepsilon > 0$ 
    existirá un polinomio $p$ cumpliendo que 
    $$||t|- p(t)| < \varepsilon \quad \forall t \in I.$$

    Puesto que $t \in I$ no son más que las posibles imágenes que puede tomar $f$ en $K$ inferimos entonces que 

    $$||f(x)| - p \circ f(x)| < \varepsilon \quad \forall x \in K.$$

    Como $f \in \mathcal{A}$ y $p$ es un polinomio, es decir, 
     $p \circ f(x)$ son sumas de potencias multiplicadas por escalares de $f(x)$, luego por la hipótesis de ser cerrado a estas operaciones tenemos que la función 
    $g = p \circ f$ pertenece a $\mathcal{A}$ de donde se deduce  
    $|f|$ se puede aproximar todo lo que queramos por elementos de $\mathcal{A}$. 

    Tenemos con esto que para $f,g \in \mathcal{A}$ también es capaz de aproximar la función
     a supremo e ínfimo  gracias a que:
    \begin{align} \label{eq:cerrado-min-max}
        & \max\{f,g\} = \frac{1}{2} \{f+g+ |f+g|\} \\
        & \min \{f,g\} = \frac{1}{2} \{f+g -|f+g|\}
    \end{align}   

 
    
    Vamos a proceder ahora con nuestro objetivo 
    principal. 
    Queremos probar que para cualquier función continua $f: K \longrightarrow \R$ y cualquier $\varepsilon > 0$ existe $g \in \mathcal{A}$ tal que 

    \begin{equation}
        |f(x) - g(x)| < \varepsilon \quad \forall  x \in K. 
    \end{equation}

    Tomamos $s,t \in K$ distintos y por la hipótesis de separación de puntos existirá $u \in \mathcal{A}$ tal que $u(s) \neq u(t)$, de esta forma definimos 
    \begin{equation}
        \tilde{h}_{s t} = 
            f(s) + (f(t) - f(s))\frac{ x - s}{ t - s} 
    \end{equation}
    Notemos que  $\tilde{h}_{s t}$ satisface que 
    $\tilde{h}_{s t}(s) = f(s)$ y $\tilde{h}_{s t}(t) = f(t)$. 
    
    Tomamos $s \in K$ fijo pero arbitrario y 
    definimos el siguiente conjunto 
    \begin{align}
        U_t = 
        \{
            u \in K \quad |  \quad
            \tilde{h}_{s t}(u) < f(u) + \varepsilon
        \}.
    \end{align}
    Notemos que $U_t$ es un conjunto abierto ya que 
    $\tilde{h}_{s t} - f$ es una función continua
    y se está tomando la imagen inversa de un abierto. 
    Además se tiene que $\{t,s\} \subset U_t$ para cualquier
    $t \in K \setminus \{s\}$. 
    % Escribimos como recubrimiento finito
    Por lo que podemos escribir 
    \begin{equation*}
        K = \bigcup_{t \in K \setminus \{s\}} U_t
    \end{equation*}
    y por ser $K$ compacto admitirá un recubrimiento finito dado por 
    $I_u = \{t_1, \ldots, t_n\} \subset K$, es decir 
    \begin{equation}\label{subrecubrimiento_t}
        K = \bigcup_{t \in I_u} U_t.
    \end{equation}
    Definimos ahora 
    \begin{equation*}
        h_s(x) = \max_{t \in I_u}\{ 
            \tilde{h}_{s t}(x) 
            \quad | \quad
            x \in U_t
        \} 
        \text{ para cada } x \in K. 
    \end{equation*}
    Así definida $h_s$ satisface que: 
    \begin{itemize}
        % Desigualdad 
        \item $\text{Para cada } x \in K$ se tiene que $h_s(x) < f(x) + \varepsilon$.
        
        Ya que por 
        (\refeq{subrecubrimiento_t}) 
        $h_s(x) =  \tilde{h}_{s t_j}(x)$ y 
        como $x \in U_{t_j}$ entonces 
        $\tilde{h}_{s t_j}(x) < f(x) + \varepsilon$.
        % continuidad 
        \item La función $h_s$ es continua.
        
        Queremos ver que para todo $\epsilon > 0$, existe un $\delta >0$ cumpliendo que 
        si $ |x-y| < \delta$ entonces $h_s(x)-h_s(y)$
        Fijamos de manera arbitraria $x \in K$, 
        por 
        (\refeq{subrecubrimiento_t}) 
        $h_s(x) =  \tilde{h}_{s t_j}(x)$ que es continua en $K$. 

        % Pertenece  a A
        \item La función $h_s$ pertenece a $\mathcal{A}$, por ser el máximo de funciones que pertenecen a $\mathcal{A}$ (\refeq{eq:cerrado-min-max}). 
    \end{itemize} 
     

    % Vamos a definir el conjunto pa sacar la otra cota. 
    Y definimos para cada $s \in K$ el conjunto 
    \begin{equation}\label{subrecubrimiento_s}
        V_s = \{
            v \in K  \quad |  \quad
            h_s(x) > f(x) - \varepsilon
            \}.
    \end{equation} 

    Y repitiendo el mismo argumento que para $U_t$ puede verse que $K$ admitirá un 
    subrecubrimiento finito por abiertos: 
    \begin{equation*}
        K = \bigcup_{s \in I_s} V_s
    \end{equation*}
    con $I_v = \{s_1, \ldots, s_m\} \subset K$. 
    % definimos ahora la función graciosa 
    Además $g$ definida como: 
    \begin{equation}
        g(x)= \min_{s \in I_v}
        \{ 
            h_{s}(x) 
            \quad | \quad
            x \in V_s
        \} 
    \end{equation}
    satisface:  
    \begin{itemize}
        \item La función $g$ es continua (mismo argumento que antes). 
        % g acotado inferiormente
        \item $\text{Para cada } x \in K$ se tiene que $g(x) >f(x) - \varepsilon$.
        
        Ya que por 
        (\refeq{subrecubrimiento_s}) 
        $g(x) =  h_{s_i}(x)$ y 
        como $x \in V_{s_i}$ entonces 
        $h_{s_i}(x) > f(x) -\varepsilon$.
        % g acotado superiormente
        \item $\text{Para cada } x \in K$ se tiene que $g(x) <f(x) +\varepsilon$.
        
        Ya que por 
        (\refeq{subrecubrimiento_s}) 
        $g(x) =  h_{s_i}(x)$ y 
        en las propiedades de $h_s$ habíamos visto que
        $h_{s}(x) < f(x) +\varepsilon$ para $s$ arbitrario.

         % Pertenece  a A
         \item La función $g$ pertenece a $\mathcal{A}$, por ser el mínimo de funciones que pertenecen a $\mathcal{A}$ (\refeq{eq:cerrado-min-max}).
    \end{itemize}
    De estas observaciones se tiene lo buscado, 
    existe $g \in \mathcal{A}$ tal que 
    \begin{equation}
        |g-f| \leq \varepsilon;
    \end{equation}
    es decir $f$ puede ser aproximada por elementos de $\mathcal{A}$. 

    \textcolor{red}{
        Hay detalles que no me terminan de convencer. 
        Como que las aproximaciones no pertenecen a A.
    }
\end{proof}

%%%%%%%%%%%%%%%%%%%%%%%%%%%%%%%%%%%%%%%%%%%%%%%%%%%%%%%%%%%
% Conclusiones teoría de la aproximación 
%%%%%%%%%%%%%%%%%%%%%%%%%%%%%%%%%%%%%%%%%%%%%%%%%%%%%%%%%%%
\section{Conclusiones teoría de la aproximación} 

Acabamos de probar en \ref{ch:TeoremaStoneWeiertrass} que cualquier función 
continua es aproximable uniformemente con polinomios en un compacto. 
Sin embargo este enfoque tiene los siguientes problemas: 

\begin{enumerate}
    \item La prueba obtenida no nos permite 
    una forma constructiva sencilla de obtener el polinomio. 
    \item Aproximando con polinomios se corre el riesgo de que fuera de la muestra 
    el error sea demasiado grande. Como muestra de ello damos un ejemplo patológico 
    reflejado en la figura \ref{fig:aproximacion-lagrage}; se pretende aproximar 
    la función $f:[-3,3] \longrightarrow \R$ definida como
    \begin{equation*}
        f(x)= \left\{ \begin{array}{lcc}
            e^{-x} + 4 &   si  & x \leq 1 \\
            \\ \log{x} &  si  & x > 1
            \end{array}
  \right.
    \end{equation*}
    usando el método de interpolación de Lagrange; en este caso el error tiende a 
    infinito conforme aumenta el número de nodos. \footnote{Puede encontrar la implementación
    del código que genera las gráficas en nuestro repositorio 
    \url{https://github.com/BlancaCC/TFG-Estudio-de-las-redes-neuronales}, 
    en el fichero \texttt{Lagrange.ipynb} que se encuentra  en el directorio de teoría de la aproximación de la memoria. } 

    \item Y otra cuestión, de corte físico o  filosófico ¿son todos los fenómenos observables continuos? 
    Sería extraño que así fueran, lo que evidencia la necesidad de formular una teoría más general. 
\end{enumerate}

\begin{figure}[H]
    \centering
    \begin{subfigure}[b]{0.475\textwidth}
        \centering
        \includegraphics[width=\textwidth]{metodo-lagrange/lagrange-3-datos.png}
        \caption[Network2]%
        {{\small Polinomio de Lagrange utilizando 3 datos}}    
    \end{subfigure}
    \hfill
    \begin{subfigure}[b]{0.475\textwidth}  
        \centering 
        \includegraphics[width=\textwidth]{metodo-lagrange/lagrange-8-datos.png}
        \caption[]%
        {{\small Polinomio de Lagrange utilizando 8 datos}}    
    \end{subfigure}
    \vskip\baselineskip
    \begin{subfigure}[b]{0.475\textwidth}   
        \centering 
        \includegraphics[width=\textwidth]{metodo-lagrange/lagrange-13-datos.png}
        \caption[]%
        {{\small Polinomio de Lagrange utilizando 13 datos}}    
    \end{subfigure}
    \hfill
    \begin{subfigure}[b]{0.475\textwidth}   
        \centering 
        \includegraphics[width=\textwidth]{metodo-lagrange/lagrange-18-datos.png}
        \caption[]%
        {{\small Polinomio de Lagrange utilizando 18 datos}}    
    \end{subfigure}
    \caption{Ejemplo de aproximación de la función $f(x)$ a partir de los polinomios de Lagrange.} 
    \label{fig:aproximacion-lagrage}
\end{figure}

El problema que evidencia este caso patológico es el tratar de abarcar todo el dominio 
con un mismo polinomio ¿y si en lugar de eso se hicieran aproximaciones 
en una partición concreta del dominio? El resultado sería una función definida a trozos.   
La cuestión es que esta aproximación sería difícil de implementar de manera eficiente;
sin embargo, es el germen y el enfoque de las \textit{funciones de activación}. 

De todas formas no abandonemos del todo esta teoría, porque como ya veremos el 
teorema de Stone Weierstrass  \ref{ch:TeoremaStoneWeiertrass} jugará 
un papel fundamental es la demostración 
de que las redes neuronales son aproximadores universales.

\subsection*{Las funciones de activación $\Gamma$ son la clave del aprendizaje} 

\label{ch03:funcionamiento-intuitivo-funcion-activacion}

Las \textit{funciones de activación} serán definidas con profundidad en la sección 
\ref{def:funcion_activacion_articulo}, pero para continuar con nuestro razonamiento 
pensemos en ellas como una función cualquiera que no sea un polinomio. 

Una vez liberados de tratar de buscar un polinomio que aproxime la función en todo
el dominio, podemos pensar en aproximar la image de acorde a intervalos.  
 
\textcolor{red}{TODO : Añadir gráficos cuando esté implementada una red neuronal}

% Ejemplo de cómo se aproxima gracias  a la forma de la función de activación
\begin{figure}[h!]
    \includegraphics[width=\textwidth]{1-Introduccion_redes_neuronales/idea-como-aproxima-redes-neuronales.jpeg}
    \caption{Cómo actúa en la aproximación una función de activación}
    \label{img:idea-como-aproxima-redes-neuronales}
   \end{figure}

La idea intuitiva es que para una capa oculta con una neurona, 
lo que se hace es \textit{colocar} por escalado y simetrías la imagen de la función de activación. 

% Ejemplo trivial de como la forma de la función de activación influye en aproximar mejor 
\begin{figure}[h!]
    \includegraphics[width=0.8\textwidth]{1-Introduccion_redes_neuronales/Idea-forma-función-Activación.jpg}
    \caption{Cómo afecta la forma de la función de activación}
    \label{img:como afecta la forma de la función de aproximación}
\end{figure}


% Las redes neuronales multicapa son aproximadores universales 
\chapter{Las redes neuronales  son aproximadores universales}  
\label{chapter:redes-neuronales-aproximador-universal}
% !TeX encoding = utf8
%
%*******************************************************
% Construcción redes neuronales  una capa 
%*******************************************************

\section{Definición de las redes neuronales \textit{Feedforward Networks} 
de una capa oculta} \label{sec:redes-neuronales-intro-una-capa}

% Nota margen aclarativa de una función medible
\reversemarginpar
\setlength{\marginparwidth}{\smallMarginSize}
\marginpar{\maginLetterSize
    \iconoAclaraciones \textcolor{dark_green}{     
        \textbf{Qué son las funciones medibles 
        y porqué las usamos en nuestra definición.}
    }
    {\maginLetterSize
        A nivel intuitivo una función medible es aquella,
        que por muy extraña que sea,  
        su imagen (los valores que toma su salida) está acotada casi siempre, lo que a nivel práctico 
        significa que \textit{podemos observar y cuantificar sus valores.}
    
        Con esto pretendemos que nuestra definición $\Gamma$ 
        sea lo menos restrictiva posible.
    }
}
\normalmarginpar
\setlength{\marginparwidth}{\bigMarginSize}


A lo largo de esta sección  explicaremos qué es una red neuronal, cómo está construida y en qué consiste el \textit{aprendizaje} de la misma, concretamente
construiremos el tipo particular \textit{Feedforward Neural Networks}, al cual nos referiremos de ahora
en adelante como red neuronal.

De acorde con nuestra filosofía de trabajo expuesta en la introducción del capítulo \ref{motivo-una-capa} partiremos de un modelo de una sola capa oculta. 

% Imagen grafo red neuronal  una capa oculta muy simple y en blanco y negro 
\begin{figure}[h!]
    \centering
    \includegraphics[width=0.85\textwidth]{1-Introduccion_redes_neuronales/Red-Neuronal-una-capa-simple.png}
    \caption{\textit{Grafo} de una red neuronal de una capa oculta}
    \label{img:grafo-red-neuronal-una-capa-oculta}
\end{figure}

\begin{definicion}[Redes neuronales de una capa oculta] \label{definition:redes_neuronales_una_capa_oculta}
    Dados $X \subseteq \R^d, Y \subseteq \R^s$ y  $\Gamma$ un conjunto no vacío de funciones medibles definidas de $\R$ a $\R$, denotaremos como 
    \begin{align}
        \mathcal{H}(X,Y) 
        =
        \{
            h : X \longrightarrow Y 
            /& \quad 
            h_k(x) = 
            \sum_{i=1}^{n} \beta_{i k} \gamma_{i}( A_{i}(x)), \\
            & \text{donde  $h_k$  es la proyección k-ésima de $h$ con 
            $k \in \{1, \ldots, s\}$}, \\
            & n \in \N,\gamma_{i} \in \Gamma , \beta_{i k} \in \R
             \text{ y }A_{i} \text{ una aplicación afín de $\R^d$ a $\R$}           
        \}.
    \end{align}
\end{definicion}
% Nota margen aclarativa de la fórmula
\marginpar{\maginLetterSize
    \iconoAclaraciones \textcolor{dark_green}{     
        \textbf{Interpretación fórmula}
    }
    {\maginLetterSize
        Observemos que $n$ es el número de neuronas de la capa oculta. Es decir lo que en el grafo \ref{img:grafo-red-neuronal-una-capa-oculta} serían las neuronas de las capas ocultas se correspondería con términos $\gamma_{i}( A_{i}(x))$.
    }
}
\normalmarginpar
\setlength{\marginparwidth}{\bigMarginSize}


Es habitual representar una red neuronal de forma matricial, veremos que tal forma es equivalente a la definición dada. 

Consideramos la aplicación inclusión 
$i: \R^r \longrightarrow \R^{r+1}$ dada por 
 $i((x_1, \ldots, x_d)) = (1,x_1, \ldots, x_d).$
Para coeficientes $w_i \in \R$ toda función afín es de la forma 
$A_{i}(x)= \sum_{j=1}^d( w_{j i} x_j) + w_{0i}$, 
tomando $W_i = (w_{0 i}, w_{1 i}, \ldots, w_{r i}) \in \R^{r+1}$ tenemos que 
$A_i(x) = W_i \cdot i(x)$ como queríamos probar. 

Además también se suele mostrar de manera pedagógica con un grafo como el mostrado en la 
figura \ref{img:grafo-red-neuronal-una-capa-oculta}.

% Elimino TODO porque ya se ha comentado.

\subsection*{Componentes de la red neuronal}  

A la vista de la definición dada notemos que cada elemento de 
$\mathcal{H}(X,Y)$ viene determinado por los coeficientes 
de las distintas $\beta$ y  $w$s de la función afín. Como veremos más adelante estos son los valores que \textit{aprende una red neuronal}.

\subsection*{Diferencia con otras definiciones}  \label{subsection:diferencia-otras-definiciones-RRNN}

En otros textos como en el capítulo cinco, páginas 227-256 del libro \cite{BishopPaterRecognition} y las notas online sobre redes neuronales de \cite{MostafaLearningFromData} se presentan las redes neuronales de una capa como 
\begin{align}
    y_k(x,w) &= \theta_k 
    \left( 
        \sum^M_{j=1} w_{ji}^{(2)}
        \sigma_j 
        \left(
            \sum_{i=1}^D w_{ji}^{(1)} x_i + w_{j0}^{(1)}
        \right)
        + w_{k0}^{(2)}
    \right) 
    \\
    & = 
    \theta_k 
    \left( 
        \sum^M_{j=1} A^{(n_k)}_{k}
        \left(
            \sigma 
            \left(
                A^{r}_{j k}
                \left(
                    x
                \right)
            \right)
        \right)
    \right)
    \text{ para cada  } k \in \{1, \ldots, K \}.
\end{align}

Donde $\theta_k$ representa una función de \textit{clasificación}, 
$\sigma_j$ lo que se suele llamar \textit{función de activación} y a la que además se le exige que sea diferenciable.

Las diferencias con nuestra definición son las siguientes 
\begin{itemize}
    \item \textbf{Desaparece la función de clasificación $\theta$}. El motivo es que es un artificio teóricamente innecesario de acorde al teorema de convergencia universal \ref{teo:MFNAUA}.
    \item \textbf{Se elimina un parámetro} de la transformación afín de la última capa, puesto que no es necesario para la convergencia de nuevo por \ref{teo:MFNAUA} lo hemos eliminado.
    \item Nuestras funciones de activación son funciones medibles en vez de diferenciables ya que a priori no existe ninguna hipótesis teórica que fuerce a tal restricción.
\end{itemize}

\subsection*{Las funciones de activación $\Gamma$ son la clave del aprendizaje}  

Notemos que de no ser por las funciones de activación se estarían haciendo transformaciones lineales de los datos, por el contrario estamos realizando \textit{cambios más fuerte} siendo capaz con esto de \textit{de diferenciar puntos claves}.

\textcolor{red}{Dicho de esta manera queda muy poco claro y habría que profundizar más.}
Vamos a mostrar un ejemplo de la importancia de la función de activación 

\textcolor{red}{TODO : Añadir gráficos cuando esté implementada una red neuronal}


% Ejemplo de cómo se aproxima gracias  a la forma de la función de activación
\begin{figure}[h!]
    \includegraphics[width=\textwidth]{1-Introduccion_redes_neuronales/idea-como-aproxima-redes-neuronales.jpeg}
    \caption{Cómo actúa en la aproximación una función de activación}
    \label{img:idea-como-aproxima-redes-neuronales}
   \end{figure}

La idea intuitiva es que para una capa oculta con una neurona, 
lo que se hace es \textit{colocar} por escalado y simetrías la imagen de la función de activación. 

% Ejemplo trivial de como la forma de la función de activación influye en aproximar mejor 
\begin{figure}[h!]
    \includegraphics[width=0.8\textwidth]{1-Introduccion_redes_neuronales/Idea-forma-función-Activación.jpg}
    \caption{Cómo afecta la forma de la función de activación}
    \label{img:como afecta la forma de la función de aproximación}
   \end{figure}

% IDEA
\iconoAclaraciones \textcolor{dark_green}{ Nota idea Blanca: Multiplicar por un coeficiente complejo sería aplicar un giro,
 lo que a priori mejoraría la convergencia
 ¿Qué pasaría se cambiáramos el cuerpo?}

Ante esta idea habría que plantearse si: 

\begin{enumerate}
    \item Formalizar si esto mejoraría.
    \item A nivel de implementar Julia tiene números complejos ¿cuánto supondría de coste computacional? ¿merecería la pena?
    \item ¿Cómo habría que actualizar los pesos?
\end{enumerate}

% IDEA
\iconoAclaraciones \textcolor{dark_green}{ Nota idea Blanca: Vista esta idea sería muy interesante plantearse a partir de los datos cómo deberían de ser las funciones de activación. Es decir, que no se fijen a priori, sino que sean de los datos donde provenga su forma.}

Para poder utilizarse la idea que acaba de plantearse debería de plantearse: 
\begin{enumerate}
    \item Formalizar beneficio teórico.
    \item Balanza costo y mejora.
    \item Una forma de mejorar que acepte esas funciones (no necesariamente \textit{backpropagation}).
\end{enumerate}

% IDEA
\iconoAclaraciones \textcolor{dark_green}{ Nota idea Blanca: Ante esto mi intuición me dice que que la función de activación sea constante por algún extremo y que no fuera constante en algún intervalo serían las únicas hipótesis para que tal función de activación sirviera para construir redes neuronales que funcionara como aproximador universal}



%%%%%%%%%%%%%%%%%%%%%%%%%%%%%%%%%%%%%%%%%%%%%%%%%%%%%%%%%%%%%%%%%%%%%%%%%%%%%%
%
% Introducción sobre algoritmos de cálculo de los pesos de una red neuronal 
%
%%%%%%%%%%%%%%%%%%%%%%%%%%%%%%%%%%%%%%%%%%%%%%%%%%%%%%%%%%%%%%%%%%%%%%%%%%%%%%

\chapter{Explicitación del cálculo de una red neuronal}

Se han concretado en los capítulos anteriores que con con redes neuronales 
podemos aproximar funciones medibles, la cuestión ahora reside en ¿cómo se consigue tales 

Ya expusimos en la sección \ref{algoritmo-forward-propagation} cómo evaluar
una red neuronal, la cuestión entonces radica en ¿cuál es la red neuronal
buscada? es decir a nivel práctico sería ¿cómo puedo calcular matrices de pesos 
adecuadas?

Explicaremos en este capítulo el método tradicional y ampliamente usado de 
\textit{backpropagation} \ref{algoritmo-de-backpropagation} y otros más novedosos como
\textcolor{red}{TODO : añadir otros métodos de actualización de pesos. La idea sería que 
los buscados tengan un coste menor para así optimizar la velocidad de aprendizaje.}


% !TeX root = ../../tfg.tex
% !TeX encoding = utf8
%
%*******************************************************
% Contenido del artículo 1: Definiciones primeras
%*******************************************************

\section{Definiciones primeras}\label{ch:articulo:sec:defincionesPrimeras}  

Comenzaremos presentando definiciones básicas sobre redes neuronales. 


\begin{definicion}[Función de activación ] \label{def:funcion_activacion_articulo}
    Una función  $\psi: \R \longrightarrow [0,1]$ es una \textbf{ función de activación} si  cumple las siguientes propiedades:
    \begin{enumerate}[label=(\roman*)]
        \item Es no decreciente.
        \item $\lim _{x \rightarrow \infty} \psi(x) = 1
        $.
        \item $\lim _{x \rightarrow -\infty} \psi(x) = 0$.
    \end{enumerate}  
   
    Ejemplos comunes de funciones de activación son

    %%% Nota sobre funciones activación más democráticas que otras
    \marginpar{\textbf{Observación sobre la idoneidad de cada función activación:}
    Se probará la convergencia de las redes neuronales independientemente de la función de activación seleccionada. Cabe entonces la pregunta
    ¿Existen funciones de activación más democráticas que otras?
    }

    % Imágenes de la función indicadora 
    \begin{figure}[h]
        \centering
        \begin{subfigure}[t]{0.47\textwidth}
            \centering
            \includegraphics[width=\textwidth]{
                articulo_rrnn_aproximadores_universales/función_indicadora_l_0.png}
            \caption{Función indicadora $\lambda_0 = 0$}  
            \label{fig:función_indicadora}
        \end{subfigure}
        \hfill
        \begin{subfigure}[t]{0.47\textwidth}  
            \centering 
            \includegraphics[width=\textwidth]{articulo_rrnn_aproximadores_universales/función_umbral_lineal.png
            }
            \caption{Función umbral $w=(2)$, $t=1$}    
            \label{fig:función_umbral_lineal}
        \end{subfigure}
        \begin{subfigure}[t]{0.47\textwidth}   
            \centering 
            \includegraphics[width=\textwidth]{articulo_rrnn_aproximadores_universales/función_rampa.png}
            \caption{Función rampa} 
            \label{fig:funciones_rampa}
        \end{subfigure}
        \hfill
        \begin{subfigure}[t]{0.47\textwidth}   
            \centering 
            \includegraphics[width=\textwidth]{articulo_rrnn_aproximadores_universales/cosineSquasherSinTitulo.png}
            \caption{\textit{Cosine Squasher}}   
            \label{fig:cosine_squasher}
        \end{subfigure}
        \caption{Ejemplos de funciones de activación} 
        \label{fig:EjemplosFunciónActivación}
    \end{figure}

    \begin{itemize}
        \item \textbf{Funciones umbral} \ref{fig:función_umbral_lineal}:
        Una función umbral, es una función booleana monótona $\psi_w: \{0,1\}^n \longleftarrow \{0,1\},$ 
        donde para $w \in \R^n$, $t \in \R$ fijos se
        satisface que 
        \begin{equation}
            \psi_w(x) = \left\{
                \begin{array}{lcc}
                    1, &   si  & w \cdot x \geq t \\
                    0, &  si & w \cdot x < t\\
                    \end{array}
            \right.
        \end{equation}
        
        \item \textbf{Funciones indicadoras} \ref{fig:función_indicadora}: $\psi(\lambda) = 1_{\{\lambda > \lambda_0\}}$ con $\lambda_0 \in \R$. 
        \item \textbf{Función rampa} \ref{fig:funciones_rampa}: $\psi(\lambda)  = \lambda 1_{\{0 \leq \lambda \leq  1\}} + 1_{\{\lambda > 1\}}.$
    
        \item \textbf{La función \textit{cosine squasher}} de Gallant and White 
        \ref{fig:cosine_squasher} (1988) \cite{Gallant88thereexists}. 
        \begin{equation*}
    \psi(\lambda )= \left(1 + cos\left(\lambda + 3 \frac{\pi}{2} \right) \frac{1}{2}\right) 
     1_{\{\frac{-\pi}{2} \leq \lambda \leq  \frac{\pi}{2}\}}
     +
     1_{\{ \frac{\pi}{2} < \lambda \}}.
    \end{equation*}
    \end{itemize}

   Notemos que así definidas las funciones de activación son medibles, ya que la imagen inversa de un abierto de $[0,1]$ siempre será un conjunto medible de  $\R$  (capítulo 7  página 77 \cite{nla.cat-vn1819421}).
    
    Cabe destacar que la definición tomada es la propuesta en \cite{HORNIK1989359} y que existen
    otras posibles definiciones menos restrictivas con las que también se ha probado la convergencia universal,
    por ejemplo podrían aceptarse funciones de activación no continuas \cite{FUNAHASHI1989183},  
    o como 
    en \cite{DBLP:journals/corr/SonodaM15} con funciones de activación no polinómicas no acotadas. 
    \textcolor{red}{A priori no debería de ser equivalentes entre ellas y será necesario un estudio más 
    profundo para establecer \textit{cuáles son las más democráticas}}
    %TODO añadir sección 

\end{definicion}



Para cualquier natural $r$ mayor que cero  denotaremos por $A^r$ al conjunto de todas 
las \textbf{funciones afines} de $\R^r$ a $\R$. Es decir el conjunto de funciones de la forma 
$A(x) = w \cdot x + b$ donde $x$ y $w$ son vectores de $\R^r$,  $b \in \R$ es un escalar y $\cdot$ representa el producto 
usual de escalares.  
    


En este contexto, $x$ corresponde al vector entrada de la red neuronal, $w$ los pesos de la red
que se multiplicarán con $x$ en la capa intermedia y $b$ el sesgo. 

Nótese que faltaría componer la función afín con una función de activación para obtener lo que hemos definido 
como un perceptrón. 
 % Nota margen sobre que abstrae esta estructura
 \marginpar{\textbf{Idea tras la definición de $\pmc$.}
 Nótese que lo que se ha definido es la clase de las redes 
 neuronales de una capa oculta y salida de una dimensión.
 Donde cada sumatoria representa una neurona en la capa oculta.
 }

\begin{definicion} [Formalización de una red neuronal de una capa oculta y salida real]
    Para cualquier función Borel medible $G$, definida de $\R$ a $\R$ y cualquier natural positivo
    $r \in \N$ se define a la clase de funciones $\pmc$ como 

    \begin{equation}
        \begin{split}
        \pmc = 
        \{ 
            & f: \R ^r \longrightarrow \R / \quad
            f(x)=\sum_{j = 1} ^q (
            \beta_j G(A_{j}(x)), \\
            & x  \in \R ^r, \beta_j \in \R, A_{j}\in A^r,q \in \N
            )
        \}.
        \end{split}
    \end{equation}
\end{definicion}

Con esta misma idea 
se define la siguiente estructura que generaliza a la familia $\pmc.$  
   
   % Idea intuitiva 
   \marginpar{\textbf{Motivación de la definición de $\pmcg$.}
   En un principio será más fácil demostrar que esta clase de funciones es densa en el espacio de funciones continuas.
   De esta manera este conjunto actuará de nexo de unión entre las funciones continuas y las redes neuronales facilitando las demostraciones.
   }
\begin{definicion} [Forma normal generalizada]\label{def:articulo_abstracción_rrnn}
    
    \begin{equation} 
        \begin{split}
        \sum \prod^r(G) = \{ 
        &f: \R^r \longrightarrow \R / \quad
        f(x) = \sum_{j = 1} ^q  \beta_j \prod_{k=1}^{l_j}
        G(A_{jk}(x)), \\
        &x  \in \R^r, \beta_j \in \R, A_{jk}\in A^r; l_j,q \in \N
        )
        \}.
    \end{split}
    \end{equation}  

 
    Notemos que $\pmc$ se recupera en el caso particular en el que $l_j = 1$ para todo $j$.
    Los elementos de $\pmcg$ son combinaciones lineales de productos finitos de neuronas. 

\end{definicion}


\subsection{ Reflexión sobre la relevancia de la función de activación}  

La función de activación $G$ es clave en el proceso de aprendizaje.
Por una parte nótese que como $A^r$ es un espacio afín, lo que se está haciendo es 
aproximar una función medible a partir de combinaciones lineales de \textit{rectas} evaluadas mediante $G$. 
 
\begin{figure}[h!]
    \includegraphics[width=\textwidth]{articulo_rrnn_aproximadores_universales/ejemploAproximaciónCurvasPorRectas.jpeg}
    \caption{Ejemplo de curva aproximada por perceptrones multicapa \cite{alma991008058419704990}.}
    \label{img:def_esenciales_ejemplo_curva_aproximada_percentrón_multicapa}
\end{figure}

Por otra parte, la función de activación acota la imagen de una aplicación afín, esta limitación 
es de total relevancia ya que permite una interpretación previamente convenida. 
Decimos por ende que es la causante del \textit{aprendizaje}.     

Por si el comentario no ha resultado claro procedamos a dar un ejemplo:   
Supongamos que nos hallamos frente a un problema de clasificación de dos clases y que la función de activación
tomada es una función umbral que toma valores cero o uno. Define por tanto esta función una separación de clases. 
De otra manera, de solo existir la función afín sin una transformación de la salida, el codominio serían 
todos los número reales donde a priori no se explicita una asignación de clase.  

Introducimos a continuación la notación de los conjuntos de funciones que seremos capaces de aproximar.  

Denotamos por  $\fC$ al conjunto de funciones continuas con dominio en $\R^r$ y codominio $\R$,
por  $\fM$ al conjunto de todas las funciones Borel medible de $\R^r$ a $\R$. 
y por $B^r$ a la $\sigma$-álgebra de Borel en $\R ^r$. 

En lo que respecta a definiciones anteriores, $\pmc$ y $\pmcg$ pertenecen a 
$\fM$ para cualquier función Borel-medible $G$. Si $G$ es continua entonces 
$\pmc$ y $\pmcg$ pertenecen a $\fC$. Tengamos presente que $\fC$ es un subconjunto
de $\fM$.  

De ahora en adelante nos referiremos a Borel-medible como medible. 
  

\subsection{ Reflexión sobre el tipo de funciones que se pueden aproximar}

La existencia de funciones no medibles manifiesta una limitación
de la formalización actual de las redes neuronales que plantea las siguientes 
preguntas: 
\begin{enumerate}
    \item ¿Supone la existencia de este tipo de funciones una verdadera limitación a nivel práctico?
    \item ¿Se podría construir alguna arquitectura que sí que las aproximara?
\end{enumerate}  

Continuando con el hilo de la segunda cuestión, si se carece de un espacio vectorial, 
de una medida,  ¿Cómo se podría construir una sucesión de funciones que se aproxime?
Quizás habría que buscar características más intrínsecas del problema en cuestión, 
razonamientos topológicos.

\begin{definicion} [Subconjunto denso]
    % Explicación de denso
    \reversemarginpar 
    \marginpar{ \textbf{Idea intuitiva conjunto denso.}
        Si $S$ es denso en $T$, 
        a nivel intuitivo sería decir que los elementos de $S$ son capaces de aproximar cualquier elemento de $T$
        con la precisión que se desee. 
    }
    \normalmarginpar
    Dado un subconjunto $S$ de un espacio métrico $(X, \rho)$, se dice que $S$ es denso por la distancia $\rho$
    en subconjunto $T$ si para todo $\epsilon$ positivo y cualquier $t \in T$ existe un $s \in S$ tal 
    que $\rho(s,t) \leq \epsilon$. 
\end{definicion}

Un ejemplo habitual sería en el espacio métrico $(\R, |\cdot|)$ con $|\cdot|$ el valor absoluto, el subconjunto 
$T = \R$ y $S$ los números irracionales, $S = \R \setminus \Q$. 


\begin{definicion} 
    Un subconjunto $S$ de $\fC$ se dice que es \textbf{uniformemente denso para compactos} en  $\fC$
    si para cada subconjunto compacto $K \subset \R^r$ se tiene que $S_K$ es denso según $\rho_K$ en $\fC$
    donde $\rho_K$ está definida como sigue.
    Para cualquier $f,g \in \fC$ 
    \begin{equation}
        \rho _ K (f,g) = \sup_{x \in K} |f(x) - g(x)|.
    \end{equation}

      % Explicación de uniformemente denso
      
      \marginpar{
          \textbf{Concepto intuitivo  de uniformemente denso para compactos }
          lo que indica es que \textit{controlamos} cuánto de cerca
          están dos funciones sea cual sea cualquier punto del compacto en que evaluemos.
      }
      
\end{definicion}

\begin{definicion}
    Una serie de funciones $\{f_n\}$ \textbf{converge uniformemente a una función $f$ sobre compactos} si para 
    cada  conjunto compacto $K \subset \R^r$  se cumple que
    \begin{equation}
        \rho_k (f_n, f) \longrightarrow 0 \text{ cuando } n \longrightarrow \infty.
    \end{equation} 
\end{definicion}


% !TeX root = ../../tfg.tex
% !TeX encoding = utf8
%
%*******************************************************
% Contenido del artículo 2: Primeros resultados
%*******************************************************


\section{Primeros resultados} 
% Introducción sección 


%%%%% primer teorema de convergencia  
% Teorema 2.1 
\begin{teorema} [Teorema de convergencia real en compactos]  \label{teo:TeoremaConvergenciaRealEnCompactosDefinicionesEsenciales}

    Sea G cualquier función continua no constante definida de $\R$ en $\R$. 
    Se tiene que $\pmcg$ es uniformemente denso para compactos en $\fC$.
\end{teorema}

% Significado del teorema 2.1
\marginpar{\maginLetterSize
    \iconoAclaraciones \textcolor{dark_green}{     
    \textbf{Qué se está probando en el teorema \ref{teo:TeoremaConvergenciaRealEnCompactosDefinicionesEsenciales}}
    }
    Podemos aproximar tanto como queramos cualquier función
continua con elementos de $\pmcg$, es decir a lo que llamamos  \textit{anillo de aproximación}. 
}

\begin{proof}
    Bastará probar que el conjunto $\pmcg$ satisface las hipótesis del teorema de
     Stone-Weierstrass \ref{ch:TeoremaStoneWeiertrass}.
    Lo primero será comprobar que $\pmcg$ es un álgebra, para ello veamos que:   

      % Claves de la demostración teorema 2.1
      \marginpar{\maginLetterSize\textbf{Clave de la demostración del teorema \ref{teo:TeoremaConvergenciaRealEnCompactosDefinicionesEsenciales}}
      Se comprueba primero que para operando elementos del \textit{anillo de aproximación} con sumas, productos y re-escalados no se \textit{salen del conjunto}, es decir que el resultado es otro elemento del \textit{anillo de aproximación}. Con estas operaciones se irá construyendo y refinando una función que aproxime cualquier función continua con un método cuya idea es \textit{con este elemento del anillo de aproximación tengo tal error porque me falla en tales puntos, si le sumo esto otro corrijo ese error y además el resultado sigue perteneciendo al conjunto.}
      }      
    \begin{enumerate}
        \item La función constante uno pertenece al conjunto. 
        Como $G$ no es constante existirá un valor de la imagen distinto de $0$, supongamos que $G(a)= b \neq 0$ para $a,b \in \R.$
        Consideremos la función afín $A(x) = 0 \cdot x + a$, está claro que $\frac{1}{b}G(A(x))$ es la función constantemente uno. 
        \item El conjunto $\pmcg$ es cerrado para sumas y producto por escalares reales. 
        En efecto, si $f,g$ pertenecen a  $\pmcg$, serán de la forma
         $f = \sum_{j = 1} ^q  \beta_{fj} \prod_{k=1}^{l_{fj}}  G(A_{fjk}(x))$ y 
        $g = \sum_{j = 1} ^p  \beta_{gj} \prod_{k=1}^{l_{gj}}G(A_{gjk}(x))$  por lo que
        \begin{equation}
            \begin{split}
                \gamma f+ \sigma g =& \gamma \sum_{j = 1} ^q  \beta_{fj} \prod_{k=1}^{l_{fj}}  G(A_{fjk}(x)) + 
                \sigma \sum_{j = 1} ^p  \beta_{gj} \ \prod_{k=1}^{l_{gj}}G(A_{gjk}(x)) \\
                & = \sum_{j = 1} ^q  (\gamma \beta_{fj})  \prod_{k=1}^{l_{fj}}  G(A_{fjk}(x)) + 
                \sum_{j = 1} ^p  (\sigma \beta_{gj}) \ \prod_{k=1}^{l_{gj}}G(A_{gjk}(x)).
            \end{split}
        \end{equation}
        
        Basta renumerar una de las sumatorias para ver $\gamma f+ \sigma g$ como una combinación 
        lineal de productos finitos y por tanto $\gamma f+ \sigma g \in \pmcg.$
        
      
        \item Cerrado para producto. Para $f,g \in \pmcg$, se tiene que $fg$ pertenece a $\pmcg$, para ello basta ver que: 
        \begin{align}
           f g & = 
           % escribimos f  en forma del anillo
           \left(
               \sum_{i \in I_f} \beta_{f_i} \prod_{k=1}^{l_i} G((x)) 
           \right) 
           % escribimos g en forma del anillo
           \left(
               \sum_{j \in I_g} \beta_{g_j} \prod_{k=1}^{l_j} G(A_{g_{j k}(x)}) 
           \right) 
           \\
           & = 
           % propiedad distributiva (con la primera suma) 
            \sum_{i \in I_f} 
            \left(
                \beta_{f_i} \prod_{k=1}^{l_i} G(A_{f_{i k}}(x)) 
           \right) 
           % escribimos g en forma del anillo
           \left(
               \sum_{j \in I_g} \beta_{g_j} \prod_{k=1}^{l_j} G(A_{g_{j k}(x)}) 
           \right) 
           \\
           & = 
           \sum_{
               \begin{subarray}{c}
                i \in I_f \\
                j \in I_g
               \end{subarray}
               }
               (\beta_{f_i} \beta_{g_j})
               \sum_{k = 1}^{l_i + l_j}G 
               \left(
                    \tilde{A}_{i j k}(x)
               \right).
        \end{align}
        Definimos $\tilde{A}_{i j k}(x) = A_{f_{i k}}$ si $k \in \{1, \ldots, l_i\}$
        y $\tilde{A}_{i j k}(x) = A_{g_{i (k - l_i)}}$ si $k \in \{l_i + 1, \ldots, l_i+l_j\}$.
        luego $fg \in \pmcg$. 
    \end{enumerate}

    Veamos que $\pmcg$ separa puntos para cada compacto $K \subset \R^r$. 

    Por ser $G$ no constante existirán $a,b \in \R$ distintos cumpliendo que $G(a) \neq G(b)$. Fijadas $x,y \in K$ tomamos entonces cualquiera de las 
    funciones afines que cumplen que $A(x) = a$ y $A(y)=b$ 
    \footnote{Sabemos que al menos una habrá, ya que podemos plantear la función afín
    como un sistema de ecuaciones lineales de $r+1$ incógnita y 2 soluciones}, 
    por lo que $G(A(x)) \neq G(A(y))$ y tenemos como buscábamos que $\pmcg$ separa los puntos de $K$. 

    Comprobemos finalmente que para todo punto de $K$ existe una función de $\pmcg$  en el que la imagen no es nula.  

    Por ser $G$ no constante volvemos a tomar un $a \in \R$ tal que $G(a) \neq 0$, consideramos ahora la aplicación lineal
    $A(x) = 0 \cdot x + a$ por lo que para todo $x \in K$, $G(A(x)) = G(a) \neq 0$. 

    Como hemos comprobado se verifican todas las hipótesis del teorema de Stone-Weierstrass, con lo que concluimos que $\pmcg |_K$ es denso en $C(K)$. 
\end{proof}

\subsection{Observaciones y reflexiones sobre el teorema de convergencia real en compactos}

Con esto lo que acabamos de probar que una estructura más general de\textit{feedforward neural networks} con tan solo una capa oculta  son capaces de aproximar cualquier 
función continua en un compacto.  Cabe destacar que a la función $G$, que haría el papel de función de activación,
 solo se le ha pedido como 
hipótesis ser continua.     

%%% Corolarios propios 

Notemos que la función de activación $G$ es única en toda la estructura,
sin embargo es habitual la combinación de éstas en una misma red neuronal (
\cite{DBLP:journals/corr/abs-1811-03378}, 
 \cite{8258768}, 
 \cite{DBLP:journals/corr/SzegedyVISW15}
). 
%Nota sobre experimentar con 
%\setlength{\marginparwidth}{\smallMarginSize}
\reversemarginpar
\marginpar{\maginLetterSize \iconoProfundizar \textcolor{blue}{\textbf{Nueva hipótesis de optimización}}
El corolario \ref{cor:se-generaliza-G-a-una-familia}
abre la puerta a preguntarse si la combinación de diferentes funciones de activación 
podría mejorar los resultados de alguna manera.
}
\normalmarginpar

\begin{aportacionOriginal}

\begin{corolario}[Pueden combinarse distintas funciones de activación en una misma red neuronal] \label{cor:se-generaliza-G-a-una-familia}

    Una misma red neuronal puede estar constituida por una familia de funciones continuas no constantes $\Gamma$, 
    bastará con generalizar $\pmcg$ a $\sum \prod ^d (\Gamma)$ donde 
    \begin{equation}
        \begin{split}
            \sum \prod^d (\Gamma) = \{ 
                &f: \R^d \longrightarrow \R /
                f(x) = \sum_{j = 1} ^q  \beta_j \prod_{k=1}^{l_j}
                G(A_{jk}(x)), \\
                &x  \in \R^d, \beta_j \in \R, A_{jk}\in \afines, l_j,q \in \N, G \in \Gamma
                )
                \}
        \end{split}
    \end{equation}
\end{corolario}
\begin{proof}
    La demostración es idéntica a la dada en el Teorema de convergencia 
    real en compactos \ref{teo:TeoremaConvergenciaRealEnCompactosDefinicionesEsenciales}.
\end{proof}
\end{aportacionOriginal}

Notemos que este resultado no da pista alguna de las ventajas de una función frente a otra,
 ni cómo afecta a la \textit{velocidad de convergencia}. 


Recordemos que de manera general se ha definido $A$ como una función afín 
$A(x) = w \cdot x + b$ donde $x$ y $w$ son vectores de $\R^d$  y $b \in \R$ es un escalar.  ¿Pero que ocurriría si trabajáramos con transformaciones más generales?  
Por ejemplo $B((x_1, ..., x_d)) = \sum_{i= 0} ^N \sum_{j= 0} ^d \alpha_{ij} x_j^i$  con $N$ natural positivo. 

% Nota sobre posible hipótesis de optimización 
\setlength{\marginparwidth}{\bigMarginSize}
\marginpar{\maginLetterSize \iconoProfundizar \textcolor{blue}{\textbf{Nueva hipótesis de optimización}}
Gracias al corolario \ref{corolario:generaliza-a}
nos podemos plantear si aumentar el espacio de búsqueda a partir de transformaciones
 no lineales es más conveniente que hacerlo mediante el número neuronas.
Abordaremos esta cuestión en \ref{hypothesis:activation-function}.
}

\begin{aportacionOriginal}
    \begin{corolario}[Generalización de de las transformaciones afines]  \label{corolario:generaliza-a}
        Se puede extender $\afines$ a conjuntos más generales como el de los polinomios de $d$ variables de grado $N$, $\mathbb{P}$.  
    \end{corolario}
    \begin{proof}
        Simplemente hay que reparar en que $\afines$ está contenido en el espacio $\mathbb{P}$. 
        Es más observando la demostración bastará con utilizar cualquier conjunto que contenga a $\afines$. 
    \end{proof}
\end{aportacionOriginal}

% Nota idea intuitiva  equivalencia 
\marginpar{\maginLetterSize\raggedright
    \iconoAclaraciones \textcolor{dark_green}{ \textbf{Idea intuitiva equivalencia de funciones:}}}
    \marginpar{\maginLetterSize
    En la definición \ref{definition:equivalencia_funciones}
Se considera que dos funciones son equivalentes si 
son iguales en casi todos sus puntos.  El objetivo de todas estas definiciones es \textbf{saber cuándo una red neuronal es equivalente a una función medible o cuánto la aproxima}.
}
%% Definiciones de equivalencia de funciones 
\begin{definicion}[Equivalencia entre funciones] \label{definition:equivalencia_funciones}
    Sea $\mu$ una medida de probabilidad en $(\R^d, B^d)$.  Dos funciones 
    $f$ y $g$ pertenecientes a $\fM$, diremos que son $\mu -$equivalentes 
    si $\mu\{ x \in \R^d : f(x)=g(x) \} = 1.$
\end{definicion}

Lo que se está diciendo es que serán iguales casi por doquier.   


% Definición distancia  
\begin{definicion} [Introducción de una distancia basada en una probabilidad] \label{definition:distancia-probabilidad}
    Dada una medida de probabilidad $\mu$ en $(\R^d, B^d)$, se define 
    la métrica $\rho_{\mu}$ definida como 
    \begin{equation}
        \begin{split}
            & \rho_{\mu} : \fM \times \fM \longrightarrow \R^+ \\
            & \rho_{\mu}(f,g) = \inf \{ \epsilon > 0: \mu \{ x : |f(x) - g(x)| > \epsilon \} < \epsilon \}.
        \end{split}
    \end{equation}
\end{definicion}  

Con esta definición lo que se está buscando es una forma de decir cuánto 
distan las funciones $f,g$ entre ellas.  


%% Lema 2.1
\begin{lema}[Caracterización de la convergencia de una sucesión]\label{lema:caracterizacionConvergenciaSucesiones2_1}
    Son equivalentes las siguientes afirmaciones: 
    \begin{enumerate}
        \item $\rho_{\mu}(f_n, f) \longrightarrow 0$.
        \item Para cualquier  $\epsilon > 0$ se tiene que $\mu \{  x : |f_n(x) - f(x)| > \epsilon \} \longrightarrow 0$.
        \item $\int \min \{ |f_n(x) - f(x)|, 1\} d\mu(x) \longrightarrow 0.$
    \end{enumerate}
\end{lema}

% Nota idea intuitiva distancia
\marginpar{\maginLetterSize\raggedright
    \iconoAclaraciones \textcolor{dark_green}{ \textbf{Idea intuitiva distancia probabilidad:}}}
\marginpar{\maginLetterSize
Este concepto de analizar la tendencia de la mayoría de los punto 
se ve reflejado en la definición de distancia basada en una probabilidad \ref{definition:distancia-probabilidad}, siendo entonces la \textbf{distancia de dos funciones la menor de las distancias que siguen la mayoría de sus puntos}.

El lema \ref{lema:caracterizacionConvergenciaSucesiones2_1} nos permitirá trabajar con mayor comodidad matemática este concepto.
}

\begin{proof}
    % 1 -> 2
    Comenzaremos probando (1) $\Rightarrow$ (2). 

    Si $\rho_{\mu}(f_n, f) \longrightarrow 0$
    Fijamos $\epsilon_0 > 0$, tenemos por definición que 
    para cualquier $0 < \delta < \epsilon_0$ existirá $n_0 \in \N$ tal que 
    $\rho_{\mu}(f_n, f) < \delta$ para cada $n$ un natural mayor que $n_0$. Es decir,  
    

    $$\inf \{ \epsilon > 0: \mu \{ x : |f_n(x) - f(x)| > \epsilon \} < \epsilon \} < \delta \quad \forall n \geq n_0$$

    entonces 

    \begin{equation}
        \mu \{ x : |f_n(x) - f(x)| > \epsilon_0 \}
        \leq
        \mu \{ x : |f_n(x) - f(x)| > \delta\}
        < \delta 
        \quad 
        \forall n \geq n_0
    \end{equation}

    lo que significa que 

    \begin{equation}
        \mu \{ x : |f_n(x) - f(x)| > \epsilon_0 \}
        \longrightarrow
        0  
    \end{equation}
    probando con ello la implicación buscada.

    % 2 -> 1
    Veamos ahora que (2) $\Rightarrow$ (1). 
    Fijamos $\epsilon_0 > 0$ y bajo la hipótesis segunda se tiene que 

    \begin{equation}
        \mu \{ x : |f_n(x) - f(x)| > \epsilon_0 \}
        \longrightarrow
        0,  
    \end{equation}
    es decir, que para cualquier real $\delta$ cumpliendo que $0 < \delta < \epsilon_0$ 
    existe un natural $n_0$ a partir del cual todo natural $n$ mayor o igual satisface que 
    
    \begin{equation}
        \mu \{ x : |f_n(x) - f(x)| > \epsilon_0 \}
        \leq
        \mu \{ x : |f_n(x) - f(x)| > \delta\}
        < \delta 
        \quad 
        \forall n \geq n_0
    \end{equation}

    lo que significa que 
    
    \begin{equation}
        \inf \{ \epsilon > 0:
         \mu \{ 
             x : |f_n(x) - f(x)| > \epsilon \} < \epsilon 
             \} 
        < \delta 
        \quad 
        \forall n \geq n_0
    \end{equation}

    que por definición de la distancia equivale a que 

    \begin{equation}
        \rho_{\mu}(f_n, f) < \delta \quad \forall n \geq n_0
    \end{equation}

    probando con ello 

    \begin{equation}
        \rho_{\mu}(f_n, f) \longrightarrow 0. 
    \end{equation}

    % 2 -> 3
    Probaremos ahora que (2) $\Longrightarrow$ (3).   

    Por (2) se tiene que sea cual sea el $\epsilon$ cumpliendo que 
    $0 < \epsilon \leq 2$ 
    existirá un natural $n_0$ a partir del cual, cualquier otro natural $n$ 
    satisface que 
    \begin{equation} 
        \mu \{  
            x : |f_n(x) - f(x)| > \frac{\epsilon}{2}  
            \}  
        < 
        \frac{\epsilon}{2},  
    \end{equation}

    Gracias a esta desigualdad, para cualquier $n > n_0$ podemos acotar la siguiente integral: 

    \begin{equation}
        \int \min \{ |f_n(x) - f(x)|, 1\} d\mu(x) 
        \leq
        \frac{\epsilon}{2} (1-\frac{\epsilon}{2}) + 1\frac{\epsilon}{2} 
         = \epsilon - \frac{\epsilon^2}{4} <  \epsilon.  
    \end{equation}
    probando con ello la implicación (2) $\Longrightarrow$ (3).

    % 3 -> 1
    Finalmente comprobaremos la implicación (3) $\Longrightarrow$ (1).

    Para cada $n\in \N$ llamamos $g_n = \min\{|f_n - f|, 1|\}$.
    Por (2), dado $0 < \epsilon < 1$, existe un $n_0 \in \N$
    de modo que si $n \geq n_0$ se cumple que 
    \begin{equation}\label{eq:definiciones_Básicas_Integral_GN_menor_Epsilon_Cuadrado}
        \int g_n d\mu < \epsilon^2
    \end{equation}
    Como $\epsilon < 1$ tenemos que 

    \begin{equation}
        \{ x; g_n(x) > \epsilon \}
         = 
         \{ x; |f_n - f| > \epsilon \}
    \end{equation}

    luego 

    \begin{equation}
        \mu\{ x; |f_n - f(x)| > \epsilon \}
        = 
        \mu\{ x; g_n(x) > \epsilon \}
        \leq
        \frac{1}{\epsilon} 
        \int_{g_n(x) > \epsilon} g_n d\mu 
        < \epsilon 
        \quad
        \forall n \geq n_0
    \end{equation}

    donde se ha usado la desigualdad de Chebyshev para $g_n$ y la desigualdad 
    (\refeq{eq:definiciones_Básicas_Integral_GN_menor_Epsilon_Cuadrado}). 

Probando con esto lo buscado que  para cualquier  $\epsilon > 0$ se tiene que 
$$\mu \{  x : |f_n(x) - f(x)| > \epsilon \} \longrightarrow 0.$$
\end{proof}


%% Lema 2.2
\begin{lema} \label{lema:2_2_convergencia_uniforme_en_compactos}  
    Si $\{f_n\}$ es una sucesión de funciones en $\fM$ que converge
    uniformemente en un compacto a $f$ entonces $\rho_{\mu}(f_n, f) \longrightarrow 0$. 
\end{lema}  
\begin{proof} Para cada $n\in \N$ llamamos $g_n = \min\{|f_n - f|, 1|\}$.
    Tengamos presente que por el  lema \ref{lema:caracterizacionConvergenciaSucesiones2_1} 
    deberemos probar que para cualquier $\epsilon > 0$, 
    existe un $n_0$ natural, tal que para cualquier otro natural $n$ mayor o igual que $n_0$ se tiene que 

    \begin{equation}
        \int \min \{ |f_n(x) - f(x)|, 1\} d\mu(x) 
        < 
        \frac{\epsilon}{2}.
    \end{equation}  

    Sea $\mu(\R^d) = M \in \R^+$  y 
    sin pérdida de generalidad puede suponerse $M = 1$
     \footnote{De otra forma bastaría con definir 
    en los pasos siguientes $\mu(K) > M - \frac{\epsilon}{2}$ y acotar con $\frac{\epsilon}{2M}$ 
    en vez de $\frac{\epsilon}{2}$.}. 
    Ya que $\R^r$ es un espacio métrico localmente compacto
    (pag 228 teorema 52.G \cite{nla.cat-vn1819421}),
    se tiene que existirá un subconjunto $K$ compacto de $\R^r$ con medida $\mu(K) > 1 - \frac{\epsilon}{2}.$
    Para el cual, por su compacidad, existirá un  $n_0$ natural 
    $\sup_{x \in K} |f_n(x) - f(x)| < \frac{\epsilon}{2}$   
    para cada natural $n$ con $n\geq n_0.$ 
    
          % Nota idea intuitiva  lema de que C es denso en M
          \marginpar{\maginLetterSize\raggedright
          \iconoAclaraciones \textcolor{dark_green}{ \textbf{Idea intuitiva lema \ref{lema:A_1_C_es_denso_en_M}:}}
          }
          \marginpar{\maginLetterSize
          Las funciones continuas pueden tomar formas muy variopintas, estando incluso no acotadas. La función de Dirichlet definida en $D(x) = 1$ si $x$ es irracional y $D(x)=0$ si $x$ es racional, es medible pero no es continua ya que presenta infinitas discontinuidades.
          }
          \marginpar{\maginLetterSize
          Sin embargo, las funciones continuas son más simples, fáciles de entender y manejar. 
          Gracias al lema \ref{lema:A_1_C_es_denso_en_M}
          acabamos de probar que \textbf{podemos aproximar en
          casi todos sus puntos  cualquier función medible a partir de una continua.} 
          }
          \marginpar{\maginLetterSize
          Por ejemplo la función constantemente uno aproxima a la función $D$ previamente definida.
          }
    De modo que para cualquier $x \in K$, 
     $n$ con $n\geq n_0$   se cumple que 
     \begin{equation}
        |f_n(x) - f(x)| 
        = 
        \min \{ |f_n(x) - f(x)|, 1\} 
        = 
        g_n.
     \end{equation}

    Por lo que  
    \begin{equation} \label{eq:lema3_2_integral_en_compacto_K}
        \int_K g_n d\mu 
        \leq
         \mu(K) \sup_{x \in K} |f_n(x) - f(x)| 
        \leq 
        \frac{\epsilon}{2} .
    \end{equation}

    Acotando el primer sumando por la medida 
    del complemento de la región integrada y en virtud de 
    (\refeq{eq:lema3_2_integral_en_compacto_K})

    \begin{equation}
        \begin{split}
            \int_{\R^d \setminus K} \min \{ |f_n(x) - f(x)|, 1\} d\mu(x) 
            +
            \int_{K} \min \{ |f_n(x) - f(x)|, 1\} d\mu(x)  \\ \leq
            \mu(\R^d \setminus K) +  \frac{\epsilon}{2}
            \leq
            \frac{\epsilon}{2} +  \frac{\epsilon}{2}
            = 
            \epsilon
        \end{split}
    \end{equation}

    para cualquier $n \geq n_0$. 
\end{proof}



% Lema A.1 
\begin{lema}\label{lema:A_1_C_es_denso_en_M}
    Para cualquier medida finita $\mu$ se tiene que $\fC$ es denso en 
    $\fM$ para la distancia $\rho_\mu$.
\end{lema}
\begin{proof}
    Dada cualquier $f \in \fM$ y un $\epsilon > 0$ arbitrario, 
    tenemos que encontrar una función $g$ que cumpla que 
    $\rho_{\mu}(f, g) < \epsilon$. 

    Tomando un $M > 1$ lo suficientemente grande, tenemos que 
    
    \begin{equation}
        \int \min \{ |f(x)\ 1_{|f(x)| < M} - f(x)|, 1\} d\mu(x)
        < \frac{\epsilon}{2}. 
    \end{equation}

    Sabemos además que podemos aproximar $f 1_{|f| < M}$ por $g$, una función continua que es límite de una sucesión de
    funciones simples ( pag 241-242,  teoremas 55C y 55D \cite{nla.cat-vn1819421}), 
    la cual satisface 
    \begin{equation}\label{eq:lema3_3_integral}
        \int \min \{ |f(x) 1_{|f(x)| < M} - g(x)|, 1\} d\mu(x) 
        < \frac{\epsilon}{2}. 
    \end{equation}
    Tomamos $M$ lo suficientemente grande, de tal forma que 
    \begin{equation} \label{eq:lema3_3_medida_conjunto}
        \mu(\{ x: |f(x)| \geq M\}) < \frac{\epsilon}{2}
    \end{equation}
    y denotamos por $\Lambda$ al conjunto $\{ x: |f(x)| < M\}.$
    
    Concluyendo por \refeq{eq:lema3_3_integral} y 
    \refeq{eq:lema3_3_medida_conjunto}
     \begin{equation}
        \begin{split}
            \int \min \{ |f  - g|, 1\} d\mu 
            = 
            \int_\Lambda \min \{ |f1_{|f(x)| < M}  - g|, 1\} d\mu
            + 
            \int_{\R^d \setminus \Lambda} \min \{ |f  - g|, 1\} d\mu 
            \\
            <
            \frac{\epsilon}{2} 
            + 
            \mu(\{ x: |f(x)| \geq M\}) 
            <
            \frac{\epsilon}{2} 
            + 
            \frac{\epsilon}{2} 
            < \epsilon. 
    \end{split}
    \end{equation}
\end{proof}









% !TeX root = ../../tfg.tex
% !TeX encoding = utf8
%
%***************************************************************
% Contenido del artículo 3: Avanzamos en la generalización
%***************************************************************

% Teorema 2.2 
\begin{teorema}\label{teo:2_2_denso_función_continua}
    Para cualquier función continua no constate $G$, $r \in \N$ y
    medida de probabilidad $\mu$ o $(\R^r, B^r)$, 
    se tiene que $\pmcg$ es $\dist$-denso en $\fM$. 
\end{teorema} 
\begin{proof}
    Debemos probar que para cualquier función $f \in \fM$ existe una 
    sucesión de funciones $\{h_n\}_{n\in \N}$ contenida en $\pmcg$ y 
    cumpliendo que $\dist(h_n, f) \longrightarrow 0.$

    Consideramos cualquier $f \in \fM$,
    por el lema \ref{lema:A_1_C_es_denso_en_M} sabemos que $\fC$ es $\dist$-denso en $\fM$; 
    es decir, existirá un sucesión $\{f_n\}_{n\in \N}$ de funciones de $\fC$ convergente a 
    $f$.  
    
    Por otra parte sabemos por el teorema \ref{teo:TeoremaConvergenciaRealEnCompactosDefinicionesEsenciales}, 
    que $\pmcg$ es uniformemente denso por compactos en $\fC$, luego en cualquier compacto 
    $K \subset \R^r$ existirá una sucesión (con $n$ fijo) $\{g(n)_m \} _{m \in \N}$ convergente 
    a $f_n$, el término n-ésimo de la sucesión convergente a $g$. 

    Así pues, denotando como $h_n$ al término $g(n)_n$, obtenemos una sucesión de funciones 
    en $\fM$ que converge uniformemente en compactos a $f$ y por el lema \ref{lema:2_2_convergencia_uniforme_en_compactos}
    tenemos que $\dist(h_n, f) \longrightarrow 0$ como queríamos probar.     
\end{proof}

% --- Faltan por demostrar -----
% Lema A.2 
\begin{lema}\label{lema:a_2_paso_previo_denso}
    Sea F una función de activación continua y $\psi$ una \textbf{función de activación} arbitraria. 
    Para cualquier $\epsilon > 0$ existe un elemento $H_{\epsilon}$ de $\sum^1(\psi)$ cumpliendo que
    \begin{equation}
        \sup_{\lambda \in \R} | F(\lambda) - H_{\epsilon}(\lambda) | < \epsilon.
    \end{equation}
\end{lema} 
\begin{proof}
    Procedamos a realizar la siguiente prueba constructiva. 
    Tomamos fijo pero arbitrario un $\epsilon > 0,$ que sin pérdida de generalidad
    supondremos menor que uno 
    \footnote{En caso de ser mayor, se tomará cualquier otro menor que la unidad y la función resultante será igual de válida.}.
    Para que la $H_\epsilon$ pertenezca a $\sum ^1 (\psi)$ deberá de ser de la 
    forma $\sum^{q-1}_{j=1} b_j \psi( A_j(\lambda))$
    debemos encontrar por ende el número de sumatorias, $q-1$; esa misma cantidad de constantes reales $b_j$ y funciones afines $A_j$. 
    

    Para ello tomamos como $q$ a cualquier número natural que cumpla que 
    \begin{equation}\label{eq:lema_a_2_def_q}
        \frac{1}{q} < \frac{\epsilon}{4}.
    \end{equation}

    Fijaremos para cada $j \in \{1,2, ...,q-1\}$ los coeficientes  $b_j$ como $\frac{1}{q}$. 

    Seleccionamos cualquier constante real $M>0$ de tal forma que 
    se cumpla que
    \begin{equation}\label{lema_a_2_psi_m}
        \psi(-M) < \frac{\epsilon}{2q}
        \quad \text{ y } \quad
        \psi(M) > 1 - \frac{\epsilon}{2q}.
    \end{equation} 
    Sabemos que esto es posible ya que por ser $\psi$ una función de activación satisface que 
    $\lim_{\lambda \longrightarrow \infty} \psi(\lambda) = 1$ y que  $\lim_{\lambda \longrightarrow -\infty} \psi(\lambda) = 0$,
    por tanto existirá una constante $M_1$ positiva tal que a partir de ella cualquier
     otra constante $n_1$ mayor o igual que cumpla que 
    $\psi(n_1) > 1 - \frac{\epsilon}{2q}$. También existirá una constante $M_2$ positiva tal que a partir de 
    ella cualquier otra constante $n_2$ mayor o igual tal que que 
    $\psi(-n_2) < \frac{\epsilon}{2q}$. Podemos tomar como $M$ al máximo de $M_1$ y $M_2$.   

    Seleccionaremos ahora los siguientes puntos del dominio
    \begin{equation}\label{lema:2_2_seleccion_r_F}
        r_j = \sup \left\{ \lambda: F(\lambda) = \frac{j}{q} \right\},
         \text{ con } j \in \{1, ..., q-1\}, 
         \quad \text{ y } \quad
        r_q = \sup \left\{ \lambda: F(\lambda) = 1 - \frac{1}{q} \right\}. 
    \end{equation}
    Que por ser $F$ continua sabemos que existen. 

    Procedemos ahora a definir las distintas aplicaciones afines. 
    Para cualquier reales $s,r$ que cumplan que $r < s$ sea $A_{rs}\in A^1$ la única aplicación afín que satisface que 
    
    \begin{equation}
        A_{rs}(r) = -M \text{ y }  A_{rs}(s) = M. 
    \end{equation} 
    
    Acabamos pues de determinar todos los elements que conforman a $H_\epsilon$, de tal forma que se tiene que
    \begin{equation}
        H_\epsilon(\lambda) = \frac{1}{q} \sum^{q-1}_{j=1} \psi( A_{r_j, r_{j+1}}(\lambda))
    \end{equation}
    y así definida cumple que: 
    \begin{itemize}
        \item Si $\lambda \in (- \infty, r_1]:$
        Se cumple que $\lambda \leq r_1 < r_2 <...< r_q$ luego  
        para todos los $j \in \{1, ..., q-1\}$ las funciones afines satisfacen que 
        $A_{r_j, r_{j+1}}(\lambda) < -M$ y por cómo se fijó la $M$ en la condición \refeq{lema_a_2_psi_m}
        resulta que  $\psi( A_{r_j, r_{j+1}}(\lambda)) < \frac{\epsilon}{2q}$ concluyendo que 
        para $\lambda \in (- \infty, r_1]$
        \begin{equation}
            H_\epsilon(\lambda) = \frac{1}{q} \sum^{q-1}_{j=1} \psi( A_{r_j, r_{j+1}}(\lambda)) 
            <
            \frac{1}{q} \sum^{q-1}_{j=1}  \frac{\epsilon}{2q}
            < 
            \frac{1}{q} (q-1) \frac{\epsilon }{2q}
            <\frac{\epsilon }{2q}
            < \frac{\epsilon }{2}
        \end{equation}
        y por ende, como además $0 \leq F(\lambda) \leq \frac{1}{q} < \frac{\epsilon}{2}$ por cómo se seleccionaron los $r_j$ en 
        \refeq{lema:2_2_seleccion_r_F} se tiene que 
        \begin{equation}
            | F(\lambda) - H_{\epsilon}(\lambda) | < \frac{\epsilon}{2} + \frac{\epsilon}{2} < \epsilon. 
        \end{equation}

        \item Si $\lambda \in (r_q, +\infty):$
        Se cumple que $r_1 < r_2 <...< r_q <\lambda$ luego  
        para todos los $j \in \{1, ..., q-1\}$ las funciones afines satisfacen que  
        $A_{r_j, r_{j+1}}(\lambda) > M$ y por cómo se fijó la $M$ en \refeq{lema_a_2_psi_m}
        resulta que  $\psi( A_{r_j, r_{j+1}}(\lambda)) > 1-\frac{\epsilon}{2q}$ concluyendo que 
        para $\lambda \in (r_q,+\infty):$
        \begin{equation}
            1 \geq
            H_\epsilon(\lambda) = \frac{1}{q} \sum^{q-1}_{j=1} \psi( A_{r_j, r_{j+1}}(\lambda)) 
            >
                \frac{1}{q} \sum^{q-1}_{j=1}  \left(1-\frac{\epsilon}{2q} \right)
            >
            \frac{(q-1)}{q}  \left(1-\frac{\epsilon}{2q} \right)   
        \end{equation}
        y por ende, como además $\frac{(q-1)}{q}  \left(1-\frac{\epsilon}{2q} \right) <  \frac{q-1}{q} \leq F(\lambda) \leq 1$ por cómo se seleccionaron los $r_j$ en 
        \refeq{lema:2_2_seleccion_r_F} se tiene que 
        \begin{equation}
            | F(\lambda) - H_{\epsilon}(\lambda) | 
            \leq
            1 - \frac{(q-1)}{q}  \left(1-\frac{\epsilon}{2q} \right)
            = \frac{1}{q} + \frac{\epsilon}{2q}
            < \epsilon.
        \end{equation}
        Donde para acotar $\frac{1}{q}$ hemos usado la desigualdad \refeq{eq:lema_a_2_def_q}.

        \item Si $\lambda \in (r_{j},r_{j+1}]:$
        
        Tenemos por una parte que $\frac{j}{q} < F(\lambda) \leq \frac{j+1}{q}$ y 
        podemos descomponer $H_\epsilon$ en las siguientes sumatorias: 
        \begin{equation}
            \begin{split}
                q H_\epsilon(\lambda) 
                = 
                 \sum^{j-1}_{i=1} \psi( A_{r_i, r_{i+1}}(\lambda))
                + 
                \psi( A_{r_j, r_{j+1}}(\lambda))
                + 
                \sum^{q-1}_{i=j+1} \psi( A_{r_i, r_{i+1}}(\lambda))
            \end{split}
        \end{equation}

        Los términos de la primera sumatoria serán mayores que $\left(1-\frac{\epsilon}{2q} \right)$ y menores o iguales que la unidad, 
        el segundo sumando satisface que 
        $0 \leq q\psi( A_{r_j, r_{j+1}}(\lambda)) \leq 1$
        y para la última sumatoria, todos sus términos serán menores que $\frac{\epsilon}{2q}$ y mayores o iguales que cero.
        De donde resulta que : 
        \begin{equation}
            \frac{j-1}{q}\left(1-\frac{\epsilon}{2q} \right)  
            <
            H_\epsilon(\lambda) 
            <
            \frac{j-1}{q} 
            + 
            \frac{1}{q} 
            + 
            \frac{q-j}{q} \frac{\epsilon}{2q} 
        \end{equation}
        Concluyendo que 
        \begin{equation}
            F(\lambda), H_\epsilon(\lambda) 
            \in 
            \left[
                \frac{j-1}{q}\left(1-\frac{\epsilon}{2q}\right),
                \frac{j+1}{q}
            \right]
        \end{equation}
        y por tanto: 
        \begin{equation}
            | F(\lambda) - H_{\epsilon}(\lambda) | 
            \leq \frac{j+1}{q} -  \frac{j-1}{q}\left(1-\frac{\epsilon}{2q}\right)
            = 
            \frac{2}{q} + \frac{j-1}{q}\frac{\epsilon}{2q}
            < \frac{\epsilon}{2} + \frac{\epsilon}{2}
            < \epsilon.
        \end{equation}
        Donde para acotar $\frac{2}{q}$ hemos usado la desigualdad \refeq{eq:lema_a_2_def_q}.
    \end{itemize}
    La acotación $| F(\lambda) - H_{\epsilon}(\lambda) | < \epsilon$ se cumple para todo
    \begin{equation}
        \lambda \in (- \infty, r_1] \cup (r_q,+\infty) \cup_{j \in \{1, ..., q-1\}} (r_{j},r_{j+1}] = \R,
    \end{equation}
    probando con ello lo buscado.
\end{proof}      

% Teorema 2.3
\begin{teorema}
    Para cualquier función de activación $\psi$, $r$ natural positivo y
    medida de probabilidad $\mu$ en $(\R^r, B^r)$, 
    se tiene que $\rrnng$ es uniformemente denso en compactos
    en $\fC$ y denso en $\fM$ de acorde a la distancia $\dist$. 
\end{teorema}
\begin{proof}
    En virtud del lema \ref{lema:2_2_convergencia_uniforme_en_compactos} y del 
    teorema \ref{teo:2_2_denso_función_continua} basta con probar que 
    $\rrnng$ es uniformemente denso en compactos de $\sum \prod^r(F)$, 
    donde $F$ es una función de activación continua 
    \footnote{el razonamiento 
    por el que con esta hipótesis es suficiente es idéntico al realizado para la 
    demostración del teorema \ref{teo:2_2_denso_función_continua}.}.

    Para ello basta ver que cualquier función de la forma $\prod_{k=1}^l F(A_k(\cdot))$
    puede ser uniformemente aproximada por una una sucesión de funciones de $\rrnng$.

    Fijamos un $\epsilon > 0$  de manera arbitraria. 
    Gracias a la continuidad de la norma y de la operación multiplicación, existirá un $\delta >0$
    tal que para cualesquiera números reales $0 \leq a_k, b_k \leq 1,$ con $k \in \{1,...,l\}$ 
    se satisfagan que $|a_k -b_k| < \delta$ se cumple que 
    \begin{equation} \label{eq:teorema_2_3__1}
        \left| 
            \prod^l_{k=1} a_k - \prod^l_{k=1} b_k 
        \right| 
        < 
        \epsilon.
    \end{equation}

    Por el lema \ref{lema:a_2_paso_previo_denso} existe una función 
    $H_{\delta}(\cdot) = \sum_{t=1}^T \beta_t \psi(A_t(\cdot))$
    cumpliendo que 

    \begin{equation}
        \sup_{\lambda \in \R} |F(\lambda) - H_{\delta}(\lambda) | < \delta.
    \end{equation}

    Se satisface con la cota suficiente de la desigualdad \refeq{eq:teorema_2_3__1} por lo que 
    resulta 
    \begin{equation}\label{eq:teorema2_3__3}
        \sup_{x \in \R^r} 
        \left| 
            \prod ^l_{k=1} F(A_k(x))
            -
            \prod ^l_{k=1} H_\delta(A_k(x))
        \right| 
        < 
        \epsilon.
    \end{equation} 
    
    Puesto que $H_\delta$ es de la forma  $\sum_{t=1}^T \beta_t \psi(A^1_t(\cdot))$ 
    y porque $A^1_t(A_k(\cdot)) \in A^r$ se tiene por la desigualdad \ref{eq:teorema2_3__3} que 
    $\prod ^l_{k=1} H_\delta(A_k(\cdot)) \in \rrnng.$

    Por lo tanto c $\prod ^l_{k=1} F(A_k(\cdot))$ puede ser 
    aproximado por elementos de $\rrnng$ y acabamos de probar con ello lo buscado. 
\end{proof} 

%% Faltan  por probar 
%Lema A.3
\begin{lema}\label{lema:A_3_función_activación_continua_con_arbitaria}
    Para cada función de activación $\psi$, cada $\epsilon >0$
    y cada $M>0$ existe una función 
    $cos_{M,\epsilon} \in \sum^1(\psi)$ tal que 
    \begin{equation}
        \sup_{ \lambda \in [-M, +M]}
        |\cos_{M,\epsilon}(\lambda) - \cos(\lambda)|
        < 
        \epsilon. 
    \end{equation}
\end{lema}
\begin{proof}
    Sea $F$ la función de activación \textit{cosine squasher} de Gallant and White (1988) definida 
    en \ref{def:funcion_activacion_articulo}

    Comenzaremos probando que para un intervalo acotado $[-M, M]$, existe $H \in \sum(F)$ 
    tal que para todo elemento $\lambda \in [-M, M]$ se cumpla que 

    \begin{equation}
        H(\lambda) = \cos(\lambda).
    \end{equation}

    Calcularemos $H \in \sum(F)$  de forma constructiva: 
    \begin{equation}
        F(\lambda )= \left(1 + \cos\left(\lambda -\frac{\pi}{2} \right) \frac{1}{2}\right) 
         1_{\{\frac{-\pi}{2} \leq \lambda \leq  \frac{\pi}{2}\}}
         +
         1_{\{ \frac{\pi}{2} < \lambda \}}.
    \end{equation}
    
    \begin{figure}[h]
        \centering
        \includegraphics[width=.7\textwidth]{articulo_rrnn_aproximadores_universales/cosineSquaser.png}
        \caption{Función \textit{cosine squaser}}
        \label{fig:cosine_squaser}
    \end{figure}

    Se tiene pues que para cualquier $\lambda \in \left[ \frac{-\pi}{2}, \frac{\pi}{2}\right]$

    \begin{equation}
        2 F(\lambda)-1 = \cos \left( \lambda - \frac{\pi}{2}\right)
    \end{equation}

    Que haciendo $\mu = \lambda - \frac{\pi}{2}$ resulta que para cualquier
    $\mu \in [-\pi, 0]$
    \begin{equation}
        \cos(\mu) = 2 F \left(\mu + \frac{\pi}{2} \right)  -1 
    \end{equation}

    Además, puesto que $F(\mu + 2 \pi M) = 1$ para todo $\mu \in [-M, M] \supset [-\pi, 0]$ tenemos que
    para cualquier $\mu \in [-\pi, 0]$
    \begin{equation}
        \cos(\mu) = 2 F \left(\mu + \frac{\pi}{2} \right)  - F(\mu + 2 \pi M) 
    \end{equation}

    \begin{figure}[h]
        \centering
        \includegraphics[width=.7\textwidth]{articulo_rrnn_aproximadores_universales/H_menos_pi_0.png}
        \caption{Comparativa $H_{[-\pi, 0]}$ con la función coseno real. }
        \label{fig:coseno_vs_H_menos_pi_cero}
    \end{figure}

    De manera generalizada denotaremos como $H_{[M_1,M_2 ]}$ a la función 
     existe $H_{[M_1,M_2 ]} \in \sum(F)$ 
    tal que para todo elemento $\lambda \in [M_1, M_2]$ se cumpla que 
    \begin{equation}
        H_{[M_1,M_2 ]}(\lambda) = \cos(\lambda)
    \end{equation}

    Por tanto $H_{[-\pi, 0]}$ viene definida como  
    \begin{equation}
        H_{[-\pi, 0]} = 2 F \left(\mu + \frac{\pi}{2} \right)  - F(\mu + 2 \pi M) 
    \end{equation}

    Por la simetría de la función coseno resulta que 
    para todo $\mu \in [0, \pi]$
    \begin{equation}
        \cos(\mu) = \cos(-\mu) = H_{[-\pi, 0]}(-\mu)
    \end{equation}
    \begin{figure}[h]
        \centering
        \includegraphics[width=.7\textwidth]{articulo_rrnn_aproximadores_universales/H_menos_pi_mas_pi.png}
        \caption{Función $H_{[-\pi, \pi]}$. }
        \label{fig:H_menos_pi_mas_pi}
    \end{figure}

    Así pues podemos definir 
    \begin{equation}
        \begin{split}
            H_{[-\pi, \pi ]}(\mu) &= H_{[-\pi, 0]}(\mu) + H_{[-\pi, 0]}(-\mu) - 1 \\
            &= H_{[-\pi, 0]}(\mu) + H_{[-\pi, 0]}(-\mu) - F(\mu + 2 \pi M). 
        \end{split} 
    \end{equation}  

    Por ser el coseno una función periódica es fácil ver que 
    considerando un natural $N$ que satisfaga que $2 \pi N \geq M$
  
    \begin{equation}
    \begin{split}
        H_{[-2\pi N, 2 \pi N]} (\lambda) = 
        \sum_{i=1}^N (H_{[-\pi, \pi ]}(\lambda + 2 \pi i) +1) 
        + H_{[-\pi, \pi ]}(\lambda)  \\
        - \sum_{i=1}^N (H_{[-\pi, \pi ]}(- \lambda + 2 \pi i - \pi) +1),
    \end{split}
\end{equation}
\begin{equation}
    \begin{split}
        H_{[-2\pi N, 2 \pi N]} (\lambda) 
        =  H_{[-\pi, \pi ]}(\lambda) + 
        \sum_{i=1}^N (
            H_{[-\pi, \pi ]}(\lambda + 2 \pi i)
            - 
            H_{[-\pi, \pi ]}(- \lambda + 2 \pi i - \pi)
        ) .         
    \end{split}
    \end{equation}

      % Ejemplos de la función H final
      \begin{figure}[h]
        \centering
        \begin{subfigure}[b]{0.45\textwidth}
            \centering
            \includegraphics[width=\textwidth]{articulo_rrnn_aproximadores_universales/H_[-2π,2π].png}
            \caption{Función $H_{[-2\pi, 2\pi]}$ con $N=1$.}
            \label{fig:H_con_M}
        \end{subfigure}
        \hfill
        \begin{subfigure}[b]{0.45\textwidth}
            \centering
            \includegraphics[width=\textwidth]{articulo_rrnn_aproximadores_universales/H_[-8π,8π].png}
            \caption{Función $H_{[-8\pi, 8\pi]}$ con $N=4$. }
        \end{subfigure}
        \hfill
        \caption{Ejemplos de funciones $H_{[-M, M]}$}
    \end{figure}

    Está claro que por ser $2 \pi N \geq M$ la función $H_{[-M, M]}$ será la restricción de la anterior, es decir: 

    \begin{equation}
        H_{[-M, M]} = H_{[-2\pi N, 2 \pi N]_{|[-M, M]}}.
    \end{equation}

    
    Así definida $H_{[-2\pi N, 2 \pi N]}$
    pertenecerá a $\sum(F)$ o lo que es de nuestro interés, estará 
    formada por una combinación finita, supongamos que $K$, 
    sumas y restas finita de $F \circ A_i$ con $A_i$ una función afín. 
    Por ser $F$ una función de activación continua gracias al
    lema \ref{lema:a_2_paso_previo_denso}, 
    existirá $W_{ \frac{\epsilon}{k}} \in \sum(\psi)$ tal que 
    podremos acotar $|F - W_{ \frac{\epsilon}{k}} | < \frac{\epsilon}{k}$ en todo $\R$.

    Además, como las transformaciones afines $A_i$ solo son traslaciones,
    para cada $i \in \{1,..K\}$ se cumple que 
     $W_{ \frac{\epsilon}{k}} \circ A_i \in \sum(\psi)$ y se mantiene que 
    \begin{equation}
        |F \circ A_i - W_{ \frac{\epsilon}{k}} \circ A_i | < \frac{\epsilon}{k}. 
    \end{equation}


    Recopilemos pues, tenemos que para $2\pi N \geq M$ y en el dominio $\lambda \in [-M, M]$: 
    \begin{equation}
        H_{[-M, M]} (\lambda) = 
        H_{[-2\pi N, 2 \pi N]}(\lambda) = 
        \sum_{i=1}^k \alpha_i F( A_i(\lambda)) \quad \alpha_i \in \{-1,1\}.
    \end{equation}
    fijando tales $\alpha_i \in \{-1,1\}$ y las traslaciones $A_i$ definimos 
    \begin{equation}
        \cos_{M, \epsilon}(\lambda) = \sum_{i=1}^k 
        \alpha_i  W_{ \frac{\epsilon}{k}}(A_i(\lambda)). 
    \end{equation}

    De esta manera, para cualquier $\lambda \in [-M, M]$

    \begin{equation}
        \begin{split}
            |\cos_{M,\epsilon}(\lambda) - \cos(\lambda)| 
            &= |\cos_{M,\epsilon}(\lambda) - \cos(\lambda) \pm  H_{[- M, M]}(\lambda)| \\
            &\leq
            |\cos(\lambda) -  H_{[- M, M]}(\lambda)|
            + 
            | \cos_{M,\epsilon}(\lambda) -  H_{[- M, M]}(\lambda)|  \\
            &\leq  0 
            + | \cos_{M,\epsilon}(\lambda) -  H_{[- M, M]}(\lambda)| \\
            & \leq  \sum_{i=1}^k 
            |  W_{ \frac{\epsilon}{k}}(A_i(\lambda)) 
            -
            F( A_i(\lambda))
             | \\
             & <   \sum_{i=1}^k \frac{\epsilon}{k} = \epsilon .
        \end{split}
    \end{equation}

    Dando con esto por concluida la demostración. 
 
\end{proof}

%Lema 4.A
\begin{lema}
    Sea $g(\cdot) = \sum_{j=1}^q \beta_j \cos(A_j(\cdot))$ con 
    $A_j \in A^r$. 
    Para cualquier función de activación $\psi$, 
    cualquier compacto $K \subset \R^r$
    y cualquier $\epsilon > 0$
    existe $f \in \sum^r(\psi)$ para el cual se cumple que 
    \begin{equation}
        \sup_{x \in K} 
        |g(x) - f(x)| < \epsilon.
    \end{equation}
\end{lema}

%Lema A.5 
\begin{lema}
    Para cualquier función de activación $\psi$ se tiene que 
    $\rrnn$ es uniformemente denso en compactos de $C^r.$
\end{lema}

% Teorema 2.4
\begin{teorema}
    Para cualquier función de activación $\psi$, $r \in \N$ y
    medida de probabilidad $\mu$ o $(\R^r, B^r)$, 
    se tiene que $\rrnn$ es uniformemente denso en compactos
    en $\fC$ y denso en $\fM$ de acorde a la distancia $\dist$. 
\end{teorema}
% !TeX root = ../../tfg.tex
% !TeX encoding = utf8
%
%***************************************************************
% Contenido del artículo 4: Colorario 2.1
%***************************************************************

% Resultado de teoría de la medida 
Trataremos ahora de generalizar la tesis expuesta en 
 el teorema \ref{teo:2_4_rrnn_densas_M} sobre las funciones medibles. 
 Para ello recordaremos el teorema de Lusin.
% Teorema de Lusin 
\begin{teorema}[Teorema de Lusin] \label{teo:Lusin}
    Si $\mu$ es una medida regular de Borel, $E$ un conjunto de medida finita 
    y $f$ una función medible en $E$ entonces
    para cualquier $\epsilon > 0$ existirá un conjunto compacto 
    $K$ en $E$ tal que $\mu(E \setminus C) \leq \epsilon$ y donde $f$ es continua en $K$. 
\end{teorema}
\begin{proof}
    Demostración en páginas 242 y 243 de \cite{nla.cat-vn1819421}.
\end{proof}  

Notemos los puntos clave de este teorema, nos va a permitir \textit{trabajar} con una función medible como si fuera continua en un compacto
todo lo parecido a $\R^r$ como se quiera. 
 
\begin{teorema}(Caracterización de normalidad de Tietze)\label{teo:Tietze}
    Sea $X$ un espacio Hausdorff. Son equivalentes las siguientes afirmaciones: 
    \begin{enumerate}
        \item $X$ es normal.
        \item Para cada conjunto cerrado $A \subset X$ y para cualquier función continua 
        $f: A \longrightarrow \R$, $f$ admite una extensión continua $F:X \longrightarrow \R.$
        Además, si para todo $a \in A$ se cumple que $|f(a)| < c \in \R$, se puede elegir $F$
        de tal forma que satisfaga que $|F(x)| < c$ para todo $x\in X.$ 
    \end{enumerate}
    (Demostración en páginas 149-151 de \cite{james1966topology})
\end{teorema}

Como el ambiente actual en el que estamos trabajando 
es el espacio $(\R^r, \mathcal{T})$ que sabemos que es normal y puesto que es habitual que nuestras funciones estén definidas
en  compactos de $\R^r$, las podremos extender a $\R^r$. 


% Corolarios del artículo 
% Corolario 2.1
\begin{corolario} \label{cor:2_1}
    Para cada función $g \in \fM$ existe un subconjunto compacto 
    $K$ de $\R^r$ y $f \in \rrnng$ tal que para cualquier 
    $\epsilon > 0$ se tiene que 
    $\mu(K) > 1- \epsilon$ y para cada $x \in K$ tenemos que 
    \begin{equation}
        |f(x) - g(x) | < \epsilon,
    \end{equation}
    independientemente de la función de activación $\psi$, dimensión $r$ o medida $\mu$. 
\end{corolario}
\begin{proof}
    Sea $\epsilon > 0$ fijo pero arbitrario.  Gracias al teorema de Lusin \ref{teo:Lusin}
    existe un subconjunto compacto $K \subset \R^r$ de medida
    $\mu(K) > 1 - \epsilon$ donde la restricción  $g_{|K}$ es continua. 

    Por otra parte, en virtud de la caracterización de Tietze 
    \ref{teo:Tietze}, 
    por estar $g_{|K}$ definida en un compacto, admite una 
    extensión continua $G:\R^r \longrightarrow \R$ tal que 
    \begin{equation}
        \begin{split}
            G_{|K} = g_{|K} .
        \end{split}
    \end{equation}

    Por ser $G$ continua en un compacto, por la densidad de las redes neuronales en compactos en $\fC$(lema \ref{lema:A_5_uniformemente_denso_compactos} ) se tiene que existirá 
     $f \in \rrnng$ tal que 
    \begin{equation}
        \sup_{x \in K} |G(x) - f(x)| < \epsilon.
    \end{equation}

    Por lo que podemos afirmar que para todo $x \in K$
    \begin{equation}
        |f(x) -g(x)| 
        \leq 
        | f(x) -G(x)| + |G(x) -g(x)|
        < \epsilon + 0 = \epsilon
    \end{equation}
    como queríamos probar.
\end{proof}

\subsubsection{Comentarios sobre el corolario \ref{cor:2_1}}  

Este resultado nos indica que existe una red neuronal con una 
capa oculta capaz de aproximar cualquier función medible con el grado 
de precisión que se desee dentro de un compacto.   

Notemos que la diferencia que presenta el corolario recién probado
\ref{cor:2_1} con respecto al teorema \ref{teo:2_4_rrnn_densas_M}
es que en el corolario se está fijando el compacto y esto nos 
permite tener convergencia uniforme en él, mientras que en el 
teorema \ref{teo:2_4_rrnn_densas_M} para funciones medibles
tendremos asegurada tan solo la convergencia. 

% !TeX root = ../../tfg.tex
% !TeX encoding = utf8
%
%***************************************************************
% Contenido del artículo 5: Colorarios LP
%***************************************************************
\begin{definicion}[Espacios Lp]
    Se llama espacio $L_p(\R^r, \mu)$ o simplemente $L_p$ al conjunto 
    de funciones $f \in \fM$ tales que 
    \begin{equation}
        \int |f(x)|^p d\mu < \infty. 
    \end{equation}

Se define la norma de $L_p$ como 
\begin{equation}
    \| f\|_p 
    =
    \left(\int |f(x)|^p d\mu \right)^\frac{1}{p}.
\end{equation}

La distancia asociada al espacio $L_p$ se define como 
\begin{equation}
    \rho_p(f,g) = \| f-g\|_p.
\end{equation}
\end{definicion}

% Corolario 2.2
\begin{corolario}
    Si existe un subconjunto compacto $K$ en $\R^r$ de medida
    $\mu(K) =1$ entonces $\rrnn$ es $\dlp$-denso en $L_p(\R^r, \mu)$
    para cualquier $p \in [1,\infty)$, independientemente de 
    $\psi$, $r$ o $\mu$.
\end{corolario}
\begin{proof}
    Se quiere probar que para cualquier $g \in L_p$ y 
    $\epsilon >0$ existe un $f \in \rrnn$ tales que 
    \begin{equation}
        \dlp(f,g) <\epsilon.
    \end{equation}   
    
    Por pertenecer $g$ a $L_p$ existe una constante $M$ real positiva
    los suficientemente grande 
    tal que si definimos la función $h =g 1_{|g|<M}$ esta satisface 
    que
    \begin{equation}\label{eq:corolario_2_2:h_compacto}
        \dlp(g,h) < \frac{\epsilon}{3}.
    \end{equation}
    
    Además como $h$ es una función acotada de $L_p$, podemos encontrar
    una función $s$ continua que es límite de una sucesión de
    funciones simples 
    ( pag 241-242,  teoremas 55C y 55D \cite{nla.cat-vn1819421})
    y la cual cumple que 

    \begin{equation}\label{eq:corolario_2_2:s_continua}
        \dlp(h,s) < \frac{\epsilon}{3}.
    \end{equation}

    Por el teorema \ref{teo:2_4_rrnn_densas_M}, al estar en un compacto $K$ y por ser $\rrnn$ uniformemente
    denso en compactos hay una $f \in \rrnn$ la cual cumple que
    \begin{equation}
        \sup_{x \in K} |f(x) -s(x)|^p 
        <
         \left( \frac{\epsilon}{3}\right) ^p.
    \end{equation}
    
    Y por hipótesis $\mu(K) =1$ y definición de la distancia $\dlp$ 
    se tiene la siguiente desigualdad: 

    \begin{equation} \label{eq:corolario_2_2:cota_rrnn}
        \dlp(f,s) = 
        \left(\int |f(x) - s(x)|^p d\mu \right)^\frac{1}{p}
        \leq 
        \left(\int  \left( \frac{\epsilon}{3}\right) ^p d\mu \right)^\frac{1}{p}
        = \left( \mu(K)  \left(\frac{\epsilon}{3} \right)^p\right) ^\frac{1}{p}
        = \frac{\epsilon}{3}.
    \end{equation}

    Gracias a la desigualdad triangular y las desigualdades
    (\refeq{eq:corolario_2_2:cota_rrnn})
    (\refeq{eq:corolario_2_2:h_compacto})
    (\refeq{eq:corolario_2_2:s_continua})

    \begin{equation}
        \dlp(f,g) 
        \leq
            \dlp(f,s)
            +\dlp(s,h)
            + \dlp(h,g)
        < 
        \frac{\epsilon}{3} + \frac{\epsilon}{3} + \frac{\epsilon}{3}
        = \epsilon.
    \end{equation}
Probando con ello lo buscado. 
\end{proof}  


% !TeX root = ../../tfg.tex
% !TeX encoding = utf8
%
%***************************************************************
% Contenido del artículo 5: Generalización a multi-output 
%***************************************************************
\section{Generalización para \textit{multi-output neuronal networks}}

En las secciones anteriores se han provisto resultados para redes 
neuronales de salida real. Vamos a generalizar los resultados vistos
para ser capaces de aproximar funciones continuas o medibles 
de $\R^r$ a $\R^s$ con $r,s \in \N.$

Denotaremos por $\fCC$ al conjunto de funciones continuas definidas de $\R^r$ a $\R^s$ y al de funciones medibles de 
$\R^r$ a $\R^s$  por $\fMM.$ 
La distancia asociada a estos espacios se define como 
\begin{equation}
    \rho_{\mu}^s(f,g) 
    =
    \sum_{i=1}^s \dist(f_i, g_i).
\end{equation}

Con la siguiente definición buscamos abstraer el modelo de una red neuronal de una capa oculta y salida múltiple.
\begin{figure}[h]
    \centering
    \includegraphics[width=.7\textwidth]{articulo_rrnn_aproximadores_universales/RedNeuronalAbstactaUnaCapaVariasSalidas.png}
    \caption{Ejemplo de red neuronal de una capa oculta con $h$ nodos, de dimensión de entrada $r$ y salida $s$.}
    \label{fig:red neuronal-r-h-s}
\end{figure}

Nótese que los vectores $(w_{0i},w_{1 i}, \ldots, w_{r i})$ representan a la aplicación afín 
$A_i((x_1, x_2, \ldots x_r)) = w_{0i} + \sum_{j=1}^r w_{ji} x_j$
con $i \in \{1,\ldots, h\}$ . 

\begin{definicion}[Abstracción de una red neuronal con una capa oculta y múltiple salida] 
    Para cualquier función Borel medible $G$, definida de $\R$ a $\R$ y cualquier natural positivo
    $r \in \N$ se define a la clase de funciones $\pmc$ como 
    \begin{equation}
        \begin{split}
        \sum^{r,s}(G) = 
        \{ 
            & f: \R ^r \longrightarrow \R^s, f= (f_1, f_2, \ldots, f_s)  / \quad 
            \\ &
            \text{ con } f_i : \R ^r\longrightarrow \R, 
            f_i(x)=\sum_{j = 1} ^h (
            \beta_{j i} G(A_{j}(x)) \quad i \in \{1,2,\ldots, s\}, \\
            & x  \in \R ^r, \beta_{j i} \in \R, A_{j}\in A^r,h \in \N
            )
        \}.
        \end{split}
    \end{equation}
\end{definicion}

No es difícil pensar que su versión generalizada sea: 

\begin{definicion} 
    Dadas las mismas hipótesis que en la definición anterior, se define la siguiente clase de funciones como 
    \begin{equation}
        \begin{split}
            \sum \prod^{r, s}(G) 
            = 
        \{ 
            & f: \R ^r \longrightarrow \R^s, f= (f_1, f_2, \ldots , f_s)  / \quad 
            \\ &
            \text{ con } f_i : \R ^r\longrightarrow \R, 
            f_i(x)=\sum_{j = 1} ^h 
            \left(
            \beta_{j i} \prod_{k=1}^{l_{j i}} G(A_{j i}(x))
            \right)
             \quad i \in \{1,2,\ldots, s\}, \\
            & x  \in \R ^r, \beta_{j i} \in \R, A_{j i}\in A^r; h,l_{j i} \in \N 
        \}.
        \end{split}
    \end{equation}
\end{definicion}


% Corolario 2.6  
\begin{corolario}\label{corolario:2_6}
    Los teoremas 
    \ref{teorema:2_3_uniformemente_denso_compactos},
    \ref{teo:2_4_rrnn_densas_M} 
    y los corolarios
    \ref{cor:2_1}, 
    \ref{corolario:2_2_rrnn},
    \ref{corolario:2_3_medida_probabilidad},
    \ref{corolario:2_4_conjunto_finito}
    y 
    \ref{corolario:2_5_función_Booleana}
    permanecen válidos si se sustituye $\rrnn$ por $\rrnnmc$
    ,$\rrnng$ por $\rrnngmc$, 
    los espacios de funciones continuas y medibles por $\fCC$ y $\fMM$ respectivamente.
\end{corolario}
\begin{proof}
    Observemos que todos los teoremas y lemas mencionados basan su tesis
    en la existencia de una red neuronal es decir, que si llamamos según 
    convenga $\mathcal{F}^{r,s}$ a $\rrnnmc$ o $\rrnngmc$ deberemos de 
    encontrar un $f \in \mathcal{F}^{r,s}$ que cumplan las respectivas tesis para salidas múltiples. 

    La prueba se construirá por inducción sobre el número de salidas $s$. 

    El caso base $s=1$ viene dado por los respectivos teoremas y lemas ya probados.
    Supuesto cierto para $s = n$ veamos que se cumple para $s=n+1$: 
    
    Se quiere encontrar 
    $f = (f_1, f_2, \ldots, f_n, f_{n+1})$ de $n+1$ salidas, 
    por hipótesis de inducción existe $g_n \in \mathcal{F}^{r,n}$ con
     $g_n = (f_1, f_2, \ldots, f_n)$ y con $h_n$ sumandos. Denotamos a los pesos de las transformaciones afines 
     $w_{i j} \in \R$ con 
     $i \in \{0, 1, \ldots , r \}$  y  $j \in \{1, \ldots ,h_n \}$ 
     y $\beta_{ k l} \in \R$ con 
     $k \in \{1, \ldots ,h_n \}$  y  $l \in \{1, \ldots ,n \}.$

    También existe $g_1 \in \mathcal{F}^{r,1}$ cumpliendo que
    $g_1 = f_{n+1}$ con $h_1$ sumandos en la capa oculta
    y pesos  
    ${w'}_{i j} \in \R$ con 
     $i \in \{0, 1, \ldots , r \}$  y  $j \in \{1, \ldots , h_1 \}$ 
     y ${\beta '}_{ k l} \in \R$ con 
     $k \in \{1, \ldots , h_1 \}$  y  $l = {n+1}$
     (Ver figura \ref{fig:red neuronal-r-h-s} para orientarse en la notación tomada).
     
    Considerando $f$ compuesta por $h_n + h_1$ sumandos y donde sus pesos son los siguientes:

    El peso $\tilde{w}$ de las funciones afines: 
    Para cuales quiera 
    $i \in \{0, 1, \ldots  , r \}$  y  
    $j \in \{1, \ldots , h_n, h_{n} + 1, \ldots, h_n + h_1\}$  determinaremos la siguiente casuística
    \begin{enumerate}
        \item Si $1 \leq j \leq h_n$ entonces $\tilde{w}_{i j} = w_{i j}.$
        \item Si $h_n < j \leq h_n + h_1$ entonces $\tilde{w}_{i j} = w_{i (j-h_n)}.$
    \end{enumerate}

    Para los pesos $\tilde{\beta}$, para cualquier
    $k \in \{1, \ldots , h_n, h_{n} + 1, \ldots, h_n + h_1\}$ y  
    $l \in \{1, \ldots ,  n+1 \} :$ 
    \begin{enumerate}
        \item Si $k \in \{1, \ldots ,  h_n \}$ y $l \in \{1, \ldots , n\}$ 
        entonces $\tilde{\beta}_{k l} = \beta_{k l}.$
        \item Si $k \in \{1, \ldots , h_n \}$ y $l=n+1$ 
        entonces $\tilde{\beta}_{k l} = 0.$
        \item Si $k \in \{h_{n} + 1, \ldots, h_n + h_1 \}$ 
        y $l \in \{1, \ldots , n\}$ 
        entonces $\tilde{\beta}_{k l} = 0.$
        \item Si $k \in \{h_{n} + 1, \ldots, h_n + h_1 \}$ 
        y $l=n+1$ 
        entonces 
        $\tilde{\beta}_{k l} = {\beta '}_{(k- h_n) l}.$
    \end{enumerate}

    Notemos que $f=(f_1, f_2, \ldots, f_n, f_{n+1})$, es decir $f \in \mathcal{F}^{r,s}$, y que para cada teorema o lema
    cada una de las proyecciones de $f$ cumple la tesis, es decir $f$ cumple lo buscado. 
\end{proof}

\subsubsection*{ Conclusión sobre el teorema anterior}  
A la vista de la demostración constructiva se nos acaba de decir de manera indirecta que si queremos construir una red neuronal 
a partir de un tamaño de entrenamiento $E$ y de salida $r$, 
con una red neuronal de $E \times r$ capas ocultas será suficiente y para este caso la reflexión expuesta en la sección \ref{subsection:reflexión_sobre_número_de_neuronas} es idéntica. 


%% Hasta aquí se aplican los cambios de Javier y JJ  


%%%%%%%%%%%%%%%%%%%%%%%%%%%%%%%
% Observación entre la diferencia de Q y R
%%%%%%%%%%%%%%%%%%%%%%%%%%%%
% Observación sobre el dominio discreto donde se está trabajando 
% y que refleja una posible fuente de mejora de las redes neuronales
% ISSUE #88
\section{Consideración sobre la capacidad de cálculo}

Suele pasar peligrosamente desapercibido que el teorema  \ref{corolario:2_6} recién probado asegura
que se podrá encontrar una red neuronal en $\rrnnmc$
que aproxime todo lo bien que queramos la función ideal; sin embargo, destaquemos que los parámetros que caracterizan a la red neuronal encontrada son reales. Es decir, 
si nuestra red neuronal toma valores irracionales difícilmente será computable en un ordenador por su infinitud. 

Esto nos impone una nueva restricción en el espacio de búsqueda, ya que no solo debe de reducirse el error, si no que los parámetros que representan la red deben de estar limitados a cierta precisión: la propia de un ordenador.  

Comenzaremos viendo que en efecto una red neuronal con parámetros en el cuerpo de los enteros es factible como aproximador universal. 

% Teorema de que podemos tener redes neuronales con parámetros racionales que también converjan. 
\begin{aportacionOriginal}
    \begin{teorema}
        El espacio $\mathcal{H}(\Q^d, \Q^s)$ es denso al espacio $\rrnnmc$. 
    \end{teorema}
    \begin{proof}
        Para todo $h_r \in \rrnnmc$, dado cualquier $\epsilon \in \R^+$, 
        gracias a la densidad de los racionales en los reales, 
        existirá $h_q \in \mathcal{H}(\Q^d, \Q^s)$ tal que 
        $\dist( h_r, h_q) < \epsilon$. 
    \end{proof}

    Si bien los números racionales tienen el potencial de ser computados por su 
    finitud, seguimos sujetos a que el número de decimales y tamaño de su parte entera \textit{sean lo suficientemente pequeños} como para poder ser representados 
    y calculados con un ordenador. Esto no tiene que ser, cierto y de hecho abre una nueva cuestión
     Al aumentar el número de neuronas, la precisión que demande una red neuronal también disminuye?  
\end{aportacionOriginal} 

De ser cierto este resultado, podría empezar a denominarse el espacio 
de las redes neuronales de cierto número de neuronas y con tal
precisión. Esto abriría la puerta a establecer un isomorfismo entre este cuerpo y el de las redes neuronales con parámetros en los 
enteros.  Lo cuál tendría su interés ya que por las arquitecturas 
actuales, los números flotantes se calculan en la GPUs, mientras que los enteros en las CPUs, siendo más rápidas la segundas\footnote{En el blog del investigador Long Zhou, se comenta sobre las injusticias que se comenten a la hora de comparar las GPUs y CPUs, dejo link a la publicación. Consultada por última vez el 23 de mayo
del 2022, URL: \url{https://long-zhou.github.io/2013/02/12/CPU-GPU-comparison.html}} \cite{CPU-vs-GPUS}. 

Si tuviéramos presente este razonamiento 


%% !TeX root = ../../tfg.tex
% !TeX encoding = utf8
%
%*******************************************************
% Observaciones artículo MFNAUA
%*******************************************************

\section{Conclusiones y observaciones} 

A falta de completar el estudio procedente del teorema \ref{teo:MFNAUA}, 
dejo reflejadas algunas observaciones. 

\subsection{Sobre la dimensión de dominio y codominio} 

Consecuencias sobre cuando en \ref{teo:MFNAUA} se estable que el dominio y codominio son finitos. 

La bondad de que el codominio sea finito depende del objetivo de función que queramos aproximar: 
Si la función pretende ser entendida o manejable es necesario y \textit{natural} su finitud. 
Si por el contrario pretende abarcar alguna 
construcción matemática más abstracta le sería imposible ¿Tienen una aplicación práctica tales definiciones?

Para la definición del dominio ocurre la misma reflexión, la disyuntiva entre tratabilidad humana y la máxima generalización de un concepto. 
¿Se podría intentar analizar si los espacios se puede ampliar?  


% Redes neuronales: métodos de evaluación y actualización de pesos 
%%%%%%%%%%%%%%%%%%%%%%%%%%%%%%%%%%%%%%%%%%%%%%%%%%%%%%%%%%%%%%%%%%%%%%%%%%%%%%%
%
% Introducción sobre algoritmos de cálculo de los pesos de una red neuronal 
%
%%%%%%%%%%%%%%%%%%%%%%%%%%%%%%%%%%%%%%%%%%%%%%%%%%%%%%%%%%%%%%%%%%%%%%%%%%%%%%

\chapter{Explicitación del cálculo de una red neuronal}

Se han concretado en los capítulos anteriores que con con redes neuronales 
podemos aproximar funciones medibles, la cuestión ahora reside en ¿cómo se consigue tales 

Ya expusimos en la sección \ref{algoritmo-forward-propagation} cómo evaluar
una red neuronal, la cuestión entonces radica en ¿cuál es la red neuronal
buscada? es decir a nivel práctico sería ¿cómo puedo calcular matrices de pesos 
adecuadas?

Explicaremos en este capítulo el método tradicional y ampliamente usado de 
\textit{backpropagation} \ref{algoritmo-de-backpropagation} y otros más novedosos como
\textcolor{red}{TODO : añadir otros métodos de actualización de pesos. La idea sería que 
los buscados tengan un coste menor para así optimizar la velocidad de aprendizaje.}


% !TeX root = ../../tfg.tex
% !TeX encoding = utf8
%
%*******************************************************
% Construcción y evaluación de las redes neuronales 
%*******************************************************

\chapter{Construcción técnica de las redes neuronales de una sola capa}  
\label{chapter:construir-redes-neuronales}
Vista la formulación teórica de una red neuronal de una sola capa 
introducida en \ref{definition:redes_neuronales_una_capa_oculta} 
se comparará y estudiará la relación teórica 
entre nuestra propuesta de modelo y los modelos usuales de redes neuronales. 

Explicaremos también una construcción técnica junto con un
análisis del costo necesario y beneficio obtenido.
Así como los algoritmos necesarios para su evaluación y aprendizaje. 

 Los algoritmos presentados son 
 la adaptación natural de técnicas como 
\textit{forward propagation} y \textit{backpropagation } explicadas en \cite{BishopPaterRecognition} a nuestro enfoque teórico. 

Además, puesto que nuestro objetivo es optimizar seremos muy meticulosos en cuanto a analizar el 
coste computacional tanto de cómputo como de 
memoria.



\section{Componentes de una red neuronal de una capa oculta} 

Como ya habíamos definido en \ref{definition:redes_neuronales_una_capa_oculta}  
para nosotros una red neuronal será  una función $h : X \longrightarrow Y$ con $X \subseteq \R^d, Y \subseteq \R^s$ 
cuya proyección $k-$ésima viene dada por
\begin{equation}
    h_k(x) =  \sum_{i=1}^{n} \beta_{i k} \gamma_{i}( A_{i}(x))
    = 
    \sum_{i=1}^{n} \beta_{i k} \gamma_{i}
    \left(
        w_{0 i} + \sum_{j=1}^d w_{j i } x_i
    \right) 
\end{equation}
donde $n$ es el número de neuronas,   $\gamma_{i} \in \Gamma$, funciones medibles definidas de $\R$ a $\R$,
$\beta_{i k} \in \R$ y $A_i \in \afines$.

% Imagen grafo red neuronal  una capa oculta muy simple y en blanco y negro 
\begin{figure}[h!]
    \centering
    \includegraphics[width=0.85\textwidth]{1-Introduccion_redes_neuronales/Red-Neuronal-una-capa-simple.png}
    \caption{\textit{Grafo} de una red neuronal de una capa oculta}
    \label{img:grafo-red-neuronal-una-capa-oculta_repeticion}
\end{figure}

Ante esta definición el número de parámetros a ajustar es: 
\begin{itemize}
    \item De la forma $\beta_{i k}$: $n s$. 
    \item De la forma $w_{i j}$: $n(d+1)$.
\end{itemize}
Lo que hace un total de $n(s+d+1)$ parámetros, done $n$ es el número de neuronas, $s$ la dimensión de salida y $d$ la dimensión de entrada. Por lo general $d$ y $s$ son fijos ya que se suponen requisitos del problema, luego si se desea reducir el coste en memoria deberá de hacerse disminuyendo el número de neuronas.

Analizaremos más a fondo su componentes. 

\subsection*{Construcción de la primera capa}
La primera capa está compuesta por el conjunto de $n$ combinaciones
lineales del vector de entrada $(x_1, \ldots, x_d)$
a las cuales denominaremos \textit{activaciones}, $n$ será el número de sumandos definidos en \ref{definition:redes_neuronales_una_capa_oculta}  que equivale al número de neuronas en la capa oculta. 

\begin{equation}
    a_i = w_{0 i} + \sum_{i=j}^d w_{j i} x_j 
    \text{ con } j \in \{1, \ldots, n \}.
\end{equation}

Nos referiremos a los  parámetros $w_{j i}$ como 
\textit{pesos} y al parámetro $w_{0 i}$ como 
\textit{sesgo}.  

La memoria necesaria será $(d+1)n \mathfrak{F}$ 
con $\mathfrak{F}$ es el espacio requerido por un peso. 
El coste computacional es además de $d$ multiplicaciones 
y $d+1$ sumas.

\subsection*{Unidades ocultas}
Cada una de esas \textit{activaciones} será transformada
utilizando una \textit{función de activación} $\gamma_j$ 

\begin{equation}
    z_i = \gamma_i(a_i).
\end{equation}
En el contexto de las redes neuronales a $z_i$ se le conoce como \textit{unidad oculta}. Ésta  podría ser de 
nuevo  transformada por una combinación lineal, en nuestro caso tan solo 
por el producto de un escalar, ya que como se adelantó en la sección \ref{subsection:diferencia-otras-definiciones-RRNN},
 frente a la transformación afín usualmente propuesta, esta transformación no es necesaria para asegurar la convergencia, profundizaremos sobre esto más adelante. 

 Finalmente la salida vendrá dada por
 \begin{equation}
    h_k = \sum_{i=1}^n \beta_{i k} z_i 
    \text{ con } k \in \{1, \ldots, s \}.
\end{equation}
Nótese que ahora el tamaño de variables de entrada es $M$
y hay un total de $s$ unidades de activación, tanto $M$ como $s$ son
valores fijados por el diseñador de la red ya que a priori no tenemos otra información. 

 % Vamos a probar que tampoco mejora el error 
\subsection{Consideraciones sobre la irrelevancia del sesgo}
\label{consideration-irrelevancia-sesgo}
\begin{aportacionOriginal}

 Dados $X \subseteq \R^d, Y \subseteq \R^s$ y  $\Gamma$ un conjunto no vacío de funciones medibles definidas de $\R$ a $\R$, denotaremos como $\mathcal{H}^+(X,Y)$ al conjunto de redes neuronales a las cuales se le ha añadido un sesgo. 

\begin{align}
    \mathcal{H}^+(X,Y) 
    =
    \{
        h : X \longrightarrow Y 
        /& \quad 
        h_k(x) = 
        \sum_{i=1}^{n} \left( \beta_{i k} \gamma_{i}( A_{i}(x)) + \alpha_{i k} \right), \\
        & \text{donde  $h_k$  es la proyección k-ésima de $h$ con 
        $k \in \{1, \ldots, s\}$}, \\
        & n \in \N,\gamma_{i} \in \Gamma , \beta_{i k} \in \R
         \text{ y }A_{i} \text{ una aplicación afín de $\R^d$ a $\R$}           
    \}.
\end{align}


Está claro que al introducir tal sesgo se añade en memoria 
un coste de $n \mathfrak{F}$ con $n$ el número de neuronas en la capa oculta y que además el costo de cómputo se ve aumentado en la misma proporción. 
Sin embargo podría obtenerse un beneficio en cuanto a precisión, 
vamos a proceder  a analizar esta idea. 
Es evidente que 
\begin{equation} \label{eq:conjuntos-redes-neuronales-con-sesgo-contiene-elemental}
    \mathcal{H}(X,Y) \subseteq \mathcal{H}^+(X,Y)
\end{equation}

Además al estar trabajando con una sola capa, se tiene que para cualquier 
$h^+ \in \mathcal{H}^+(X,Y)$

\begin{align}
    h^+ = \sum_{i=1}^{n} \left(\beta_{i k} \gamma_{i}( A_{i}(x)) + \alpha_{i k} \right)
    = \sum_{i=1}^{n} \left(\beta_{i k} \gamma_{i}( A_{i}(x))\right) + O, 
\end{align}
con $O \in \R$ un parámetro libre.

Ante esto, al igual que se hacía en el lema \ref{lema:A_3_función_activación_continua_con_arbitaria}
es fácil obtener una neurona de valor constante y por tanto, para un conjunto fijo de neuronas $n$, se tiene que 
\begin{equation}
    \mathcal{H}^+_n(X,Y) \subsetneq  \mathcal{H}_{n+1}(X,Y),
\end{equation}
de esta relación se obtienen dos cosas: 
puesto que $n$ era arbitrario y 
\begin{equation}
    \mathcal{H}(X,Y) = \bigcup_{n \in \N} \mathcal{H}_n (X,Y)
\end{equation}
entonces como espacios de funciones 
\begin{equation}
    \mathcal{H}(X,Y) = \mathcal{H}^+ (X,Y).
\end{equation}
Por otra parte, la precisión que pueda aportar el sesgo es
superada añadiendo una neurona más a un modelo sin sesgo, es más, también se obtiene una mejora en memoria, ya que 
$\mathcal{H}^+_n(X,Y)$ requiere de $n \mathfrak{F}$ espacio de memoria adicional con respecto a $\mathcal{H}_n(X,Y)$
mientras que $\mathcal{H}_{n+1}(X,Y)$ de $(d +1) \mathfrak{F}$
y por lo visto en el teorema de convergencia universal \ref{teo:MFNAUA}, la precisión se consigue añadiendo neuronas (la dimensión de los datos es fija),
luego podemos suponer que $n$ será mayor que $d+1$. 

Hasta ahora hemos comparado la capacidad de expresión 
por la forma de los elementos de los conjuntos, para comparar su bondad aproximando, vamos a fijar  un 
 número de neuronas en la capa oculta $n$ y 
 y una función de error cualquiera que mida el error dentro 
 del conjunto de entrenamiento
 $E_{\mathcal{D}}:\mathcal{H}^+(X,Y) \longrightarrow \R^+_0$.
 
 Todas las normas en $\R^n$ son equivalentes luego no hay pérdida de generalidad fijando una cualquiera.  

 Definimos también el error dentro del espacio $\Lambda$ como 
 \begin{equation}
    \mathcal{E}_{\mathcal{D}} (\Lambda)
    = \inf \{ E_{\mathcal{D}}(h) : h \in \Lambda\}.
 \end{equation}

Está claro por la relación  
 (\refeq{eq:conjuntos-redes-neuronales-con-sesgo-contiene-elemental})
 que 
 \begin{equation}
    \mathcal{E}_{\mathcal{D}}(\mathcal{H}^+(X,Y))
    \leq
    \mathcal{E}_{\mathcal{D}}(\mathcal{H}(X,Y))
 \end{equation}

 La clave ahora reside en si se satisface la desigualdad opuesta
Es decir, dada cualquier $h^+ \in \mathcal{H}^+_n(X,Y)$ con un error de $E_D(h^+)$ existe $h \in \mathcal{H}_n(X,Y)$  tal que $E_D(h) \leq E_D(h^+).$  

Esto no es posible y para darse cuenta basta con considerar el caso de una neurona, con $G(x)$ su función de activación y $k \in \R^+$ definimos la función ideal como $f(x) = G(x) +k$. 
Tomando tan solo dos puntos convenientemente seleccionados (por ejemplo $M$ y $-M$ tal que $G(-M) = 0 y G(M) = 1$) aprecia que $H_1(\R, \R)$ no puede aproximar $f$ y sin embargo $f \in H^+_1(\R, \R)$.

% Fin de la demostración 

Concluimos tras todo esto que aunque la precisión
 que se pueda obtener con funciones de
 $\mathcal{H}_n(X,Y)$ y $\mathcal{H}^+_n(X,Y)$ es
  diferente para un mismo número de neuronas $n$,
  añadiendo una más el sesgo es irrelevante  y 
  además $\mathcal{H}^+_n(X,Y)$ tiene mayor coste
   computacional, por lo que afirmamos que 
es un artificio de las redes neuronales multicapa para enlazar una capa con otra y que en redes neuronales de una capa oculta carece de sentido.

\end{aportacionOriginal} % fin aportación original irrelevancia del sesgo


% Consideración sobre componer la red neuronal con una función
\subsection{\textit{Modus operandi} ante problemas que requieran un dominio de salida discreto}
\label{ch05:dominio-discreto}

Es usual en la literatura presentar las redes neuronales con la salida compuesta con una función $\theta$, de tal manera que una red neuronal sea de la forma

\begin{equation}\label{red-neuronal-con-compuesta}
    h_k(x) = \theta_k 
    \left(
        \sum_{i=1}^{n} \beta_{i k} \gamma_{i}
    \left(
        w_{0 i} + \sum_{j=1}^d w_{j i } x_i
    \right) 
    \right)
    \text{ para cada  } k \in \{1, \ldots, s \}.
\end{equation}
El teorema universal de convergencia \ref{teo:MFNAUA} nos asegura 
que dado un número lo suficientemente grande de neuronas tal 
composición no es necesaria. Sin embargo, a nivel práctico ese 
número de neuronas puede no alcanzarse y, por cómo se produce la convergencia, el resultado serán funciones con imágenes con forma de dominios contenidos en $\R^n$.

Por la naturaleza de la imagen de ciertos problemas no sería 
necesaria mayor modificación y con nuestra definición sería 
más que suficiente. Pero en caso de que la salida tenga que cumplir alguna restricción, como por ejemplo en 
problemas de clasificación que necesiten de una salida discreta; sería necesario la composición con una función 
$\theta$ que discretice la imagen como observábamos en \ref{corolario:2_5_función_Booleana}.  


La situación por tanto es la siguiente, si procedemos como en \refeq{red-neuronal-con-compuesta}, se tendría que si $f$ es el patrón subyacente de la clasificación y $\theta$ la función que discretiza el dominio se deberá de encontrar 
$h \in \rrnnmc$ tal que $h$ aproxime a $\theta^{-1} \circ f$.

Lo cual exige que $\theta$ tenga inversa y que sea medible, además de la cuestión de qué $\theta$ es la más conveniente $\dots$

\begin{aportacionOriginal} % aportación original salida discreta
    
Es por ello que en nuestra aproximación descartamos este modo de proceder y proponemos el siguiente: 

\begin{enumerate}
    \item Entrenamos $h\in \rrnnmc$, esto dará una salida que no necesariamente será discreta. 
    \item Una vez que se ha calculado $h$,  $\theta$ se determina a partir de $h$. 
\end{enumerate}

\subsubsection*{Construcción de  $\theta$}

Una vez fijada $h$,  para $A \subseteq \mathcal{D}$ se define  el conjunto 
\begin{equation}
    \Lambda = \{(h(x), y) : (x,y) \in A\}.
\end{equation}

A partir del cual $\theta$ será una función que se ha obtenido por el método de clasificación supervisada de los $k$-vecinos más cercanos ($k-$NN) donde el conjunto de entrenamiento es $\Lambda$.

El motivo por el que se ha optado por esta aproximación es que si consideráramos $\theta$ como una función a trozos usual definida como 

\begin{equation}
    \tilde{\theta}(x) = y \text{ si } h(x) \in I_y.
\end{equation}

Donde $I_y$ puede no ser un intervalo conexo y que introduciría complejidad al cálculo y determinación de $\tilde{\theta}$, es por ello que lo natural sería entonces construir un función que discretice a partir de los valores conocidos más cercanos, esto es el algoritmo $k-$NN.  

\end{aportacionOriginal}  % aportación original salida discreta

\subsubsection*{Construcción de $k-$NN}  

Si la dimensión de salida con la que se ha entrenado $h$ es uno, es decir $s=1$. Por ser $\R$ de una dimensión lo natural sería hacer $k=1$, es decir 

\begin{align}
    & \theta(h(x)) = y_{cercano}, \\
    &\text{donde } 
    (h(a),  y_{cercano}) \in A 
    \text{ y }
    h(a) = \min \{|h(x) - h(b)| :  (h(b),  y) \in A  \}.
\end{align}

La implementación de la fórmula anterior sería la siguiente: 

% Algoritmo 1-NN
\begin{algorithm}[H]
   \caption{ Cálculo de la función $\theta$ como clasificador $1-$NN} 
   \textbf{Input: } Conjunto de datos $\Lambda$.  \\
   \textbf{Output: } Función $\theta : \R \longrightarrow \R$
   \begin{algorithmic}[1]
    \STATE \COMMENT{Definimos $\theta$ como} \\
    \SetKwFunction{FSum}{$\theta$}
    \SetKwProg{Fn}{Function}{:}{}
    \Fn{\FSum{$h_x$}}{
        \STATE \textit{distancia mínima} $\gets\infty$
        \STATE $clase \leftarrow NADA$ \\
        \STATE \COMMENT{Encontramos el más cercano}\\
        \ForAll{cada $(h, y)$}{
            \If{$distancia(h,h_x) <$\textit{distancia mínima}}{
               $clase \leftarrow y$
            }
        }
        \Return{$clase$}
    } 
    \Return{$\theta$}
   \end{algorithmic}
\end{algorithm}

%%%%%%%%%%% Fin de lo observado

\subsection{Qué función de activación seleccionar}

Los aspectos a tener en cuenta a la hora de seleccionar una función 
de activación frente a otra de una red neuronal serían los siguientes:
\begin{enumerate}
    \item Espacio de memoria.
    \item Coste computacional.
    \item Efectividad en cuanto a reducir el error de aproximación.
\end{enumerate}

Sobre la primera consideración está claro que el uso de una única función ahorraría el tener que almacenar el tipo de función que se va emplear en cada neurona.

Respecto al coste computacional puede uno basarse en una análisis teórico del número de 
operaciones y su complejidad o realizar un estudio empírico se ha realizado 
en al sección \ref{ch06:coste-computacional-funciones-activacion}.

% Nota margen sobre idea 
\marginpar{\maginLetterSize
    \iconoClave \textcolor{darkRed}{     
        \textbf{
        Estrategia de selección de funciones de activación.
        }
    }

    Esta idea abre la puerta a determinar mediante algoritmos 
    genéticos qué funciones de activación podrían resultar más 
    beneficiosas.
    
}
Fijado cierto número de neuronas, en lo que respecta a la precisión que nos puedan aportar las funciones de activación; por la idea intuitiva 
del funcionamiento de las funciones de activación mostrado en mostrada en \ref{ch03:funcionamiento-intuitivo-funcion-activacion} es un factor que repercute en los resultados 
pero desconocido a priori. 


%% Formulación técnica 

\section{Construcción explícita y evaluación de una red neuronal}

Ante todas las consideraciones expuestas y puesto que 
no existe ningún resultado o hipótesis a favor de combinar funciones de activación, en pos de simplificar el estudio, vamos a suponer a priori que todos los nodos están compuestos con la misma. Entonces,  una red neuronal para nosotros 
vendrá determinada dos matrices de pesos y una función de evaluación. 

Es decir siguiendo la idea constructiva expuesta en manuales tales como \cite{learning-from-data-1-2}
 cualquier $h \in \rrnnsmn$ de $n$ neuronas en la capa oculta se implementa como 
$(\gamma, M_{n \times (1+r)}, M_{(n+1) \times s})$ con $M_{f \times c}$ matrices de dimensiones $f$ filas y $c$ columnas. 

$\gamma$ representa a función de activación en cada nodo. 
$M_{n \times (1+r)}$ son los pesos de la primera capa 
y $M_{(n+1) \times s}$ son los pesos de la salida. 

Tomando como $\mathfrak{I}_{r+1}: \R^r \longrightarrow M_{(r+1) \times 1}(\R)$ a la aplicación inyección que a cada vector $(x_1, \ldots, x_d)$ le hace corresponder el vector columna $(1, x_1, \ldots x_d)^T.$
Además entendemos para una función $\gamma : \R \longrightarrow \R$ 
que la imagen de la composición de una matriz con $\gamma$ es evaluar cada una de sus entradas por $\gamma$. 
Con esta representación evaluar una red neuronal consistiría en el producto matricial de 

\begin{equation}
    h(x) =  M_{(n+1) \times s} \cdot
    \mathfrak{I}_{n+1}\left(
         \gamma \left( 
             M_{n \times (1+d)} 
            \cdot 
            \mathfrak{I}_{d+1}(x)
        \right)
    \right).
\end{equation}

El coste computacional de tal operación es 
\begin{table}[h]
    \begin{center}
    \begin{tabular}{| c | c |}
    \hline
    Operación & Apariciones  \\ \hline
    $+$ & $n d+n s = n(d+s)$  \\
    $\times$ & $n(d+1)+(n+1)s = n(d+s+2)$  \\
    $\gamma$ & $n$  \\
    $\mathfrak{I}$ & $2$  \\
    \hline
    \end{tabular}
    \caption{Coste computacional de la implementación propuesta en \cite{MostafaLearningFromData}}
    \label{tab:coste computacional de la implementación de Mustafa}
    \end{center}
\end{table}

\subsection*{Primera optimización reformulando la implementación de una red neuronal}

Sin embargo de acorde a nuestro modelo definido en \ref{definition:redes_neuronales_una_capa_oculta}, nuestra propuesta de representación es al siguiente: 

\begin{equation}
    (\gamma, A, S, B) 
    \text{ donde } 
    A \in M_{n \times d}(\R), 
    S\in M_{n \times 1}(\R) 
    \text{ y }
    B \in M_{s \times n}(\R).
\end{equation}
y con esta representación la evaluación sería

\begin{equation}
    h(x) =  B \cdot
        \gamma \left( 
            A
            \cdot 
            x
            + S
        \right),
\end{equation}
que tiene un coste computacional de 
\begin{table}[h]
    \begin{center}
    \begin{tabular}{| c | c |}
    \hline
    Operación & Apariciones  \\ \hline
    $+$ & $n d+n(s-1) = n(d+s-1)$  \\
    $\times$ & $n d+n s = n(d+s)$  \\
    $\gamma$ & $n$  \\
    $\mathfrak{I}$ & $0$  \\
    \hline
    \end{tabular}
    \caption{Coste computacional de nuestra implementación de red neuronal de una capa}
    \label{tab:coste computacional nuestr modelo red neuronal}
    \end{center}
\end{table}

Que como vemos ha supuesto una mejora de 

\begin{table}[h]
    \centering
    \resizebox{\textwidth}{!}{
        \begin{tabular}{| c | c | c | c |}
        \hline
        % cabecera 
        Operación 
        & Coste formulación usual \ref{tab:coste computacional de la implementación de Mustafa}
        & Coste nuestra propuesta \ref{tab:coste computacional nuestr modelo red neuronal} 
        & Operaciones reducidas  \\ \hline
        $+$ & $n(d+s)$ & $n(d+s-1)$ & $n$\\
        $\times$ & $n(d+s+2)$ & $n(d+s)$  & $2n$\\
        $\gamma$ & $n$  & $n$  & $0$ \\
        $\mathfrak{I}$  & $2$ & $0$ & $2$ \\
        \hline
        \end{tabular}
    }
    \caption{Comparativas de coste computacional entre la implementación usual de una red neuronal y la nuestra}
    \label{tab:comparativas coste red neuronal }
\end{table}

Además se ha reducido el espacio de memoria en $n$ unidades del tipo de datos que utilicen las redes neuronales. 

\subsubsection*{Ejemplo de evaluación de una red neuronal}

\begin{figure}[h!]
    \includegraphics[width=0.7\textwidth]{4-Actualizacion_redes_neuronales/evaluacion-1-capa.drawio.png}
    \centering
    \caption{Ejemplo de evaluación de una red neuronal de una capa con entrada y salida de tamaño dos y dos neuronas en la capa oculta}
    \label{img:Ejemplo-evaluación-red-neruonal-una-capa}
\end{figure}

A continuación vamos a dar un ejemplo basado en la red neuronal mostrada en la imagen \ref{img:Ejemplo-evaluación-red-neruonal-una-capa}, 
La red neuronal viene dada por las matrices: 
la matriz de la primera capa asociada es 
\begin{equation}
     A = 
        \begin{bmatrix}
            0.1 & -0.2 \\
            -0.3 & -0.1 
        \end{bmatrix}  
\end{equation}
la matriz de sesgos 
\begin{equation}
    S = 
        \begin{bmatrix}
            0.3  \\
            0.5 
        \end{bmatrix}  
\end{equation}
La matriz de pesos de la capa oculta sería 
\begin{equation}
     B = 
    \begin{bmatrix}
        0.1 & -0.6 \\
        -0.6 & 0.5
    \end{bmatrix}. 
\end{equation}

Para la entrada  $x = (0.3, 0.1)$ los sucesivos cálculos serían
\begin{equation}
    A x = 
    \begin{bmatrix}
        0.1 & -0.2 \\
        -0.3 & -0.1 
    \end{bmatrix}  
    \begin{bmatrix}
        0.3  \\
        0.1 
    \end{bmatrix}
    = 
    \begin{bmatrix}
        0.01  \\
        -0.1 
    \end{bmatrix}
    . 
\end{equation}

\begin{equation}
    A x +S= 
    \begin{bmatrix}
        0.01  \\
        -0.1 
    \end{bmatrix}
    + 
    \begin{bmatrix}
        0.3  \\
        0.5 
    \end{bmatrix}
    = 
    \begin{bmatrix}
        0.31  \\
        0.4 
    \end{bmatrix}
    . 
\end{equation}

\begin{equation}
    \gamma (A x +S) = 
    \gamma \left(
    \begin{bmatrix}
        0.31  \\
        0.4 
    \end{bmatrix}
    \right)
    = 
    \begin{bmatrix}
        \gamma(0.31)  \\
        \gamma(0.4) 
    \end{bmatrix}
    = 
    \begin{bmatrix}
        0.3004 \\
        0.3799  \\
    \end{bmatrix}.
\end{equation}

Finalmente las respectivas salidas de las capas ocultas serían
\begin{equation}
    B  \gamma (A x +S)  
    = 
    \begin{bmatrix}
        0.1 & -0.6 \\
        -0.6 & 0.5
    \end{bmatrix}
    \begin{bmatrix}
        0.3004 \\
        0.3799  \\
    \end{bmatrix}
    = 
    \begin{bmatrix}
        -0.1979 \\
       0.0097  \\
    \end{bmatrix}.
\end{equation}

\subsection{Implementación de una red neuronal y evaluación}
\label{section:rrnn_implementation}
Como conclusión a todo lo explicado una red neuronal no 
sería más que una estructura que almacenara $(\gamma, A, S, B)$, esto es 

% Pseudo código que refleja la estructura de una red neuronal 

\begin{algorithm}[H]
    \caption{Estructura de una red neuronal}
    \label{algoritmo:estructura de una red neuronal}
    \DontPrintSemicolon
    \hspace*{\algorithmicindent} 
        \textbf{Entrada}:
        \begin{itemize}
            \item $d$: Dimensión de los datos de entrada.
            \item $s$: Dimensión que define una red neuronal.
            \item $n$: Número de nodos en la capa oculta. 

        \end{itemize}
        \hspace{\algorithmicindent} 
        \Struct{Red neuronal(d,s,n)}{
            $A \gets$ Matriz $n$ filas $d$ columnas\;
            $S \gets$ Matriz $n$ filas $1$ columnas\;
            $B \gets$ Matriz $s$ filas $n$ columnas\;
        }
        \hspace{\algorithmicindent} 
  \end{algorithm}

y si quisiéramos evaluar la red neuronal el algoritmo sería el siguiente 
  %% Algoritmo de evaluación de redes neuronales 

  \begin{algorithm}[H]
    \caption{Evaluación de una red neuronal, \textit{Forward propagation}}
    \label{algoritmo:evaluar red neuronal}

    \hspace*{\algorithmicindent} 
        \textbf{Entrada}:
        \begin{itemize}
            \item $x$: Vector de atributos que se desea evaluar con la red neuronal.
            \item $h = (\gamma, A, S, B)$ representa la red neuronal que se desea evaluar y la función de activación $\gamma$ con la que se va a evaluar. 
        \end{itemize}
        \hspace{\algorithmicindent} 
        \SetKwFunction{Main}{ForwardPropagation}
        \SetKwProg{F}{Function}{:}{}
        \F{\Main{$h, x$}}{ 

            $sensibilidad \gets A \cdot x + S$ \;
            $primeraCapa \gets$ Se crea vector del mismo tamaño que $sensibilidad$ \;
            \ForEach{ entrada-Iésima-Neurona $\in$ sensibilidad}{
                $primeraCapa[i] \gets \gamma($ entrada-Iésima-Neurona $)$
            }
            \;
            $salida \gets B \cdot primeraCapa$ \;
            \KwRet $salida$ \;
        }
  \end{algorithm}
  \newpage







\section{Aprendizaje}  

Se entiende por aprendizaje de una red neuronal como el proceso 
por el cual se determina el valor sus pesos, es decir, lo que en el ejemplo \ref{img:Ejemplo-evaluación-red-neruonal-una-capa} consistía en las matrices $A,S$ y $B$.

%%%%%%%%%%%%%%%%%%% algoritmo de gradiente descendente 

\subsection{Método de gradiente descendente y \textit{backpropagation}} \label{sec:gradiente-descendente}

De acorde a los capítulos uno y dos del libro  \cite{learning-from-data-1-2},
 una vez concretado el problema y sus elementos 
(\ref{sub:componentes_aprendizaje}) es necesario definir un método con 
el que aproximar la función ideal $f,$ para ello introduciremos el algoritmo de gradiente descendente.  

El gradiente descendente es un método iterativo de minimización de funciones diferenciables. 

% Nota sobre el algoritmo de gradiente descendente 
\reversemarginpar
\marginpar{
    \textcolor{dark_green}{    
        \textbf{
            Aclaración gradiente red neuronal
        }
    }

    Puede a priori uno confundirse con la notación,  
    pero recordemos que las redes neuronales estaban determinadas por sus parámetros (matrices), luego lo único que se está haciendo es derivar con respecto a tales parámetros.
} 
%Fin nota margen
% Idea  sobre el algoritmo de gradiente descendente 
\marginpar{
    \textcolor{dark_green}{    
        \textbf{
            Idea general del algoritmo gradiente descendente
        }
    }

    Cada iteración se obtendrá una nueva red neuronal con un error menor dentro de los datos de entrenamiento del conjunto.
} 

% Nota margen sobre diferenciabilidad
\normalmarginpar
\setlength{\marginparwidth}{\smallMarginSize}
\marginpar{
    \textcolor{red}{    
        \textbf{
            Consecuencia del requisito de diferenciabilidad 
            de $E(h)$
        }
    }
    $E(h)$ será diferenciable si y sólo si las funciones de activación lo son. 
}


%Fin nota margen
En nuestro caso particular se quiere aproximar la función ideal desconocida $f$ a partir de funciones (redes neuronales) $h \in \rrnnmc$, concretamente se fijará un número $q$ de neuronas en la capa oculta. 
Dada también una función de error diferenciable y que no presente puntos de inflexión
$E: \mathcal{H}_q (\R^d, \R^s) \longrightarrow \R,$
se toma red neuronal cualquiera $h_0 \in \mathcal{H}_q (\R^d, \R^s)$ y 
fijado $\eta \in \R^+$. 

Se define la sucesión 
\begin{equation}\label{eq:descenso-gradiente}
    h_{t+1}  = h_t - \eta \nabla E(h_t).
\end{equation}  

Donde $h_n$ es una sucesión cuyos términos convergen a un mínimo local.
\subsubsection*{Observaciones sobre el algoritmo }

\begin{itemize}
    \item El algoritmo solo encuentra óptimos locales con una dependencia crucial del valor de inicio. 
    \item La convergencia no es segura en un tiempo finito y requiere de criterios de parada. 
    \item Debe de fijarse el número de neuronas en la capa oculta a priori.
    \item Si la función es convexa el mínimo será global.
    \item El parámetro $\eta$ puede ser cualquiera y debe de ser fijado o controlado por el diseñador.  
\end{itemize}


Con el fin de reducir el coste del cálculo del gradiente, 
se utiliza el algoritmo conocido como \textit{backpropagation} que fue publicado en 
1989 en el artículo \cite{backpropagation-Hinton}. 

Denotaremos como $w$ al conjunto de parámetros que determinan el valor de una red neuronal. 

Sea $E_{in}(h_w)$ la función de error habitualmente usada, la cual tomaremos como el error dentro de conjunto de entrenamiento, esto es,  si el conjunto 
de entrenamiento está constituido por $N$ datos de la forma $(x_n, y_n)$ con $x_n$ el vector de entrada y $y_n$ el estado o valor deseado para cualquier $n\in \{1, \ldots, N\}.$
\begin{equation}
    E_{in}(h_w) = \frac{1}{N} \sum^N_{n=1} (h_w(x)- y_n)^2. 
\end{equation}
que es una métrica para medir error entre, en nuestro caso  
la red neuronal $h_w$ y los valores de entrenamiento.

Para nuestro caso de una sola capa oculta el resultado sería el siguiente: 
\begin{equation}
    \nabla_w E_{in}(h) = \frac{2}{N} \nabla_w h_w(x)
\end{equation}
y puesto que $\frac{2}{N}$ no es más que una constante que 
puede ser corregida en \refeq{eq:descenso-gradiente} con $\eta$ con el fin de ahorrar coste computacional la 
omitiremos de ahora en adelante. Es decir, podemos suponer que 
nuestra función de error a minimizar es 

\begin{equation}
    E_{in}(h) = \frac{1}{2} \sum^N_{n=1} (h_w(x)- y_n)^2. 
\end{equation}

Tengamos presente que hemos definido una red neuronal  $h_w \in \mathcal{H}_q (\R^d, \R^s)$ como
\begin{equation}
    h_w(x) = 
    \sum_{i=1}^q \beta_i 
    \sigma
    \left(  
        \alpha_{0 i} +
        \sum_{j=1}^d \alpha_{j i}x_j
    \right)
\end{equation}
para la cual hemos impuesto que la función de activación $\sigma$ sea diferenciable.

Así pues a la hora de calcular el gradiente tendríamos tres tipos de derivadas parciales, las dependientes de $\beta_i$, 
las de $\alpha_{0 i}$ y las de $\alpha_{j i}$, para cada caso concreto y en virtud de la regla de la cadena, se tiene: 
\begin{itemize}
    \item Derivada parcial del error con respecto a $\beta_i$:
    \begin{align} \label{eq:parcial_beta}
        \frac{\partial E(h)}{\partial \beta_i} 
        =
        \frac{\partial h(x)}{\partial \beta_i} 
        = 
        \sigma
    \left(  
        \alpha_{0 i} +
        \sum_{j=1}^d \alpha_{j i}x_j
    \right).
    \end{align}

    \item Derivada parcial del error con respecto a $\alpha_{0 i}$:
    \begin{align} \label{eq:parcial_alpha_cero}
        \frac{\partial E(h)}{\partial \alpha_{0 i}} 
        =
        \frac{\partial h(x)}{\partial \alpha_{0 i}} 
        = 
        \beta_i \sigma'
    \left(  
        \alpha_{0 i} +
        \sum_{j=1}^d \alpha_{j i}x_j
    \right).
    \end{align}

    \item Derivada parcial del error con respecto a $\alpha_{j i}$:
    
    \begin{align} \label{eq:parcial_alpha_i}
        \frac{\partial E(h)}{\partial \alpha_{j i}} 
        =
        \frac{\partial h(x))}{\partial \alpha_{j i}} 
        = 
        \beta_i \sigma'
    \left(  
        \alpha_{0 i} +
        \sum_{j=1}^d \alpha_{j i}x_j
    \right) x_j.
    \end{align}
\end{itemize}  

Si el cálculo se hiciera sin tener más consideración alguna que la propia expresión supondría un coste computacional de: 

% tabla con coste en multiplicación 


\begin{table}[h]
    \begin{center}
    \begin{tabular}{| c | c | c | c | c | c| }
    \hline
    % cabecera
       & Número de parámetros & $+ / -$ & $\times / \div$ & $\sigma$ & $\sigma'$
    \\ \hline
    % Para betas
    (\ref{eq:parcial_beta}) $\frac{\partial E(h)}{\partial \beta_i}$ 
    & $n s$ & $n s (d+1)$ & $n s d$ & $n s$ & 0
    \\
    \hline
    (\ref{eq:parcial_alpha_cero}) $\frac{\partial E(h)}{\partial \alpha_{0 i}}$ 
    & $n$ & $n (d+1)$ & $n s d$ & $n s$ & 0
    \\
    \hline
    % Para los segos alpha 0i 
    \end{tabular}
    \caption{Coste computacional aplicar para el cálculo directo para actualizar $h \in \mathcal{H}_n(\R^d, \R^s)$}
    \label{tab:coste-computacional-directa}
    \end{center}
\end{table}

A la vista de estas expresiones debemos de definir un algoritmo que busque ahorro en memoria y en coste computacional.

Para el coste computacional notaremos que hay cálculos que se repiten en (\refeq{eq:parcial_beta}), (\refeq{eq:parcial_alpha_cero}) y  
(\refeq{eq:parcial_alpha_i}). 

\begin{table}[h]
    \begin{center}
    \begin{tabular}{| c | c | c | c | c| }
    \hline
    % cabecera
    Apariciones de cierta expresión en 
    & $\frac{\partial E(h)}{\partial \beta_i}$ 
    & $\frac{\partial E(h)}{\partial \alpha_{0 i}}$ 
    &$\frac{\partial E(h)}{\partial \alpha_{j i}}$ 
    & Total apariciones 
    \\ \hline
    % primer cálculo repetido 
    $\mathfrak{S} =  \alpha_{0 i} \sum_{j=1}^d \alpha_{j i}x_j$ 
    & 1 & 1& $d$ & $d+2$
    \\
    % Segundo cálculo repetido 
    $\mathfrak{D} = \sigma'
    \left(  
        \mathfrak{S}
    \right) = \sigma'
    \left(  
     \alpha_{0 i} 
     \sum_{j=1}^d \alpha_{j i}x_j
    \right)$
    & 0 & 1 & $d$  & $d+1$
    \\
    % Tercer cálculo repetido 
    $\beta_i \mathfrak{D} = \beta_i \sigma'
    \left(  
     \alpha_{0 i} 
     \sum_{j=1}^d \alpha_{j i}x_j
    \right)$
    & 0 & 1 & $d$ & $d+1$
    \\ \hline
    \end{tabular}
    \caption{Veces que se calcula cierta expresión fijado un $i$}
    \label{tab:expresiones_repetidas_en_descenso_gradiente}
    \end{center}
\end{table}

A la vista de los resultados expuesto en la tabla \ref{tab:expresiones_repetidas_en_descenso_gradiente}, exponemos la siguiente relación coste computacional coste en memoria


%%%%%%%%%%%%%%%%%%%%%%%%%%%%%%%%%%%
%%% Alternativas al algoritmo de gradiente descendente 
%%%%%%%%%%%%%%%%%%%%%%%%%

\subsection{Otras alternativas al algoritmo de gradiente descendente}  
\label{ch05:alternativas-gradiente-descendente}
Recordemos que nuestro enfoque partía de
fundamentar toda decisión de diseño de manera rigurosa. Destaquemos por tanto que en el algoritmo de gradiente descendente se introduce la restricción de que 
las funciones de activación deben de ser diferenciables.

Esto, a priori, no es de extrañar, ya que si buscamos optimizar algún aspecto, en algún momento deberemos de particularizar o reenfocar algunos de los componentes del problema, sin embargo ¿exigir tales restricciones está sustentado teóricamente? 

Desde un punto de vista analítico la forma por excelencia de minimizar radica en la derivación. Pudiéndose relajar el concepto de derivada a derivada 
débil o incluso trabajar en el ambiente de teoría de distribuciones, donde \textit{casi todo} se puede 
derivar \cite{teoriaDistribuciones}.
 La clave sería poder implementar los cálculos y obtener buenos resultados experimentales.

La cuestión reside entonces si existen otras alternativas al algoritmo de gradiente descendente 
que podrían reducir el costo computacional del proceso de aprendizaje. 

Para ello se ha consultado el estado del arte actual encontrando publicaciones como: 

\subsubsection{\textit{Gradients without Backpropagations}}

A principios de este mismo año, se publicó el artículo \textit{Gradients without Backpropagations} \cite{forwardGradient}, en él se introduce el algoritmo al que acuñan como 
\textit{forward propagation} y al que posicionan los propios autores como una alternativa \textit{más eficiente en coste} que el algoritmo de \textit{Backpropagation}. 

Por desgracia, en este artículo no se da una demostración formal que verdaderamente explique el beneficio computacional, sino que se basan en meras experimentaciones. 

Sin embargo, tiene su interés y es por ello que lo mencionamos, en que indica que existe margen de mejora 
imponiendo como restricción el que las funciones de activación sean diferenciables. 
Esta tendencia se puede ver también en artículos como \cite{TransactionsOnNeuralNetworks}.




%%%%%%%%%%%%%%%%%%%
%% Optimización de la inicialización de los pesos de una red neuronal 
%%%%%%%%%%%%%%%%%%%%%%%%%

\section{ Inicialización de los pesos de una red neuronal}  
\label{section:inicializar_pesos}
Como observábamos en la sección \ref{sec:gradiente-descendente}, el gradiente descendente pretende en cada 
cada iteración mejorar la solución encontrada, pero es 
totalmente sensible a la posición inicial 
de los pesos. 
Presentamos por tanto la siguiente propuesta para inicializar una red neuronal de modo que sus pesos se encuentre ya lo suficientemente cerca de la solución, 
no solo servirá exclusivamente para el método de gradiente descendente 
sino para cualquier otro dependiente del punto inicial. 

\subsection{ Estado del arte relacionado con esto} 

\textcolor{red}{TODO: Estado del arte. Buscar sobre backbones y otros
sistema y cómo influye el estado inicial. Argumentar con esto 
que este algoritmo permitiría generar nuestro propio backbone}

\subsection{Descripción del método propuesto}

\begin{aportacionOriginal} % método de construción
    
La idea proviene de la demostración casi constructiva del teorema \ref{teorema:2_5_entrenamiento_redes_neuronales}.

Se desea inicializar los pesos de $h \in \rrnnsmn$, para la cual, una vez fijado el número $n$ de neuronas de nuestra red neuronal, será necesario  determinar un subconjunto $\Lambda \mathcal{D}$ de datos de entrenamiento. 

La bondad del resultado depende en gran medida de $\Lambda$, 
puesto que a priori se carece de hipótesis, se seleccionará 
de manera aleatoria bajo supuesto de una distribución 
independiente e idénticamente distribuida de los datos. 

Como apunta la demostración, debe de encontrarse un 
$p \in \R^d$ satisfaciendo que $p \cdot (x_i-x_j) \neq 0$ para cualesquiera
atributos $x_i,x_j$ distintos de $\Lambda$.  

Es decir que se estaría considerando un vector que no 
pertenezca a una unión finita de hiperplanos ortogonales de $\R^r$. 
De manera teórica la probabilidad de seleccionar un $p$ y 
que pertenezca al espacio ortogonal es $0$, sin embargo esto 
no quiere decir que no pueda pasar. 

Tomaremos por tanto un $p$ aleatorio y a partir de él 
seleccionaremos $\Lambda$ lo suficientemente grande para que
 al menos $n$ vectores admitan de manera estricta la ordenación: 

\begin{equation}\label{eq:method_inicializar_condition_desigualdad}
    p \cdot x_1 < 
    p \cdot x_2 
    < \cdots <
    p \cdot x_n.
\end{equation}
Para continuar, para la función de activación 
seleccionada $\gamma$, por cómo se definen 
existirá un $M \in \R^+$ tal que 
\begin{equation} \label{eq:method_inicializar_M}
    \gamma(K)=1 \text{ y } \gamma(-K)=0 
    \text{ sean constantes para todo }K \geq M.
\end{equation}

Una vez concretados los valores $p$, $\Lambda$ y $M$ que satisfagan las condiciones 
(\refeq{eq:method_inicializar_condition_desigualdad}) 
y (\refeq{eq:method_inicializar_M})  
falta concretar los valores iniciales de la red neuronal. 

Para ello debemos de calcular el valor de las matrices $(A,S,B)$ que definen a una red neuronal y que presentamos en la sección \ref{section:rrnn_implementation}.

Recordemos que $A$ y $S$ tienen tantas filas como neuronas  y $B$ tantas columnas como neuronas. 

Usado la notación vectorial
$p_{[i,j]} = (p_i, p_{i+1}, \ldots, p_{j})$ donde $(p_0, p_1, \ldots, p_d)=p$, comenzaremos definiendo el valor de la primera fila como

\begin{align}
    &S_1 = M p_0, \\
    & A_{1 *} = M p_{[1,d]}, \\
    & B_{* 1} = y_1.
\end{align}

Los valores de la fila  k-ésimas de las matrices $(A,S)$, vendrán determinados por la única función afín $A \in \afines$, 
dada por $A_k(x)=B_k(p \cdot x)$, con $B_{k}$ como la única función afín de $\R$ en $\R$ que cumple que 
\begin{equation}
    B_k(p \cdot x_{k-1}) = -M 
    \quad \text{y} \quad 
     B_{k}(p \cdot x_k)= M.
\end{equation}

Que esto equivale a calcular las constantes reales $\tilde {\alpha}$. 

Si tenemos presente que 
\begin{equation}
    \tilde{\alpha}_{k p} (p \cdot x_{k-1}) + \tilde{\alpha}_{k s} = -M 
    \quad \text{y} \quad 
    \tilde{\alpha}_{k p}(p \cdot x_k) + \tilde{\alpha}_{k s}= M.
\end{equation} 
Resolviendo el sistema resulta que 

\begin{equation}
    \left\{ 
        \begin{array}{l}
            \tilde{\alpha}_{k p} = \frac{2 M}{p \cdot (x_k - x_{k-1})}
            \\
            \tilde{\alpha}_{k s} 
            = M -  \tilde{\alpha}_{k p}(p \cdot x_{k-1})
            = M -  \frac{2 M}{p \cdot (x_k - x_{k-1})}(p \cdot x_{k-1}) 
        \end{array}
    \right.
\end{equation}

Luego los coeficientes de la red neuronal $A$, $S$ se deduciría de 
\begin{equation}
    \left\{ 
        \begin{array}{l}
            \alpha_{k 0} = \tilde{\alpha}_{k s} =
            M -  \frac{2 M}{p \cdot (x_k - x_{k-1})}(p \cdot x_{k-1}) 
            \\
            \alpha_{k i} =  \tilde{\alpha}_{k p} p_{i}
            = 
            \frac{2 M}{p \cdot (x_k - x_{k-1})}
            p_i 
        \end{array}
        \right.
\end{equation}

Esto define un sistema lineal compatible
cuya solución son las respectivas filas y columnas: 

\begin{equation}
    \left\{ 
        \begin{array}{l}
            S_{k} = M -  \frac{2 M}{p \cdot (x_k - x_{k-1})}(p \cdot x_{k-1})\\
            A_{k i} = \frac{2 M}{p \cdot (x_k - x_{k-1})}
            p_{i}  
            \\
            B_{* k} = y_k - y_{k-1}
        \end{array}
    \right.
\end{equation}  

Con todo esto el proceso algorítmico resultante es: 

\end{aportacionOriginal} % método de construción

% Algoritmo de inicialización de pesos de una red neuronal

\begin{algorithm}[H]
    \caption{Inicialización de pesos de una red neuronal}
    \textbf{Input:} Tamaño red neuronal $n$, conjunto de datos de entrenamiento $\mathcal{D}$, constate $M$ involucrada en \refeq{eq:method_inicializar_M}.

    \textbf{Input:} Red neuronal, representada con las matrices $(A,S,B)$.
    \hspace*{\algorithmicindent} 
    \begin{algorithmic}[1]
        %selección de p
       \STATE \textit{Inicializamos $p$}. \\
       $p \gets$ vector de $\R^{d+1}$. 
       \COMMENT{Como heurística será generado con distribución uniforme en el intervalo $[0,1]$} 
       % Cálculo de Lambda
       \STATE \textit{Selección datos inicialización
       $\Lambda \subset \mathcal{D}$}. \\
       \begin{equation}
           \Lambda \gets \{ \emptyset \}
       \end{equation}
       \While{tamaño de $\Lambda < n$}{
            Tomamos de manera aleatoria $(x,y)$ de $\mathcal{D}$.   \\
        \If{para todo $(a,b) \in \Lambda$ se satisface que 
        $p \cdot (x-a)$}{ 
           \begin{equation}
                \Lambda  \gets \Lambda \cup \{(x,y)\}.
           \end{equation} 
           \COMMENT{$\Lambda$ está ordenado conforme a la propiedad 
           \refeq{eq:method_inicializar_condition_desigualdad} 
           }
        }
       }
       \STATE \textit{Cálculo de los parámetros base de la red neuronal.} \\
       
       Para el primer $(x_1, y_1) \in \Lambda$ \\
       \begin{align}
            &S_1 = M p_0, \\
            & A_{1 *} = M p_{[1,d]}, \\
            & B_{* 1} = y_1.
        \end{align}
       $\Lambda \gets \Lambda \setminus \{(x_1, y_1)\} $ \\
       \STATE \textit{Cálculo del resto de neuronas}. 
       \For{ cada $(x_k, y_k) \in \Lambda$}{
        \begin{align}
            &S_{k} = M -  \frac{2 M}{p \cdot (x_k - x_{k-1})}(p \cdot x_{k-1})\\
            & A_{k i} = \frac{2 M}{p \cdot (x_k - x_{k-1})}
            p_{i}  \quad i \in \{1, \ldots d\},\\
            & B_{* k} = y_k - y_{k-1}.
        \end{align} 
       }
       \STATE \textbf{return $(A,S,B)$}.
    \end{algorithmic}  
\end{algorithm}

\textcolor{red}{TODO: Añadir coste computacional}

\textcolor{red}{TODO: Hacer observaciones sobre que sería igual de válido cambiando la $M$ y $p$ ¿hay alguna selección mejor que otra? }

\textcolor{red}{TODO: Hacer experimentaciones }


% Hipótesis
\part{Exploración de las hipótesis planteadas y estudio experimental de las mismas}
\include{capitulos/5-Estudio_experimental/1_funciones_activacion}
%%%%%%%%%%%%%%%%%%%%%%%%%%%%%%%%%%%%%%%%%%%%%%%%%%%%%%%%%%%%%%%%%%%%
% Experimentación con ALGORITMO INICIALIZACIÓN DE PESOS
%%%%%%%%%%%%%%%%%%%%%%%%%%%%%%%%%%%%%%%%%%%%%%%%%%%%%%%%%%%%%%%%%%%%

\chapter{Algoritmo de inicialización pesos de una red neuronal}  

Se desea conocer la bondad del algoritmo propuesto 
en \ref{section:inicializar_pesos} que plantea dos objetivos múltiples en lo que respecta a la inicialización de pesos de una red neuronal:

\begin{itemize}
    \item Su inicialización reporta un beneficio considerable con respecto a una inicialización aleatoria. 
    \item A mismo tiempo es más ventajoso que utilizar un descenso de gradiente. 
\end{itemize}

Para ello se han diseñado dos experimentos: 
 
\section{Experimento 1: Contraste de hipótesis con inicialización aleatoria} 
\label{ch07:experimento-1} 

Las preguntas a resolver son ¿mejora nuestro algoritmo? ¿Cuánto mejora?

La primera observación  es que como
hemos observado en el modelado de una red neuronal 
en la sección \ref{ch05:construction-evaluation-nnnn}
una red neuronal depende de varios parámetros:
la dimensión de entrada $d$, el número de neuronas en la capa oculta $n$, la dimensión de salida $s$ 
y la funciones de activación de cada neurona.  

Por simplicidad fijaremos una función de activación 
Así que deberemos de formular el test 
para diferentes tamaños $n$, $d$, $s$. 

\textcolor{red}{Ahora mismo no tengo muy claro 
los tamaños porque tampoco quiero que dure mucho tiempo la realización del experimento, los concretaré tras unas primeras pruebas}. 

\subsection{Descripción experimento}

El experimento costa de los siguientes pasos: 

\begin{enumerate}
% Paso 0: Selección de data sets 
\item Dado un conjunto de datos de entrenamiento $\D$  se separará el conjunto en
\begin{itemize}
    \item $\D_i$ \textbf{Conjunto de 
    datos de inicialización.} Debe de ser mayor que 
    $n$ y lo suficientemente grande para que el algoritmo diseñado funcione correctamente. 

    \item $\D_t$ \textbf{Conjunto de 
    datos de test.} Se utilizarán para el cálculo del error. 
\end{itemize}  

% Paso 1: Construcción 
\item Fijados $n, d$ y $s$ se generarán dos redes neuronales: 

\begin{itemize}
    \item Una inicializada de manera aleatoria con valores dentro de un rango de valores. 
    
    \item  Otra inicializada con nuestro algoritmo. 
\end{itemize}

% Paso 2: Evaluación del error
\item Utilizando $\D_t$ deberá de tomarse un registro del error dentro de tal muestra. 
\end{enumerate}

Los pasos 2 y 3 se repetirán tantas veces como 
muestras se desee tomar. 

\subsection{Contraste de hipótesis}

Se desea comparar si los errores observados efectivamente son notables: 

Para ello se realizará un test de Wilcoxon, con las siguientes hipótesis

\begin{itemize}
    \item $H_0$: La mediana de las diferencia de cada par de muestras es $0$. 
    \item $H_a$: La mediana de las diferencia entre cada par de muestras es diferente de cero. 
\end{itemize}

La utilidad de este test es que si rechaza la hipótesis la hipótesis nula sabremos que con un $95 \%$ de certeza tendrán medianas diferentes, es decir, \textbf{existe una 
diferencia en los errores}. En caso de que no se rechace no podremos afirmar nada.
Puede encontrar la implementación en el repositorio del
 proyecto \footnote{En el directorio de experimentos 
 de \url{https://github.com/BlancaCC/TFG-Estudio-de-las-redes-neuronales}.}.

\subsection{Requisitos técnicos}  

A la vista de todo el proceso es descrito surgen las siguientes necesidades técnicas que deberemos de implementar:  

\subsubsection{Capacidad de crear una red neuronal aleatoria}  

Deberá de crearse una red neuronal con entradas dentro de un rango $[a,b]$ con $a < b$ reales,
que tenga una entrada de tamaño $d$,
$n$ neuronas en la capa oculta y
una dimensión de salida $d$.

\subsubsection{Implementación del algoritmo de inicialización}

Deberá de implementarse del algoritmo  \ref{algo:algoritmo-iniciar-pesos} con todos los requisitos y atributos que ahí se describe.  

\subsubsection{Función para medir el error}

Deberá implementarse una función para medir el
 error, no es lo mismo problemas de clasificación 
que de regresión, así que deberemos de ir con 
cuidado. 

Además deberá de realizarse una busca de los datos 

\subsubsection{ Forma de evaluar las redes neuronales}  

Dado una red neuronal, una función de evaluación y un datos ser capaz de aplicar el algoritmo de \textit{forward propagation} descrito en \ref{algoritmo:evaluar red neuronal}.


\subsubsection{Bases de datos de prueba}
\textcolor{red}{Nota:
Comenzaremos probando con bases de datos de juguete 
y en función de tiempo y prestaciones ya veremos si merece la pena plantearse el uso de datos reales
}

\subsubsection{Implementación del experimento} 
Deberá de implementarse una función que realice el 
experimento tal cual hemos descrito en \ref{ch07:experimento-1}.

Podrían utilizarse alguno de estos: 

\begin{itemize}
    \item \href{https://github.com/JuliaStats/RDatasets.jl}{Julia contiene las base de datos estándar de R}.
    \item \href{https://juliaml.github.io/MLDatasets.jl/stable/}{Otros paquetes básicos provistos también por la comunidad.}
\end{itemize}

% !TeX root = ../../tfg.tex
% !TeX encoding = utf8
%
%*******************************************************
% Hipótesis planteadas 
%*******************************************************

\chapter{Hipótesis de optimización }
\textcolor{red}{ATENCIÓN: Todos este capítulo está como notas personales}

\section{A qué nos referimos con optimización}
ES necesario decir qué queremos optimizar
Ejemplo: 
- Mejores resultados para mismo tiempo. 

Medir error y tiempo de cálculo. 

Es por ello que es necesario establecer cómo lo vamos a medir.



\textcolor{red}{ATENCIÓN: Todo este capítulo está como notas personales}  


En esta sección recopilaremos las posibles ideas que podrían optimizar las 
redes neuronales, describiremos una experimentación para contrastar los resultados y mostraremos sus conclusiones. 

\section{Democratización de la función de activación}\label{hypothesis:activation-function}

La primera pregunta, existe alguna función de activación 
claramente mejor en algún sentido que las otras. 

Haciendo un estudio computacional de evaluaciones concretas sí. 
(TODO: hacer experimento)

Pero eso no significaría que fuera mejor para
evaluar los resultados en una red neuronal real. 
(hacer experimento)

Este experimento depende de los datos y da lugar a la siguiente pregunta. 

¿Existe una dependencia en la mejora de los resultados 
con respecto de los datos?

Es decir si tenemos dos redes neuronales $f$ y $g$  de mismo número de neuronas y distintas funciones de activación y dos conjuntos de entrenamiento $D_1$, $D_2$

¿Podría darse el caso de que para $D_1$ $f$ aprenda mejor pero que para $D_2$ $g$ sea mejor?. 


Vamos a tratar de encontrar de encontrar un ejemplo de esto.

\section{Inicialización de la pesos red neuronal}\label{hypothesis:pesos-iniciales}

\section{Construcción dinámica del número de neuronas}




% --------------------------------------------------------------------
% APPENDIX: Opcional
% --------------------------------------------------------------------

\appendix % Reinicia la numeración de los capítulos y usa letras para numerarlos
\pdfbookmark[-1]{Apéndices}{appendix} % Alternativamente podemos agrupar los apéndices con un nuevo \part{Apéndices}


%% !TeX root = ../libro.tex
% !TeX encoding = utf8

\chapter{Documentación}\label{ap:documentacion}

En este apéndice se deja la documentación en estilo \emph{python} de la documentación de las distintas clases, métodos y funciones más importantes implementados en el proyecto.

\section{Selección de Modelos}

Las clases implementadas para la parte de selección de modelos que se pueden encontrar en la carpeta $PV/src$.

\subsection{Perturbated Validation}

Esta clase y sus métodos se encuentran en el archivo $PV.py$.

\paragraph{PV}

Clase PV que implementa el método para calcular y manejar la heurística PV. Guarda los datos de las series originales, las perturbaciones realizadas, los ratio de error, el nombre del dataset que se está perturbando, y valores auxiliares para imprimir gráficas del cálculo del PV.

\begin{lstlisting}
class PV:
    """
        Clase que implementa Perturbation Validation (PV).

        Attributes
        ----------
        X : np.array
            Dataset
        y : np.array
            Conjuto de etiquetas perturbadas
        ds_name : str
            Nombre del dataset
        errs : np.array
            Errores tomados
        counter : int
            Contador auxiliar
        fig : Figure
            Figura actual
        ax : Axes
            Ejes actuales
    """
\end{lstlisting}

\paragraph{Constructor}

Constructor de la clase PV que necesita los datos originales, el número de perturbaciones, el nombre del \emph{dataset}, y el inicio y fin de los ratio de error. Crea los conjuntos de etiquetas perturbadas.

\begin{lstlisting}
def __init__(self, X, y, n_pv = 5, ds_name = "", err_ini = 0.1,
                 err_fin = 0.3):
        """
            Inicializa la clase creando las etiquetas perturbadas.

            Las perturbaciones se realizan tomando un %err de cada
            clase, poniendole otra etiqueta distinta.

            Se toman "n_pv" puntos entre [err_ini, err_fin].

            Parameters
            ----------
            X : np.array
                Dataset
            y : np.array
                Etiquetas
            n_pv: int
                Número de puntos/errores
            ds_name: str
                Nombre del datases
            err_ini : float
                Error inicial
            err_fin : float
                Error final
        """
\end{lstlisting}

\paragraph{Cálculo PV}

Método para calcular el valor PV de un modelo dado.

\begin{lstlisting}
def get_pv(self, clf, clf_name = "", plot = True):
        """
            Calcula el PV score para el clasificador.

            Parameters
            ----------
            clf : Classifier
                Clasificador
            clf_name : str
                Nombre del clasificador

            Returns
            -------
            pv : float
                PV score
            accs : list(float)
                accs obtenidos
        """
\end{lstlisting}

\paragraph{Dibujar cálculo PV}

Método para representar en una gráfica los valores de la métrica $acc$ obtenidos en el cálculo de PV junto a la recta de regresión obtenida.

\begin{lstlisting}
def plot_pv(self, errs, accs, poly, pv, clf_name = ""):
        """
            Dibuja los puntos y la recta de regresión en la figura actual.

            Parameters
            ----------
            errs : np.array
                Errores
            accs : np.array
                acc obtenidos
            poly : np.array
                Recta de regresión
            pv : float
                Valor PV
            clf_name : str
                Nombre del clasificador
        """
\end{lstlisting}

\paragraph{Guardar gráfica}

Método para guardar en una imagen .png el gráfico del método $plot\_pv$.

\begin{lstlisting}
def save_graph(self, name_fig):
        """
            Guarda el gráfico de los resultados en un .png

            Parameters
            ----------
            name_fig : str
                Nombre (ruta) de la imagen a guardar.
        """
\end{lstlisting}

\subsection{Clasificador LSTM}

Esta clase y sus métodos se encuentran en el archivo $LSTM.py$.

\paragraph{LSTM}

La clase LSTM que implementa el clasificador LSTM. Guarda el modelo LSTM, el número de clases, la longitud de las series, opciones de entrenamiento y para gráficas de entrenamiento.

\begin{lstlisting}
class LSTM(BaseEstimator):
    """
        Implementación de una red neuronal con capas LSTM.

        Attributes
        ----------
        counter : int, static
            Valor auxiliar para ruta de imagen
        model : Sequential
            Modelo red neuronal
        history : list
            Historial del entrenamiento
        n_clases : int
            Número de clases de las etiquetas
        input_shape : tuple
            Forma de los datos
        epochs: int
            Número de épocas para entrenamiento
        verbose : int
            Información sobre el entrenamiento
        save_hist : boolean
            Si guardar las gráficas de los entrenamientos
    """
\end{lstlisting}

\paragraph{Constructor}

Constructor de la clase LSTM que guarda opciones de entrenamiento.

\begin{lstlisting}
def __init__(self, epochs, n_neurs = 80, verbose = 0, save_hist = False,
             n_clases = -1):
        """
            Inicializamos la red LSTM.

            Attributes
            ----------
            epochs : int
                Número de épocas para entrenamiento
            n_neurs : int
                Número de neuronas LSTM
            verbose : int
                Información sobre el entrenamiento
            save_hist : boolean
                Si guardar las gráficas de los entrenamientos
            n_clases : int
                Número de clases a predecir
        """
\end{lstlisting}

\paragraph{Creación del modelo}

Método para crear el modelo LSTM.

\begin{lstlisting}
def create_model(self):
        """
            Crea el modelo LSTM.
        """
\end{lstlisting}

\paragraph{Compilar el modelo}

Compila el modelo con el optimizador ADAM y la función de pérdida entropía cruzada categórica.

\begin{lstlisting}
def compile_model(self):
        """
            Compila el modelo con optimizador ADAM y función de pérdida
            categorical_crossentropy.
        """
\end{lstlisting}

\paragraph{Entrenamiento}

Método para entrenar el modelo con el conjunto de datos, con las épocas guardadas, validación al 10\% y con parada temprana.

\begin{lstlisting}
def fit(self, X, y):
        """
            Entrenamos el modelo.

            Parameters
            ----------
            X : numpy.array
                Datos de entrenamiento
            y : numpy.array
                Etiquetas de entrenamiento
        """
\end{lstlisting}

\paragraph{Cálculo métrica}

Método para calcular la métrica $acc$ en el conjunto de datos pasado.

\begin{lstlisting}
def score(self, X, y):
        """
            Calcula el acc con los datos que se le pasan.

            Parameters
            ----------
            X : numpy.array
                Datos test
            y : numpy.array
                Etiquetas test

            Returns
            ----------
            acc : float
                accuracy obtenida
        """
\end{lstlisting}

\paragraph{Guardar gráfica de entrenamiento}

Método para guardar el historial de entrenamiento en una imagen.

\begin{lstlisting}
def save_history(self):
        """
            Guarda el historial en una imagen.
        """
\end{lstlisting}

\subsection{Clasificadores}

Los clasificadores adicionales que usamos para comparar modelos, comparten dos métodos generales: entrenamiento y cálculo de la métrica.

\paragraph{Entrenamiento}

Método que se encarga de entrenar el modelo usando una muestra de datos.

\begin{lstlisting}
def fit(self, X, y):
    """
        Entrena el modelo.

        Parameters
        ----------
        X : numpy.array
            Datos de entrenamiento
        y : numpy.array
            Etiquetas de entrenamiento
    """
\end{lstlisting}

\subparagraph{Cálculo de métrica}

Método que se encarga de calcular la métrica ($accuracy$) de un modelo en un conjunto de datos.

\begin{lstlisting}
def score(self, X, y):
    """
        Calcula el acc con los datos que se le pasan.

        Parameters
        ----------
        X : numpy.array
            Datos test
        y : numpy.array
            Etiquetas test

        Returns
        ----------
        acc : float
            accuracy obtenida
    """
\end{lstlisting}

\subsubsection{C4.5}

Esta clase y sus métodos se encuentran en el archivo $RClassifiers.py$.

\paragraph{C45}

La clase C45 que usa el árbol de decisión C4.5 que guarda el modelo.

\begin{lstlisting}
class C45(BaseEstimator):
    """
        Implementa el árbol de decisión C4.5.

        Attributes
        ----------
        model : clasificador en R
            El clasificador (clase en R)
    """
\end{lstlisting}

\subsubsection{C5.0}

Esta clase y sus métodos se encuentran en el archivo $RClassifiers.py$.

\paragraph{C50}

La clase C50 que usa el árbol de decisión C5.0 que guarda el modelo, y también el valor del \emph{boosting}.

\begin{lstlisting}
class C50(BaseEstimator):
    """
        Implementa el árbol de decisión C5.0 (con boosting o no).

        Attributes
        ----------
        model : clasificador en R
            El clasificador (clase en R)
        boosting : int
            El valor del boosting
    """
\end{lstlisting}

\paragraph{Constructor}

Constructor de la clase C50 que se le pasa el número de \emph{boosting} que se necesite.

\begin{lstlisting}
def __init__(self, boosting = 10):
        """
            Inicializa el clasificador.

            Parameters
            ----------
            boosting : int
                El valor del boosting
        """
\end{lstlisting}

\subsubsection{Recursive Partioning Tree}

Esta clase y sus métodos se encuentran en el archivo $RClassifiers.py$.

\paragraph{RPart}

Clase que usa el árbol RPart.

\begin{lstlisting}
class RPart(BaseEstimator):
    """
        Implementa el árbol de decisión RPart (Recursive Partioning Tree).

        Attributes
        ----------
        model : clasificador en R
            El clasificador (clase en R)
    """
\end{lstlisting}

\subsubsection{Condicional Tree}

Esta clase y sus métodos se encuentran en el archivo $RClassifiers.py$.

\paragraph{CTree}

La clase CTree implementa el uso del árbol de decisión Condicional Tree.

\begin{lstlisting}
class CTree(BaseEstimator):
    """
        Implementa el árbol de decisión CTree (Conditional Inference Tree).

        Attributes
        ----------
        model : clasificador en R
            El clasificador (clase en R)
    """
\end{lstlisting}


\subsubsection{$k$-NN}

Esta clase y sus métodos se encuentran en el archivo $KNN.py$.


\paragraph{Clase KNN}

Clase que implementa el clasificador $k$-NN que se le puede pasar el $k$ fijo o que lo calcule automáticamente tomado como la raíz cuadrada del número de datos.

\begin{lstlisting}
class KNN(BaseEstimator):
    """
        Implementa el clasificador KNN (K-Nearest neighbors).

        Parameters
        ----------
        k : int
            Número de vecinos
        model : KNeighborsClassifier
            Modelo k-NN
        metric : str, metric
            Métrica que usar con KNN
        n_jobs : int
            Número de procesadores usados
    """
\end{lstlisting}

\paragraph{Constructor}

Constructor de la clase KNN que necesita el número de vecinos, la métrica y el número de procesadores.

\begin{lstlisting}
def __init__(self, k = None, metric = "euclidean", n_jobs = 1):
        """
            Inicializa el clasificador.

            Parameters
            ----------
            k : int
                Número de vecinos
            metric : str, metric
                Métrica que usar con KNN
            n_jobs : int
                Número de procesadores
        """
\end{lstlisting}

\subsubsection{$k$-NN + DTW}

Esta clase y sus métodos se encuentran en el archivo $RClassifiers.py$.

\paragraph{DTW}

Clase que implementa el clasificador $k$-NN con métrica DTW, que guarda los datos de entrenamiento, el número de vecinos y el tamaño de la ventana para aplicar DTW.

\begin{lstlisting}
class DTW(BaseEstimator):
    """
        Clase que implementa K-Nearest Neighbors con la distancia DTW
        usando la implementación del paquete "IncDTW".

        Attributes
        ----------
        data : R.DataFrame
            Datos transformados en un objeto dataframe de R
        k : int
            Número de vecinos
        window_shift : int
            Tamaño de la ventana para aplicar DTW
    """
\end{lstlisting}

\paragraph{Constructor}

Constructor de la clase DTW que necesita el número de vecinos y el tamaño de la ventana para el cálculo de la métrica DTW.

\begin{lstlisting}
def __init__(self, k = 1, window_shift = 5):
    """
        Constructor de la clase, debe hacerse solo una vez por dataset.

        Parameters
        ----------
        k : int
            Números de vecinos
        window_shift : int
            Tamaño de la ventana para aplicar DTW
    """
\end{lstlisting}

\section{Detección de anomalías}

Funciones y clases relativas a la parte de detección de anomalías que se encuentran en la carpeta $AD/src$.

\subsection{Alteración de series}

Funciones para la creación de anomalías en base a las series normales implementadas en el archivo $alteraciones.py$.

\paragraph{Tramo aleatorio}

Función para escoger un tramo aleatorio de la serie en función a la longitud indicada (máxima, mínima, fija).

\begin{lstlisting}
def random_slice(x, max_length = None, min_length = None,
                   length = None, pos = None, border = 0):
    """
        Se encarga de elegir un tramo aleatorio de una serie que queda
        determinado por una posición y longitud, de manera que el tramo
        elegido es [posición, posición + longitud).

        Se puede determinar una longitud máxima o mínima, o incluso
        especificar una longitud o posición fijada. También se puede
        indicar si excluir los extremos (añadir borde).

        Parameters
        ----------
        x : np.numpy
            Serie temporal que alterar
        max_length : int, None
            Longitud máxima de la perturbación
        min_length : int, None
            Longitud mínima de la perturbación
        length : int, None
            Longitud fija de la perturbación
        pos : int, None
            Posición fija de la perturbación
        border : int
            Borde para excluir la perturbación

        Returns
        -------
        pos : int
            Posición de la perturbación
        length : int
            Longitud de la perturbación
    """
\end{lstlisting}

\paragraph{Ruido gaussiano}

Método para alterar un tramo aleatorio de la serie añadiendo ruido gaussiano mediante un parámetro $\sigma$ que controla la intensidad de esta perturbación, y la longitud máxima y mínima de esta.

\begin{lstlisting}
def gaussian_noise(x, max_length, min_length = 3, std = 3, neg = False,
                   border = 0, neg_random = True):
    """
        Crea una perturbación de ruido gaussiano añadiendo en un
        tramo aleatorio un muestreo de la función de densidad normal.
        Se puede controlar la intensidad.

        Además se puede activar aleatoriamente (50%) o de manera fija que la
        alteración gaussiana sea negativa.

        Parameters
        ----------
        x : np.numpy
            La serie para alterar
        max_length : int
            Longitud máxima de la alteración
        min_length : int
            Longitud minima de la alteración
        std : float
            Controla la intensidad de la alteración
        neg : boolean
            Si invertir la señal gaussiana
        border : int
            El borde para excluir la perturbación
        neg_random : boolean
            Si se invierte aleatoriamente las señales

        Returns
        -------
        x : np.numpy
            Una copia de la señal perturbada
    """
\end{lstlisting}

\paragraph{Pulso sinusoidal-gaussiano}

Método para alterar un tramo aleatorio de la serie añadiendo un pulso sinusoidal-gaussiano mediante su frecuencia $fc$, un parámetro $\sigma$ que controla la intensidad de la perturbación y la longitud máxima y mínima de esta.

\begin{lstlisting}
def gaussian_sine_pulse(x, max_length, min_length = 3, fc = 1.5, std = 3,
                        border = 0):
    """
        Crea una perturbación con un pulso sinusoidal-gaussiano añadido en un
        tramo aleatorio. Se puede controlar la intensidad y la frecuencia
        del pulso.

        Parameters
        ----------
        x : np.numpy
            La serie para alterar
        max_length : int
            Longitud máxima de la alteración
        min_length : int
            Longitud minima de la alteración
        fc : float
            Frecuencia de la señal del pulso
        std : float
            Controla la intensidad de la alteración
        border : int
            El borde para excluir la perturbación

        Returns
        -------
        x : np.numpy
            Una copia de la señal perturbada
    """
\end{lstlisting}

\paragraph{Estacionalidad}

Método para alterar un tramo aleatorio de la serie modificando la estacionalidad de la descomposición STL (dada con un periodo) por un parámetro $\sigma$ que controla la intensidad y la longitud máxima y mínima de esta.

\begin{lstlisting}
def modify_season(x, period, max_length, min_length = 3, std = 1, border = 0):
    """
        Crea una perturbación multiplicando por un real la estacionalidad
        de un tramo aleatorio de la serie. Se necesita el periodo para
        realizar la descomposición STL.

        Parameters
        ----------
        x : np.numpy
            La serie para alterar
        period : int
            Periodo de repetición de la serie para descomposición STL
        max_length : int
            Longitud máxima de la alteración
        min_length : int
            Longitud minima de la alteración
        std : float
            Controla la intensidad de la alteración
        border : int
            El borde para excluir la perturbación
    """
\end{lstlisting}

\paragraph{Tendencia}

Método para alterar un tramo aleatorio de la serie modificando la tendencia de la descomposición STL (dada con un periodo) por un parámetro $\sigma$ que controla la intensidad y la longitud máxima y mínima de esta.

\begin{lstlisting}
def modify_trend(x, period, max_length, min_length = 3, std = 1, border = 0):
    """
        Crea una perturbación multiplicando por un real la tendencia
        de un tramo aleatorio de la serie. Se necesita el periodo para
        realizar la descomposición STL.

        Parameters
        ----------
        x : np.numpy
            La serie para alterar
        period : int
            Periodo de repetición de la serie para descomposición STL
        max_length : int
            Longitud máxima de la alteración
        min_length : int
            Longitud minima de la alteración
        std : float
            Controla la intensidad de la alteración
        border : int
            El borde para excluir la perturbación
    """
\end{lstlisting}

\subsection{Detector}

Clase y sus métodos implementados para crear el detector de anomalías, implementado en $detector.py$

\paragraph{Clase LSTM\_AD}

Clase que implementa el detector de anomalías basado en autoencoder LSTM. Mantiene el modelo LSTM, la probabilidad estimada y otros parámetros de entrenamiento.

\begin{lstlisting}
class LSTM_AD:
    """
        Clase que implementa un detector de anomalías usando
        un modelo autoencoder con capas LSTM.

        Attributes
        ----------
        model : keras.Sequential
            Autoencoder LSTM
        n_neur : int
            Número de neuronas base para las capas
        alpha : float
            Parámetro de regularización L2
        lr : float
            Learning rate
        epochs : int
            Número de épocas de entrenamiento
        mode : int
            Si incluir espacio de codificación (1) o no (2)
        hist : keras.Historial
            Historial de entrenamiento
        kernel : scipy.gaussian_kde
            Distribución de errores estimada
    """
\end{lstlisting}

\paragraph{Constructor}

Constructor de la clase LSTM\_AD que guarda los parámetros relativos al entrenamiento y al modo de arquitectura.

\begin{lstlisting}
def __init__(self, n_neur = 32, alpha = 0, lr = 0.001, epochs = 300,
             mode = 2):
    """
        Constructor de la clase

        Parameters
        ----------
        n_neur : int
            Número de neuronas base para las capas
        alpha : float
            Parámetro de regularización L2
        lr : float
            Learning rate
        epochs : int
            Número de épocas de entrenamiento
        mode : int
            Si incluir espacio de codificación (1) o no (2)
    """
\end{lstlisting}

\paragraph{Creación del modelo}

Función para crear la arquitectura del modelo autoencoder LSTM.

\begin{lstlisting}
def create_model(self, X):
    """
        Crea la arquitectura del autoencoder LSTM con los atributos
        de la clase.

        Parameters
        ----------
        X : np.numpy
            Series temporales
    """
\end{lstlisting}

\paragraph{Compilación}

Función para compilar el modelo autoencoder LSTM.

\begin{lstlisting}
def compile_model(self):
    """
        Compila el modelo con ADAM añadiendo un clip de 1, learning
        rate especificado y minimizando el error cuadrático medio.
    """
\end{lstlisting}

\paragraph{Entrenamiento}

Función para entrenar el modelo autoencoder LSTM.

\begin{lstlisting}
def load_model(self, path):
    """
        Carga el modelo de unos pesos guardados en un archivo

        Parameters
        ----------
        path : str
            Ruta donde está el archivo de los pesos
    """
\end{lstlisting}

\paragraph{Reconstrucción}

Función para obtener las reconstrucciones de un conjunto de series temporales.

\begin{lstlisting}
"""
    Obtiene las reconstrucciones del autoencoder para las series.

    Parameters
    ----------
    X : numpy.array
        Datasets de series temporales

    Returns
    -------
    reconstrucciones : numpy.array
        Reconstrucciones de las series temporales
"""
\end{lstlisting}

\paragraph{Estimar distribución}

Función para estimar la distribución de los errores de reconstrucción.

\begin{lstlisting}
def fit_kernel(self, X):
    """
        Ajustamos la distribución de los errores de reconstrucción
        con los datos de entrenamiento.

        Parameters
        ----------
        X : numpy.array
            Dataset de series temporales
    """
\end{lstlisting}

\paragraph{Calcular probabilidades}

Función para obtener las probabilidades de ser serie anómala para un conjunto de series temporales.

\begin{lstlisting}
def predict_prob(self, X):
    """
        Devolvemos las probabilidades de ser serie anómala para
        cada serie del dataset

        Parameters
        ----------
        X : numpy.array
            Dataset de series temporales

        Returns
        -------
        probs : numpy.array
            Probabilidades de anomalía para cada serie
    """
\end{lstlisting}

\subsection{Cálculo Curva Precision-Recall}

Las funciones para calcular la métrica $AUC$-$PR$ (curva precisión-recall) que se encuentran en el archivo $calc\_pr.py$.

\paragraph{Contar anomalías}

Se cuentan el número de anomalías detectadas en función de las probabilidades de las series de ser anómalas y de un umbral de probabilidad al partir del cual se considera que es anómala.

\begin{lstlisting}
def count_anomalies(probs, threshold):
    """
        Cuenta cuantas anomalías hay en función a la probabilidad de serlo
        y un umbral de probabilidad.

        Parameters
        ----------
        probs : np.numpy
            Array con probabilidades de cada serie de ser anómala
        threshold : float
            Umbral de probabilidad a partir del cual se considera anómala

        Returns
        -------
        n_anomalies : int
            Número de anomalías detectadas
    """
\end{lstlisting}

\paragraph{Calcular sensibilidad}

Se calcula la sensibilidad del modelo en base a las probabilidades de las series anómalas y un umbral.

\begin{lstlisting}
def calc_recall(probs_anomalies, threshold):
    """
        Calcula la sensibilidad (recall) de un modelo en base a las
        probabilidades de las series anómalas.

        Parameters
        ----------
        probs_anomalies : np.numpy
            Array con probabilidades de anomalías de las series anómalas
        threshold : float
            Umbral de probabilidad

        Returns
        -------
        recall : float
            Sensibilidad del modelo
    """
\end{lstlisting}

\paragraph{Calcular precisión}

Se calcula la precisión del modelo en base a las probabilidades de las series anómalas y normales junto a un umbral.

\begin{lstlisting}
def calc_precision(probs_normal, probs_anomalies, threshold):
    """
        Calcula la precisión de un modelo en base a las probabilidades
        de las series anómalas y normales.

        Parameters
        ----------
        probs_normal : np.numpy
            Array con probabilidades anomalías de las series normales
        probs_anomalies : np.numpy
            Array con probabilidades anomalías de las series anómalas
        threshold : float
            Umbral de probabilidad

        Returns
        -------
        precision : float
            Precisión del modelo
    """
\end{lstlisting}

\paragraph{Curva Precision-Recall}

Se calcula la métrica $PR$ tomando el área debajo de la curva Precision-Recall integrando en el cuadrado $[0, 1]^2$. Además se imprime una figura mostrando la curva que se forma.

\begin{lstlisting}
def recall_precision_curve(X_normal, X_anomalies, model, clf_name = "clf",
                           title = "recall-precision curve", axis = None,
                           plot = True):
    """
        Calcula la métrica PR y además muestra la curva Precision-Recall
        del modelo.

        Parameters
        ----------
        X_normal : np.numpy
            Series normales
        X_anomalies : np.numpy
            Series anómalas
        model : detector
            Detector de anomalías
        clf_name : str
            Nombre del detector
        title : str
            Título de la gráfica
        axis : matplotlib.axis
            Objeto para imprimir las gráficas
        plot : boolean
            Si imprimir cosas opcionales de la gráfica

        Returns
        -------
        pr_score : float
            Valor de la métrica PR
    """
\end{lstlisting}


\endinput
%------------------------------------------------------------------------------------
% FIN DEL APÉNDICE.
%------------------------------------------------------------------------------------


% Añadir tantos apéndices como sea necesario

% --------------------------------------------------------------------
% GLOSARIO: Opcional
% --------------------------------------------------------------------

%\include{glosario}


% -------------------------------------------------------------------
% BACKMATTER
% -------------------------------------------------------------------

\backmatter % Desactiva la numeración de los capítulos
\pdfbookmark[-1]{Referencias}{BM-Referencias}

% BIBLIOGRAFÍA
%-------------------------------------------------------------------

\setbibpreamble{Las referencias se listan por orden alfabético. Aquellas referencias con más de un autor están ordenadas de acuerdo con el primer autor.\par\bigskip}
\bibliographystyle{alphaurl}
\begin{small} % Normalmente la bibliografía se imprime en un tamaño de letra más pequeño.
\bibliography{library.bib}
\end{small}


% ÍNDICE TERMINOLÓGICO  (Opcional)
%-------------------------------------------------------------------

\cleardoublepage
\begin{footnotesize} % Normalmente el índice se imprime en un tamaño de letra más pequeño.
\printindex
\end{footnotesize}
% !TeX root = ../libro.tex
% !TeX encoding = utf8

%*******************************************************
% Agradecimientos
%*******************************************************

\chapter*{Agradecimientos}

No todos los días termina una de escribir su trabajo fin de grado y en esta euforia casi descomedida me parece oportuno volver la vista atrás y agradecer a todos aquellos que me han acompañado durante el camino. 

Comenzaré por 
las personas que más quiero del mundo, gracias papá y mamá por vuestra infinita paciencia y vuestro amor inconmensurable, sin vosotros no hubiera sido posible (de hecho nada lo sería). 
Gracias también a mis dos tutores, JJ y Javier por toda la atención y consideraciones que me han dedicado, que sepáis que sois dos \textit{soletes} y el cariño que me inspiráis no es poco.

Por supuesto también a todas las personas que me han insuflado ganas de aprender e ir a clase, ya sean esos profesores inspiradores y cercanos o 
todas las personas que me han sacado una sonrisa alguna vez. 

Quiero además añadir una mención especial a mis \textit{algebristas recalcitrantes favoritos} Daniel y Ricardo por haberme aguantado durante tantísimas horas; y como no podía ser de otra forma: a mi fiel compañero de aventuras, a mi Sancho para su Quijote (o su Sancho para mi Quijote, según se tercie), a mi archiamigo Jose, gracias por todos los momentos que hemos compartido y nos quedan por vivir. 


\endinput

\end{document}

% DEFINICIÓN DE COMANDOS Y ENTORNOS

% CONJUNTOS DE NÚMEROS

  \newcommand{\N}{\mathbb{N}}     % Naturales
  \newcommand{\R}{\mathbb{R}}     % Reales
  \newcommand{\Z}{\mathbb{Z}}     % Enteros
  \newcommand{\Q}{\mathbb{Q}}     % Racionales
  \newcommand{\C}{\mathbb{C}}     % Complejos

  %%%%%%%%% Mis comandos %%%%%%%%%
% Para escribir código y pseudo código  
\usepackage{minted}

\usepackage{algorithmic}
% Para la definición de redes neuronales de una sola capa 
\newcommand{\Hu}{\mathcal{H}(X,Y)}  % Espacio de las redes neuronales

% Notas en el margen
\usepackage{sidenotes}
  \newcommand{\afines}{\mathcal{A}(\R^d)}
  \newcommand{\pmc}{\mathcal{H}_G(\R^d,\R)}%{\sum ^r (G)}  % Red neurona una capa una salida
  \newcommand{\pmcg}{ \sum \prod^d (G)} % Generalización red neuronal
  \newcommand{\fC}{\mathcal{C}(\R^d)} %conjunto de funciones continuas en R^r -> R
  \newcommand{\fM}{\mathcal{M}(\R^d)} % Conjunto funiones medibles
  \newcommand{\rrnn}{ \mathcal{H}(\R^d,\R)} % Red neuronal  sin subíndice
  \newcommand{\rrnng}{ \sum \prod^d (\psi)} % Red neuronal  generalizado
  \newcommand{\dist}{\rho_{\mu}}     % Distancia de una medida
  \newcommand{\dlp}{\rho_{p}} % Distancia de los espacios Lp
  % Múltiples salidas 
  \newcommand{\fCC}{\mathcal{C}(\R^d ,\R^s)}
  \newcommand{\fMM}{\mathcal{M}(\R^d , \R^s)}
  \newcommand{\rrnnmc}{ \mathcal{H}(\R^d,\R^s)} 
  \newcommand{\rrnnsmn}{ \mathcal{H}_n(\R^d,\R^s)} % Red neuronal salida múltiple con n neuronas
  \newcommand{\rrnngmc}{ \sum \prod^{d,s} (\psi)} 
  %%%%%%%%% Mis comandos %%%%%%%%%5
\usepackage{sidenotes} % Notas en el margen
\newcommand{\margenimagen}{
  \newgeometry{
      left=2.5cm, % Margen izquierdo
    right=5cm, % Margen derecho
    bottom=2.5cm % Margen inferior}
  }
}
\usepackage{caption}
\usepackage{subcaption}

% TEOREMAS Y ENTORNOS ASOCIADOS

  % \newtheorem{theore<m}{Theorem}[chapter]
  \newtheorem*{teorema*}{Teorema}
  \newtheorem{teorema}{Teorema}[chapter]
  \newtheorem{proposicion}{Proposición}[chapter]
  \newtheorem{lema}{Lema}[chapter]
  \newtheorem{corolario}{Corolario}[chapter]

    \theoremstyle{definition}
  \newtheorem{definicion}{Definición}[chapter]
  \newtheorem{ejemplo}{Ejemplo}[chapter]

    \theoremstyle{remark}
  \newtheorem{observacion}{Observación}[chapter]


\DeclareMathOperator{\sign}{signo}
\usepackage[inline]{enumitem}
\usepackage{mathtools}
\usepackage[spanish,onelanguage,linesnumbered,ruled,vlined]{algorithm2e}
\usepackage{listingsutf8}
\lstset{language=Python,
        literate=
          {ó}{{\'o}}1
          {í}{{\'i}}1
          {á}{{\'a}}1
          {ú}{{\'u}}1
          {é}{{\'e}}1
          {ñ}{{\v{n}}}1
}
\usepackage{tocloft}
\setlength{\cftfignumwidth}{2.55em}
\DeclareMathOperator*{\argmin}{arg\,min}

\SetKwRepeat{Struct}{struct \{}{\}}%

% Para las notas del margen 
%Nota los colores seleccionados han sido creados con una paleta inclusiva
% https://palett.es/6a94a8-013e3b-7eb645-31d331-26f27d
\definecolor{darkRed}{rgb}{ 0.149, 0.99, 0.49}%{1,0.1,0.1}
\definecolor{dark_green}{rgb}{0, 0.24, 0.23} %{0.2, 0.7, 0.2}
\definecolor{blue}{rgb}{0.41, 0.58, 0.659} % sobreeescribimos el azul
\newcommand{\smallMarginSize}{1.8cm}
\newcommand{\bigMarginSize}{3cm}
\newcommand{\maginLetterSize}{\footnotesize} %{\scriptsize}%
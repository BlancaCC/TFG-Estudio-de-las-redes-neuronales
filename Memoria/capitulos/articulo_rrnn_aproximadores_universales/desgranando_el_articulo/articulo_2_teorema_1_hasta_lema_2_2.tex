% !TeX root = ../../tfg.tex
% !TeX encoding = utf8
%
%*******************************************************
% Contenido del artículo 2: Primeros resultados
%*******************************************************


\section{Primeros resultados} 
% Introducción sección 


%%%%% primer teorema de convergencia  
% Teorema 2.1 
\begin{teorema} [Teorema de convergencia real en compactos]  \label{teo:TeoremaConvergenciaRealEnCompactosDefinicionesEsenciales}

    Sea G cualquier función continua no constante definida de $\R$ en $\R$. 
    Se tiene que $\pmcg$ es uniformemente denso para compactos en $\fC$.
\end{teorema}

\begin{proof}
    Bastará probar que el conjunto $\pmcg$ satisface las hipótesis del teorema de
     Stone-Weierstrass \ref{ch:TeoremaStoneWeiertrass}.
    Lo primero será comprobar que $\pmcg$ es un álgebra, para ello veamos que:         
    \begin{enumerate}
        \item La función constante uno pertenece al conjunto. 
        Como $G$ no es constante existirá un valor de la imagen distinto de $0$, supongamos que $G(a)= b \neq 0$ para $a,b \in \R.$
        Consideremos la función afín $A(x) = 0 \cdot x + a$, está claro que $\frac{1}{b}G(A(x))$ es la función constantemente uno. 
        \item El conjunto $\pmcg$ es cerrado para sumas y producto por escalares reales. 
        En efecto, si $f,g$ pertenecen a  $\pmcg$, serán de la forma
         $f = \sum_{j = 1} ^q  \beta_{fj} \prod_{k=1}^{l_{fj}}  G(A_{fjk}(x))$ y 
        $g = \sum_{j = 1} ^p  \beta_{gj} \prod_{k=1}^{l_{gj}}G(A_{gjk}(x))$  por lo que
        \begin{equation}
            \begin{split}
                \gamma f+ \sigma g =& \gamma \sum_{j = 1} ^q  \beta_{fj} \prod_{k=1}^{l_{fj}}  G(A_{fjk}(x)) + 
                \sigma \sum_{j = 1} ^p  \beta_{gj} \ \prod_{k=1}^{l_{gj}}G(A_{gjk}(x)) \\
                & = \sum_{j = 1} ^q  (\gamma \beta_{fj})  \prod_{k=1}^{l_{fj}}  G(A_{fjk}(x)) + 
                \sum_{j = 1} ^p  (\sigma \beta_{gj}) \ \prod_{k=1}^{l_{gj}}G(A_{gjk}(x)).
            \end{split}
        \end{equation}
        
        Basta renumerar una de las sumatorias para ver $\gamma f+ \sigma g$ como una combinación 
        lineal de productos finitos de perceptrones y por tanto $\gamma f+ \sigma g \in \pmcg.$
        
        \item Cerrado para producto. Para $f,g \in \pmcg$, se tiene que $fg$ pertenece a $\pmcg$. 
        Renombrando los índices de la sumatoria con $\Lambda = i\{1..l_i\} \cup j\{1..l_j\}$ basta ver que 
        \begin{equation}
            \begin{split}
                fg &= \left(\sum_{i \in I_f} \beta_{j}  \prod_{k=1}^{l_{i}}  G(A_{ik}(x))\right)
                    \left(\sum_{j \in I_g}   \beta_{j}  \prod_{k=1}^{l_{j}} G(A_{jk}(x)) \right) \\
                    & = \sum_{i \in I_f} \left(  \beta_{j}  \prod_{k=1}^{l_{i}}  G(A_{ik}(x))
                        \left( \sum_{j \in I_g}  \beta_{j} \prod_{k=1}^{l_{j}} G(A_{jk}(x))  \right)  
                     \right) \\
                    & =  \sum_{(i,j) \in I_f \times I_g} (\beta{i}\beta{j}) \prod_{k \in \Lambda} G(A_{k}(x))
            \end{split}
        \end{equation}
        luego $fg \in \pmcg$. 
    \end{enumerate}

    Veamos que $\pmcg$ separa puntos cada compacto $K \subset \R^r$. 

    Por ser $G$ no constante existirán $a,b \in \R$ distintos cumpliendo que $G(a) \neq G(b)$. Fijadas $x,y \in K$ tomamos entonces cualquiera de las 
    funciones afines que cumplen que $A(x) = a$ y $A(y)=b$ 
    \footnote{Sabemos que al menos una habrá, ya que podemos plantear la función afín
    como un sistema de ecuaciones lineales de $r+1$ incógnita y 2 soluciones}, 
    por lo que $G(A(x)) \neq G(A(y))$ y tenemos como buscábamos que $\pmcg$ separa los puntos de $K$. 

    Veamos finalmente que para todo punto de $K$ existe una función de $\pmcg$  en el que la imagen no es nula.  

    Por ser $G$ no constante volvemos a tomar un $a \in \R$ tal que $G(a) \neq 0$ , consideramos ahora la aplicación lineal
    $A(x) = 0 \cdot x + a$ por lo que para todo $x \in K$, $G(A(x)) = G(a) \neq 0$. 

    Como hemos comprobado se verifican todas las hipótesis del teorema de Stone-Weierstrass, con lo que concluimos, como queríamos probar que $\pmcg |_K$ es denso en $C(K). $ 
\end{proof}

\subsection{Observaciones y reflexiones sobre el teorema de convergencia real en compactos}

Con esto lo que acabamos de probar que \textit{feedforward neural networks} con tan solo una capa oculta  son capaces de aproximar cualquier 
función continua en un compacto.  Cabe destacar que a la función $G$, la función de activación,
 solo se le ha pedido como 
hipótesis ser una función continua.     

Además, solo se está demostrando para el caso de una capa oculta, como veremos a continuación  de 
manera intuitiva se explica que sea extrapolable también a redes neuronales 
con varias capas ocultas, sin embargo; esto pone de manifiesto, si se quieren formular nuevos teoremas en el campo de las redes neuronales
multicapas a la necesidad de una definición más abstracta de las mismas. 


Se aportan las siguientes generalizaciones del método. 
%%% Corolarios propios 

Notemos que la función de activación $G$ es única en toda la estructura,
sin embargo es habitual la combinación de éstas en una misma red neuronal (
\cite{DBLP:journals/corr/abs-1811-03378}, 
 \cite{8258768}, 
 \cite{DBLP:journals/corr/SzegedyVISW15}
). 

\begin{corolario}[Pueden combinarse distintas funciones de activación en una misma red neuronal]

    Una misma red neuronal puede estar constituida por una familia de funciones continuas no constantes $\Gamma$, 
    bastará con generalizar $\pmcg$ a $\sum \prod ^r (\Gamma)$ donde 
    \begin{equation}
        \begin{split}
            \sum \prod^r (\Gamma) = \{ 
                &f: \R^r \longrightarrow \R /
                f(x) = \sum_{j = 1} ^q  \beta_j \prod_{k=1}^{l_j}
                G(A_{jk}(x)), \\
                &x  \in \R^r, \beta_j \in \R, A_{jk}\in A^r, l_j,q \in \N, G \in \Gamma
                )
                \}
        \end{split}
    \end{equation}
    Es decir, combinaciones lineales de perceptrones cuyas funciones 
    de activación pueden diferir unas de otras. 
\end{corolario}

\begin{proof}
    La demostración es idéntica a la dada en el Teorema de convergencia 
    real en compactos \ref{teo:TeoremaConvergenciaRealEnCompactosDefinicionesEsenciales}.
\end{proof}

Notemos que este resultado no da pista alguna de las ventajas de una función frente a otra,
 ni cómo afecta a la \textit{velocidad de convergencia}. 

\begin{corolario}[Extensión a múltiples capas ocultas]

    Sea $\Gamma$ cualquier familia de funciones continuas definidas de $\R$ en $\R$. 
    Se tiene que $\sum \prod ^r (\Gamma)$ es uniformemente denso por compactos en $\fC$  
    Es decir, las redes neuronales con varias capas son densas en el  espacio de la funciones continuas de una variable en un compacto. 
\end{corolario}

    Como con una capa ya se nos asegura la convergencia bastará con asegurar que exista 
    en el espacio de las redes neuronales profundas capas que transmitan la información sin cambiarla. 

    Una vez concretada la estructura de la red neuronal,  su estructura algebraica podría permitir esa transmisión. 


Recordemos que de manera general se ha definido $A$ como una función afín 
$A(x) = w \cdot x + b$ donde $x$ y $w$ son vectores de $\R^r$  y $b \in \R$ es un escalar.  ¿Pero que ocurriría si trabajáramos con transformaciones más generales?  
Por ejemplo $B((x_1, ..., x_r)) = \sum_{i= 0} ^N \sum_{j= 0} ^r \alpha_{ij} x_j^i$  con $N$ natural positivo. 

\begin{corolario}[Generalización de A]  
    Se puede extender $A^r$ a conjuntos más generales como el de los polinomios de $r$ variables de grado $N$, $\mathbb{P}$.  
\end{corolario}
\begin{proof}
    Simplemente hay que reparar que $A^r$ está contenido en el espacio $\mathbb{P}$. 
    Es más observando la demostración bastará con utilizar cualquier conjunto que contenga a $A^r$. 
\end{proof}

La utilidad de este corolario a nivel práctico es cuestionable, ya que aumentaría considerablemente el número de 
parámetros que ajustar de la red neuronal ocasionando: (1) la necesidad de mayor número de datos que aprender, 
(2) mayor costo computacional, (3) probablemente peores resultados a igual número de iteraciones en comparativa 
con otros modelos de menor número de neuronas (ya que el espacio de búsqueda ha aumentado).

Podría tener el siguiente interés:
el teorema nos dice que podemos aproximar cualquier función continua de variable real, sin embargo, desconocemos el 
número de neuronas, por capa. Supongamos una situación en la que el número de neuronas esté restringido, en tal caso,
generalizar $A^r$ sí que podría tener un papel importante en cuanto a mejoras. 


%% Definiciones de equivalencia de funciones 
\begin{definicion}[Equivalencia entre funciones]
    Sea $\mu$ una medida de probabilidad en $(\R^r, B^r)$.  Dos funciones 
    $f$ y $g$ pertenecientes a $\fM$, diremos que son $\mu -$equivalentes 
    si $\mu\{ x \in \R^r : f(x)=g(x) \} = 1.$
\end{definicion}

Lo que se está diciendo es que serán iguales casi por doquier.   

% Definición distancia  
\begin{definicion} [Introducción de una distancia basada en una probabilidad]
    Dada una medida de probabilidad $\mu$ en $(\R^r, B^r)$, se define 
    la métrica $\rho_{\mu}$ definida como 
    \begin{equation}
        \begin{split}
            & \rho_{\mu} : \fM \times \fM \longrightarrow \R^+ \\
            & \rho_{\mu}(f,g) = \inf \{ \epsilon > 0: \mu \{ x : |f(x) - g(x)| > \epsilon \} < \epsilon \}.
        \end{split}
    \end{equation}
\end{definicion}  

Con esta definición lo que se está buscando es una forma de decir cuánto 
distan las funciones $f,g$ entre ellas.  

%% Lema 2.1
\begin{lema}[Caracterización de la convergencia de una sucesión]\label{lema:caracterizacionConvergenciaSucesiones2_1}
    Son equivalentes las siguientes afirmaciones: 
    \begin{enumerate}
        \item $\rho_{\mu}(f_n, f) \longrightarrow 0$.
        \item Para cualquier  $\epsilon > 0$ se tiene que $\mu \{  x : |f_n(x) - f(x)| > \epsilon \} \longrightarrow 0$.
        \item $\int \min \{ |f_n(x) - f(x)|, 1\} d\mu(x) \longrightarrow 0.$
    \end{enumerate}
\end{lema}

\begin{proof}
    % 1 -> 2
    Comenzaremos probando (1) $\Rightarrow$ (2). 

    Si $\rho_{\mu}(f_n, f) \longrightarrow 0$
    Fijamos $\epsilon_0 > 0$, tenemos por definición que 
    para cualquier $0 < \delta < \epsilon_0$ existirá $n_0 \in \N$ tal que 
    $\rho_{\mu}(f_n, f) < \delta$ para cada $n$ un natural mayor que $n_0$. Es decir,  
    

    $$\inf \{ \epsilon > 0: \mu \{ x : |f_n(x) - f(x)| > \epsilon \} < \epsilon \} < \delta \quad \forall n \geq n_0$$

    entonces 

    \begin{equation}
        \mu \{ x : |f_n(x) - f(x)| > \epsilon_0 \}
        \leq
        \mu \{ x : |f_n(x) - f(x)| > \delta\}
        < \delta 
        \quad 
        \forall n \geq n_0
    \end{equation}

    lo que significa que 

    \begin{equation}
        \mu \{ x : |f_n(x) - f(x)| > \epsilon_0 \}
        \longrightarrow
        0  
    \end{equation}
    probando con ello la implicación buscada.

    % 2 -> 1
    Veamos ahora que (2) $\Rightarrow$ (1). 
    Fijamos $\epsilon_0 > 0$ y bajo la hipótesis segunda se tiene que 

    \begin{equation}
        \mu \{ x : |f_n(x) - f(x)| > \epsilon_0 \}
        \longrightarrow
        0,  
    \end{equation}
    es decir, que para cualquier real $\delta$ cumpliendo que $0 < \delta < \epsilon_0$ 
    existe un natural $n_0$ a partir del cual todo natural $n$ mayor o igual satisface que 
    
    \begin{equation}
        \mu \{ x : |f_n(x) - f(x)| > \epsilon_0 \}
        \leq
        \mu \{ x : |f_n(x) - f(x)| > \delta\}
        < \delta 
        \quad 
        \forall n \geq n_0
    \end{equation}

    lo que significa que 
    
    \begin{equation}
        \inf \{ \epsilon > 0:
         \mu \{ 
             x : |f_n(x) - f(x)| > \epsilon \} < \epsilon 
             \} 
        < \delta 
        \quad 
        \forall n \geq n_0
    \end{equation}

    que por definición de la distancia equivale a que 

    \begin{equation}
        \rho_{\mu}(f_n, f) < \delta \quad \forall n \geq n_0
    \end{equation}

    probando con ello 

    \begin{equation}
        \rho_{\mu}(f_n, f) \longrightarrow 0. 
    \end{equation}

    % 2 -> 3
    Probaremos ahora que (2) $\Longrightarrow$ (3).   

    Por (2) se tiene que sea cual sea el $\epsilon$ cumpliendo que 
    $0 < \epsilon \leq 2$ 
    existirá un natural $n_0$ a partir del cual, cualquier otro natural $n$ 
    satisface que 
    \begin{equation} 
        \mu \{  
            x : |f_n(x) - f(x)| > \frac{\epsilon}{2}  
            \}  
        < 
        \frac{\epsilon}{2},  
    \end{equation}

    Gracias a esta desigualdad, para cualquier $n > n_0$ podemos acotar la siguiente integral: 

    \begin{equation}
        \int \min \{ |f_n(x) - f(x)|, 1\} d\mu(x) 
        \leq
        \frac{\epsilon}{2} (1-\frac{\epsilon}{2}) + 1\frac{\epsilon}{2} 
         = \epsilon - \frac{\epsilon^2}{4} <  \epsilon.  
    \end{equation}
    probando con ello la implicación (2) $\Longrightarrow$ (3).

    % 3 -> 1
    Finalmente comprobaremos la implicación (3) $\Longrightarrow$ (1).

    Para cada $n\in \N$ llamamos $g_n = \min\{|f_n - f|, 1|\}$.
    Por (2), dado $0 < \epsilon < 1$, existe un $n_0 \in \N$
    de modo que si $n \geq n_0$ se cumple que 
    \begin{equation}\label{eq:definiciones_Básicas_Integral_GN_menor_Epsilon_Cuadrado}
        \int g_n d\mu < \epsilon^2
    \end{equation}
    Como $\epsilon < 1$ tenemos que 

    \begin{equation}
        \{ x; g_n(x) > \epsilon \}
         = 
         \{ x; |f_n - f| > \epsilon \}
    \end{equation}

    luego 

    \begin{equation}
        \mu\{ x; |f_n - f(x)| > \epsilon \}
        = 
        \mu\{ x; g_n(x) > \epsilon \}
        \leq
        \frac{1}{\epsilon} 
        \int_{g_n(x) > \epsilon} g_n d\mu 
        < \epsilon 
        \quad
        \forall n \geq n_0
    \end{equation}

    donde se ha usado la desigualdad de Chebyshev para $g_n$ y la desigualdad 
    (\refeq{eq:definiciones_Básicas_Integral_GN_menor_Epsilon_Cuadrado}). 

Probando con esto lo buscado que  para cualquier  $\epsilon > 0$ se tiene que 
$$\mu \{  x : |f_n(x) - f(x)| > \epsilon \} \longrightarrow 0.$$
\end{proof}


%% Lema 2.2
\begin{lema} \label{lema:2_2_convergencia_uniforme_en_compactos}  
    Si $\{f_n\}$ es una sucesión de funciones en $\fM$ que converge
    uniformemente en un compacto a $f$ entonces $\rho_{\mu}(f_n, f) \longrightarrow 0$. 
\end{lema}  
\begin{proof} Para cada $n\in \N$ llamamos $g_n = \min\{|f_n - f|, 1|\}$.
    Tengamos presente que por el  lema \ref{lema:caracterizacionConvergenciaSucesiones2_1} 
    deberemos probar que para cualquier $\epsilon > 0$, 
    existe un $n_0$ natural, tal que para cualquier otro natural $n$ mayor o igual que $n_0$ se tiene que 

    \begin{equation}
        \int \min \{ |f_n(x) - f(x)|, 1\} d\mu(x) 
        < 
        \frac{\epsilon}{2}.
    \end{equation}  

    Sea $\mu(\R^r) = M \in \R^+$  y 
    sin pérdida de generalidad puede suponerse $M = 1$
     \footnote{De otra forma bastaría con definir 
    en los pasos siguientes $\mu(K) > M - \frac{\epsilon}{2}$ y acotar con $\frac{\epsilon}{2M}$ 
    en vez de $\frac{\epsilon}{2}$.}. 
    Ya que $\R^r$ es un espacio métrico localmente compacto
    (pag 228 teorema 52.G \cite{nla.cat-vn1819421}),
    se tiene que existirá un subconjunto $K$ compacto de $\R^r$ con medida $\mu(K) > 1 - \frac{\epsilon}{2}.$
    Para el cual, por su compacidad, existirá un  $n_0$ natural 
    $\sup_{x \in K} |f_n(x) - f(x)| < \frac{\epsilon}{2}$   
    para cada natural $n$ con $n\geq n_0.$  
    De modo que para cualquier $x \in K$, 
     $n$ con $n\geq n_0$   se cumple que 
     \begin{equation}
        |f_n(x) - f(x)| 
        = 
        \min \{ |f_n(x) - f(x)|, 1\} 
        = 
        g_n.
     \end{equation}

    Por lo que  
    \begin{equation} \label{eq:lema3_2_integral_en_compacto_K}
        \int_K g_n d\mu 
        \leq
         \mu(K) \sup_{x \in K} |f_n(x) - f(x)| 
        \leq 
        \frac{\epsilon}{2} .
    \end{equation}

    Acotando el primer sumando por la medida 
    del complemento de la región integrada y en virtud de 
    (\refeq{eq:lema3_2_integral_en_compacto_K})

    \begin{equation}
        \begin{split}
            \int_{\R^r \setminus K} \min \{ |f_n(x) - f(x)|, 1\} d\mu(x) 
            +
            \int_{K} \min \{ |f_n(x) - f(x)|, 1\} d\mu(x)  \\ \leq
            \mu(\R^r \setminus K) +  \frac{\epsilon}{2}
            \leq
            \frac{\epsilon}{2} +  \frac{\epsilon}{2}
            = 
            \epsilon
        \end{split}
    \end{equation}

    para cualquier $n \geq n_0$. 
\end{proof}

% Lema A.1 
\begin{lema}\label{lema:A_1_C_es_denso_en_M}
    Para cualquier medida finita $\mu$ se tiene que $\fC$ es denso en 
    $\fM$ para la distancia $\rho_\mu$.
\end{lema}
\begin{proof}
    Dada cualquier $f \in \fM$ y un $\epsilon > 0$ arbitrario, 
    tenemos que encontrar una función $g$ que cumpla que 
    $\rho_{\mu}(f, g) < \epsilon$. 

    Tomando un $M > 1$ lo suficientemente grande, tenemos que 
    
    \begin{equation}
        \int \min \{ |f(x)\ 1_{|f(x)| < M} - f(x)|, 1\} d\mu(x)
        < \frac{\epsilon}{2}. 
    \end{equation}

    Sabemos además que podemos aproximar $f 1_{|f| < M}$ por $g$, una función continua que es límite de una sucesión de
    funciones simples ( pag 241-242,  teoremas 55C y 55D \cite{nla.cat-vn1819421}), 
    la cual satisface 
    \begin{equation}\label{eq:lema3_3_integral}
        \int \min \{ |f(x) 1_{|f(x)| < M} - g(x)|, 1\} d\mu(x) 
        < \frac{\epsilon}{2}. 
    \end{equation}
    Tomamos $M$ lo suficientemente grande, de tal forma que 
    \begin{equation} \label{eq:lema3_3_medida_conjunto}
        \mu(\{ x: |f(x)| \geq M\}) < \frac{\epsilon}{2}
    \end{equation}
    y denotamos por $\Lambda$ al conjunto $\{ x: |f(x)| < M\}.$
    
    Concluyendo por \refeq{eq:lema3_3_integral} y 
    \refeq{eq:lema3_3_medida_conjunto}
     \begin{equation}
        \begin{split}
            \int \min \{ |f  - g|, 1\} d\mu 
            = 
            \int_\Lambda \min \{ |f1_{|f(x)| < M}  - g|, 1\} d\mu
            + 
            \int_{\R^r \setminus \Lambda} \min \{ |f  - g|, 1\} d\mu 
            \\
            <
            \frac{\epsilon}{2} 
            + 
            \mu(\{ x: |f(x)| \geq M\}) 
            <
            \frac{\epsilon}{2} 
            + 
            \frac{\epsilon}{2} 
            < \epsilon. 
    \end{split}
    \end{equation}
\end{proof}









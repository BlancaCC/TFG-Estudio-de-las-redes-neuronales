% !TeX root = ../../tfg.tex
% !TeX encoding = utf8
%
%*******************************************************
% Polinomios de Bernstein
%*******************************************************

\section{Polinomios de Bernstein}\label{ch:Bernstein}  

%% Resumen capítulo 
En esta sección introduciremos los polinomios de Bernstein,   
que vistos como una serie nos asegurarán una convergencia uniformemente a cualquier 
función continua en un compacto y serán esenciales para nuestra prueba del teorema de Stone-Weierstrass, 
las pruebas se basan en el manual \cite{the-elements-of-real-analysis}.  

% Nota margen sobre Idea intuitiva convergencia uniforme
\marginpar{\maginLetterSize
\iconoAclaraciones \textcolor{dark_green}{     
\textbf{¿Qué es la convergencia uniforme?}}.
    Cuando se hable de \textbf{convergencia} nos referiremos 
    a encontrar un elemento que \textbf{aproxime todo lo bien que queramos};
    es decir, fijado cualquier error podemos encontrar un elemento 
    que lo aproxime cometiendo un error menor. 
    
    Tengamos presente que en el contexto de funciones
    el error se puede medir evaluando la imagen de cada punto del dominio; entonces  
    con \textbf{uniforme} lo que se dice es que independientemente del error que se busque, 
    se puede encontrar una función que en \textbf{para todos los puntos del dominio
     aproxima al objetivo
    por debajo de ese error}.  
}

Comenzaremos recordando el Teorema del Binomio de Newton. 

%% Teorema Binomio de Newton
\begin{teorema}[Binomio de Newton]
    Cualquier potencia de un binomio $x+y$ con $x,y \in \R$,  puede ser expandido en una suma de la forma
    \[
        (x+y)^n = \sum_{k=0}^n \binom{n}{k} x^{k}y^{n-k}. 
    \]
\end{teorema} 

% Nota margen sobre Idea intuitiva sobre porqué se está procediendo de esta manera 
\marginpar{\maginLetterSize
\iconoAclaraciones \textcolor{dark_green}{     
\textbf{¿Con qué idea se está avanzando?}}.

    Nuestro objetivo es ser capaces de aproximar una función 
    a partir de una muestra de sus preimágenes e imágenes. 
    Comenzaremos probando primero que dada una función continua, podemos
    aproximarla a partir de una muestra concreta. 
    Por eso se está tratando de reescribir la función en 
    términos de cierto conjunto de sus imágenes. 
}
%%% Idea intuitiva y desigualdad 
Tomando ahora para esta igualdad $x \in \R, y= 1-x$ se tiene que 

\begin{equation}\label{eq:uno_igual_binomio}
    1 = (x+ (1-x))^n = \sum_{k=0}^n \binom{n}{k} x^{k} (1-x)^{n-k}.
\end{equation}

Dada cualquier función continua $f: I \longrightarrow \R$ podríamos multiplicar la ecuación 
\eqref{eq:uno_igual_binomio} por $f(x)$ resultando 

\begin{equation}\label{eq:f_igual_binomio}
    f(x) = \sum_{k=0}^n f(x) \binom{n}{k} x^{k} (1-x)^{n-k}.
\end{equation} 

Y tomando como dominio $I=[0,1]$ de $f: I \longrightarrow \R$,
 nos encontramos
frente a una ecuación muy sugerente para introducir $B_n(x)$, el \textit{Polinomio n-ésimo  de Bernstein }. 
El cual pretende  aproximar la función $f$ a través de los puntos $\frac{k}{n}$ con $n \in \N$ fijo
y $k \in \{0,...,n \}.$

\begin{definicion}[Polinomios de Bernstein] \label{def:Bernstein}
    Dada cierta función $f: [0,1] \rightarrow \R$, se define el n-ésimo polinomio de Bernstein para $f$ como 

    $$B_n(x) = B_n(x;f)=\sum_ {k=0}^{n} \left( f \left( \frac{k}{n} \right) \binom{n}{k} x^k (1-x)^{n-k} \right).$$

\end{definicion}

Faltaría por ver si efectivamente nuestro polinomio construido
 \textit{aproxima lo suficientemente bien} a la $f$ originaria. 
Basándonos en la igualdad \eqref{eq:f_igual_binomio} y 
la diferencia entre $f(x)$ y $B_n(x)$ se concluye que

%\begin{equation}
%    f(x)-B_n(x) = \sum_{k=0}^n \left(f(x) - f \left( \frac{k}{n} \right)\right)
%    \binom{n}{k} x^{k} (1-x)^{n-k}
%\end{equation} 

%Tomando valor absoluto resulta 
\begin{equation} \label{eq:eq:Bernstein_diferencia}
    |f(x)-B_n(x)| \leq \sum_{k=0}^n \left|f(x) - f \left( \frac{k}{n} \right)\right|
    \binom{n}{k} x^{k} (1-x)^{n-k}
\end{equation} 

Ante tal igualdad la intuición ya nos
hace pensar que sea convergente al aumentar el tamaño de la partición.
 En efecto, en nuestro sucesivo teorema, nos cercioraremos que \ref{def:Bernstein}
 es uniformemente convergente a $f$ en un compacto. 

Para la demostración será necesario el siguiente resultado técnico: 
% Lema ecuación que se usa para una de las demostraciones del lema de Bernstein
\begin{lema}
    Se desea probar la siguiente igualdad 

    \begin{equation} \label{eq:binomio_segunda_suma}
        \frac{1}{n} x (1-x)= \sum_{k=0}^{n}  \left( x-\frac{k}{n} \right)^2  \binom{n}{k} x^{k} (1-x)^{n-k}.
     \end{equation}
\end{lema}
% Nota margen sobre el lema
\marginpar{\maginLetterSize
    \iconoAclaraciones \textcolor{dark_green}{     
        \textbf{
            Resultado técnico. 
        }
    }
    La demostración del lema \ref{eq:binomio_segunda_suma}
    se ha incluido para que todo resultado esté fundamentado, solo se utiliza para establecer una cota necesaria en una desigualdad del teorema \ref{teo:aproximacion_bernstein}.  
     
}
\begin{proof}

    %%% propiedades coeficientes binomial 

    Tengamos ahora presente las siguientes igualdades 
    \begin{equation} \label{eq:binomio_menos_uno}
        \binom{n-1}{k-1} = \frac{(n-1)!}{(k-1)! (n-1-(k-1))!} = \frac{k}{n} \binom{n}{k}
    \end{equation}
    \begin{equation} \label{eq:binomio_menos_dos}
        \binom{n-2}{k-2} = \frac{(n-2)!}{(k-2)! (n-2-(k-2))!} = \frac{k(k-1)}{n(n-1)} \binom{n}{k}
    \end{equation}

    Partiendo de la igualdad \eqref{eq:uno_igual_binomio}:
    \begin{equation}
        1 = (x+ (1-x))^n = \sum_{k=0}^n \binom{n}{k} x^{k} (1-x)^{n-k}.
    \end{equation}

    Reemplazamos la $n$ por $n-1$ y la $k$ por $j$ y tenemos 
    \begin{equation}
        1 = \sum_{j=0}^{n-1} \binom{n-1}{j} x^{j} (1-x)^{(n-1)-j}.
    \end{equation}
    Multiplicamos por $x$ y aplicamos la igualdad \eqref{eq:binomio_menos_uno} resultando 

    \begin{equation}
        x = \sum_{j=0}^{n-1} \frac{j+1}{n} \binom{n}{j+1} x^{j+1} (1-x)^{n-(j+1)}.
    \end{equation}

    Renombramos $k= j+1$, por lo que resulta
    \begin{equation}
        x = \sum_{k=1}^{n} \frac{k}{n} \binom{n}{k} x^{k} (1-x)^{n-k}.
    \end{equation}

    Como el término con $k=0$ es nulo podemos añadirlo a la sumatoria
    
    \begin{equation} \label{eq:desarrollo_binomio_uno}
        x = \sum_{k=0}^{n} \frac{k}{n} \binom{n}{k} x^{k} (1-x)^{n-k}.
    \end{equation}
    %------------------ caso 2, no te confundas --------------------
    Haremos ahora un razonamiento similar sustituyendo $n$ por $n-2$.
    Partiendo de \eqref{eq:uno_igual_binomio} se tiene que 
    \begin{equation}
        1 = (x+ (1-x))^n = \sum_{k=0}^n \binom{n}{k} x^{k} (1-x)^{n-k}.
    \end{equation}
    Reemplazamos la $n$ por $n-2$ y la $k$ por $j$ y tenemos 
    \begin{equation}
        1 = \sum_{j=0}^{n-2} \binom{n-2}{j} x^{j} (1-x)^{(n-2)-j}.
    \end{equation}
    Multiplicamos por $x^2$ y aplicamos la igualdad \eqref{eq:binomio_menos_dos} resultando 
    \begin{equation}
        x^2 = \sum_{j=0}^{n-2} \frac{(j+2)(j+1)}{n(n-1)} \binom{n}{j+2} x^{j+2} (1-x)^{n-(j+2)}.
    \end{equation}
    Renombramos $k= j+2$, por lo que resulta
    \begin{equation}
        x^2 = \sum_{k=2}^{n} \frac{k(k-1)}{n(n-1)} \binom{n}{k} x^{k} (1-x)^{n-k}.
    \end{equation}
    Como con los términos $k=0$ y $k=1$ se anula, podemos añadir dichos índices sin modificar la suma 
    \begin{equation}
        x^2 = \sum_{k=0}^{n} \frac{k(k-1)}{n(n-1)} \binom{n}{k} x^{k} (1-x)^{n-k}.
    \end{equation}
    Podemos reescribir la ecuación resultando \label{eq:desarrollo_binomio_dos}: 
    \begin{equation}
      (n^2 - n)  x^2 = \sum_{k=0}^{n} (k^2 - k) \binom{n}{k} x^{k} (1-x)^{n-k}.
    \end{equation}
    
    
%--------------- fin de las igualdades del binomio de Newton 

Sumando las dos expresiones que hemos obtenido
 \eqref{eq:desarrollo_binomio_uno} y \eqref{eq:desarrollo_binomio_dos}
 resultando 
 \begin{equation} 
    (n^2 - n)  x^2 + nx= \sum_{k=0}^{n} ((k^2 - k)+k) \binom{n}{k} x^{k} (1-x)^{n-k}.
  \end{equation}
  Dividimos todo entre $n$. 
  \begin{equation} \label{eq:binomio_tras_suma}
    (1 - \frac{1}{n})  x^2 + \frac{1}{n}x= \sum_{k=0}^{n} \left( \frac{k}{n} \right)^2 \binom{n}{k} x^{k} (1-x)^{n-k}.
  \end{equation}
  A continuación sumamos a la igualdad \eqref{eq:binomio_tras_suma} la ecuación \eqref{eq:uno_igual_binomio} multiplicada por $x^2$ y la ecuación \eqref{eq:desarrollo_binomio_uno}
  multiplicada por $-2x$ resultando: 
  \begin{equation} 
    \left(1 - \frac{1}{n} + 1 -2 \right)  x^2 + \frac{1}{n}x= \sum_{k=0}^{n} \left( \left( \frac{k}{n} \right)^2 + x^2 -2x \right) \binom{n}{k} x^{k} (1-x)^{n-k}.
  \end{equation}
  Sacando factor común en el miembro de la izquierda y escribiendo como un cuadrado el factor de la derecha,  resulta la igualdad \ref{eq:binomio_segunda_suma} que buscábamos
  \begin{equation} 
     \frac{1}{n} x (1-x)= \sum_{k=0}^{n}  \left( x-\frac{k}{n} \right)^2  \binom{n}{k} x^{k} (1-x)^{n-k}.
  \end{equation}

    
\end{proof}

\begin{teorema}[Teorema de aproximación de Bernstein]\label{teo:aproximacion_bernstein}

    Sea $f$ una función continua de un intervalo $I = [0,1]$ con imágenes en los reales. 
    La secuencia de polinomio de Bernstein
    \ref{def:Bernstein} converge uniformemente a $f$ en $I.$
    
\end{teorema}
Recordaremos antes la definición de convergencia uniforme: 

\begin{definicion}[Convergencia uniforme para funciones reales]

    Dado $E$ un conjunto y $\{f_n\}_{n \in \N}$ una sucesión de funciones de $E$
    a los reales, se dice 
    que dicha sucesión converge uniformemente si para cualquier $\varepsilon > 0$ existe un número natural $m$ tal que 
    para todo $x   \in E$ y cualquier natural $n$ que cumpla $n \geq m$ se tiene que 

    \begin{equation*}
        |f_n(x) - f(x) | < \varepsilon
    \end{equation*}
    
\end{definicion}

% Demostración de la convergencia de los polinomios de Bernstein
Comencemos pues con la demostración del teorema \ref{teo:aproximacion_bernstein}.
\begin{proof}
    
    Para cualquier $\varepsilon > 0$ queremos probar que existe un $m_\varepsilon  \in \N$ tal que para 
    todo $x \in I$ e $n \geq m_\varepsilon$  se tiene que 
    $|f(x) - B_n(x)| < \varepsilon$.
    
     Para ello por estar $f$ definida en un intervalo, 
    se tienen dos consecuencias claves: 
    \begin{enumerate}
        \item Está acotada, supongamos por $M \in \R$, esto es $|f(x)| \leq M$. \label{consecuencia:M}
        \item En virtud del teorema de Heine-Cantor $f$ es uniformemente continua, es decir, por estar $f$ definida en un compacto,  para cualquier $\varepsilon >0$ existirá un $\delta_\varepsilon$
        tal que para cualesquiera $x,y \in I$ que cumplan $|x-y| < \delta_\varepsilon$ entonces $|f(x)-f(y)| < \varepsilon$. \label{consecuencia:delta}
    \end{enumerate}
    %% Cota 2M para  cada sumando
    En virtud de la consecuencia \refeq{consecuencia:M}. 
    Dado $N \in  \N$ fijo pero arbitrario, para cualquier $k \in \{1, ..., N\}$ se tiene que
    $\frac{k}{N} \in I$ y tomando $x \in I$ podemos acotar por la desigualdad triangular

    $$\left|f(x)- f\left( \frac{k}{N} \right) \right| \leq |f(x)| + \left|f \left( \frac{k}{N}\right) \right|\leq 2M.$$  
    Por lo que 
    \begin{equation*}
        |f(x)-B_n(x)| \sum_{k=0}^n \left|f(x) - f \left( \frac{k}{n} \right)\right| \leq
     \binom{n}{k} x^{k} (1-x)^{n-k}.
    \end{equation*}

    Puesto que tenemos que $f$ acotada por $M$ y es uniformemente continua, para valores de $k$ tales que $\frac{k}{n}$ 
    esté próxima a $x$, tal término de la sumatoria será pequeño por la continuidad de $f$ en $x$. Por otro lado
    si está lo suficientemente alejado, tan solo podremos acotar tal término por $2M$.  

    Para $\varepsilon > 0$ y para $\delta_\varepsilon$ de la definición de continuidad uniforme de $f$ 
    podemos encontrar un 
    \begin{equation} \label{eq:cota-de-la-n}
        n \geq \sup \left\{ (\delta_\varepsilon)^{-4}, \frac{M^2}{ \varepsilon^2}\right\}.
    \end{equation}
    Separaremos pues nuestra sumatoria en los siguientes dos conjuntos. 
    % Conjunto de los que son menores
    \begin{equation}
        \mathcal{A}_{n x} = \left\{ k \text{ tales que } k \in \{0,..., n\} \text{ y  } \left|x - \frac{k}{n}\right| < n^{\frac{-1}{4}} \leq \delta_\varepsilon \right\},
    \end{equation}
    % Conjunto de los que son mayores
    \begin{equation}
        \mathcal{B}_{n x} = \{0,..., n\} - \mathcal{A}_{n x}. 
    \end{equation}

     Para los elementos de $\mathcal{A}_{n x}$ se obtiene la siguiente estimación: 

     \begin{equation*}
        \begin{split}
        \sum_{k \in \mathcal A } \left|f(x) - f \left( \frac{k}{n} \right)\right|
     \binom{n}{k} x^{k} (1-x)^{n-k}
     \leq 
     \sum_{k \in \mathcal{A}_{n x} } \varepsilon \binom{n}{k} x^{k} (1-x)^{n-k} 
     =  \\
      = \varepsilon \sum_{k \in \mathcal{A}_{n x} }  \binom{n}{k} x^{k} (1-x)^{n-k} 
     \leq 
     \varepsilon \sum_{k = 0} ^ n  \binom{n}{k} x^{k} (1-x)^{n-k} = 
     \varepsilon
        \end{split}
    \end{equation*}

    Para el resto de sumandos para los que $|x - \frac{k}{n}| \geq  n^{\frac{-1}{4}}$ se tiene que 
    $(x - \frac{k}{n})^2 \geq  n^{\frac{-1}{2}}$ cota. Procederemos a acotar tales términos, para ello partimos de la ecuación (\refeq{eq:eq:Bernstein_diferencia}) y  la acotación de $f$. 

    \begin{equation*}
        |f(x)-B_n(x)| \sum_{k=0}^n \left|f(x) - f \left( \frac{k}{n} \right)\right|  
     \leq \sum_{k=0}^n 2M 
    \binom{n}{k} x^{k} (1-x)^{n-k}
    \end{equation*}
    
    \begin{align*}
        & \sum_{k \in \mathcal{B}_{n x}} 2M 
    \binom{n}{k} x^{k} (1-x)^{n-k}
    \\ 
    & 
    = 2M \sum_{k \in \mathcal{B}_{n x}} 
    % multiplicamos y dividimos por diferencias 
    \frac{
        \left(x- \frac{k}{n}\right)^2
    }{
        \left(x- \frac{k}{n}\right)^2
    }
    \binom{n}{k} x^{k} (1-x)^{n-k}
    \\
    &
    % cota de la fracción introducida 
    % y más sumandos
    \leq 2M 
    \sqrt{n}
    \sum_{k \in \mathcal{B}_{n x}} 
    \left(x- \frac{k}{n}\right)^2
    \binom{n}{k} x^{k} (1-x)^{n-k}
    \\
    & % cota del lema
    \leq
    2M \sqrt{n}
    \left(
        \frac{1}{n} x(1-x)
    \right)
    \leq 
    \frac{M}{2 \sqrt{n}},
    \end{align*}

    Donde para la penúltima desigualdad se ha usado 
    la desigualdad (\refeq{eq:binomio_segunda_suma}) del lema previo y para la última que en el intervalo $I$ se satisface que 
    \begin{equation*}
        x (1-x) \leq \frac{1}{4}.
    \end{equation*}

    Además recordemos que se puede tomar un  $n$ que satisfaga (\refeq{eq:cota-de-la-n}) y entonces se concluye que para los valores de
    $\mathcal{B}_{n x}$ se puede acotar la desigualdad por $\varepsilon$. 
    
    Por tanto para un $n$ convenientemente seleccionado, se ha acotado la desigualdad para los  indices $\mathcal{A}_{n x}$ y $\mathcal{B}_{n x}$, es decir todos, por lo que concluimos que 
    \begin{equation*}
        |f(x) - B_n(x)| \leq 2 \varepsilon,
    \end{equation*}
    independiente del valor de $x$, por lo que se prueba que la secuencia de polinomios de Bernstein converge uniformemente a $f$ en $I$.
\end{proof}
% !TeX root = ../../tfg.tex
% !TeX encoding = utf8
%%%%
% OBJETIVOS SOBRE EL CAPÍTULO DE TEORÍA DE LA PROXIMACIÓN 
%%%%%%%%

\chapter{Teoría de la aproximación}
 \label{chapter:teoria-aproximar}
Teniendo siempre presente la busca de una comprensión última de las redes neuronales
con el fin de poder encontrar alguna clave con las que optimizarlas, debemos de 
notar que el teorema clave que nos asegura la convergencia se trata del Teorema de Stone Weierstrass
usado en el teorema \ref{teo:TeoremaConvergenciaRealEnCompactosDefinicionesEsenciales}.
El desarrollo de los capítulos comprendidos entre Polinomios de Bernstein \refeq{ch:Bernstein}, 
a la demostración del teorema de Stone Weierstrass \refeq{ch:TeoremaStoneWeiertrass} es múltiple.
Se pretende primeramente construir las herramientas esenciales para la demostración del 
Teorema Universal de redes neuronales por propagación hacia delante y hacia detrás; 
mas comprendiendo la naturaleza del fundamento es posible entender la bondad, alcance e imposición
de las estructuras elementales que conforman las redes neuronales, luego se hará simultáneamente
un análisis y estudio de las implicaciones de la teoría demostrada. 





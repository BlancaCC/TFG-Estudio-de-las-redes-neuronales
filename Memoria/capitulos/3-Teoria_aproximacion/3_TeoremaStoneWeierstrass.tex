% !TeX root = ../../tfg.tex
% !TeX encoding = utf8
%
%*******************************************************
% Teorema de Stone Weiertrass 
%*******************************************************

\section{Teorema de Stone-Weierstrass }\label{ch:TeoremaStoneWeiertrass}

\begin{teorema}[Teorema de Stone-Weierstrass] 

    Sea $K$ un subconjunto compacto de $\R^p$ y sea $\mathcal{A}$ una colección de 
    funciones continuas de $K$ a $\R$ cumpliendo $\mathcal{A}$ y 
    que separa puntos en $K$, es decir cumpliendo las siguientes propiedades: 

    \begin{enumerate}
        \item La función constantemente uno, definida como $e(x)=1$, para cualquier $x\in K$ pertenece a $\mathcal{A}$.
        \item $\mathcal{A}$ es cerrado para sumas y producto para escalares reales. Si $f,g$ pertenece a  $\mathcal{A}$, entonces $\alpha f + \beta g$ pertenece a $\mathcal{A}$ . 
        \item $\mathcal{A}$ es cerrado para productos. Para $f,g \in \mathcal A$, se tiene que $fg$ pertenece a $\mathcal{A}$. 
        \item Separación de $K$, es decir, si $x \neq y$ pertenecen a $K$, entonces existe una función $f$ en $\mathcal{A}$  de tal manera que $f(x) \neq f(y)$. 
    \end{enumerate}
    
    Se tiene que toda función continua de $K$ a $\R$ puede ser aproximada en $K$ por funciones de $\mathcal A$. 

\end{teorema}  

\begin{proof}
    Comenzaremos probando que para cualquier  $f \in \mathcal{A}$, 
    la función $|f|$ se puede aproximar todo lo que queramos por elementos de $\mathcal{A}$, 
    esto 
    Para cualquier $\varepsilon > 0$ y $f \in \mathcal{A}$, 
    existe $g \in \mathcal{A}$  tal que 
    \begin{equation*}
        |g(x) - |f(x)|| < \varepsilon \quad \forall x \in K.
    \end{equation*}

     Por el teorema de Heine, para $f \in \mathcal A$ está acotada por tomar imagen en un compacto, es decir $|f(x)| \leq M$ para $x \in K.$  

    Consideremos ahora la función valor absoluto, $\phi(t)=|t|$ definida en el dominio $I = [-M, M].$
    Por el teorema de aproximación de Weierstrass 
    \ref{teo:Teorema-Weierstrass}
    para cualquier $\varepsilon > 0$ 
    existirá un polinomio $p$ cumpliendo que 
    $$||t|- p(t)| < \varepsilon \quad \forall t \in I.$$

    Puesto que $t \in I$ no son más que las posibles imágenes que puede tomar $f$ en $K$ inferimos entonces que 

    $$||f(x)| - p \circ f(x)| < \varepsilon \quad \forall x \in K.$$

    Como $f \in \mathcal{A}$ y $p$ es un polinomio, es decir, 
     $p \circ f(x)$ son sumas de potencias multiplicadas por escalares de $f(x)$, luego por la hipótesis de ser cerrado a estas operaciones tenemos que la función 
    $g = p \circ f$ pertenece a $\mathcal{A}$ de donde se deduce  
    $|f|$ se puede aproximar todo lo que queramos por elementos de $\mathcal{A}$. 

    Tenemos con esto que para $f,g \in \mathcal{A}$ también es capaz de aproximar la función
     a supremo e ínfimo  gracias a que:
    \begin{align} \label{eq:cerrado-min-max}
        & \max\{f,g\} = \frac{1}{2} \{f+g+ |f+g|\} \\
        & \min \{f,g\} = \frac{1}{2} \{f+g -|f+g|\}
    \end{align}   

 
    
    Vamos a proceder ahora con nuestro objetivo 
    principal. 
    Queremos probar que para cualquier función continua $f: K \longrightarrow \R$ y cualquier $\varepsilon > 0$ existe $g \in \mathcal{A}$ tal que 

    \begin{equation}
        |f(x) - g(x)| < \varepsilon \quad \forall  x \in K. 
    \end{equation}

    Tomamos $s,t \in K$ distintos y por la hipótesis de separación de puntos existirá $u \in \mathcal{A}$ tal que $u(s) \neq u(t)$, de esta forma definimos 
    \begin{equation}
        \tilde{h}_{s t} = 
            f(s) + (f(t) - f(s))\frac{ x - s}{ t - s} 
    \end{equation}
    Notemos que  $\tilde{h}_{s t}$ satisface que 
    $\tilde{h}_{s t}(s) = f(s)$ y $\tilde{h}_{s t}(t) = f(t)$. 
    
    Tomamos $s \in K$ fijo pero arbitrario y 
    definimos el siguiente conjunto 
    \begin{align}
        U_t = 
        \{
            u \in K \quad |  \quad
            \tilde{h}_{s t}(u) < f(u) + \varepsilon
        \}.
    \end{align}
    Notemos que $U_t$ es un conjunto abierto ya que 
    $\tilde{h}_{s t} - f$ es una función continua
    y se está tomando la imagen inversa de un abierto. 
    Además se tiene que $\{t,s\} \subset U_t$ para cualquier
    $t \in K \setminus \{s\}$. 
    % Escribimos como recubrimiento finito
    Por lo que podemos escribir 
    \begin{equation*}
        K = \bigcup_{t \in K \setminus \{s\}} U_t
    \end{equation*}
    y por ser $K$ compacto admitirá un recubrimiento finito dado por 
    $I_u = \{t_1, \ldots, t_n\} \subset K$, es decir 
    \begin{equation}\label{subrecubrimiento_t}
        K = \bigcup_{t \in I_u} U_t.
    \end{equation}
    Definimos ahora 
    \begin{equation*}
        h_s(x) = \max_{t \in I_u}\{ 
            \tilde{h}_{s t}(x) 
            \quad | \quad
            x \in U_t
        \} 
        \text{ para cada } x \in K. 
    \end{equation*}
    Así definida $h_s$ satisface que: 
    \begin{itemize}
        % Desigualdad 
        \item $\text{Para cada } x \in K$ se tiene que $h_s(x) < f(x) + \varepsilon$.
        
        Ya que por 
        (\refeq{subrecubrimiento_t}) 
        $h_s(x) =  \tilde{h}_{s t_j}(x)$ y 
        como $x \in U_{t_j}$ entonces 
        $\tilde{h}_{s t_j}(x) < f(x) + \varepsilon$.
        % continuidad 
        \item La función $h_s$ es continua.
        
        Queremos ver que para todo $\epsilon > 0$, existe un $\delta >0$ cumpliendo que 
        si $ |x-y| < \delta$ entonces $h_s(x)-h_s(y)$
        Fijamos de manera arbitraria $x \in K$, 
        por 
        (\refeq{subrecubrimiento_t}) 
        $h_s(x) =  \tilde{h}_{s t_j}(x)$ que es continua en $K$. 

        % Pertenece  a A
        \item La función $h_s$ pertenece a $\mathcal{A}$, por ser el máximo de funciones que pertenecen a $\mathcal{A}$ (\refeq{eq:cerrado-min-max}). 
    \end{itemize} 
     

    % Vamos a definir el conjunto pa sacar la otra cota. 
    Y definimos para cada $s \in K$ el conjunto 
    \begin{equation}\label{subrecubrimiento_s}
        V_s = \{
            v \in K  \quad |  \quad
            h_s(x) > f(x) - \varepsilon
            \}.
    \end{equation} 

    Y repitiendo el mismo argumento que para $U_t$ puede verse que $K$ admitirá un 
    subrecubrimiento finito por abiertos: 
    \begin{equation*}
        K = \bigcup_{s \in I_s} V_s
    \end{equation*}
    con $I_v = \{s_1, \ldots, s_m\} \subset K$. 
    % definimos ahora la función graciosa 
    Además $g$ definida como: 
    \begin{equation}
        g(x)= \min_{s \in I_v}
        \{ 
            h_{s}(x) 
            \quad | \quad
            x \in V_s
        \} 
    \end{equation}
    satisface:  
    \begin{itemize}
        \item La función $g$ es continua (mismo argumento que antes). 
        % g acotado inferiormente
        \item $\text{Para cada } x \in K$ se tiene que $g(x) >f(x) - \varepsilon$.
        
        Ya que por 
        (\refeq{subrecubrimiento_s}) 
        $g(x) =  h_{s_i}(x)$ y 
        como $x \in V_{s_i}$ entonces 
        $h_{s_i}(x) > f(x) -\varepsilon$.
        % g acotado superiormente
        \item $\text{Para cada } x \in K$ se tiene que $g(x) <f(x) +\varepsilon$.
        
        Ya que por 
        (\refeq{subrecubrimiento_s}) 
        $g(x) =  h_{s_i}(x)$ y 
        en las propiedades de $h_s$ habíamos visto que
        $h_{s}(x) < f(x) +\varepsilon$ para $s$ arbitrario.

         % Pertenece  a A
         \item La función $g$ pertenece a $\mathcal{A}$, por ser el mínimo de funciones que pertenecen a $\mathcal{A}$ (\refeq{eq:cerrado-min-max}).
    \end{itemize}
    De estas observaciones se tiene lo buscado, 
    existe $g \in \mathcal{A}$ tal que 
    \begin{equation}
        |g-f| \leq \varepsilon;
    \end{equation}
    es decir $f$ puede ser aproximada por elementos de $\mathcal{A}$. 

    \textcolor{red}{
        Hay detalles que no me terminan de convencer. 
        Como que las aproximaciones no pertenecen a A.
    }
\end{proof}

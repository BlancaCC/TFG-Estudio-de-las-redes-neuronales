% !TeX root = ../../tfg.tex
% !TeX encoding = utf8
%
%*******************************************************
% Teorema de Stone Weiertrass 
%*******************************************************

\section{Teorema de Stone-Weierstrass }\label{ch:TeoremaStoneWeiertrass}

\begin{lema}\label{ch03:lema:min-max}
    Sean tres funciones $f_1,f_2,g : K \longrightarrow \R$ tales que 
    \begin{equation}
        |f_2 - g| \leq \varepsilon.
    \end{equation}
    Entonces la diferencia entre los máximos 
    y respectivamente mínimos satisface que: 
    \begin{equation}
        |
            \max\{f_1, f_2\}
            -
            \max\{f_1, g\}
        |
        \leq \epsilon
    \end{equation}
    y 
    \begin{equation}
        |
            \min\{f_1, f_2\}
            -
            \min\{f_1, g\}
        |
        \leq \epsilon.
    \end{equation}
\end{lema}
\begin{proof}
    Sabemos que 
    \begin{align}
        & \max\{f,g\} = \frac{1}{2} (f+g+ |f-g|) \\
        & \min \{f,g\} = \frac{1}{2} (f+g -|f-g|).
    \end{align}
    Por lo que 
    \begin{align}
        |
            \max\{f_1, f_2\}
            -
            \max\{f_1, g\}
        |   
        & = 
        \left|
            \frac{1}{2} (f_1+f_2+ |f_1-f_2|)
            -
            \frac{1}{2} (f_1+g+ |f_1-g|)
        \right|
        \\
        &= 
        \frac{1}{2}
        \big|
             f_1+f_2+ |f_1-f_2|
            -
            f_1 - g - |f_1-g|
        \big|
        \\
        &= 
        \frac{1}{2}
        \big|
             f_2 - g + |f_1-f_2|
           - |f_1-g|
           \big|
        \\
        & \leq
        \frac{1}{2}
        (|f_2 - g |
         +
         |f_2 - g |
        )
        \\
        & \leq
        |f_2 - g |
        \\
        & \leq
        \varepsilon.
    \end{align} 

    Para el mínimo se razona igual: 
    \begin{align}
        |
            \min\{f_1, f_2\}
            -
            \min\{f_1, g\}
        |   
        & = 
        \left|
            \frac{1}{2} (f_1+f_2- |f_1-f_2|)
            -
            \frac{1}{2} (f_1+g - |f_1-g|)
        \right|
        \\
        &= 
        \frac{1}{2}
        \big|
             f_1+f_2- |f_1-f_2|
            -
            f_1 - g + |f_1-g|
        \big|
        \\
        &= 
        \frac{1}{2}
        \big|
             f_2 - g - |f_1-f_2|
           + |f_1-g|
        \big|
        \\
        & \leq
        \frac{1}{2}
        (
        |f_2 - g |
         +
         |f_2 - g |
        )
        \\
        & \leq
        |f_2 - g|
        \\
        & \leq
        \varepsilon.
    \end{align} 
    
\end{proof}

\begin{teorema}[Teorema de Stone-Weierstrass] 

    Sea $K$ un subconjunto compacto de $\R^p$ y sea $\mathcal{A}$ una colección de 
    funciones continuas de $K$ a $\R$ 
    que separa puntos en $K$, es decir cumpliendo las siguientes propiedades: 

    \begin{enumerate}
        \item La función constantemente uno, definida como $e(x)=1$, para cualquier $x\in K$ pertenece a $\mathcal{A}$.
        \item $\mathcal{A}$ es cerrado para sumas y productos por escalares reales. Si $f,g$ pertenece a  $\mathcal{A}$, entonces $\alpha f + \beta g$ pertenece a $\mathcal{A}$ . 
        \item $\mathcal{A}$ es cerrado para productos. Para $f,g \in \mathcal A$, se tiene que $fg$ pertenece a $\mathcal{A}$. 
        \item Separación de $K$, es decir, si $x \neq y$ pertenecen a $K$, entonces existe una función $f$ en $\mathcal{A}$  de tal manera que $f(x) \neq f(y)$. 
    \end{enumerate}
    
    Entonces se tiene que toda función continua de $K$ a $\R$ puede ser aproximada en $K$ por funciones de $\mathcal A$. 

\end{teorema}  

\begin{proof}
    \subsubsection*{Paso 1: Para $f \in \mathcal{A}$ la función valor absoluto de $f$ puede ser aproximada por elementos
     de $\mathcal{A}$.}
    Probaremos que 
    para cualquier $\varepsilon > 0$ y $f \in \mathcal{A}$, 
    existe $g \in \mathcal{A}$  tal que 
    \begin{equation*}
        \big| g(x) - |f(x)|\big| < \varepsilon \quad \forall x \in K.
    \end{equation*}

     Por el teorema de Heine $f$ es acotada, es decir existe $M \in R^+$ tal que  $|f(x)| \leq M$ para $x \in K.$  

    Consideremos ahora la función valor absoluto, $\phi(t)=|t|$ definida en el dominio $I = [-M, M].$
    Por el teorema de aproximación de Weierstrass 
    \ref{teo:Teorema-Weierstrass}
    para cualquier $\varepsilon > 0$ 
    existirá un polinomio $p$ cumpliendo que 
    \begin{equation}
        \big||t|- p(t)\big| < \varepsilon \quad \forall t \in I.
    \end{equation}

    Puesto que $t \in I$ no son más que las posibles imágenes que puede tomar $f$ en $K$ inferimos entonces que 

    \begin{equation}
        \big||f(x)| - p \circ f(x)\big | < \varepsilon \quad \forall x \in K
    \end{equation}

    Como $f \in \mathcal{A}$ y $p$ es un polinomio, 
     $p \circ f$ son sumas de potencias  de $f$ multiplicadas por escalares.
    Como $\mathcal{A}$ es cerrado para estas operaciones tenemos que la función 
    $p \circ f$ pertenece a $\mathcal{A}$ de donde se deduce  
    $|f|$ se puede aproximar todo lo que queramos por elementos de $\mathcal{A}$. 

    Tenemos con esto que para $f,g \in \mathcal{A}$ también es capaz de aproximar la función
     a máximo e mínimo gracias a que:
    \begin{align} \label{eq:cerrado-min-max}
        & \max\{f,g\} = \frac{1}{2} (f+g+ |f-g|) \\
        & \min \{f,g\} = \frac{1}{2} (f+g -|f-g|). 
    \end{align}  
    

    \subsubsection*{Paso 2: Para cada familia finita $\mathcal{F} \subset \mathcal{A}$ las funciones $\max\{\mathcal{F}\}$ y $\min\{\mathcal{F}\}$ pueden ser aproximadas por elementos de
    de $\mathcal{A}$}
    \label{ch03:weiertrass-paso-2}
     Razonando por inducción: para el caso base $n=2$ ya se ha probado. 
    Si $n >2$ entonces podemos escribir (los razonamientos son idénticos para la función mínimo):
    \begin{equation}
        \max\{f_1, f_2, \ldots, f_n\}
        =
        \max\{f_1, \max\{f_2, \ldots, f_n\}\},
    \end{equation}
    por hipótesis de inducción existe
    $g \in \mathcal{A}$ tal que 
    \begin{equation}
        |\max\{f_2, \ldots, f_n\} - g|
        < 
        \frac{\varepsilon}{2}.
    \end{equation}
    Considerando entonces 
    $\max\{f_1,g\}$, por el caso base existirá 
    un $g' \in \mathcal{A}$ tal que
    \begin{equation}\label{cota:max-g}
        |\max\{f_1,g\} - g'| 
        < \frac{\varepsilon}{2}.
    \end{equation} 
    Por desigualdad triangular, el lema \ref{ch03:lema:min-max} y la ecuación (\refeq{cota:max-g}) se tiene que 
    \begin{align}
        |\max\{f_1, f_2, \ldots, f_n\} 
        -
        g'
        | 
        &\leq
        |\max\{f_1, f_2, \ldots, f_n\} 
        -
        \max\{f_1,g\}
        |
        + 
        |
        \max\{f_1,g\}-
        g'
        | 
        \leq
        \varepsilon.
    \end{align}

    Probando con ello lo buscado.
    
    \subsubsection*{Paso 3: Cualquier función continua $f \in \mathcal{C}(K, \R)$ puede ser aproximada con elementos
    de $\mathcal{A}$}
    Vamos a proceder ahora con nuestro objetivo 
    principal. 
    Queremos probar que para cualquier función continua $f: K \longrightarrow \R$ y cualquier $\varepsilon > 0$ existe $g \in \mathcal{A}$ tal que 

    \begin{equation}
        |f(x) - g(x)| < \varepsilon \quad \forall  x \in K. 
    \end{equation}

    Tomamos $s,t \in K$ distintos y por la hipótesis de separación de puntos existirá $u \in \mathcal{A}$ tal que $u(s) \neq u(t)$, de esta forma definimos 
    \begin{equation}
        \tilde{h}_{s t}(x) = 
            f(s) + (f(t) - f(s))\frac{ u(x) - u(s)}{ u(t) - u(s)}, 
    \end{equation}
    se tiene que  $\tilde{h}_{s t} \in \mathcal{A}$
    y satisface que 
    $\tilde{h}_{s t}(s) = f(s)$ y $\tilde{h}_{s t}(t) = f(t)$. 
    
    Tomamos $s \in K$ fijo pero arbitrario y 
    definimos el siguiente conjunto 
    \begin{align}
        U_t = 
        \left\{
            u \in K \colon
            \tilde{h}_{s t}(u) < f(u) + \frac{\varepsilon}{2}
        \right\}.
    \end{align}
    Notemos que $U_t$ es un conjunto abierto ya que 
    $\tilde{h}_{s t} - f$ es una función continua
    y se está tomando la imagen inversa de un abierto. 
    Además se tiene que $\{t,s\} \subset U_t$ para cualquier
    $t \in K \setminus \{s\}$. 
    % Escribimos como recubrimiento finito
    Por lo que podemos escribir 
    \begin{equation*}
        K = \bigcup_{t \in K \setminus \{s\}} U_t
    \end{equation*}
    y por ser $K$ compacto admitirá un recubrimiento finito dado por 
    $I_u = \{t_1, \ldots, t_n\} \subset K$, es decir 
    \begin{equation}\label{subrecubrimiento_t}
        K = \bigcup_{t \in I_u} U_t.
    \end{equation}
    Definimos ahora 
    \begin{equation*}
        h_s(x) = \max_{t \in I_u}\{ 
            \tilde{h}_{s t}(x) 
            \colon
            x \in U_t
        \} 
        \text{ para cada } x \in K. 
    \end{equation*}
    Así definida $h_s$ satisface que: 
    \begin{itemize}
        % Desigualdad 
        \item $\text{Para cada } x \in K$ se tiene que $h_s(x) < f(x) + \frac{\varepsilon}{2}$.
        
        Ya que por 
        (\refeq{subrecubrimiento_t}) 
        $h_s(x) =  \tilde{h}_{s t_j}(x)$ y 
        como $x \in U_{t_j}$ entonces 
        $\tilde{h}_{s t_j}(x) < f(x) + \frac{\varepsilon}{2}$.
        % continuidad 
        \item La función $h_s$ puede ser aproximada con la precisión que se desee por elementos de $\mathcal{A}$ como ya hemos visto en el paso 2.
        Sea $h_s'\in \mathcal{A}$ tal que 
        $|h_s' - h_s| < \frac{\varepsilon}{2}$.

        % Desigualdad para aproximación 
        \item $\text{Para cada } x \in K$ se tiene que $h_s(x) < f(x) + \frac{\varepsilon}{2}$ y que $\frac{\varepsilon}{2} > |h_s' - h_s| \geq h_s' - h_s$; sumando ambas ecuaciones se tiene que 
        \begin{equation}
            h_s'(x) < f(x) + \varepsilon.
        \end{equation}
        
    \end{itemize} 
     

    % Vamos a definir el conjunto pa sacar la otra cota. 
    Y definimos para cada $s \in K$ el conjunto 
    \begin{equation}\label{subrecubrimiento_s}
        V_s = \left\{
            v \in K  \colon
            h_s'(x) > f(x) - \frac{\varepsilon}{2}
            \right\}.
    \end{equation} 

    Y repitiendo el mismo argumento que para $U_t$ puede verse que $K$ admitirá un 
    subrecubrimiento finito por abiertos: 
    \begin{equation*}
        K = \bigcup_{s \in I_s} V_s
    \end{equation*}
    con $I_v = \{s_1, \ldots, s_m\} \subset K$. 
    % definimos ahora la función graciosa 
    Además $g$ definida como: 
    \begin{equation}
        g(x)= \min_{s \in I_v}
        \{ 
            h_{s}'(x) 
            \colon
            x \in V_s
        \} 
    \end{equation}
    satisface:  
    \begin{itemize}

        \item Por el paso 2 sabemos que existe $g' \in \mathcal{A}$ tal que   $|g' - g| < \frac{\varepsilon}{2}$.
         
        % g acotado inferiormente
        \item $\text{Para cada } x \in K$ se tiene que $g(x) >f(x) - \frac{\varepsilon}{2}$.
        Ya que por 
        (\refeq{subrecubrimiento_s})   
        $g(x) =  h_{s_i}'(x)$ y 
        como $x \in V_{s_i}$ entonces 
        $h_{s_i}'(x) > f(x) - \frac{\varepsilon}{2}$.
        % g acotado superiormente
        \item $\text{Para cada } x \in K$ se tiene que $g(x) <f(x) + \frac{\varepsilon}{2}$.
        
        Ya que por 
        (\refeq{subrecubrimiento_s}) para cada $x$ existe $i$ tal que 
        $g(x) =  h_{s_i}'(x)$ y 
        en las propiedades de $h_s'$ habíamos visto que
        $h_{s}'(x) < f(x) + \frac{\varepsilon}{2}$ para $s$ arbitrario.
        \item De las últimas observaciones se deduce que $|g-f| < \frac{\varepsilon}{2}$.

    \end{itemize}
    De estas observaciones se tiene que  $g' \in \mathcal{A}$ cumple 
    \begin{equation}
        |g'-f| \leq |g-f| + |g'-g| 
        <
         \frac{\varepsilon}{2} + \frac{\varepsilon}{2} 
        = 
        \varepsilon.
    \end{equation}
    Es decir, $f$ puede ser aproximada por elementos de $\mathcal{A}$.
\end{proof}

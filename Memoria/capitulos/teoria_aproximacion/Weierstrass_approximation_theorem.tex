% !TeX root = ../../tfg.tex
% !TeX encoding = utf8
%
%*******************************************************
% Teorema de Aproximación Weierstass 
%*******************************************************

Realizando un repaso global habiendo acabado el teorema, se pueden extraer que conjunto a un ingenioso manejo de operaciones y acotaciones; la clave del resultado reside  en las consideraciones
en \refeq{consecuencia:M} \refeq{consecuencia:delta} y estas a su vez en la 
compacidad de $I$.

Por su parte, la selección del dominio de $I = [0,1]$ viene determinada ya que 
 los nodos $\{ \frac{k}{N} | k\in \{0,..., N\} |  \}$ sobre los que se construye el \textit{N-ésimo polinomio de Bernstein}  deben pertenecer a $I$.

Sin embargo, tal dificultad es fácilmente salvable con un homeomorfismo. 

Como resultado de relajar el dominio donde se define $f$, pidiéndole tan solo
compacidad nace el siguiente corolario.  

\begin{corolario}[Teorema de aproximación de Weierstass] \label{teo:TeoremaAproximaciónWeierstrass}
    Sea $f$ una función continua definida en un intervalo real y con valores reales. Se tiene que $f$ puede ser aproximada uniformemente con polinomios. 
\end{corolario}  

\begin{proof}
    Si $f$ se encuentra definida en $[a,b]$ con $a<b$ y bastará considerar la función
    \begin{equation*}
        g(t) = f( (b-a)t + a) \text{ con } t \in [0,1]
    \end{equation*}

    Esta función está definida en $[0,1]$, tiene la misma imagen que $f$ y 
    mantiene la continuidad ya que se ha construido a través del homeomorfismo 
    $H:[0,1] \longrightarrow [a,b]$, con $H(t) = (b-a)t + a$. 

    En virtud del teorema de convergencia \refeq{teo:aproximación_Bernstein}
    $g$ podrá ser aproximada uniformemente por una sucesión de polinomios de Bernstein $\{S_n\}_{n \in \N}$. Gracias a ella se puede
    construir la sucesión $\{S_n \circ H^{-1}\}_{n \in \N}$, que aproxima uniformemente a $f$. 
\end{proof}
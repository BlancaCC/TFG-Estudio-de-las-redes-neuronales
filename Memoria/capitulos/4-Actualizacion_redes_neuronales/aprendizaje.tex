\section{Aprendizaje}  

Se entiende por aprendizaje de una red neuronal como el proceso 
por el cual se determina el valor sus pesos, es decir, lo que en el ejemplo \ref{img:Ejemplo-evaluación-red-neruonal-una-capa} consistía en las matrices $A,S$ y $B$.

%%%%%%%%%%%%%%%%%%% algoritmo de gradiente descendente 

\subsection{Método de gradiente descendente y \textit{backpropagation}} \label{sec:gradiente-descendente}

De acorde a los capítulos uno y dos del libro  \cite{learning-from-data-1-2},
 una vez concretado el problema y sus elementos 
(\ref{sub:componentes_aprendizaje}) es necesario definir un método con 
el que aproximar la función ideal $f,$ para ello introduciremos el algoritmo de gradiente descendente.  

El gradiente descendente es un método iterativo de minimización de funciones diferenciables. 

% Nota sobre el algoritmo de gradiente descendente 
\reversemarginpar
\marginpar{
    \textcolor{dark_green}{    
        \textbf{
            Aclaración gradiente red neuronal
        }
    }

    Puede a priori uno confundirse con la notación,  
    pero recordemos que las redes neuronales estaban determinadas por sus parámetros (matrices), luego lo único que se está haciendo es derivar con respecto a tales parámetros.
} 
%Fin nota margen
% Idea  sobre el algoritmo de gradiente descendente 
\marginpar{
    \textcolor{dark_green}{    
        \textbf{
            Idea general del algoritmo gradiente descendente
        }
    }

    Cada iteración se obtendrá una nueva red neuronal con un error menor dentro de los datos de entrenamiento del conjunto.
} 

% Nota margen sobre diferenciabilidad
\normalmarginpar
\setlength{\marginparwidth}{\smallMarginSize}
\marginpar{
    \textcolor{red}{    
        \textbf{
            Consecuencia del requisito de diferenciabilidad 
            de $E(h)$
        }
    }
    $E(h)$ será diferenciable si y sólo si las funciones de activación lo son. 
}


%Fin nota margen
En nuestro caso particular se quiere aproximar la función ideal desconocida $f$ a partir de funciones (redes neuronales) $h \in \rrnnmc$, concretamente se fijará un número $q$ de neuronas en la capa oculta. 
Dada también una función de error diferenciable y que no presente puntos de inflexión
$E: \mathcal{H}_q (\R^d, \R^s) \longrightarrow \R,$
se toma red neuronal cualquiera $h_0 \in \mathcal{H}_q (\R^d, \R^s)$ y 
fijado $\eta \in \R^+$. 

Se define la sucesión 
\begin{equation}\label{eq:descenso-gradiente}
    h_{t+1}  = h_t - \eta \nabla E(h_t).
\end{equation}  

Donde $h_n$ es una sucesión cuyos términos convergen a un mínimo local.
\subsubsection*{Observaciones sobre el algoritmo }

\begin{itemize}
    \item El algoritmo solo encuentra óptimos locales con una dependencia crucial del valor de inicio. 
    \item La convergencia no es segura en un tiempo finito y requiere de criterios de parada. 
    \item Debe de fijarse el número de neuronas en la capa oculta a priori.
    \item Si la función es convexa el mínimo será global.
    \item El parámetro $\eta$ puede ser cualquiera y debe de ser fijado o controlado por el diseñador.  
\end{itemize}


Con el fin de reducir el coste del cálculo del gradiente, 
se utiliza el algoritmo conocido como \textit{backpropagation} que fue publicado en 
1989 en el artículo \cite{backpropagation-Hinton}. 

Denotaremos como $w$ al conjunto de parámetros que determinan el valor de una red neuronal. 

 Sea $E_{in} : \rrnnmc \longrightarrow \R^+_0$ la función para medir error habitualmente usada, la cual tomaremos como el error dentro de conjunto de entrenamiento, esto es,  si el conjunto 
de entrenamiento $\mathcal{D}$ está constituido por $N$ datos de la forma $(x_n, y_n)$ con $x_n$ el vector de entrada o atributos y $y_n$ el estado o valor deseado para cualquier $n\in \{1, \ldots, N\}.$ Para cualquier $h \in \rrnnmc$ denotando por $h_k$ a su $k-$ésima proyección, se define como métrica de error dentro de $\mathcal{D}$ como
\begin{equation}
    E_{in}(h) = \frac{1}{N} \sum_{(x,y) \in \mathcal{D}} \sum_{k=1}^s(h_k(x)- y_k)^2. 
\end{equation}

puesto que $\frac{1}{N}$ no es más que una constante que 
puede ser corregida en \refeq{eq:descenso-gradiente} con $\eta$, con el fin de ahorrar coste computacional la 
omitiremos de ahora en adelante. Es decir, podemos suponer que 
nuestra función de error a minimizar es 

\begin{equation}
    E_{in}(h) = \frac{1}{2} \sum_{(x,y) \in \mathcal{D}} \sum_{k=1}^s (h_k(x)- y_k)^2. 
\end{equation}

Antes de adentrarnos en los cálculos tengamos presente que hemos definido una red neuronal  $h \in \mathcal{H}_n (\R^d, \R^s)$ como
\begin{equation}
    h_k(x) = 
    \sum_{j=1}^n \beta_{j k}
    \sigma
    \left(  
        \alpha_{0 j} +
        \sum_{i=1}^d \alpha_{i j}x_j
    \right)
\end{equation}
para la cual hemos impuesto que la función de activación $\sigma$ sea diferenciable.

Denotaremos como $\chi_{[c]}$ a la función características
\begin{equation}
    \chi_{[c]}= \left\{ \begin{array}{lcc}
        1 &   si  &  \text{se satisface c} \\
        \\ 0 &  si  & \text{no se satisface c} 
        \end{array}
\right.
\end{equation}

Así pues a la hora de calcular el gradiente tendríamos tres tipos de derivadas parciales, las dependientes de $\beta_{i  k}$, 
las de $\alpha_{0 i}$ y las de $\alpha_{j i}$, para cada caso concreto y en virtud de la regla de la cadena, se tiene: 
\begin{itemize}
    \item Derivada parcial del error con respecto a $\beta_{u v}$ donde $u \in \{1, \ldots, n\}$ y $v \in \{1, \ldots, s\}$:
    \begin{align} \label{eq:parcial_beta}
        \frac{\partial E(h)}{\partial \beta_{u v}} 
        = &
        \frac{\partial}{\partial \beta_{u v}}
        \left[
            \frac{1}{2}
            \sum_{(x,y) \in \mathcal{D}}
            \sum_{k = 1}^s 
            \left(h_k(x) - y_k \right)^2
        \right]
        \\ % primer paso regla de la cadena
        = &
        \frac{1}{2}
        \sum_{(x,y) \in \mathcal{D}}
        \sum_{k = 1}^s 
        2 \left(h_k(x) - y_k \right)
        \frac{\partial h_k(x)}{\partial \beta_{u v}} 
        \\ 
        = & % desarrollamos h
        \sum_{(x,y) \in \mathcal{D}}
        \sum_{k = 1}^s 
        \left(h_k(x) - y_k \right)
        \frac{\partial}{\partial \beta_{u v}} 
        \left[
            \sum_{j = 1}^n 
                \beta_{j k}
                \sigma
                \left(  
                    \alpha_{0 i} +
                    \sum_{i=1}^d \alpha_{i j}x_i
                \right)
        \right] 
        \\ 
        = & % parcial de la suma es suma de parciales 
        \sum_{(x,y) \in \mathcal{D}}
        \sum_{k = 1}^s 
        \left(h_k(x) - y_k \right)
        \left(
            \sum_{j = 1}^n 
            \frac{\partial}{\partial \beta_{u v}} 
            \left[
                \beta_{j k}
                \sigma
                \left(  
                    \alpha_{0 i} +
                    \sum_{i=1}^d \alpha_{i j}x_i
                \right)
            \right]
        \right) 
        \\ 
        = & % Expresión a partir de la función caraterísticas
        \sum_{(x,y) \in \mathcal{D}}
        \sum_{k = 1}^s 
        \left(h_k(x) - y_k \right)
        \left(
            \sum_{j = 1}^n 
                \chi_{[j = u \wedge k = v]}
                \sigma
                \left(  
                    \alpha_{0 i} +
                    \sum_{j=1}^d \alpha_{i j}x_i
                \right)
        \right)
        \\ 
        = & % Expresión final quitando términos nulos 
        \sum_{(x,y) \in \mathcal{D}}
        \left(h_v(x) - y_v \right)
        \left(
            \sigma
            \left(  
                \alpha_{0 u} +
                \sum_{i=1}^d \alpha_{i u}x_i
            \right)
        \right).
    \end{align}

    \item Derivada parcial del error con respecto a $\alpha_{0 v}$ donde $v \in \{1, \ldots, n\}$:
    \begin{align} \label{eq:parcial_alpha_cero}
        \frac{\partial E(h)}{\partial \alpha_{0 v}} 
        = &
        \frac{\partial}{\partial \alpha_{0 v}}
        \left[
            \frac{1}{2}
            \sum_{(x,y) \in \mathcal{D}}
            \sum_{k = 1}^s 
            \left(h_k(x) - y\right)^2
        \right]
        \\ % primer paso regla de la cadena
        = &
        \frac{1}{2}
        \sum_{(x,y) \in \mathcal{D}}
        \sum_{k = 1}^s 
        2 \left(h_k(x) - y_k \right)
        \frac{\partial h_k(x)}{\partial \alpha_{0 v}} 
        \\ 
        = & % desarrollamos h
        \sum_{(x,y) \in \mathcal{D}}
        \sum_{k = 1}^s 
        \left(h_k(x) - y_k \right)
        \frac{\partial}{\partial \alpha_{0 v}} 
        \left[
            \sum_{j = 1}^n 
                \beta_{j k}
                \sigma
                \left(  
                    \alpha_{0 i} +
                    \sum_{i=1}^d \alpha_{i j}x_i
                \right)
        \right] 
        \\ 
        = & % parcial de la suma es suma de parciales 
        \sum_{(x,y) \in \mathcal{D}}
        \sum_{k = 1}^s 
        \left(h_k(x) - y_k \right)
        \left(
            \sum_{j = 1}^n 
            \beta_{j k}
            \frac{\partial}{\partial \alpha_{0 v}} 
            \left[
                \sigma
                \left(  
                    \alpha_{0 i} +
                    \sum_{i=1}^d \alpha_{i j}x_i
                \right)
            \right]
        \right) 
        \\ 
        = & %regla de la cadena 
        \sum_{(x,y) \in \mathcal{D}}
        \sum_{k = 1}^s 
        \left(h_k(x) - y_k \right)
        \left(
            \sum_{j = 1}^n 
            \beta_{j k}
            \sigma '
            \left(  
                \alpha_{0 j} +
                \sum_{i=1}^d \alpha_{i j}x_i
            \right)
            \frac{\partial}{\partial \alpha_{0 v}}    
            \left[
                \alpha_{0 j} +
                \sum_{i=1}^d \alpha_{i j}x_i
            \right]
        \right) 
        \\ 
        = & % función característica
        \sum_{(x,y) \in \mathcal{D}}
        \sum_{k = 1}^s 
        \left(h_k(x) - y_k \right)
        \left(
            \sum_{j = 1}^n 
            \beta_{j k}
            \sigma '
            \left(  
                \alpha_{0 j} +
                \sum_{i=1}^d \alpha_{i j}x_i
            \right)   
            \chi_{[j = v]}
        \right) 
        \\ 
        = & % Expresión final quitando términos nulos 
        \sum_{(x,y) \in \mathcal{D}}
        \sum_{k = 1}^s 
        \left(h_k(x) - y_k \right)
        \left(
            \beta_{v k}
            \sigma '
            \left(  
                \alpha_{0 v} +
                \sum_{i=1}^d \alpha_{i v}x_i
            \right)   
        \right). 
    \end{align}

    \item Derivada parcial del error con respecto a $\alpha_{u v}$ donde $u \in \{1, \ldots, d\}$ y $v \in \{1, \ldots, n\}$:
    
    \begin{align} \label{eq:parcial_alpha_i}
        \frac{\partial E(h)}{\partial \alpha_{u v}} 
        =&
        \frac{\partial}{\partial \alpha_{u v}}
        \left[
            \frac{1}{2}
            \sum_{(x,y) \in \mathcal{D}}
            \sum_{k = 1}^s 
            \left(h_k(x) - y_k \right)^2
        \right]
        \\ % primer paso regla de la cadena
        = &
        \frac{1}{2}
        \sum_{(x,y) \in \mathcal{D}}
        \sum_{k = 1}^s 
        2 \left(h_k(x) - y_k \right)
        \frac{\partial h_k(x)}{\partial \alpha_{u v}} 
        \\ 
        = & % desarrollamos h
        \sum_{(x,y) \in \mathcal{D}}
        \sum_{k = 1}^s 
        \left(h_k(x) - y_k \right)
        \frac{\partial}{\partial \alpha_{u v}} 
        \left[
            \sum_{j = 1}^n 
                \beta_{j k}
                \sigma
                \left(  
                    \alpha_{0 i} +
                    \sum_{i=1}^d \alpha_{i j}x_i
                \right)
        \right] 
        \\ 
        = & % parcial de la suma es suma de parciales 
        \sum_{(x,y) \in \mathcal{D}}
        \sum_{k = 1}^s 
        \left(h_k(x) - y_k \right)
        \left(
            \sum_{j = 1}^n 
            \beta_{j k}
            \frac{\partial}{\partial \alpha_{u v}} 
            \left[
                \sigma
                \left(  
                    \alpha_{0 i} +
                    \sum_{i=1}^d \alpha_{i j}x_i
                \right)
            \right]
        \right) 
        \\ 
        = & %regla de la cadena 
        \sum_{(x,y) \in \mathcal{D}}
        \sum_{k = 1}^s 
        \left(h_k(x) - y_k \right)
        \left(
            \sum_{j = 1}^n 
            \beta_{j k}
            \sigma '
            \left(  
                \alpha_{0 j} +
                \sum_{i=1}^d \alpha_{i j}x_i
            \right)
            \frac{\partial}{\partial \alpha_{u v}}    
            \left[
                \alpha_{0 j} +
                \sum_{i=1}^d \alpha_{i j}x_i
            \right]
        \right) 
        \\ 
        = & % función característica
        \sum_{(x,y) \in \mathcal{D}}
        \sum_{k = 1}^s 
        \left(h_k(x) - y_k \right)
        \left(
            \sum_{j = 1}^n 
            \beta_{j k}
            \sigma '
            \left(  
                \alpha_{0 j} +
                \sum_{i=1}^d \alpha_{i j}x_i
            \right)   
            \chi_{[i = u\wedge j = v]}x_i
        \right) 
        \\ 
        = & % Expresión final quitando términos nulos 
        \sum_{(x,y) \in \mathcal{D}}
        \sum_{k = 1}^s 
        \left(h_k(x) - y_k \right)
        \left(
            \beta_{v k}
            \sigma '
            \left(  
                \alpha_{0 v} +
                \sum_{i=1}^d \alpha_{i v}x_i
            \right)x_u   
        \right).
    \end{align}
\end{itemize}  

Recapitulando los resultados, el cálculo de las derivadas parciales con $u \in \{1, \ldots, d\}$, $v \in \{1, \ldots, n\}$ y $w \in \{1, \ldots, s\}$ consiste en: 

% Resumen desarrollo anterior: 
\begin{itemize}
    \item Por el desarrollo (\ref{eq:parcial_beta}) la derivada parcial del error con respecto a $\beta_{v w}$ es:
    \begin{align} 
        \frac{\partial E(h)}{\partial \beta_{v w}} 
        = & % Expresión final quitando términos nulos 
        \sum_{(x,y) \in \mathcal{D}}
        \left(h_w(x) - y_w \right)
        \left(
            \sigma
            \left(  
                \alpha_{0 v} +
                \sum_{i=1}^d \alpha_{i v}x_i
            \right)
        \right).
    \end{align}

    \item Por el desarrollo (\ref{eq:parcial_alpha_cero}) derivada parcial del error con respecto a $\alpha_{0 v}$ es :
    \begin{align} 
        \frac{\partial E(h)}{\partial \alpha_{0 v}} 
        =  % Expresión final quitando términos nulos 
        \sum_{(x,y) \in \mathcal{D}}
        \sum_{k = 1}^s 
        \left(h_k(x) - y_k \right)
        \left(
            \beta_{v k}
            \sigma '
            \left(  
                \alpha_{0 v} +
                \sum_{i=1}^d \alpha_{i v}x_i
            \right)   
        \right). 
    \end{align}

    \item Por el desarrollo (\ref{eq:parcial_alpha_i}) derivada parcial del error con respecto a $\alpha_{u v}$:
    \begin{align} 
        \frac{\partial E(h)}{\partial \alpha_{u v}} 
        =
        \sum_{(x,y) \in \mathcal{D}}
        \sum_{k = 1}^s 
        \left(h_k(x) - y_k\right)
        \left(
            \beta_{v k}
            \sigma '
            \left(  
                \alpha_{0 v} +
                \sum_{i=1}^d \alpha_{i v}x_i
            \right)x_u   
        \right).
    \end{align}
\end{itemize}  

% Consideración sobre los costes 
Si el cálculo se hiciera sin tener más consideración alguna que la propia expresión e incluso obviando el coste de aplicar \textit{forward propagation}, la diferencia y el producto
supondría el cómputo que mostramos en la tabla \ref{tab:coste-computacional-directa}.

% tabla con coste en multiplicación 
\begin{table}[H]
    \begin{center}
    \resizebox{\textwidth}{!}{
    \begin{tabular}{| c | c | c | c | c | c| }
    \hline
    % cabecera
       & Número de parámetros & $+ / -$ & $\times / \div$ & $\sigma$ & $\sigma'$
    \\ \hline
    % Para betas
    (\ref{eq:parcial_beta}) $\frac{\partial h(x))}{\partial \beta_{j k}}$ 
    & $n s$ & $n s (d+1)$ & $n s d$ & $n s$ & 0
    \\
    \hline
    % Para los segos alpha 0i 
    (\ref{eq:parcial_alpha_cero}) $\frac{\partial h(x)}{\partial \alpha_{0 i}}$ 
    & $n$ & $n d$ & $n (d + 1)$ & 0 & $n$
    \\
    \hline
    % Para los segos alpha ji 
    (\ref{eq:parcial_alpha_i}) $\frac{\partial h(x)}{\partial \alpha_{i j}}$ 
    & $n d$ & $n d^2$ & $n d (d + 2)$ & 0 & $n d$
    \\
    \hline
    % Para los segos alpha ji 
    \textit{forward propagation}
    & $n (s+1+d)$ & $n^2 (d+s)(s-1+d)$ & $ n^2 (s+1+d)(d + s)$ & n & 0
    \\
    \hline
    \end{tabular}
    }
    \caption{Coste computacional de aplicar directamente el  algoritmo de gradiente 
    descendente para actualizar $h \in \mathcal{H}_n(\R^d, \R^s)$}
    \label{tab:coste-computacional-directa}
    \end{center}
\end{table}

Abordar el problema de manera directa es muy ineficiente, ya que como mostramos en la tabla 
\ref{tab:expresiones_repetidas_en_descenso_gradiente}
hay cálculos que se repiten en (\refeq{eq:parcial_beta}), (\refeq{eq:parcial_alpha_cero}) y  
(\refeq{eq:parcial_alpha_i}). 

\textcolor{red}{Actualizar la tabla con los valores reales}
\begin{table}[H]
    \begin{center}
    \begin{tabular}{| c | c | c | c | c| c | }
    \hline
    % cabecera
    Expresión  repetida en 
    & $\frac{\partial E(h)}{\partial \beta_i}$ 
    & $\frac{\partial E(h)}{\partial \alpha_{0 i}}$ 
    &$\frac{\partial E(h)}{\partial \alpha_{j i}}$ 
    & Total apariciones 
    & Coste en memoria 
    \\ \hline
    % Primer cálculo repetido 
    $\alpha_{0 i} \sum_{j=1}^d \alpha_{j i}x_j$ 
    & 1 & 1& $d$ & $d+2$ & $n$
    \\
    % Segundo cálculo repetido
    $\beta_{i k} \sigma'
    \left(  
     \alpha_{0 i} 
     \sum_{j=1}^d \alpha_{j i}x_j
    \right)$
    & 0 & 1 & $d$ & $d+1$ & $n s$
    \\ % Veces que se repite el algoritmo de forward propagation 
    $h_k(x)-y_k$
    & $n s$ & $n$ & $n d$ & $n(s+1+d)$ & $s$
    \\ \hline
    \end{tabular}
    \caption{Veces que se calcula una misma expresión 
    para el cálculo de gradiente descendente fijado un
     $i$ y coste en memoria en unidades si se almacenara el cálculo de todos esos parámetros.}
    \label{tab:expresiones_repetidas_en_descenso_gradiente}
    \end{center}
\end{table}

Notemos además que las expresiones del tipo 
$\alpha_{0 i} + \sum_{j=1}^d \alpha_{j i}x_j$  son las que denotábamos como $sensibilidades$ en el algoritmo de \textit{forward propagation} \ref{algoritmo:evaluar red neuronal}.
 

\subsection{ Conclusión balance almacenamiento memoria y coste computacional}  
A la vista de las repeticiones que se producen recogidas en la tabla  
\ref{tab:expresiones_repetidas_en_descenso_gradiente} ofrece un gran beneficio frente a coste de cálculo y memoria almacenar $h_k(x)-y_k$, ya que el coste de cálculo aumenta de manera lineal conforme aumenta el número de neuronas mientras que se mantiene constante el costo en memoria. Además aunque hubiera limitaciones de memoria, podría plantearse el uso el de algoritmo de compresión que si $n$ es lo suficientemente grande, ya que el coste de éste sería menor que calcular todas las apariciones.   

Para el resto de expresiones estamos en una situación en la que el coste en memoria aumenta de manera lineal frente  a que el beneficio
\textcolor{red}{Completar con las conclusiones cuando la tabla esté completa y bien hecha}. 

Resultando la siguiente implementación del algoritmo de cálculo. 

\subsection{Algoritmos de actualización de pesos de una neurona}

Primero aplicaremos algoritmo de gradiente descendente 

Sea $W = (A,S,B)$ 
la matrices que definen a una red neuronal $h \in \rrnnsmn$, denotando a estas como: 
\begin{align}\label{eq:representation red neuronal}
    A &= (\alpha)_{i,j} \text{ con } 1 \leq i \leq n , 1 \leq j \leq d. \\
    S &= (\alpha)_{0, i} \text{ con } 1 \leq i \leq n. \\
    B &= (\beta)_{i,j} \text{ con } 1 \leq i \leq s, 1 \leq j \leq n. \\
\end{align}
 $h$ modificará el valor de sus pesos de acorde al algoritmo de gradiente descendente \ref{eq:descenso-gradiente} que viene dado por

\begin{algorithm}[H]
    \caption{Algoritmo gradiente descendente.}
    \hspace*{\algorithmicindent} \textbf{Input}:$h$ red neuronal  y conjunto de entrenamiento \\
    \hspace*{\algorithmicindent} \textbf{Output:} $h$ actualizada de acorde al algoritmo de gradiente descendente. 
    \begin{algorithmic}[1]
        \STATE Debe de calcularse el previamente el $\nabla E(h)$, es decir cada una de la parciales.
        % actualizamos los pesos de a 
        \STATE Actualización de los pesos de $A$ \\  
        \For{
                $i \in \{1,\ldots n\}$
            }{
                \For{
                $j \in \{1,\ldots d\}$
            }{
                \begin{equation}
                    \alpha_{i j} 
                    \gets 
                    \eta 
                    \frac{\partial E(h)}{\partial \alpha_{i j}}
                \end{equation}
            }
        }
         %  Actualizamos los pesos de S               
        \STATE Actualización de los pesos de $S$ \\   
        \For{
                $i \in \{1,\ldots n\}$
            }{
                \begin{equation}
                    \alpha_{i 0} 
                    \gets 
                    \eta 
                    \frac{\partial E(h)}{\partial \alpha_{i 0} }
                \end{equation}
        } 
        % actualizamos los pesos de B
        \STATE Actualización de los pesos de $B$ \\ 
        \For{
                $i \in \{1,\ldots s\}$
            }{
                \For{
                $j \in \{1,\ldots d\}$
            }{
                \begin{equation}
                    \beta_{i j} 
                    \gets 
                    \eta 
                    \frac{\partial E(h)}{\partial \beta_{i j}}
                \end{equation}
            }
        }
\end{algorithmic}
\end{algorithm}

Notemos la necesidad de una variable $\eta \in \R$ que es prefijada. 

\textcolor{red}{$\eta$ podría ser incluso una vector de pesos que se aplicara de manera distinta a cualquier coeficiente o una función variable de ciertas condiciones. No sé hasta qué punto sería interesante plantearse el uso de algoritmos genéticos para actualizarlo ya que estos carecen de valor teórico. ¿Existe algún resultado de análisis que nos de una cota mejor?}


Procederemos ahora a determinar el cálculo del gradiente, como hemos visto este vienen determinado por la expresión. 

El algoritmo que se muestra a continuación supone que la memoria no es un problema así que se plantea para que el no haya ningún cálculo repetido. 


% Cálculo de los gradientes 
\begin{algorithm}[H]
    \caption{Algoritmo cálculo del gradiente $\nabla E(h)$.}
    \hspace*{\algorithmicindent} \textbf{Input}:$h \in  \rrnnsmn$ red neuronal que representaremos como en \ref{eq:representation red neuronal} y $\mathcal{D}$ conjunto de entrenamiento, con pares $(x, y)$ \\
    \hspace*{\algorithmicindent} \textbf{Output:} Gradiente del error, esto es $\nabla E(h)$ que se almacena en las siguientes variables: 
    \begin{itemize}
        \item $\text{parcial} \alpha_{i j}$ hace referencia a $\frac{\partial E(h)}{\partial \alpha_{i j}}$. 
   
        \item $\text{parcial} \beta _{i j}$ hace referencia a $\frac{\partial E(h)}{\partial \beta _{i j}}$. 
    \end{itemize} 
    \begin{algorithmic}[1]
        \STATE Inicializamos respectivamente las variable que contendrán a las derivadas parciales
        \begin{itemize}
            \item $\text{parcial} \alpha_{i j} \gets$ matriz de reales de tamaño $n+1 \times d$ con entradas a cero. 
            \item $\text{parcial} \beta_{i j} \gets$ matriz de reales de tamaño $s \times n$ con entradas a cero. 
        \end{itemize}
        \STATE 
        \For{ cada para $(x,y) \in \mathcal{D}$}{
            \STATE Calculamos \textit{forward propagation }
            \For{ cada $i \in \{1, \ldots n\}$}{
                \STATE Cálculo de las sensibilidades vector $\delta$ donde su componente $i-$ésima viene dada por \\
                $\delta_i \gets \alpha_{0 i} + \sum_{j = 1}^d \alpha_{j i} x_j$.
                \STATE Cálculo de la salida de los nodos, vector $s$ donde cada componente es \\
                $s_i \gets \sigma (\delta_i)$. 
                \STATE Almacenamos las derivadas \\
                $\text{ds}_i \gets \sigma' (\delta_i)$. \\
                \For{cada $k \in \{1, \ldots s\}$}{
                    $\text{derivada} \beta_{i k } \gets
                     \beta_{i k} \text{ds}_i$
                }
            }
            \STATE $h_{x} \gets B s$  (se multiplica la matriz de coeficientes de la red neuronal por la salida previamente calculada)
            \STATE $\text{diferencia} \gets h_{x} - y$  
        
            % Cálculo del  gradiente
            % cálculo de los betas
            \STATE Cálculo de gradiente de $\frac{\partial E(h)}{\partial \beta _{u v}}$\\  
            \For{
                    $u \in \{1,\ldots n\}$
                }{
                    \For{
                    $v \in \{1,\ldots s\}$
                }{
                 
                        $\text{parcial} \beta _{u v} 
                        \gets 
                        \text{parcial} \beta _{u v} 
                        + 
                        \text{diferencia}_v  s_i$.
                    
                }
            }
            % Cálculo de alpha
            \STATE Cálculo del tipo $\frac{\partial E(h)}{\partial \alpha _{u v}}$\\  
            \For{
                    $v \in \{1,\ldots n\}$
                }{
                    \For{
                        $k \in \{1,\ldots s\}$
                    }{
                        
                        $\text{auxiliarDiferenciaPorDerivada}
                            \gets
                                \text{diferencia}_k 
                                \,
                                \text{derivada} \beta_{v k}$. \\
                        
                        $\text{parcial} \alpha _{0 v} 
                            \gets 
                            \text{parcial} \alpha _{0 v} 
                            + 
                            \text{auxiliarDiferenciaPorDerivada}$.
                        

                        \For{
                            $u \in \{1,\ldots d\}$
                        }{
                           
                            $\text{parcial} \alpha _{u v} 
                                \gets 
                                \text{parcial} \alpha _{u v} 
                                + 
                                \text{auxiliarDiferenciaPorDerivada}
                                \, 
                                x_u$.
                            
                        }
                    }
                }
            
        }
\end{algorithmic}
\end{algorithm}



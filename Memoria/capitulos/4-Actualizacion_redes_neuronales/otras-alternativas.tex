%%%%%%%%%%%%%%%%%%%%%%%%%%%%%%%%%%%
%%% Alternativas al algoritmo de gradiente descendente 
%%%%%%%%%%%%%%%%%%%%%%%%%

\subsection{Otras alternativas al algoritmo de gradiente descendente}  

Recordemos que nuestro enfoque partía de
fundamentar toda decisión de diseño de manera rigurosa. Destaquemos por tanto que en el algoritmo de gradiente descendente se introduce la restricción de que 
las funciones de activación deben de ser diferenciables.

Esto, a priori, no es de extrañar, ya que si buscamos optimizar algún aspecto, en algún momento deberemos de particularizar o reenfocar algunos de los componentes del problema, sin embargo ¿exigir tales restricciones está sustentado teóricamente? 

Desde un punto de vista analítico la forma por excelencia de minimizar radica en la derivación. Pudiéndose relajar el concepto de derivada a derivada 
débil o incluso trabajar en el ambiente de teoría de distribuciones, donde \textit{casi todo} se puede 
derivar \cite{teoriaDistribuciones}.
 La clave sería poder implementar los cálculos y obtener buenos resultados experimentales.

La cuestión reside entonces si existen otras alternativas al algoritmo de gradiente descendente 
que podrían reducir el costo computacional del proceso de aprendizaje. 

Para ello se ha consultado el estado del arte actual encontrando publicaciones como: 

\subsubsection{\textit{Gradients without Backpropagations}}

A principios de este mismo año, se publicó el artículo \textit{Gradients without Backpropagations} \cite{forwardGradient}, en él se introduce el algoritmo al que acuñan como 
\textit{forward propagation} y al que posicionan los propios autores como una alternativa \textit{más eficiente en coste} que el algoritmo de \textit{Backpropagation}. 

Por desgracia, en este artículo no se da una demostración formal que verdaderamente explique el beneficio computacional, sino que se basan en meras experimentaciones. 

Sin embargo, tiene su interés y es por ello que lo mencionamos, en que indica que existe margen de mejora 
imponiendo como restricción el que las funciones de activación sean diferenciables. 
Esta tendencia se puede ver también en artículos como \cite{TransactionsOnNeuralNetworks}.




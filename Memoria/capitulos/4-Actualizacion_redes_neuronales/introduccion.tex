%%%%%%%%%%%%%%%%%%%%%%%%%%%%%%%%%%%%%%%%%%%%%%%%%%%%%%%%%%%%%%%%%%%%%%%%%%%%%%
%
% Introducción sobre algoritmos de cálculo de los pesos de una red neuronal 
%
%%%%%%%%%%%%%%%%%%%%%%%%%%%%%%%%%%%%%%%%%%%%%%%%%%%%%%%%%%%%%%%%%%%%%%%%%%%%%%

\chapter{Explicitación del cálculo de una red neuronal}


Se han concretado en los capítulos anteriores que con con redes neuronales 
podemos aproximar funciones medibles, la cuestión ahora reside en ¿cómo se consigue tales 

Ya expusimos en la sección \ref{algoritmo-forward-propagation} cómo evaluar
una red neuronal, la cuestión entonces radica en ¿cuál es la red neuronal
buscada? es decir a nivel práctico sería ¿cómo puedo calcular matrices de pesos 
adecuadas?

Explicaremos en este capítulo el método tradicional y ampliamente usado de 
\textit{backpropagation} \ref{algoritmo-de-backpropagation} y otros más novedosos como
% Este TODO debería de estar hecho en otro PR, lo elimino y ya lo mergearé


%%%%%%%%%%%%%%%%%%%%%%%%%%%%%%%%%%%%%%%%%%%%%%%%%%%%%%%%%%%%%%%%%%%%%%%%%%%%%%
%
% Introducción sobre algoritmos de cálculo de los pesos de una red neuronal 
%
%%%%%%%%%%%%%%%%%%%%%%%%%%%%%%%%%%%%%%%%%%%%%%%%%%%%%%%%%%%%%%%%%%%%%%%%%%%%%%

\chapter{Cálculo de la red neuronal}

Ya expusimos en la sección \ref{algoritmo-forward-propagation} cómo evaluar
una red neuronal, la cuestión entonces radica en ¿cuál es la red neuronal
buscada? es decir a nivel práctico sería ¿cómo puedo calcular matrices de pesos 
adecuadas?

Explicaremos en este capítulo el método tradicional y ampliamente usado de 
\textit{backpropagation} \ref{algoritmo-de-backpropagation} y otros más novedosos como
\textcolor{red}{TODO : añadir otros métodos de actualización de pesos. La idea sería que 
los buscados tengan un coste menor para así optimizar la velocidad de aprendizaje.}


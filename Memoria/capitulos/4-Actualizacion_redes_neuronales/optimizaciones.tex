%%%%%%%%%%%%%%%%%%%
%% Optimización de la inicialización de los pesos de una red neuronal 
%%%%%%%%%%%%%%%%%%%%%%%%%

\section{ Inicialización de los pesos de una red neuronal}  
\label{section:inicializar_pesos}
Como observábamos en la sección \ref{sec:gradiente-descendente}, el gradiente descendente pretende en cada 
cada iteración mejorar la solución encontrada, pero es 
totalmente sensible a la posición inicial 
de los pesos. 
Presentamos por tanto la siguiente propuesta para inicializar una red neuronal de modo que sus pesos se encuentre ya lo suficientemente cerca de la solución, 
no solo servirá exclusivamente para el método de gradiente descendente 
sino para cualquier otro dependiente del punto inicial. 

\subsection{ Estado del arte relacionado con esto} 

\textcolor{red}{TODO: Estado del arte. Buscar sobre backbones y otros
sistema y cómo influye el estado inicial. Argumentar con esto 
que este algoritmo permitiría generar nuestro propio backbone}

\subsection{Descripción del método propuesto}

La idea proviene de la demostración casi constructiva del teorema \ref{teorema:2_5_entrenamiento_redes_neuronales}.

Se desea inicializar los pesos de $h \in \rrnnsmn$, para la cual, una vez fijado el número $n$ de neuronas de nuestra red neuronal, será necesario  determinar un subconjunto $\Lambda \mathcal{D}$ de datos de entrenamiento. 

La bondad del resultado depende en gran medida de $\Lambda$, 
puesto que a priori se carece de hipótesis, se seleccionará 
de manera aleatoria bajo supuesto de una distribución 
independiente e idénticamente distribuida de los datos. 

Como apunta la demostración, debe de encontrarse un 
$p \in \R^d$ satisfaciendo que $p \cdot (x_i-x_j) \neq 0$ para cualesquiera
atributos $x_i,x_j$ distintos de $\Lambda$.  

Es decir que se estaría considerando un vector que no 
pertenezca a una unión finita de hiperplanos ortogonales de $\R^r$. 
De manera teórica la probabilidad de seleccionar un $p$ y 
que pertenezca al espacio ortogonal es $0$, sin embargo esto 
no quiere decir que no pueda pasar. 

Tomaremos por tanto un $p$ aleatorio y a partir de él 
seleccionaremos $\Lambda$ lo suficientemente grande para que
 al menos $n$ vectores admitan de manera estricta la ordenación: 

\begin{equation}\label{eq:method_inicializar_condition_desigualdad}
    p \cdot x_1 < 
    p \cdot x_2 
    < \cdots <
    p \cdot x_n.
\end{equation}
Para continuar, para la función de activación 
seleccionada $\gamma$, por cómo se definen 
existirá un $M \in \R^+$ tal que 
\begin{equation} \label{eq:method_inicializar_M}
    \gamma(K)=1 \text{ y } \gamma(-K)=0 
    \text{ sean constantes para todo }K \geq M.
\end{equation}

Una vez concretados los valores $p$, $\Lambda$ y $M$ que satisfagan las condiciones 
(\refeq{eq:method_inicializar_condition_desigualdad}) 
y (\refeq{eq:method_inicializar_M})  
falta concretar los valores iniciales de la red neuronal. 

Para ello debemos de calcular el valor de las matrices $(A,S,B)$ que definen a una red neuronal y que presentamos en la sección \ref{section:rrnn_implementation}.

Recordemos que $A$ y $S$ tienen tantas filas como neuronas  y $B$ tantas columnas como neuronas. 

Usado la notación vectorial
$p_{[i,j]} = (p_i, p_{i+1}, \ldots, p_{j})$ donde $(p_0, p_1, \ldots, p_d)=p$, comenzaremos definiendo el valor de la primera fila como

\begin{align}
    &S_1 = M p_0, \\
    & A_{1 *} = M p_{[1,d]}, \\
    & B_{* 1} = y_1.
\end{align}

Los valores de la fila  k-ésimas de las matrices $(A,S)$, vendrán determinados por la única función afín $A \in \afines$, 
dada por $A_k(x)=B_k(p \cdot x)$, con $B_{k}$ como la única función afín de $\R$ en $\R$ que cumple que 
\begin{equation}
    B_k(p \cdot x_{k-1}) = -M 
    \quad \text{y} \quad 
     B_{k}(p \cdot x_k)= M.
\end{equation}

Que esto equivale a calcular las constantes reales $\tilde {\alpha}$. 

Si tenemos presente que 
\begin{equation}
    \tilde{\alpha}_{k p} (p \cdot x_{k-1}) + \tilde{\alpha}_{k s} = -M 
    \quad \text{y} \quad 
    \tilde{\alpha}_{k p}(p \cdot x_k) + \tilde{\alpha}_{k s}= M.
\end{equation} 
Resolviendo el sistema resulta que 

\begin{equation}
    \left\{ 
        \begin{array}{l}
            \tilde{\alpha}_{k p} = \frac{2 M}{p \cdot (x_k - x_{k-1})}
            \\
            \tilde{\alpha}_{k s} 
            = M -  \tilde{\alpha}_{k p}(p \cdot x_{k-1})
            = M -  \frac{2 M}{p \cdot (x_k - x_{k-1})}(p \cdot x_{k-1}) 
        \end{array}
    \right.
\end{equation}

Luego los coeficientes de la red neuronal $A$, $S$ se deduciría de 
\begin{equation}
    \left\{ 
        \begin{array}{l}
            \alpha_{k 0} = \tilde{\alpha}_{k s} =
            M -  \frac{2 M}{p \cdot (x_k - x_{k-1})}(p \cdot x_{k-1}) 
            \\
            \alpha_{k i} =  \tilde{\alpha}_{k p} p_{i}
            = 
            \frac{2 M}{p \cdot (x_k - x_{k-1})}
            p_i 
        \end{array}
        \right.
\end{equation}

Esto define un sistema lineal compatible
cuya solución son las respectivas filas y columnas: 

\begin{equation}
    \left\{ 
        \begin{array}{l}
            S_{k} = M -  \frac{2 M}{p \cdot (x_k - x_{k-1})}(p \cdot x_{k-1})\\
            A_{k i} = \frac{2 M}{p \cdot (x_k - x_{k-1})}
            p_{i}  
            \\
            B_{* k} = y_k - y_{k-1}
        \end{array}
    \right.
\end{equation}  

Con todo esto el proceso algorítmico resultante es: 

% Algoritmo de inicialización de pesos de una red neuronal

\begin{algorithm}[H]
    \caption{Inicialización de pesos de una red neuronal}
    \textbf{Input:} Tamaño red neuronal $n$, conjunto de datos de entrenamiento $\mathcal{D}$, constate $M$ involucrada en \refeq{eq:method_inicializar_M}.

    \textbf{Input:} Red neuronal, representada con las matrices $(A,S,B)$.
    \hspace*{\algorithmicindent} 
    \begin{algorithmic}[1]
        %selección de p
       \STATE \textit{Inicializamos $p$}. \\
       $p \gets$ vector de $\R^{d+1}$. 
       \COMMENT{Como heurística será generado con distribución uniforme en el intervalo $[0,1]$} 
       % Cálculo de Lambda
       \STATE \textit{Selección datos inicialización
       $\Lambda \subset \mathcal{D}$}. \\
       \begin{equation}
           \Lambda \gets \{ \emptyset \}
       \end{equation}
       \While{tamaño de $\Lambda < n$}{
            Tomamos de manera aleatoria $(x,y)$ de $\mathcal{D}$.   \\
        \If{para todo $(a,b) \in \Lambda$ se satisface que 
        $p \cdot (x-a)$}{ 
           \begin{equation}
                \Lambda  \gets \Lambda \cup \{(x,y)\}.
           \end{equation} 
           \COMMENT{$\Lambda$ está ordenado conforme a la propiedad 
           \refeq{eq:method_inicializar_condition_desigualdad} 
           }
        }
       }
       \STATE \textit{Cálculo de los parámetros base de la red neuronal.} \\
       
       Para el primer $(x_1, y_1) \in \Lambda$ \\
       \begin{align}
            &S_1 = M p_0, \\
            & A_{1 *} = M p_{[1,d]}, \\
            & B_{* 1} = y_1.
        \end{align}
       $\Lambda \gets \Lambda \setminus \{(x_1, y_1)\} $ \\
       \STATE \textit{Cálculo del resto de neuronas}. 
       \For{ cada $(x_k, y_k) \in \Lambda$}{
        \begin{align}
            &S_{k} = M -  \frac{2 M}{p \cdot (x_k - x_{k-1})}(p \cdot x_{k-1})\\
            & A_{k i} = \frac{2 M}{p \cdot (x_k - x_{k-1})}
            p_{i}  \quad i \in \{1, \ldots d\},\\
            & B_{* k} = y_k - y_{k-1}.
        \end{align} 
       }
       \STATE \textbf{return $(A,S,B)$}.
    \end{algorithmic}  
\end{algorithm}

\textcolor{red}{TODO: Añadir coste computacional}

\textcolor{red}{TODO: Hacer observaciones sobre que sería igual de válido cambiando la $M$ y $p$ ¿hay alguna selección mejor que otra? }

\textcolor{red}{TODO: Hacer experimentaciones }
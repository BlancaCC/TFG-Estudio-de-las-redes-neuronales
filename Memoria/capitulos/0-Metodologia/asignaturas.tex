%%%%%%%%%%%%%%%%%%%%%%%%%%%%%%%%%%%%%%%%%%%%%%%%%%%%%%%%
%           Asignaturas del grado relacionadas 
%%%%%%%%%%%%%%%%%%%%%%%%%%%%%%%%%%%%%%%%%%%%%%%%%%%%%%%%

\section{Asignaturas de grado relacionadas con el trabajo }  
\label{ch01:asignaturas}
Si bien, es casi imposible enumerar de manera
exhaustiva todas las asignaturas involucradas en este trabajo,
ya que todas han influido en menor o mayor medida en la comprensión 
y formulación de ideas; las principales han
sido: 
\begin{itemize}
    \item \textbf{Análisis Matemático}: todas las asignaturas del departamento de análisis matemático
    han tenido relevancia, ya sea en el modelado de espacios de funciones,
    para probar que las redes neuronales son aproximadores universales
    y para la elaboración de nuestros propios resultados.  
    \item El \textbf{Aprendizaje Automático} y \textbf{Visión por Computador} sientan las bases de lo que son problemas de aprendizaje, 
    tratamiento de los datos y evaluación del error, así como el uso práctico de las redes neuronales. 
    \item \textbf{Estructura de Datos}: diseño e implementación de la modelización de las redes neuronales y 
    sus algoritmos concernientes. 
\end{itemize}

En menor medida han tenido también relevancia: 
\begin{itemize}
    \item \textbf{Inferencia Estadística}, la Inferencia Estadística está estrechamente ligada con 
    la ciencia de datos, también ha sido utilizada para los test de hipótesis. 
    \item Otras asignaturas que han intervenido \textbf{Programación Orientada a Objeto} \textbf{Diseño y Desarrollo de Sistemas Informáticos}
    \item Nociones de \textbf{Topología} se han requerido para probar ciertos resultados analíticos. 
    \item \textbf{Álgebra, Métodos Numéricos I, Modelos I, Geometría III y Metaheurística} esta agrupación de asignaturas 
    han ayudado a la comprensión  y servido como germen de ideas y relaciones a lo largo de todo el desarrollo de la memoria, 
    por poner algunos ejemplos: la relación entre grafo y una matriz proviene de la asignatura de Modelos, 
    la existencia de funciones cuyo error tiende a infinito de Métodos Numéricos,
    resultados constructivos  que después han podido ser implementados (Álgebra y Métodos Numéricos), algunos resultados propios sobres las funciones de activación (Geometría), 
    conocimiento sobre algoritmos de optimización como los genéticos y \textit{KNN} (Metaheurística).
\end{itemize}
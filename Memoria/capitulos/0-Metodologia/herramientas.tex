%%%%%%%%%%%%%%%%%%%%%%%%%%%%%%%%%%%%%%%%%%%%%%%%%%%%%%%%%%%%%%%%
% Justificación de porqué se ha seleccionado Julia como lenguaje de programación 
%%%%%%%%%%%%%%%%%%%%%%%%%%%%%%%%%%%%%%%%%%%%%%%%%%%%%%%%%%%%%%%%

\section{Herramientas utilizadas}

\subsection{GitHub}
Como servicio externo hemos usado \href{https://github.com}{GitHub}, ya que permite implementar de manera eficaz
 todo el desarrollo ágil: 
desde la planificación, comunicación, test e incluso difusión y acceso a los resultados. 

\subsection{Lenguaje de programación Julia}
Hemos seleccionado como lenguaje de programación \href{https://julialang.org}{Julia}, 
esto se debe a que nos ofrece \textit{benchmarks} muy competitivos\footnote{Véase los resultado expuestos en 
\url{https://julialang.org/benchmarks/}, web consultada por última vez el 22 de mayo de 2022.} 
, al nivel de C. Así como la disponibilidad  de bibliotecas usadas en ciencia de datos de  
la de lenguajes como \textit{R} y \textit{Python}. 
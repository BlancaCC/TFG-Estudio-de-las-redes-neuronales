%%%%%%%%%%%%%%%%%%%%%%%%%%%%%%%%%%%%%%%%%%%%%%%%%%%%%%%%%%%%%%%%
% Justificación de porqué se ha seleccionado Julia como lenguaje de programación 
%%%%%%%%%%%%%%%%%%%%%%%%%%%%%%%%%%%%%%%%%%%%%%%%%%%%%%%%%%%%%%%%

\section{Herramientas utilizadas}

\subsection{GitHub}
Como servicio externo hemos usado \href{https://github.com}{GitHub}, ya que permite implementar de manera eficaz
 todo el desarrollo ágil: 
desde la planificación, comunicación, test e incluso difusión y acceso a los resultados. 

\subsection{Lenguaje de programación Julia}
Hemos seleccionado como lenguaje de programación \href{https://julialang.org}{Julia} 
por los siguientes motivos: 
\begin{itemize}
    \item Ofrece \textit{benchmarks} 
    muy competitivos\footnote{Véase los resultado expuestos en 
    \url{https://julialang.org/benchmarks/}, 
    web consultada por última vez el 22 de mayo de 2022.},
    al nivel de C.
    \item Variedad de bibliotecas usadas para ciencia de datos que podríamos encontrar 
    en otros lenguajes muy utilizados en el sector como son Python y R. 
    \item Otras características del lenguaje que facilitan 
    la optimización del código como veremos en la sección \ref{ch06:activation-function-implementation}. 
\end{itemize}
 
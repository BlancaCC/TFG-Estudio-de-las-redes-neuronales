%%%%%%%%%%%%%%%%%%%%%%%%%%%%%%%%%%%%%%%%%%%%%%%%%%%%%%%%%%%%%%%%
% Justificación de porqué se ha seleccionado Julia como lenguaje de programación 
%%%%%%%%%%%%%%%%%%%%%%%%%%%%%%%%%%%%%%%%%%%%%%%%%%%%%%%%%%%%%%%%

\section{Herramientas utilizadas}
\label{ch01:Herramientas}
\subsection{GitHub}
Como servicio externo hemos usado \href{https://github.com}{GitHub}, ya que permite implementar de manera eficaz
 todo el desarrollo ágil: 
desde la planificación, comunicación, test e incluso difusión y acceso a los resultados. 

\subsection{Lenguaje de programación Julia}
Hemos seleccionado como lenguaje de programación \href{https://julialang.org}{Julia} 
por los siguientes motivos (\cite{virtudes-de-julia}): 
\begin{itemize}
    \item Ofrece \textit{benchmarks} 
    muy competitivos\footnote{Véase los resultado expuestos en 
    \url{https://julialang.org/benchmarks/}, 
    web consultada por última vez el 22 de mayo de 2022.},
    al nivel de C.
    \item Variedad de bibliotecas usadas para ciencia de datos que podríamos encontrar 
    en otros lenguajes muy utilizados en el sector como son Python y R. 
    \item Otras características del lenguaje que facilitan 
    la optimización del código como veremos en la sección \ref{ch06:activation-function-implementation}. 
\end{itemize}

\subsection{\textit{Notebooks}}
% Comentario sobre qué es un scripts
\marginpar{\maginLetterSize
    \iconoAclaraciones \textcolor{dark_green}{     
        \textbf{
            Significado del término \textit{script}
        }
    }

    Término para designar un programa relativamente simple y que por lo general
    es interpretado (se va traduciendo y ejecutando línea  a línea).
}

Todas las gráficas que se muestran en la memoria han sido creadas por nosotros. 
Las hemos generado con \textit{scripts} o la mayoría de la veces con 
\href{https://jupyter.org}{\textit{notebooks} de Jupyter}, el motivo de esto ha 
sido el tener una interacción y visualización más cómoda y compacta de los resultados.

El lenguaje utilizado en los \textit{notebooks} ha sido Julia o Python indistintamente,
a conveniencia de cuál ofrecía la función o biblioteca más cómoda para el propósito.  


 
%%%%%%%%%%%%%%%%%%%%%%%%%%%%%%%%%%
%% Comentarios previos 
%%%%%%%%%%%%%%%%%%%%%%%%%%%%%%%%%%
% Convenio colores notas
%%%%%%%%%%%%%%%%%%%%%%%%%%%%%%%%%%

\section*{Comentario previo}

Se pretende con este documento presentar un  trabajo de fin de grado de Ingeniería Informática y Matemáticas y que cualquier miembro del tribunal; 
independientemente de la vertiente a la que pertenezca sea capaz de comprenderlo en su totalidad.  
Para ello se ha acompañado la exposición con notas en los márgenes aclaratorias, que siguen el siguiente código de color y icono: 

\begin{itemize}
    \item  \iconoAclaraciones \textcolor{dark_green}{  Color 1}: Comentarios para 
    aclarar conceptos matemáticos o ofertar la idea intuitiva que 
    se esconde, donde no se presuponen conocimientos avanzados en 
    matemáticas. 
    \item  \iconoProfundizar \textcolor{blue}{  Color 2}: Comentarios para una reflexión más profunda o que indique nuevas áreas que explorar. 
    \item  \iconoClave  \textcolor{darkRed}{  Color 3}: Concepto clave y destacable que tendrá un papel fundamental a posteriori.  
\end{itemize}

%Nota los colores seleccionados han sido creados con una paleta inclusiva
% https://palett.es/6a94a8-013e3b-7eb645-31d331-26f27d

% !TeX root = ../../tfg.tex
% !TeX encoding = utf8

\chapter{Introducción a las redes neuronales} 
\label{chapter:Introduction-neuronal-networks}
Toma en la actualidad un papel clave en el desarrollo tecnológico el aprendizaje automático
\cite{importancia-arte-aprendizaje-automatico}, siendo 
las redes neuronales un modelo ampliamente usado en este contexto y por ello 
nuestro objetivo de democratización. 

Comenzaremos pues  sentando las bases y 
formalizando  
qué significa 
que una máquina \textit{aprenda}  y cómo puede conseguirse, 
todo ello en la sección \ref{sec:Aprendizaje}.
Continuaremos en las secciones \ref{sec:redes-neuronales-intro-una-capa} 
y \ref{chapter:construir-redes-neuronales}
definiendo las redes neuronales, cómo se construyen y evalúan. 
\label{motivo-una-capa}
\textbf{Centraremos además nuestros esfuerzos en el estudio de las redes neuronales de una sola capa}, 
esto se debe a que a pesar de la tendencia actual de utilizar redes neuronales profundas con 
un aumento de parámetros que ajustar 
\cite{a-universal-law-of-Robustness} \cite{CHAI2021100134} el sustento de esto no deja de ser experimental 
o basado en cotas de carácter \textit{en el peor de los casos y por el tamaño del espacio de búsqueda}.
Pero estos motivos no constituyen una demostración formal ni rigurosa de porqué decantarnos verdaderamente por 
ello y es más, otros artículos experimentales demuestran que aumentar el número de capas no mejora los resultado 
\cite{DBLP:conf/iwann/Linan-Villafranca21}. 

Así pues, sustentados con la demostración de convergencia universal \cite{HORNIK1989359}
de las redes neuronales, la cual asegura la convergencia en igualdad de condiciones, 
tanto para una capa como para varias
y en pos de simplificar el modelo. 
A priori, 
para establecer nuestras propias hipótesis de optimización desprovistos
de sesgos empíricos, comenzaremos nuestro trabajo tratando tan solo de redes neuronales de una sola capa oculta.
%%%%%%%%%%%%%%%%%%%%%%%%%%%%%%%%%%%%%%%%%%%%%%%%
%    Combinación de distintas funciones 
%              de activación 
%%%%%%%%%%%%%%%%%%%%%%%%%%%%%%%%%%%%%%%%%%%%%%%%
\newpage
\chapter{Selección genética de las funciones de activación }
\label{ch08:genetic-selection}

\textcolor{red}{TODO: issue 90}
A modo de borrador de lo que tendrá este capítulo: 
\begin{itemize}
    \item Referencia a la idea intuitiva.
    \item Comentario sobre complejidad.
    
    Más funciones de activación -> aumento espacio de búsqueda.
    Si se usan bien -> disminuye número de neuronas 
    -> se converge más rápido con métodos. 
    \item Cómo introducir los algoritmos genéticos para desarrollar esta idea. 
    
    El algoritmo genético selecciona qué funciones de activación se usan. Tras esto se entrena el algoritmo de de manera usual.
\end{itemize}
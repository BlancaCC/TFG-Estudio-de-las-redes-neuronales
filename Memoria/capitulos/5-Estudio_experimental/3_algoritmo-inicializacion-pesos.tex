%%%%%%%%%%%%%%%%%%%%%%%%%%%%%%%%%%%%%%%%%%%%%%%%%%%%%%%%%%%%%%%%%%%%
% Experimentación con ALGORITMO INICIALIZACIÓN DE PESOS
%%%%%%%%%%%%%%%%%%%%%%%%%%%%%%%%%%%%%%%%%%%%%%%%%%%%%%%%%%%%%%%%%%%%


En la siguiente sección trataremos sobre la bondad del algoritmo expuesto

\section{Contraste de hipótesis con inicialización aleatoria} 
\label{ch07:experimento-1} 

Las preguntas a resolver son ¿mejora nuestro algoritmo? ¿Cuánto mejora?

La primera observación  es que como
hemos observado en el modelado de una red neuronal 
en la sección \ref{ch05:construction-evaluation-nnnn}
una red neuronal depende de varios parámetros:
la dimensión de entrada $d$, el número de neuronas en la capa oculta $n$, la dimensión de salida $s$ 
y la funciones de activación de cada neurona.  

Por simplicidad fijaremos una función de activación 
Así que deberemos de formular el test 
para diferentes tamaños $n$, $d$, $s$. 

\textcolor{red}{Ahora mismo no tengo muy claro 
los tamaños porque tampoco quiero que dure mucho tiempo la realización del experimento, los concretaré tras unas primeras pruebas}. 

\subsection{Descripción experimento}

El experimento costa de los siguientes pasos: 

\begin{enumerate}
% Paso 0: Selección de data sets 
\item Dado un conjunto de datos de entrenamiento $\D$  se separará el conjunto en
\begin{itemize}
    \item $\D_i$ \textbf{Conjunto de 
    datos de inicialización.} Debe de ser mayor que 
    $n$ y lo suficientemente grande para que el algoritmo diseñado funcione correctamente. 

    \item $\D_t$ \textbf{Conjunto de 
    datos de test.} Se utilizarán para el cálculo del error. 
\end{itemize}  

% Paso 1: Construcción 
\item Fijados $n, d$ y $s$ se generarán dos redes neuronales: 

\begin{itemize}
    \item Una inicializada de manera aleatoria con valores dentro de un rango de valores. 
    
    \item  Otra inicializada con nuestro algoritmo. 
\end{itemize}

% Paso 2: Evaluación del error
\item Utilizando $\D_t$ deberá de tomarse un registro del error dentro de tal muestra. 
\end{enumerate}

Los pasos 2 y 3 se repetirán tantas veces como 
muestras se desee tomar. 

\subsection{Contraste de hipótesis}

Se desea comparar si los errores observados efectivamente son notables: 

Para ello se realizará un test de Wilcoxon, con las siguientes hipótesis

\begin{itemize}
    \item $H_0$: La mediana de las diferencia de cada par de muestras es $0$. 
    \item $H_a$: La mediana de las diferencia entre cada par de muestras es diferente de cero. 
\end{itemize}

La utilidad de este test es que si rechaza la hipótesis la hipótesis nula sabremos que con un $95 \%$ de certeza tendrán medianas diferentes, es decir, \textbf{existe una 
diferencia en los errores}. En caso de que no se rechace no podremos afirmar nada.
Puede encontrar la implementación en el repositorio del
 proyecto \footnote{En el directorio de experimentos 
 de \url{https://github.com/BlancaCC/TFG-Estudio-de-las-redes-neuronales}.}.

\subsection{Requisitos técnicos}  

A la vista de todo el proceso es descrito surgen las siguientes necesidades técnicas que deberemos de implementar:  

\subsubsection{Capacidad de crear una red neuronal aleatoria}  

Deberá de crearse una red neuronal con entradas dentro de un rango $[a,b]$ con $a < b$ reales,
que tenga una entrada de tamaño $d$,
$n$ neuronas en la capa oculta y
una dimensión de salida $d$.

\subsubsection{Implementación del algoritmo de inicialización}

Deberá de implementarse del algoritmo  \ref{algo:algoritmo-iniciar-pesos} con todos los requisitos y atributos que ahí se describe.  

\subsubsection{Función para medir el error}

Deberá implementarse una función para medir el
 error, no es lo mismo problemas de clasificación 
que de regresión, así que deberemos de ir con 
cuidado. 

Además deberá de realizarse una busca de los datos 

\subsubsection{ Forma de evaluar las redes neuronales}  

Dado una red neuronal, una función de evaluación y un datos ser capaz de aplicar el algoritmo de \textit{forward propagation} descrito en \ref{algoritmo:evaluar red neuronal}.


\subsubsection{Bases de datos de prueba}
\textcolor{red}{Nota:
Comenzaremos probando con bases de datos de juguete 
y en función de tiempo y prestaciones ya veremos si merece la pena plantearse el uso de datos reales
}

\subsubsection{Implementación del experimento} 
Deberá de implementarse una función que realice el 
experimento tal cual hemos descrito en \ref{ch07:experimento-1}.

Podrían utilizarse alguno de estos: 

\begin{itemize}
    \item \href{https://github.com/JuliaStats/RDatasets.jl}{Julia contiene las base de datos estándar de R}.
    \item \href{https://juliaml.github.io/MLDatasets.jl/stable/}{Otros paquetes básicos provistos también por la comunidad.}
\end{itemize}

\subsection*{Estructura de datos}
Para mantener eficientemente ordenados los puntos 
lo normal sería utilizar un \textit{conjunto ordenado} donde ir añadiendo los datos, que la propia estructura los ordene y tras esto sacar
los datos ya ordenados. 
El problema es que el tipo de dato \text{set} en Julia está basado en diccionarios
\footnotetext{Véase \url{https://discourse.julialang.org/t/can-you-sort-a-set/47948/2} 
link accedido por última vez el 3 de junio de 2022.}

Por lo tanto hemos recurrido a una biblioteca externa
utilizado las estructuras de datos propias de Julia 
\href{https://juliacollections.github.io/DataStructures.jl/v0.9/sorted_containers.html}{estructuras}. 
La documentación del lenguaje está bastante regular. 
No hemos encontrado en ese parque un conjunto ordenado. 

Por lo que 


Por lo tanto lo que vamos es a almacenar los datos y luego guardarlos
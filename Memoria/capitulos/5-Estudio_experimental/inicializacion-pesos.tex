%%%%%%%%%%%%%%%%%%%
%% Optimización de la inicialización de los pesos de una red neuronal 
%%%%%%%%%%%%%%%%%%%%%%%%%
\chapter{ Mejora en la inicialización de los pesos de una red neuronal}  
\label{section:inicializar_pesos}
Como observábamos en la sección \ref{sec:gradiente-descendente}, el gradiente descendente pretende en cada 
iteración mejorar la solución encontrada, pero es 
totalmente sensible a la posición inicial 
de los pesos. 
Presentamos por tanto la siguiente propuesta para inicializar una red neuronal con el objetivo de que sus pesos se encuentren ya cerca de la solución. Destaquemos que este algoritmo 
no solo servirá exclusivamente para el método de gradiente descendente 
sino para cualquier otro dependiente del punto inicial. 

\section{ Estado del arte relacionado } 

% Nota informativa de lo que es un backbone 
\setlength{\marginparwidth}{\bigMarginSize}
\marginpar{\maginLetterSize
    \iconoAclaraciones \textcolor{dark_green}{     
        \textbf{
            Qué es un \textit{backbone}
        }
    }

    Se denomina \textit{backbone} a los nodos de una red neuronal que han sido \textbf{inicializados con valores 
    de otras redes neuronales ya entrenadas}. Por ejemplo si quisiéramos clasificar insectos y partiéramos de una red neuronal que se entrenó para clasificar frutas.  
    
}

Se ha comprobado en la práctica que el uso de 
\textit{backbones} reporta mejoras sustanciales en el 
entrenamiento de redes neuronales, véanse artículos como \cite{backbone-object-detection}, \cite{backbone-Architecture} y sus referencias. 

Cuando se carece de \textit{backbones} o no es posible utilizarlos, una buena técnica para dar un valor inicial  a los pesos consiste en basarse en metaheurísticas tales como algoritmos genéticos,
ver \cite{Montana2002NeuralNW}, sin embargo, éstos 
tienen un coste computacional importante. 

Ante esta situación, nuestro algoritmo propone una inicialización de bajo coste computacional y que no necesita de más valores que los del conjunto de entrenamiento. 



\section{Descripción del método propuesto}

\begin{aportacionOriginal} % método de construción
    
La idea proviene de la demostración casi constructiva del teorema \ref{teorema:2_5_entrenamiento_redes_neuronales}.

Se desea inicializar los pesos de $h \in \rrnnsmn$, para la cual, una vez fijado el número $n$ de neuronas de nuestra red neuronal, será necesario  determinar un subconjunto $\Lambda \mathcal{D}$ de datos de entrenamiento. 

La bondad del resultado depende en gran medida de $\Lambda$, 
puesto que a priori se carece de hipótesis, se seleccionará 
de manera aleatoria bajo el supuesto de una distribución 
independiente e idénticamente distribuida de los datos. 

Como apunta la demostración, debe encontrarse un 
$p \in \R^d$ satisfaciendo que $p \cdot (x_i-x_j) \neq 0$ para cualesquiera
atributos $x_i,x_j$ distintos de $\Lambda$.  

Es decir que se estaría considerando un vector que no 
pertenezca a una unión finita de hiperplanos ortogonales de $\R^r$. 
De manera teórica la probabilidad de seleccionar un $p$ y 
que pertenezca al espacio ortogonal es $0$, sin embargo esto 
no quiere decir que no pueda pasar. 

Tomaremos por tanto un $p$ aleatorio y a partir de él 
seleccionaremos $\Lambda$ lo suficientemente grande para que
 al menos $n$ vectores admitan de manera estricta la ordenación: 

\begin{equation}\label{eq:method_inicializar_condition_desigualdad}
    p \cdot x_1 < 
    p \cdot x_2 
    < \cdots <
    p \cdot x_n.
\end{equation}
Para continuar, para la función de activación 
seleccionada $\gamma$, por cómo se definen 
existirá un $M \in \R^+$ tal que 
\begin{equation} \label{eq:method_inicializar_M}
    \gamma(K)=1 \text{ y } \gamma(-K)=0 
    \text{ sean constantes para todo }K \geq M.
\end{equation}

Una vez concretados los valores $p$, $\Lambda$ y $M$ que satisfagan las condiciones 
(\refeq{eq:method_inicializar_condition_desigualdad}) 
y (\refeq{eq:method_inicializar_M})  
falta concretar los valores iniciales de la red neuronal. 

Para ello debemos calcular el valor de las matrices $(A,S,B)$ que definen a una red neuronal y que presentamos en la sección \ref{section:rrnn_implementation}.

Recordemos que $A$ y $S$ tienen tantas filas como neuronas  y $B$ tantas columnas como neuronas. 

Usado la notación vectorial
$p_{[i,j]} = (p_i, p_{i+1}, \ldots, p_{j})$ donde $(p_0, p_1, \ldots, p_d)=p$, comenzaremos definiendo el valor de la primera fila como

\begin{align}
    &S_1 = M p_0, \\
    & A_{1 *} = M p_{[1,d]}, \\
    & B_{* 1} = y_1.
\end{align}

Los valores de la fila  k-ésima de las matrices $(A,S)$, vendrán determinados por la única función afín $A \in \afines$, 
dada por $A_k(x)=B_k(p \cdot x)$, con $B_{k}$ como la única función afín de $\R$ en $\R$ que cumple que 
\begin{equation}
    B_k(p \cdot x_{k-1}) = -M 
    \quad \text{y} \quad 
     B_{k}(p \cdot x_k)= M.
\end{equation}

Esto equivale a calcular las constantes reales $\tilde {\alpha}$. 

Si tenemos presente que 
\begin{equation}
    \tilde{\alpha}_{k p} (p \cdot x_{k-1}) + \tilde{\alpha}_{k s} = -M 
    \quad \text{y} \quad 
    \tilde{\alpha}_{k p}(p \cdot x_k) + \tilde{\alpha}_{k s}= M.
\end{equation} 
Resolviendo el sistema resulta que 

\begin{equation}
    \left\{ 
        \begin{array}{l}
            \tilde{\alpha}_{k p} = \frac{2 M}{p \cdot (x_k - x_{k-1})}
            \\
            \tilde{\alpha}_{k s} 
            = M -  \tilde{\alpha}_{k p}(p \cdot x_{k-1})
            = M -  \frac{2 M}{p \cdot (x_k - x_{k-1})}(p \cdot x_{k-1}) 
        \end{array}
    \right.
\end{equation}

Luego los coeficientes de la red neuronal $A$, $S$ se deducirían de 
\begin{equation}
    \left\{ 
        \begin{array}{l}
            \alpha_{k 0} = \tilde{\alpha}_{k s} =
            M -  \frac{2 M}{p \cdot (x_k - x_{k-1})}(p \cdot x_{k-1}) 
            \\
            \alpha_{k i} =  \tilde{\alpha}_{k p} p_{i}
            = 
            \frac{2 M}{p \cdot (x_k - x_{k-1})}
            p_i 
        \end{array}
        \right.
\end{equation}

Esto define un sistema lineal compatible
cuya solución son las respectivas filas y columnas: 

\begin{equation}
    \left\{ 
        \begin{array}{l}
            S_{k} = M -  \frac{2 M}{p \cdot (x_k - x_{k-1})}(p \cdot x_{k-1})\\
            A_{k i} = \frac{2 M}{p \cdot (x_k - x_{k-1})}
            p_{i}  
            \\
            B_{* k} = y_k - y_{k-1}
        \end{array}
    \right.
\end{equation}  

Con todo esto el proceso algorítmico resultante es: 

\end{aportacionOriginal} % método de construcción

% Algoritmo de inicialización de pesos de una red neuronal

\begin{algorithm}[H]
    \caption{Inicialización de pesos de una red neuronal}
    \textbf{Input:} Tamaño red neuronal $n$, conjunto de datos de entrenamiento $\mathcal{D}$, constate $M$ involucrada en \refeq{eq:method_inicializar_M}.

    \textbf{Input:} Red neuronal, representada con las matrices $(A,S,B)$.
    \hspace*{\algorithmicindent} 
    \begin{algorithmic}[1]
        %selección de p
       \STATE \textit{Inicializamos $p$}. \\
       $p \gets$ vector de $\R^{d+1}$. 
       \COMMENT{Como heurística será generado con distribución uniforme en el intervalo $[0,1]$} 
       % Cálculo de Lambda
       \STATE \textit{Selección  de los datos de inicialización
       $\Lambda \subset \mathcal{D}$}. \\
       \begin{equation}
           \Lambda \gets \{ \emptyset \}
       \end{equation}
       \While{tamaño de $\Lambda < n$}{
            Tomamos de manera aleatoria $(x,y)$ de $\mathcal{D}$.   \\
        \If{para todo $(a,b) \in \Lambda$ se satisface que 
        $p \cdot (x-a) \neq 0$}{ 
           \begin{equation}
                \Lambda  \gets \Lambda \cup \{(x,y)\}.
           \end{equation} 
           \COMMENT{$\Lambda$ está ordenado conforme a la propiedad 
           \refeq{eq:method_inicializar_condition_desigualdad} 
           }
        }
       }
       \STATE \textit{Cálculo de los parámetros base de la red neuronal.} \\
       
       Para el primer $(x_1, y_1) \in \Lambda$ \\
       \begin{align}
            &S_1 = M p_0, \\
            & A_{1 *} = M p_{[1,d]}, \\
            & B_{* 1} = y_1.
        \end{align}
       $\Lambda \gets \Lambda \setminus \{(x_1, y_1)\} $ \\
       \STATE \textit{Cálculo del resto de neuronas}. 
       \For{ cada $(x_k, y_k) \in \Lambda$}{
        \begin{align}
            &S_{k} = M -  \frac{2 M}{p \cdot (x_k - x_{k-1})}(p \cdot x_{k-1})\\
            & A_{k i} = \frac{2 M}{p \cdot (x_k - x_{k-1})}
            p_{i}  \quad i \in \{1, \ldots d\},\\
            & B_{* k} = y_k - y_{k-1}.
        \end{align} 
       }
       \STATE \textbf{return $(A,S,B)$}.
    \end{algorithmic}  
\end{algorithm}

\subsection{Coste computacional algoritmo de inicialización de pesos}

Antes de comenzar con el análisis notemos que el algoritmo tiene un componente aleatorio introducido en la selección del $p$; 
y que en el peor (y con probabilidad nula) de los casos podría acabar con coste $\mathcal{O}(|\mathcal{D}|)$ devolviendo un resultado erróneo. 

Sin embargo, en virtud de las observaciones hechas en la 
descripción del método, la mayoría de los datos del conjunto de entrenamiento tomados para inicializar la red neuronal cumplirán la propiedad de ortogonalidad y por tanto una buena 
heurística es suponer 
que  el paso 2 del algoritmo, \textit{Selección  de los datos de inicialización} se va a realizar con un coste de orden lineal a $n$. De esta manera el algoritmo de inicialización de pesos propuesto tendría una complejidad $\mathcal{O}(n)$.

Notemos que este coste es bastante menor al de realizar \textit{backpropagation} y que además éste debería de realizarse repetidamente pare mejorar el error considerablemente,  mientras que el nuestro se realiza tan solo una vez. 


\subsection{Observaciones }

Observemos que nuestro algoritmo, para un mismo conjunto de entrenamiento  es capaz de producir infinitas soluciones 
diferentes ya que 
existen dos variables libres $M$ y $p$. 

La  constante $M$ depende de la función de activación seleccionada y recordemos que  debe escogerse para que satisfaga la condición \refeq{eq:method_inicializar_M}. 

Presentamos a continuación en la tabla \ref{table:M-activation-function} el valor $M$ para algunas funciones de activación. 

\begin{table}[H]
    \centering
    \begin{tabular}{|c|c|}
    \hline
        Función de activación  & Valor mínimo de $M$ \\ \hline
        Función rampa & 1 \\ \hline
        \textit{Cosine Squasher} & $\frac{\pi}{2}$ \\ \hline
        Función indicadora 0 & 0 \\ \hline
    \end{tabular}
    \caption{Valor mínimo del parámetro $M$ en algoritmo de inicialización de redes neuronales según la función de activación seleccionada.}
    \label{table:M-activation-function}
\end{table}

\subsection{Generalización del método para funciones de activación }

A priori si  la función de activación no cumple las propiedades 
demandadas no podría ser utilizada en el algoritmo.  Sin embargo, es 
posible ver que se puede generalizar para otras
funciones de activación menos restrictivas como las definidas en \ref{table:funciones-de-activation}. 

\begin{itemize}
    \item Por el teorema \ref{teo:eficacia-funciones-activation} también será valido para \textbf{funciones de activación que sean afines a una que satisface \refeq{eq:method_inicializar_M}}. El proceso constructivo consistiría en: (1) hacer que la red aprenda con la función que cumple los requisitos \refeq{eq:method_inicializar_M}. (2) Los pesos obtenidos transformarlos con la misma técnica que se aplica en la demostración del teorema \ref{teo:eficacia-funciones-activation}. 
    
    \item  \textbf{Funciones de activación asintóticas a 0 o 1}, esto es funciones que satisfacen que: 
    \begin{enumerate}
        \item $\lim _{x \rightarrow \infty} \psi(x) = 1
        $.
        \item $\lim _{x \rightarrow -\infty} \psi(x) = 0$.
        \item Que cumpla que $\psi(x) \neq 1$ para todo  $x\in \R$  o $\psi(x) \neq 0$ para todo $x\in \R$ .
    \end{enumerate}
    Bastará con tomar un $M$ lo suficientemente grande. 
    \item Para funciones del tipo anterior, pero asintóticas a $a,b \in \R$, con alguno de los extremos $a,b$ distintos de $0$ y $1$, bastará con realizar una transformación afín de $f$ cumpliendo que $f(a)= 0$ y $f(b)= 1$ y aplicar el teorema \ref{teo:eficacia-funciones-activation}. 
\end{itemize}

De esta manera se pueden ampliar las funciones de activación válidas, añadiendo algunas como: 

\begin{table}[H]
    \centering
    \begin{tabular}{|c|c|}
    \hline
        Función de activación  & Valor mínimo de $M$ \\ \hline
        Función rampa & 1 \\ \hline
        \textit{Cosine Squasher} & $\frac{\pi}{2}$ \\ \hline
        Función indicadora 0 & 0 \\ \hline
        Función de activación  & Valor mínimo de $M$ \\ \hline
        Sigmoidea  & con $M=10$ el error menor de $10^{-5}$\\ \hline
        Tangente hiperbólica  &  con $M=7$ el error menor de $10^{-5}$\\ \hline
        \textit{Hardtanh} & 1 \\ \hline
    \end{tabular}
    \caption{Valor mínimo del parámetro $M$ en algoritmo de inicialización de redes neuronales según la función de activación seleccionada (con más resultados).}
    \label{table:M-activation-function-2}
\end{table}

%\textcolor{red}{TODO issue 107: Hacer experimentaciones }
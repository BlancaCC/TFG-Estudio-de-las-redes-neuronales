%%%%%%%%%%%%%%%%%%%%%%%%%%%%%%%%%%%%%%%%%%%%
% Implementación del algoritmo
%%%%%%%%%%%%%%%%%%%%%%%%%%%%%%%%%%%%%%%%%%%%

\section{Implementación}
\label{ch07:Implementar} 

Los requisitos mínimos necesarios para una implementación adecuada son los siguientes
\begin{itemize}
    \item Implementación de red neuronal y tipos de constructores.
    \item Implementación del algoritmo. 
\end{itemize}

\subsection*{Implementación de redes neuronal}

El modelo  a implementar es el presentado en el algoritmo \ref{algoritmo:estructura-de-una-red-neuronal}. En virtud del \textit{composite type} de Julia \footnotetext{ Véase la \href{https://docs.julialang.org/en/v1/manual/types/}{documentación oficial}} la forma más simple y eficiente de 
declarar una red neuronal es como una tipo de dato \textit{red neuronal} cuyos atributos sean las matrices que definen el modelo. 

En vista a la optimización en evaluación y 
entrenamiento más eficiente las matrices 
$A, S$ se han escrito en una sola permitiendo así una evaluación más compacta. 

\subsection*{Implementación del algoritmo de \textit{Forward propagation}}  
La evaluación de una red neuronal se realizará por 
medio de una función que recibe como parámetros un 
tipo de dato \textit{red neuronal}. 

\subsection*{Implementación del algoritmo}

Con el objetivo de optimizar: 
- Cuestión de tipos composite dispatch. 
- Necesidad de una lista ordenada: comentar conjunto y tipos
y argucia del diccionario. 
 

%%%%%%%%%%%%%%%%%%%%%%%%%%%%%%%%%%%%%%%%%%%%%%%%%%%%%%%%%%%%%%%%%%%%%%%%%%%
%% Introducción para comenzar la teoría hablando de la filosofía que se va a seguir 
%%
%%%%%%%%%%%%%%%%%%%%%%%%%%%%%%%%%%%%%%%%%%%%%%%%%%%%%%%%%%%%%%%%%%%%%

\part{Teoría subyacente}

No es usual en un manual que trate sobre redes neuronales encontrarse en su interior con un 
capítulo sobre teoría de la aproximación, pero tampoco es nuestra intención
hacer de este documento una recopilación de todo lo usual, sino todo lo contrario.

Existe en la actualidad un desequilibrio entre resultados empíricos y teóricos de redes neuronales llegando incluso a contradicción (como se comenta en la introducción del capítulo \ref{chapter:Introduction-neuronal-networks}), será por tanto
nuestro primer objetivo conseguir una revisión y purga de cualquier artificio existente sobre redes neuronales carente de fundamento matemático. 

El fin de esto no es más que construir una teoría sólida que de cabida a optimizaciones de 
fundamento teórico.  Para ello nuestro \textit{modus operandi} será el siguiente: 
Se describirá el conjunto y características de problemas que pretendemos abarcar  en el capítulo \ref{chapter:Introduction-neuronal-networks}. 
Se comentará las limitaciones e inconvenientes que presenta un enfoque clásico 
basado en teoría de la aproximación en el capítulo \ref{chapter:teoria-aproximar}.
A continuación en el capítulo \ref{chapter4:redes-neuronales-aproximador-universal}, 
presentarán las redes neuronales como un modelo eficiente.
 
Al final del mismo capítulo se introduce la definición que hemos determinado por conveniente de red neuronal y que es producto de los capítulos 
\ref{chapter:teoria-aproximar} y \ref{chapter4:redes-neuronales-aproximador-universal}.
 
Tras todo el fundamento teórico en \ref{chapter:construir-redes-neuronales} se explicitará el diseño de la red neuronal modelizada así como los algoritmo de evaluación y aprendizaje.

En los capítulos \ref{funciones-activacion-democraticas-mas-demoscraticas} y \ref{section:inicializar_pesos} se explican además otros resultados para optimizar el coste computacional.
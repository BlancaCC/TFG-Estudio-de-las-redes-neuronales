%%%%%%%%%%%%%%%%%%%%%%%%%%%%%%%%%%%%%%%%%%%%%%%%%%%%%%%%%%%%%%%%%%%%%%%%%%%
%% Introducción para comenzar la teoría hablando de la filosofía que se va a seguir 
%%
%%%%%%%%%%%%%%%%%%%%%%%%%%%%%%%%%%%%%%%%%%%%%%%%%%%%%%%%%%%%%%%%%%%%%

\chapter*{Introducción}

\section*{Origen}
En nuestros días, el aprendizaje automático es un campo en
continua expansión. Desde procesamiento de imágenes hasta predicciones de
acciones en bolsa, las redes neuronales se van implantando en todos los
campos del conocimiento. \\


A pesar del pragmatismo actual que prevalece en las redes 
neuronales, el concepto de neurona artificial nace en el primer tercio del siglo XX con el artículo  \textit{ Logical calculus of the ideas immanent in nervous activity} \cite{primer-articulo}, como
intento de modelar el pensamiento humano en forma de proceso lógico. Esta
primera etapa de investigación culmina con la invención del perceptrón en
1958 por Frank Rosenblatt con el artículo \textit{The perceptron: a probabilistic model for information storage and organization in the brain} \cite{rosenblatt1958perceptron}. Es importante notar que todas estas investigaciones
tenían un objetivo más allá de la resolución de problemas complejos utilizando
máquinas u ordenadores: comprender el proceso de aprendizaje y cognición del
ser humano.\\

Estos primeros modelos tenían sus carencias y reservas por parte
de la comunidad científica y cuando en 1969 la publicación de la demostración de que
los modelos eran incapaces de resolver problemas lógicos simples (véase \cite{minsky69perceptrons}) hizo que el
campo casi muriera. No fue hasta 1986 con el descubrimiento de un modelo en más complejo,
que se superarían estas limitaciones y que darían lugar a las actuales redes neuronales
\cite{10.5555/104279.104293}.\\

El modelos de 1986, no obstante, usaba un método que sus autores eran incapaces de
relacionar con el mundo de la cognición, suponiendo con esto el 
inicio de la separación de las redes neuronales con el campo de la psicología
y la neurociencia y el inicio del enfoque actual de resolución de problemas.
Estos nuevos descubrimientos, sumados a una mejora en las capacidades
computacionales de los ordenadores y que en 1989 se demostrara formalmente su eficacia \cite{HORNIK1989359} provocaron un auge en el interés acerca de
las redes, que perdura hasta hoy.\\


\section*{Descripción del problema,  motivación y objetivos} 
Si bien las redes neuronales abandonaron ya la psicología 
y a pesar de su uso extensivo en la actualidad, es un área incipiente de la 
matemática donde los grandes teoremas aún están por descubrir. 
Ante tal desequilibrio entre resultados empíricos y teóricos,
 será 
nuestro primer objetivo conseguir una revisión y purga de cualquier artificio
 existente sobre redes neuronales carente de fundamento matemático. 
Siendo el fin de esto construir una teoría sólida que de cabida a 
optimizaciones de fundamento teórico.

\section*{Técnicas, áreas matemáticas y fuentes utilizadas}  

El artículo principal que membrana el proyecto es el artículo 
\textit{Multilayer Feedforward Networks are Universal Approximators} \cite{HORNIK1989359}
 escrito por Kurt Hornik, Maxwell Stinchcombe y Halber White. Como pilar a la 
 sección de teoría de aproximación se ha utilizad el manual \textit{The Elements of Real Analysis} \cite{the-elements-of-real-analysis} y finalmente los manuales de referencia seguidos sobre el aprendizaje automático han sido: 
 \textit{Learning From Data} \cite{MostafaLearningFromData} y \textit{Pattern Recognition and Machine Learning} \cite{BishopPaterRecognition}.

 Sobre las técnicas y áreas empleadas pueden consultarse con detalle en la secciones \ref{ch01:Herramientas} y \ref{ch01:asignaturas}.


% !TeX root = ../../tfg.tex
% !TeX encoding = utf8
%
%***************************************************************
% Contenido del artículo 5: Colorarios LP
%***************************************************************
\section{Generalización a espacios $L_p$}  
\label{ch04:espacios-Lp}
Hasta ahora habíamos considerado el espacio de funciones continuas 
$\fC$ 
como subespacio dentro del espacio de funciones medibles $\fM$. 
Sin embargo, ser continua es una hipótesis muy estricta ya que existe una amplia gama de subespacios que contienen al de 
las funciones continuas y están contenidos en el de funciones medibles. 
Es por ello que vamos a realizar una generalización de los teoremas
para espacios $L_p$. De manera intuitiva estos espacios nos van a 
permitir considerar funciones que no necesariamente sean continuas
y que incluso no están acotadas. 

 % Nota intuitiva sobre la definición  LP
 \normalmarginpar
 \marginpar{\maginLetterSize\raggedright
 \iconoAclaraciones \textcolor{dark_green}{ 
     \textbf{Idea intuitiva de espacio $L_p$}
 }
 }
 \marginpar{\maginLetterSize
 Con esta definición lo que se trata es de considerar funciones 
 que tomen un valor real en la mayoría de sus puntos.
 Ya que la integral no varía su valor si se aplican cambios puntuales. 
 }
\begin{definicion}[Espacios Lp]
    Se llama espacio $L_p(\R^d, \mu)$ o simplemente $L_p$ al conjunto 
    de funciones $f \in \fM$ tales que 
    \begin{equation}
        \int |f(x)|^p d\mu < \infty. 
    \end{equation}  
Se define la norma de $L_p$ como 
\begin{equation}
    \| f\|_p 
    =
    \left(\int |f(x)|^p d\mu \right)^\frac{1}{p}
\end{equation}
y la distancia asociada al espacio $L_p$ se define como 
\begin{equation}
    \rho_p(f,g) = \| f-g\|_p.
\end{equation}
\end{definicion}


% Corolario 2.2
\begin{corolario}\label{corolario:2_2_rrnn}
    Si existe un subconjunto compacto $K$ en $\R^d$ de medida
    $\mu(K) =1$ entonces $\rrnn$ es $\dlp$-denso en $L_p(\R^d, \mu)$
    para cualquier $p \in [1,\infty)$, independientemente de 
    $\psi$, $d$ o $\mu$.
\end{corolario}

% Nota intuitiva sobre el corolario 2.2
\marginpar{\maginLetterSize\raggedright
    \iconoAclaraciones \textcolor{dark_green}{ 
        \textbf{Idea intuitiva corolario \ref{corolario:2_2_rrnn}}
    }
}
\marginpar{\maginLetterSize
Se prueba que las redes neuronales son capaces de aproximar 
cualquier función de la clase recién introducida.
}


\begin{proof}
    Se quiere probar que para cualquier $g \in L_p$ y 
    $\varepsilon >0$ existe $f \in \rrnn$ tal que 
    \begin{equation}
        \dlp(f,g) <\varepsilon.
    \end{equation}   
    
    Por pertenecer $g$ a $L_p$ existe una constante $M$ real positiva
    lo suficientemente grande 
    tal que si definimos la función $h =g 1_{|g|<M}$ esta satisface 
    que
    \begin{equation}\label{eq:corolario_2_2:h_compacto}
        \dlp(g,h) < \frac{\varepsilon}{3}.
    \end{equation}
    
    Además como $h$ es una función acotada de $L_p$, podemos encontrar
    una función $s$ continua que es límite de una sucesión de
    funciones simples 
    ( pag 241-242,  teoremas 55C y 55D \cite{nla.cat-vn1819421})
    y la cual cumple que 

    \begin{equation}\label{eq:corolario_2_2:s_continua}
        \dlp(h,s) < \frac{\varepsilon}{3}.
    \end{equation}

    Por el teorema \ref{teo:2_4_rrnn_densas_M}, al estar en un compacto $K$ y por ser $\rrnn$ uniformemente
    denso en compactos hay una $f \in \rrnn$ la cual cumple que
    \begin{equation}
        \sup_{x \in K} |f(x) -s(x)|^p 
        <
         \left( \frac{\varepsilon}{3}\right) ^p.
    \end{equation}
    
    Y por hipótesis $\mu(K) =1$ y definición de la distancia $\dlp$ 
    se tiene la siguiente desigualdad: 

    \begin{equation} \label{eq:corolario_2_2:cota_rrnn}
        \dlp(f,s) = 
        \left(\int |f(x) - s(x)|^p d\mu \right)^\frac{1}{p}
        \leq 
        \left(\int  \left( \frac{\varepsilon}{3}\right) ^p d\mu \right)^\frac{1}{p}
        = \left( \mu(K)  \left(\frac{\varepsilon}{3} \right)^p\right) ^\frac{1}{p}
        = \frac{\varepsilon}{3}.
    \end{equation}

    Gracias a la desigualdad triangular y las desigualdades
    (\refeq{eq:corolario_2_2:cota_rrnn}),
    (\refeq{eq:corolario_2_2:h_compacto}) y 
    (\refeq{eq:corolario_2_2:s_continua})
    tenemos
    \begin{equation}
        \dlp(f,g) 
        \leq
            \dlp(f,s)
            +\dlp(s,h)
            + \dlp(h,g)
        < 
        \frac{\varepsilon}{3} + \frac{\varepsilon}{3} + \frac{\varepsilon}{3}
        = \varepsilon.
    \end{equation}
Probando con ello lo buscado. 
\end{proof}  

% Corolario 2.3
\begin{corolario}\label{corolario:2_3_medida_probabilidad}
    Si $\mu$ es una medida de probabilidad en $[0,1]^d$
    entonces 
    $\rrnn$ es $\dlp$-denso en 
    $L_p([0,1]^d, \mu)$ para todo $p \in [1, \infty)$,
    independientemente de $\psi, d, \mu$. 
\end{corolario}

% Nota intuitiva sobre el corolario 2.3
\marginpar{\maginLetterSize\raggedright
    \iconoAclaraciones \textcolor{dark_green}{ 
        \textbf{Idea intuitiva corolario 
        \ref{corolario:2_3_medida_probabilidad}}
    }
}
\marginpar{\maginLetterSize
    A diferencia que con el corolario \ref{corolario:2_2_rrnn},
aquí se afirma que las redes neuronales son capaces de aproximar 
cualquier función de la clase recién introducida cuyo dominio parta de $[0,1]^d$, es decir \textbf{se está concretando los valores de entrada que pueden tomar las funciones}.
}


\begin{proof}
    Es consecuencia directa del corolario previo \ref{corolario:2_2_rrnn}
    donde para este caso particular $K = [0,1]^d$ un compacto
    de $\R^d$
    que cumple que $\mu(K) = 1.$
\end{proof}

%Corolario 2.4 
\begin{corolario} \label{corolario:2_4_conjunto_finito}
    Sea $\mu$ una medida, que para
    un conjunto finito de puntos $O$ cumple que $\mu(O)=1$, 
    entonces, para cualquier función medible $g \in \fM$
    y sea cual sea $\varepsilon >0$ 
    existe $f \in \rrnn$ la cual cumple que 
    \begin{equation}
        \mu\{ 
            x:
            |f(x) - g(x)| 
            < \varepsilon
        \}
        = 1.
    \end{equation}

\end{corolario}
\begin{proof}
    Por el teorema \ref{teo:2_4_rrnn_densas_M} existe 
    $f \in \rrnn$ tal que para cualquier 
    $\varepsilon_1, \varepsilon_2 >0$ se cumple que 
    $\mu \{x: |f(x) - g(x)| > \varepsilon_1\} < \varepsilon_2.$
    Sea $O$ el conjunto de puntos tal que $\mu(O) = 1.$
    Por ser finito $O$ podemos encontrar
    \begin{equation} \label{eq:2_4:definición_epsilon}
        \delta = \min_{x \in O} \{ 
            \mu(x) : \mu(x)>0
        \}. 
    \end{equation}

    Sin pérdida de generalidad tomamos $\varepsilon < \delta$ y entonces
    para  que la $f$ fijada cumpla que
    \begin{equation}
        \dist(f,g) = \varepsilon
    \end{equation}
    debe de cumplirse que 
    \begin{equation}
        \dist(f,g) =  \inf 
        \{
           \varepsilon_1 > 0:
           \mu\{ 
            x:
            |f(x) - g(x)| 
            > \varepsilon_1
        \}
        < \varepsilon_1
        \} 
        = \varepsilon,
    \end{equation}
    pero por cómo tomamos $\varepsilon$ en (\refeq{eq:2_4:definición_epsilon}) se
    tiene que $\mu\{ 
        x:
        |f(x) - g(x)| 
        > \varepsilon
    \} = 0.$
    Por lo que acabamos de probar, como queríamos, que 
    \begin{equation}
        \mu\{ 
            x:
            |f(x) - g(x)| 
            < \varepsilon
        \}
        = 1.
    \end{equation}
\end{proof}

Nótese que con este corolario lo que se está indicando es que dado
un conjunto finito y sus respectivas imágenes, se puede encontrar una red neuronal que para tales puntos \textit{devuelva} el valor exacto. 
%Nota aclarativa sobre la relevancia del corolario
\marginpar{\maginLetterSize
    \iconoAclaraciones \textcolor{dark_green}{     
        \textbf{
            Relevancia práctica del corolario 
            \ref{corolario:2_5_función_Booleana}
        }
    }
     
    Este resultado nos permite \textbf{aproximar funciones que actúen como clasificadores discretos}. Por ejemplo sea $g$ como una función que dada una imagen $x$ indica si hay un perro haciendo valer $g(x)=1$ y en caso contrario $g(x)=0$.
      
}

%Nota aclarativa sobre la implementación del corolario
\marginpar{\maginLetterSize
    \textcolor{red}{    
        \textbf{
            Observación sobre la implementación
            del corolario
            \ref{corolario:2_5_función_Booleana}
        }
    }
    
    El resultado nos   indica que podemos obtener una red neuronal $h$ que aproxime tal clasificador, 
    pero \textbf{tal red neuronal no necesariamente tomará valores discretos}, es decir,
     pudiera darse el caso en que las imágenes 
     a un rango, por ejemplo: 
     $$h( \{ x : g(x)=0 \})  \subset [-0.2,0.3]$$ 
     y que 
     $$h(\{ x : g(x)=1 \})  \subset [0.9,1.2],$$ 
     por lo que se pone de manifiesto en este
      resultado, 
     que en caso de requerirse
      de una salida completamente
     discreta debería de componerse 
     con otra función $\theta$ 
     tal que 
     $$\theta \circ h(\{ x : g(x)=0 \})=0$$ y 
     $$\theta \circ h(\{ x : g(x)=1 \})=1$$.
    
}

\begin{definicion}[Función Booleana]
    Decimos que una función es Booleana si su dominio es 
    $\{0,1\}^d$  para algún $d \in \N \setminus \{0\}$
    y su codominio es $\{0,1\}$. Es decir,
    \begin{equation}
      f:\{0,1\}^d \longrightarrow \{0,1\}.
    \end{equation}
\end{definicion}

Ejemplos conocidos son la función 
$or: \{0,1\}^d \longrightarrow \{0,1\}$  que vale 
uno si alguno de su entrada es uno y la función 
$and: \{0,1\}^d \longrightarrow \{0,1\}$
que se define como $and(x_1, \ldots, x_d) = \prod_{i=1}^d x_i.$

% Corolario 2.5  
\begin{corolario}\label{corolario:2_5_función_Booleana} 
    Para cada función Booleana 
    $g: \{0,1\}^d \longrightarrow \{0,1\}$
     y 
    cada $\varepsilon >0$ existe una red neuronal
    $f \in \rrnn$ tal que 
    \begin{equation}
        \max_{x \in \{ 0,1\}^d} |g(x) - f(x)|
        < \varepsilon.
    \end{equation}
\end{corolario}



\begin{proof}
    Se define la función $\mu : \R^d \longrightarrow [0,1]$ de forma que 
    \begin{equation}
        \mu(x) = 
      \left \{
    \begin{aligned}
      \frac{1}{2^d} \quad &\text{ si } x \in \{0,1\}^d \\
      0 \quad & \text{ si } x \notin \{0,1\}^d 
    \end{aligned}
  \right .
    \end{equation}

    Se tiene que $\mu$ es una medida ya que cumple que 
    \begin{enumerate}
        \item Hipótesis de acotación: $0 \leq \mu(A) \leq 1$ para $A \in \mathcal{P}(\R^d).$
        \item La probabilidad del vacío es nula y la del  total es la unidad. 
        \item La probabilidad de la unión es la suma de la probabilidades. 
        \begin{equation}
            P\left(
                \cup_{i=1}^n A_i
            \right)
            = \sum_{i=1}^n P(A_i).
            \quad
            \forall A_i \in  \mathcal{P}(\R^d).
        \end{equation}
    \end{enumerate}  

    Como la cardinalidad de $\{0,1\}^d$ es $2^d$
    podemos aplicar el corolario \ref{corolario:2_4_conjunto_finito}
    y entonces sabemos que  existe $f\in \rrnn$ tal que 
    \begin{equation}
            \mu\{ 
                x:
                |f(x) - g(x)| 
                < \varepsilon
            \}
            = 1,
    \end{equation} 
    es decir que 
    \begin{equation}
        \max_{x \in \{ 0,1\}^d} |g(x) - f(x)|
        < \varepsilon
    \end{equation}
    como queríamos probar. 
\end{proof}


% Lemas propios  previos al teorema 2.5
\begin{lema} \label{lema:propio_1_antes_teorema_2_5}
    Si una función de activación  $\psi$ alcanza el cero y el uno, esto es 
    si existen dos constantes reales $M_1, M_2$ 
    tales que 
    \begin{equation}
        \psi(M_1) = 0 \text{ y } \psi(M_2)=1
    \end{equation}
    entonces existe una constante real positiva $M$ tal que 
    \begin{equation}
        \psi(-M) = 1- \psi(M) = 0.
    \end{equation}
\end{lema}
\begin{proof}
Sea $M = \max \{|M_1|,|M_2|\}$ y por ser $\psi$ una función de activación sabemos que
es no decreciente y que su imagen pertenece al intervalo $[0,1].$

Por tanto
\begin{align}
      0 &\leq \psi(-M) \leq \psi(M_1) = 0 \quad \text{ luego } \quad \psi(-M) = 0, \\
      1 &\geq \psi(M) \geq \psi(M_2) = 1 \quad \text{ luego } \quad\psi(M) = 1.
\end{align}

Gracias a estas desigualdades es fácil ver que 
\begin{equation}
    \psi(-M) = 1 - \psi(M) = 0
\end{equation}
como queríamos probar. 
\end{proof}   

Es interesante percatarse de que de no exigirse la hipótesis de 
que $\psi$ alcanza el cero y el uno no puede 
asegurarse la igualdad demostrada. Pongamos como ejemplo la siguiente función de activación

\begin{equation}
    \psi(x)= \left\{ \begin{array}{lcc}
        0 &   si  & x \leq 0 \\
        \frac{| x |}{1+ | x |}&  si & 0< x. 
        \end{array}
    \right. 
\end{equation}


% Otro lema propio antes de probar el teorema 2.5
\begin{lema}\label{lema:previo_propio_2_al_teorema_2_5}
    Dado un conjunto finito de vectores $\Lambda \subset \R^d$ con 
    $d$ natural positivo. 
    Existe un vector $p \in \R^d$ que satisface que 
    para cualesquiera $x,y \in \Lambda$ diferentes 
    \begin{equation}
        p \cdot(x-y) \neq 0.
    \end{equation}
\end{lema}
% Nota intuitiva sobre el Teorema 2.5
\marginpar{\maginLetterSize\raggedright
    \iconoAclaraciones \textcolor{dark_green}{ 
        \textbf{Idea clave teorema
        \ref{teorema:2_5_entrenamiento_redes_neuronales}}
    }
}
\marginpar{\maginLetterSize
   Podemos conseguir una red neuronal que \textit{valga} lo que queramos en un conjunto finito de puntos.
}
% Prueba lema propio antes de probar el teorema 2.5
\begin{proof}
    Si $n$ es el cardinal de $\Lambda$,
    consideramos el conjunto $U$ definido como la unión de 
    $n (n-1)$ hiperplanos de $\R^d$

    \begin{equation}
        U = \bigcup_{ 
            \substack{
                x,y \in \Lambda \\
                x \neq y
            }
        }
        \{ 
            p \in \R^d: p \cdot (x-y) = 0
        \}.
    \end{equation}

    Puesto que $\R^d$ no puede ser expresado como unión finita de hiperplanos,
    $U \subsetneq \R^d$ y por tanto existirá $p \in \R^d \setminus U$ tal que 
    \begin{equation}
        p \cdot(x-y) \neq 0 
        \text{ para cualesquiera }
         x,y \in \Lambda \text{ diferentes, }
    \end{equation}
    como queríamos probar. 
\end{proof}

% Nota intuitiva sobre la demostración del Teorema 2.5
\setlength{\marginparwidth}{\smallMarginSize}
\marginpar{\maginLetterSize\raggedright
    \iconoAclaraciones \textcolor{dark_green}{ 
        \textbf{Idea de la demostración del teorema
        \ref{teorema:2_5_entrenamiento_redes_neuronales}}
    }
}
\marginpar{\maginLetterSize
    {
  Es interesante reparar en que la demostración se basa
  en añadir una neurona por cada punto que queramos que tome
  un valor concreto, esa neurona se activará (es decir, no será nula) cuando la entrada $x$ \textit{sea mayor} que el valor que la activa $x_i$ y vale la diferencia con el valor anterior $x_{i-1}$, es decir $g(x_{i}) - g(x_{i-1})$, como el nodo $x_{i-1}$
  también se activará por ser menor, el término $g(x_{i-1})$ se suma a la salida de la red y así como una serie telescópica al final solo resultará el valor $g(x_i)$.
    }
}
\setlength{\marginparwidth}{\bigMarginSize}
% fin de la nota


% Teorema 2.5  
\begin{teorema}[Sobre el entrenamiento práctico de redes neuronales]
    \label{teorema:2_5_entrenamiento_redes_neuronales}
    Sea $ \Lambda = \{x_1, \ldots, x_n\}$ un conjunto de puntos distintos de 
    $\R^d$ y sea 
    $g: \R^d \longrightarrow \R$ una función arbitraria. 
    Si $\psi$ alcanza el cero y el uno, 
    entonces
    existe una red neuronal $f \in \rrnn$ con $n$
    neuronas ocultas tal que 
    \begin{equation}
        f(x_i) = g(x_i) \text{ para todo } i \in \{1, \ldots, n \}.
    \end{equation}
\end{teorema}

\begin{proof}
Con el fin de facilitar la comprensión dividiremos la demostración en dos casos, 
primero uno particular, cuando $d=1$ y después el caso general.

\textbf{Caso primero}

Suponemos que $\{x_1, \ldots, x_n\} \subset \R$ y tras renombrar 
podemos suponer que 
\begin{equation}
    x_1 < x_2 < \cdots < x_n. 
\end{equation}

Por alcanzar la función de activación $\psi$ el cero y el uno, 
gracias al lema  \ref{lema:propio_1_antes_teorema_2_5} existe una constante $M$ tal que $\psi(-M) = 1-\psi(M) = 0.$

Definiremos de manera recursiva la red neuronal buscada $f_n$.

% Nota nueva hipótesis de optimización del Teorema 2.5

\marginpar{\maginLetterSize\raggedright
     \iconoProfundizar \textcolor{blue}{
        \textbf{Nueva hipótesis de optimización}
    }
}
\marginpar{\maginLetterSize
    \textbf{
    El teorema 
        \ref{teorema:2_5_entrenamiento_redes_neuronales}
        nos brinda una heurística de inicialización de los pesos
        de una red neuronal.
    }
    La cual se tratará en la sección \ref{section:inicializar_pesos}.
\setlength{\marginparwidth}{\bigMarginSize}
 
}

\begin{itemize}
    \item Red neuronal $f_1$. 

Sea $A_1$ la función afín constante $A_1 = M.$
Fijamos $\beta_1 = g(x_1)$. 
De esta manera la red neuronal $f_1$ queda
definida como $f_1(x) = \beta_1 \psi(A_1(x)).$

\item Red neuronal $f_k$ con $1 < k \leq n$. 

Se define $A_{k}$ como la única función afín que cumple que 
\begin{equation}
    A_k(x_{k-1}) = -M \quad \text{y} \quad  A_{k}(x_k)= M.
\end{equation}
Fijamos $\beta_k = g(x_k) - g(x_{k-1})$. 
La red neuronal $f_k$ se calcula como 
\begin{equation}
    f_k(x) 
    = 
    \sum_{j=1}^k \beta_j \psi(A_j(x))
     = 
    (g(x_k)-g(x_{k-1})) \psi(A_k(x)) + f_{k-1}(x) .  
\end{equation}
Observemos que así construida se tiene que para cualquier
 $y > x_k$ la evaluación con la red neuronal resulta $f_k(y) = g(x_k).$
\end{itemize}

Veamos por inducción sobre $n$ que así definida para cualquier $1 \leq i \leq n$ se tiene que     
$f_n(x_i) = g(x_i)$. 
Además de que tiene un total de $n$ neuronas ocultas. 


\begin{itemize}
    \item Caso base, $n=1$. 
    \begin{equation}
        f_1(x_1)= \beta_1 \psi(A_1(x_1)) = g(x_1)\psi(M) = g(x_1).
    \end{equation}
    \item Supuesto que es cierto para $n-1$ veamos que lo es para $n$.      
    Evaluación de $x_n$
    \begin{align}
        f_n(x_n) 
        &= 
        (g(x_n) - g(x_{n-1}))\psi(A_n(x_n)) + f_{n-1}(x_n)
        \\
        & = (g(x_n) - g(x_{n-1}))\psi(M) + g(x_{n-1}) 
        \\
        & = (g(x_n) - g(x_{n-1})) + g(x_{n-1}) 
        \\
        & = g(x_n).
    \end{align}

Evaluación de $x_i$ con $1 \leq i < n$. 

Usando que $0 \leq \psi(A_n(x_i)) < \psi(A_n(-M)) = 0$ y la hipótesis de inducción se tiene que 
\begin{align}
    f_n(x_i) 
        &= 
        (g(x_n) - g(x_{n-1}))\psi(A_n(x_i)) + f_{n-1}(x_i)
        \\
        & = 0 + g(x_{i}) 
        \\
        &= g(x_i).
\end{align}
\end{itemize}

Acabamos de probar por inducción  que
$f_n(x) = g(x)$ para cualquier $x \in \Lambda$, lo que termina la demostración.

\textbf{Caso segundo}  

Se tiene para este caso que $\Lambda \subset \R^d$ con $d >1$. 
Seleccionamos $p \in \R^d$ cumpliendo que 
para cualesquiera $x,y \in \Lambda$ distintos 
$p(x-y) \neq 0$ (es posible de 
encontrar por el lema \ref{lema:previo_propio_2_al_teorema_2_5}).
Gracias a esta condición cada producto $p \cdot x$ con $x \in \Lambda$ es distinto y podemos establecer con ello una relación de orden, que tras 
renombrar los elementos de $\Lambda$ queda
\begin{equation}
    p \cdot x_1 < p \cdot x_2 < \cdots < p \cdot x_n,
\end{equation}
y como procedimos en el caso primero, definimos de manera
 recursiva la red neuronal $f_n$ buscada:

 \begin{itemize}
 \item Red neuronal $f_1$. 

Sea $B_1$ 
la función afín de $\R$ a $\R$ constante $B_1 = M.$
Fijamos $\beta_1 = g(x_1)$. 
De esta manera la red neuronal $f_1$ 
definida como $f_1(x) = \beta_1 \psi(B_1(p \cdot x)).$

\item Red neuronal $f_k$ con $1 < k \leq n$. 

Se define $B_{k}$ como la única función afín de $\R$ en $\R$ que cumple que 
\begin{equation}
    B_k(p \cdot x_{k-1}) = -M 
    \quad \text{y} \quad 
     B_{k}(p \cdot x_k)= M.
\end{equation}
Podemos definir entonces $A \in \afines$ por 
$A_k(x)=B_k(p \cdot x).$
Fijamos  también
$$\beta_k = g(x_k) - g(x_{k-1}).$$ 
La red neuronal $f_k$ se calcula como 
\begin{align}
    f_k(x) 
    = &
    \sum_{j=1}^k \beta_j \psi(B_j(p \cdot x))
     = 
    (g(x_k)-g(x_{k-1})) \psi(B_k(p \cdot x)) + f_{k-1}(x) \\
    \\
    & = 
    \sum_{j=1}^k \beta_j \psi(A_j(x))
    = 
   (g(x_k)-g(x_{k-1})) \psi(A_k(x)) + f_{k-1}(x).  
\end{align}
\end{itemize}

Así definida, la prueba por inducción es idéntica a la del caso primero y hemos encontrado por tanto la red neuronal $f_n$ buscada.
\end{proof}

\subsection{Reflexión sobre el número de neuronas} \label{subsection:reflexión_sobre_número_de_neuronas}

Ante la pregunta natural de 
\textit{¿cuántas neuronas son necesarias?} el 
recién probado teorema nos responde que 
si estamos entrenando con $n$ datos, con $n$ neuronas es suficiente para volver a reproducir esos datos. Pero esto carece de sentido a nivel práctico por los siguientes motivos. 

\subsubsection*{Naturaleza de los datos}  
Recordemos que el problema al que nos enfrentamos es el siguiente:
queremos ser capaces de predecir cierto fenómeno regido por $g$ una \textit{función} desconocida. 
Para ello tenemos un \textit{conjunto de muestras} 
compuestas por pares $(x, g(x))$, es decir, ante la situación $x$
hemos observado que el fenómeno se comporta como $g(x)$. Todo esta 
situación experimental puede producir incoherencias teóricas, como por ejemplo que se tengan dos muestras $(x_1, y_1), (x_2, y_2)$ 
tales que $x_1 = x_2$ pero $y_1 \neq y_2$ o que la medición contenga errores, es decir $(x, g(x)+\delta)$. 

Se podría paliar esta situación con un
 preprocesado de los datos. 

\subsubsection*{ Naturaleza de la regla subyacente}  

Supongamos que los datos de entrenamiento son perfectos. Existen infinitas aplicaciones 
que evalúan de la misma manera un conjunto finito de puntos. 
¿Cuáles tomar?

Podríamos, siguiendo el principio de economía de Ockham optar
por modelos que reduzcan el número de neuronas, pero entonces la 
pregunta sería  ¿qué número mínimo de neuronas serían necesarios para representar el modelo? 
Una solución sería tomar un número de neuronas menor que el tamaño del conjunto de 
entrenamiento y utilizar esa \textit{redundancia} de datos para 
\textit{afinar}. 

\subsubsection*{Coste computacional inasumible}  
Supongamos que los datos son idílicos, el modelo se conoce y se establece un tamaño de datos de entrenamiento suficiente y
un número de neuronas acorde  ¿podríamos asumir el coste computacional?


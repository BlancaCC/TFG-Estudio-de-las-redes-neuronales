%%%%%%%%%%%%%%%%%%%%%%%%%%%%%%%
% Observación entre la diferencia de Q y R
%%%%%%%%%%%%%%%%%%%%%%%%%%%%
% Observación sobre el dominio discreto donde se está trabajando 
% y que refleja una posible fuente de mejora de las redes neuronales
% ISSUE #88
\section{Consideración sobre la capacidad de cálculo}

Suele pasar peligrosamente desapercibido que el teorema  \ref{corolario:2_6} recién probado asegura
que se podrá encontrar una red neuronal en $\rrnnmc$
que aproxime todo lo bien que queramos la función ideal; sin embargo, destaquemos que los parámetros que caracterizan a la red neuronal encontrada son reales. Es decir, 
si nuestra red neuronal toma valores irracionales difícilmente será computable en un ordenador por su infinitud. 

Esto nos impone una nueva restricción en el espacio de búsqueda, ya que no solo debe de reducirse el error, si no que los parámetros que representan la red deben de estar limitados a cierta precisión: la propia de un ordenador.  

Comenzaremos viendo que en efecto una red neuronal con parámetros en el cuerpo de los enteros es factible como aproximador universal. 


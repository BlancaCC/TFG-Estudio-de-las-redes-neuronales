%%%%%%%%%%%%%%%%%%%%%%%%%%%%%%%
% Observación entre la diferencia de Q y R
%%%%%%%%%%%%%%%%%%%%%%%%%%%%
% Observación sobre el dominio discreto donde se está trabajando 
% y que refleja una posible fuente de mejora de las redes neuronales
% ISSUE #88
\section{Consideración sobre la capacidad de cálculo}

Suele pasar peligrosamente desapercibido que el teorema  \ref{corolario:2_6} recién probado asegura
que se podrá encontrar una red neuronal en $\rrnnmc$
que aproxime todo lo bien que queramos la función ideal; sin embargo, destaquemos que los parámetros que caracterizan a la red neuronal encontrada son reales. Es decir, 
Si nuestra red neuronal toma valores irracionales difícilmente será computable en un ordenador por su infinitud. 

Esto nos impone una nueva restricción en el espacio de búsqueda, ya que no solo debe de reducirse el error, si no que los parámetros que representan la red deben de estar limitados a cierta precisión: la precisión propia de un ordenador.  

Comenzaremos viendo que en efecto una red neuronal con parámetros en el cuerpo de los enteros es factible como aproximador universal. 

% Teorema de que podemos tener redes neuronales con parámetros racionales que también converjan. 
\begin{aportacionOriginal}
    \begin{teorema}
        El espacio $\mathcal{H}(\Q^d, \Q^s)$ es denso al espacio $\rrnnmc$. 
    \end{teorema}
    \begin{proof}
        Para todo $h_r \in \rrnnmc$, dado cualquier $\epsilon \in \R^+$, 
        gracias a la densidad de los racionales en los reales,  existirá $h_q \in \mathcal{H}(\Q^d, \Q^s)$ tal que 
        $\dist( h_r, h_q) < \epsilon$. 
    \end{proof}

    Si bien los números racionales tienen el potencial de ser computados por su finitud, seguimos sujetos a que el número de decimales y tamaño de su parte entera \textit{sean lo suficientemente pequeños} como para poder ser representados y calculados con un ordenador. Esto no tiene que ser, cierto y de hecho abre una nueva cuestión Al aumentar el número de neuronas, la precisión que demande una red neuronal también disminuye?  
\end{aportacionOriginal} 

De ser cierto este resultado, podría empezar a denominarse el espacio de las redes neuronales de cierto número de neuronas y con tal precisión. Esto abriría la puerta a establecer un isomorfismo entre este cuerpo y el de las reden neuronales con parámetros en los enteros.  Lo cuál tendría su interés ya que por las arquitecturas actuales, los números flotantes se calculan en la GPUs, mientras que los enteros en las CPUs, siendo mucho más rápidas la segundas. 

Si tuviéramos presente este razonamiento 
% Comienzan las notas antiguas del primer PR
Si tenemos presente la definición de red neuronal dada en \ref{definition:redes_neuronales_una_capa_oculta}, notaremos que lo que se intenta construir no es más que una función $h: \R^d \longrightarrow \R^s$. Sin embargo, los números reales no dejan de ser una entelequia, ya que en un ordenador es imposible representarlos por su infinitud. Esto nos impone una restricción del espacio de búsqueda, es decir nos está acotando el espacio. 

 Si la precisión máxima de se pueda o quiera alcanzar es de $t$ decimales;
 y por la naturaleza de los datos,
 % hacer referencia limitaciones en el muestreo
 % hacer referencia a que converge mejor en compactos 
 cada componente esté acotada por $M \in \N$
 entonces se podría 
establecer un isomorfismo entre \textit{la salida en flotante implementable} y el espacio cociente 

\begin{equation}%\label{ch05:espacio_cociente_isomorfo}
    \left( \frac{\N}{<10^{t + ceil( \log_{10}(2 M))}>} \right)^s. 
\end{equation}


Donde la deducción de esta forma es sencilla, 
fijamos una de las $s$ componentes de la salida, si los datos de salida estuvieran acotados en el 
intervalo real $[0,1]$ y si la precisión máxima 
posible a representar es de $t$ decimales; 
multiplicando cada salida por $10^{t+1}$ se tendría siempre un número natural de $t+1$ cifras, este sería el isomorfismo. 

Si par el contrario está acotado por $M \in N$, multiplicando cada salida por $10^{t+1}$ se tendrá número de a lo sumo $ceil(\log_{10}(2 M))+ t$ cifras, es decir lo que matemáticamente se expresa con el espacio cociente formulado en
%(\refeq{ch05:espacio_cociente_isomorfo}). 
 
Esto supondría abrir la puerta a trabajar con enteros en vez de con flotantes. 

Los cálculos realizados con enteros son más eficientes % TODO: argumentar
, la cuestión entonces reside en si sería factible modelizar el aprendizaje y evaluación de una red neuronal en los enteros 
y si supondría verdaderamente un beneficio. 
% Fin de la zona en que se está trabajando 

%%%%%%%%%%%%%%%%%%%%%%%%%%%%%%%
% Observación entre la diferencia de Q y R
%%%%%%%%%%%%%%%%%%%%%%%%%%%%
% Observación sobre el dominio discreto donde se está trabajando 
% y que refleja una posible fuente de mejora de las redes neuronales
% ISSUE #88
\section{Consideración sobre la capacidad de cálculo}

Suele pasar peligrosamente desapercibido que el teorema  \ref{corolario:2_6} recién probado asegura
que se podrá encontrar una red neuronal en $\rrnnmc$
que aproxime todo lo bien que queramos la función ideal; sin embargo, destaquemos que los parámetros que caracterizan a la red neuronal encontrada son reales. Es decir, 
si nuestra red neuronal toma valores irracionales difícilmente será computable en un ordenador por su infinitud. 

Esto nos impone una nueva restricción en el espacio de búsqueda, ya que no solo debe de reducirse el error, si no que los parámetros que representan la red deben de estar limitados a cierta precisión: la propia de un ordenador.  

Comenzaremos viendo que en efecto una red neuronal con parámetros en el cuerpo de los enteros es factible como aproximador universal. 

% Teorema de que podemos tener redes neuronales con parámetros racionales que también converjan. 
\begin{aportacionOriginal}
    \begin{teorema}
        El espacio $\mathcal{H}(\Q^d, \Q^s)$ es denso al espacio $\rrnnmc$. 
    \end{teorema}
    \begin{proof}
        Se quiere probar que para todo $h^r \in \rrnnmc$,
         dado cualquier $\epsilon \in \R^+$  
        existirá $h^q \in \mathcal{H}(\Q^d, \Q^s)$ tal que 
        $\dist( h^r, h^q) < \epsilon$. 

        La red neuronal $h^r$ está determinada por un conjunto finito
        de parámetros supongamos que hay $q$ y que están determinados por un índice del conjunto $\Lambda$. 

        Sea $\alpha^0_i \in R$, el primer índice $h^r$ definimos la red neuronal $h^1$ con coeficientes  $\alpha^1_i$ con $i\in \Lambda$
        de tal forma que los parámetros que la determinan vienen dados 
        \begin{equation*}
            \alpha^1_i = \alpha^0_i 
            \text{ con } i \in \Lambda
            \text{ y } i \neq 1.
        \end{equation*}

        El primer parámetro $\alpha^1_1$, será un número racional 
        seleccionado de tal forma que se satisfaga que 
        \begin{equation*}
            \dist( h^r, h^1) < \frac{\epsilon}{q}.
        \end{equation*}
        Por la densidad de los racionales en los reales sabemos que  $\alpha^1_1$.

        Siguiendo la misma idea, para $j \in \{2,\ldots, q\}$ se define la red neuronal 
        $h^j$ cuyos parámetro vienen dados por 
        \begin{equation*}
            \alpha^j_i = \alpha^{j-1}_i 
            \text{ con } i \in \Lambda
            \text{ y } i \neq j;
        \end{equation*}
        y $\alpha^j_j \in \Q$ seleccionado de tal forma que  
        \begin{equation*}
            \dist( h^j, h^{j-1}) < \frac{\epsilon}{q}.
        \end{equation*}

        De esta forma, en virtud de la desigualdad triangular,  se ha encontrado una red neuronal $h^q$ que todos sus parámetros son racionales y 
        satisface 
        $\dist( h^r, h^q) < \epsilon$.
    \end{proof}

    Si bien los números racionales tienen el potencial de ser computados por su 
    finitud, seguimos sujetos a que el número de decimales y tamaño de su parte entera \textit{sean lo suficientemente pequeños} como para poder ser representados 
    y calculados con un ordenador. Esto no tiene que ser, cierto y de hecho abre una nueva cuestión:
     Al aumentar el número de neuronas, ¿la precisión demandada por una red neuronal también disminuye?  
\end{aportacionOriginal} 

De ser cierto este resultado, podría empezar a denominarse el espacio 
de las redes neuronales de $n$  neuronas y $d$ bits de precisión. 

Es más, una vez limitados los decimales con los que se puede representar una red neuronal; puesto que los números racionales son isomorfos a los enteros, se podría plantear establecer un isomorfismo entre las redes neuronales con parámetros
en los racionales y el 
y el de las redes neuronales con entradas en los enteros. 

Todas estas pesquisas tienen su interés ya que por las arquitecturas 
actuales, los números flotantes (racionales con un límite de decimales) se calculan en la GPUs, mientras que los enteros en las CPUs, siendo más rápidas la segundas\footnote{En el blog del investigador Long Zhou, se comenta se da una visión favorable sobre las CPUs y cómo en la actualidad se comenten errores a la hora de comparar las GPUs y CPUs, dejo link a la publicación. Consultada por última vez el 23 de mayo
del 2022, URL: \url{https://long-zhou.github.io/2013/02/12/CPU-GPU-comparison.html}} \cite{CPU-vs-GPUS}. 


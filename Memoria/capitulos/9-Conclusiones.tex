%%%%%%%%%%%%%%%%%%%%%%%%%%%%%%%%%%%%%%%%%%%%%%%%%%%%
% Conclusiones del trabajo
%%%%%%%%%%%%%%%%%%%%%%%%%%%%%%%%%%%%%%%%%%%%%%%%%%%%

\chapter{Conclusiones} 
\label{ch09:conclusion}
Era nuestro objetivo con este trabajo esclarecer 
el motivo y funcionamiento de las redes neuronales y 
a partir de ahí optimizar algún aspecto de ellas. 

El sustento teórico queda expuesto en los capítulos 
\ref{ch00:methodology}, \ref{chapter:Introduction-neuronal-networks},
\ref{ch03:teoria-aproximar}, \ref{chapter4:redes-neuronales-aproximador-universal}
y \ref{chapter:construir-redes-neuronales}. 
Hemos contribuido al estado del arte actual con los 
resultados: 

\begin{itemize}
    \item La propuesta del uso de modelo de red neuronal (definición \ref{img:grafo-red-neuronal-una-capa-oculta}).
    \item La demostración teórica del uso de distintas funciones de activación en 
    el modelo seleccionado (corolario \ref{cor:se-generaliza-G-a-una-familia}). 
    \item La demostración de la densidad del espacio de las redes neuronales racionales en el espacio de las funciones medibles (teorema \ref{teo:densidad-racional}).
    \item Resultados sobre la irrelevancia del sesgo en las redes neuronales (sección \ref{consideration-irrelevancia-sesgo}).
    \item Una alternativa al uso de funciones de clasificación (sección \ref{ch05:dominio-discreto}).
    \item Un criterio de selección de funciones de activación (capítulo \ref{funciones-activacion-democraticas-mas-demoscraticas}).
    \item Resultados teóricos sobre la equivalencia de funciones de activación (teorema \ref{teo:equivalencia-grafos-activation-function} y 
    corolario \ref{corolario:afine-activation-function}).
    \item Un algoritmo de inicialización aprendida de los pesos de una red neuronal que acelera los métodos de aprendizaje iterativos (capítulo \ref{section:inicializar_pesos}).
    \item La biblioteca \textit{OptimizedNeuralNetwork.jl} que aporta un modelo y métodos optimizados para el uso de redes neuronales. 
\end{itemize}

y proponemos como posibles vías de investigación en proyectos futuros: 

\begin{itemize}
    \item Una revisión de la selección genética de funciones de activación con nuestro modelo (capítulo \ref{ch08:genetic-selection}).
    \item Una investigación sobre la repercusión en la convergencia de la delimitación de la precisión en los coeficientes de las redes neuronales (sección \ref{ch04:capacidad-calculo}). 
\end{itemize}

Pero finalmente y sobretodo, me llevo la grata experiencia de 
todo el proceso que ha conllevado este Trabajo Fin de Grado;
con las habilidades de gestión bibliográfica, comprensión y expresión rigurosa que ello implica;
así como el método adquirido, constancia y paciencia;
y por supuesto la satisfacción personal de haber sido capaz de acabar un proyecto 
de estas características. 



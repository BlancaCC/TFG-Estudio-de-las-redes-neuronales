% \manualmark
% \markboth{\textsc{Introducción}}{\textsc{Introducción}}

\chapter{Introducción}\label{ch:introduccion}

Estimado lector, podría comenzar una amigable y espectacular introducción mostrando algunos de los incontables ejemplos 
de problemas para los que 
el aprendizaje automático o las redes neuronales aportan soluciones exitosas e incluso sorprendentes. (Esto se tratará en 
[insertar referencia, habrá que hacerlo para que no me digan nada)
Pero en este trabajo me gustaría principalmente aportar una visión a más bajo nivel, partiendo de los pilares matemáticos 
que sostienen
las redes neuroles y explicar qué son exactamente, por qué funcionan y cómo se pueden optimizar. 

\section{Filosofía}
Las matemáticas a mi parecer se rigen sobre un equilibrismo férreo, entrañable y absorvente que son los axiomas;
con solo aceptar la verosimilitud de un axioma una vez, uno ya puede dejarse llevar por los derroteros
lógicos y confiar en que todo lo probado es completamente cierto (centro de dicho fonambulismo).   

Es por ello que me gustaría empezar con las siguiente pregunta que pulula en mi interior más crítico: 

¿Todo fenómeno observable guarda una ley que lo explique? 

---
Relacionar la realidad con lógica -> Gödel nos diría que no. 
¿Es la realidad lógica? 
[TODO buscar más información] 
---
Si 
Desconozco la respuesta, pero de tal manera imitando en cierta manera al razonamiento sobre la existencia
de Dios de Pascal: 
Si la respuesta fuera negativa: ¿Debería ser esto un motivo para frenar nuestra busca? 
¿Se podría separar la realidad en explicable o no? ¿Cómo se podría crear un método?
Si fuera positiva bastaría seguir cómo vamos. 

Mientras alguien encuentra respuestra, ya que la curiosidad humana es indómita
 entretengámonos pues pensando que sí y escarbemos en la inmensidad del conocimiento 
aún por descubrir. 

\section{Motivación del aprendizaje automático}


Al igual que un niño pequeño es capaz de distinguir entre un árbol y su progenitor sin ser
capaz de dar una descripción matemática de su deducción o una mera explicación.

De entre todo el desconocimiento existente podría presentarse la siguiente situación: 
Qué ocurriría si no hubiéramos encontrado una manera analítica viable de resolver un problema,
pero sin embargo 
contáramos con "los datos, ejemplos o muestras suficientes" como para dar una solución empírica. 


En resonancia con las ideas platónicas, los problemas de aprendizaje tratan de encontrar soluciones empíricas donde todavía 
no se conocen métodos analíticos. 

Estos no darán una explicación de porqué funcionan, pero como ya se ha comprobado [] pueden llegar a 
soluciones exitosas. 

Y si bien, así expuesto puede en un principio parecer de poco interés para el lector más matemático, 
la teoría que subyace bajo éste instrumento de la Ingeniería que se encuentra en plena esfervesciencia y  creación 
es realmente bello y apasionanto, espero a lo largo de este libro poder transmitirlo. 

Pregunta para mí ¿podría esta estructura guardar alguna relación o expliación con el ser humano?

\subsection{Componentes del aprendizaje}  

\endinput
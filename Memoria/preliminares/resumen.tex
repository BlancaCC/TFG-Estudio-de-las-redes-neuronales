% !TeX root = ../libro.tex
% !TeX encoding = utf8
%
%*******************************************************
% Resumen
%*******************************************************

% \manualmark
% \markboth{\textsc{Introducción}}{\textsc{Introducción}}

\chapter*{Resumen}\label{ch:resumen}
%\addcontentsline{toc}{chapter}{Resumen}

Existe en la actualidad un desequilibrio entre resultados empíricos 
y teóricos de redes neuronales llegando incluso a contradicción
 (como se comenta en la introducción del capítulo 
 \ref{chapter:Introduction-neuronal-networks}), será por tanto
nuestro primer objetivo construir una teoría sólida
que de cabida a 
 optimizaciones de fundamento teórico; 
una revisión y
 purga de cualquier artificio existente sobre 
 redes neuronales carente de fundamento matemático. 

Como resultado de ello se ha creado e implementado 
un nuevo modelo de red neuronal así como sus 
métodos de aprendizaje y evaluación. 
Además se ha propuesto un criterio de selección de 
funciones de activación y un algoritmo de 
inicialización de pesos que mejora los ya existentes. Todos los resultados han conducido a la creación de 
la biblioteca \textit{OptimizedNeuralNetwork.jl}, que contiene la implementación de nuestros modelos y métodos optimizados. 


La estructura de la memoria es la siguiente: 

\begin{itemize}
    \item \textbf{Capítulo \ref{ch00:methodology}: Descripción de la metodología seguida.} Se ha organizado el proyecto de acorde a una metodología ágil, basada en personas, historias de usuario, hitos y test. Tal método ha conducido e hilado desde el comienzo tanto el desarrollo teórico como el técnico a la par que  salvaguardaba la corrección de cada paso. 
    \item \textbf{Capítulo \ref{chapter:Introduction-neuronal-networks}: Descripción del problema de aprendizaje.} Se introduce las características y tipo de problemas del aprendizaje automático. Además se clarifica cuáles tratan de resolver las redes neuronales. 
    \item \textbf{Capítulo \ref{ch03:teoria-aproximar}: Teoría de la aproximación.} Se muestran los problemas y virtudes que presenta un enfoque clásico  de teoría de la aproximación frente a problemas de aprendizaje. En pos de solventar tales impedimentos,  
    se sitúa esta teoría como el germen de 
    las redes neuronales.
    Concretamente se desarrolla la teoría necesaria hasta demostrar el teorema de \textit{Stone-Weierstrass} y se explicarán las trabas que presentan este tipo de aproximaciones. 
    \item \textbf{Capítulo \ref{chapter4:redes-neuronales-aproximador-universal}: Introducción de las redes neuronales como aproximadores universales.} Se presenta nuestra propuesta de modelo de red neuronal y se compara con los modelos actuales. Se demuestra que nuestra definición actúa como un aproximador universal a cualquier función medible basándonos en el artículo 
    \textit{Multilayer Feedforward Networks are Universal Approximators} (\cite{HORNIK1989359}). Además se demuestran unas serie de resultados sobre cómo es la convergencia en problema de regresión y clasificación. Finalemnte se plantea si en la práctica las redes neuronales 
    verdaderamente son aproximadores universales.
    \item \textbf{Capítulo \ref{chapter:construir-redes-neuronales}: Diseño y construcción de las redes neuronales.} Se describe la implementación de las redes neuronales; esto nos permitirá una comparación  
    más profunda de nuestro modelo frente a los usuales. Producto de ello son dos resultados originales sobre el sesgo y dominio de la imagen. 
    Una vez determinado el modelo concreto se
    han diseñado un algoritmo de aprendizaje, basado en \textit{Backpropagation} y otro de evaluación de redes neuronales. Además se han comparado los resultados de nuestro modelado con los utilizado usualmente. 
    \item \textbf{Capítulo \ref{funciones-activacion-democraticas-mas-demoscraticas}: Democratización de las funciones de activación.} Se presenta un resultado propio que establece una equivalencia entre distintas funciones de activación. A partir de él se da un criterio de selección de funciones de activación y se estudian experimentalmente los beneficios obtenidos. 
    \item \textbf{Capítulo \ref{section:inicializar_pesos}: Algoritmo de inicialización de pesos.} Se propone un algoritmo de inicialización de pesos de una red neuronal.
    \item \textbf{Capítulo \ref{ch08:genetic-selection}: Selección genética de las funciones de activación.} Se propone un algoritmo de selección de la función de activación basado en algoritmos genéticos. 
\end{itemize}

\paragraph{PALABRAS CLAVE:}
\begin{itemize*}[label=,itemsep=1em,itemjoin=\hspace{1em}]
  \item redes neuronales
  \item optimización 
  \item funciones de activación 
  \item inicialización de pesos
\end{itemize*}

\endinput

% !TeX root = ../libro.tex
% !TeX encoding = utf8
%
%*******************************************************
% Resumen
%*******************************************************

% \manualmark
% \markboth{\textsc{Introducción}}{\textsc{Introducción}}

\chapter*{Resumen}\label{ch:resumen}
%\addcontentsline{toc}{chapter}{Resumen}

Existe en la actualidad un desequilibrio entre resultados empíricos 
y teóricos de redes neuronales llegando incluso a contradicción
 (como se comenta en la introducción del capítulo 
 \ref{chapter:Introduction-neuronal-networks}), será por tanto
nuestro primer objetivo construir una teoría sólida
que de cabida a 
 optimizaciones de fundamento teórico; 
una revisión y
 purga de cualquier artificio existente sobre 
 redes neuronales carente de fundamento matemático. 

Como resultado de ello se ha propuesto e implementado 
un nuevo modelo de red neuronal así como sus 
métodos de aprendizaje y evaluación. 
Además se ha dado un criterio de selección de 
funciones de activación y un algoritmo de 
inicialización de pesos que mejora los ya existentes. 
Todos los resultados han conducido a la creación de 
la biblioteca \textit{OptimizedNeuralNetwork.jl}, 
que contiene la implementación de nuestros modelos y
 métodos optimizados. 


La estructura de la memoria es la siguiente: 


\paragraph{PALABRAS CLAVE:}
\begin{itemize*}[label=,itemsep=1em,itemjoin=\hspace{1em}]
  \item redes neuronales
  \item optimización 
  \item funciones de activación 
  \item inicialización de pesos
  \item Biblioteca de aprendizaje automático
\end{itemize*}

\endinput

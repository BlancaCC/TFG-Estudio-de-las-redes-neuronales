% !TeX root = ../libro.tex
% !TeX encoding = utf8
%
%*******************************************************
% Resumen
%*******************************************************

% \manualmark
% \markboth{\textsc{Introducción}}{\textsc{Introducción}}

\chapter*{Resumen}\label{ch:resumen}
%\addcontentsline{toc}{chapter}{Resumen}
Con este trabajo se ha pretendido construir una teoría sólida que de cabida a optimizar el modelado actual de las redes neuronales.

Como resultado de ello se ha creado e implementado 
un nuevo modelo de red neuronal así como sus 
métodos de aprendizaje y evaluación. 
Además se ha propuesto un criterio de selección de 
funciones de activación y un algoritmo de 
inicialización de pesos que mejora los ya existentes. 

La estructura de la memoria es la siguiente: 

\begin{itemize}
    \item \textbf{Capítulo \ref{ch00:methodology}: Descripción de la metodología seguida.} 
    \item \textbf{Capítulo \ref{chapter:Introduction-neuronal-networks}: Descripción del problema de aprendizaje.} Caracterización de los problemas de aprendizaje.
    \item \textbf{Capítulo \ref{ch03:teoria-aproximar}: Teoría de la aproximación.} Ejemplificación de los problemas que presenta un enfoque clásico frente a problemas de problema de aprendizaje.  Demostración del teorema de \textit{Stone-Weierstrass}.
    \item \textbf{Capítulo \ref{chapter4:redes-neuronales-aproximador-universal}: Introducción de las redes neuronales como aproximadores universales.} Se presenta nuestro modelo de red neuronal. Se demuestra que es un aproximador universal basado en el artículo 
    \textit{Multilayer Feedforwar Networks are Universal Approximatos} de Kurt Hornik, 
    Maxwell Stinchcombe y Halber White. Se plantea si en la práctica las redes neuronales se pueden implementar.
    \item \textbf{Capítulo \ref{chapter:construir-redes-neuronales}: Construcción de las redes neuronales.} Descripción de la implementación de las redes neuronales. 
    Comparación de nuestro modelo con los usuales bajo la introducción de resultados originales sobre el sesgo y dominio de la imagen. Motivación, desarrollo e implementación de los algoritmos de aprendizaje y evaluación propuestos.
    \item \textbf{Capítulo \ref{funciones-activacion-democraticas-mas-demoscraticas}: Democratización de las funciones de activación.} Se presenta un resultado propio que establece una equivalencia entre distintas funciones de activación. A partir de él se da un criterio de selección de funciones de activación y se estudia experimentalmente los beneficios obtenidos. 
    \item \textbf{Capítulo \ref{section:inicializar_pesos}: Algoritmo de inicialización de pesos.} Se propone un algoritmo de inicialización de pesos de una red neuronal.
    \item \textbf{Capítulo \ref{ch08:genetic-selection}: Selección genética de las funciones de activación.} Se propone un algoritmo de selección de la función de activación basado en algoritmos genéticos. 
\end{itemize}

\paragraph{PALABRAS CLAVE:}
\begin{itemize*}[label=,itemsep=1em,itemjoin=\hspace{1em}]
  \item Redes neuronales
  \item Optimización 
  \item Funciones de activación 
  \item Inicialización de pesos
\end{itemize*}

\endinput

% !TeX root = ../../tfg.tex
% !TeX encoding = utf8
%
%*******************************************************
% Teorema de Stone Weiertass 
%*******************************************************

\chapter{Teorema de Stone-Weierstass }\label{ch:TeoremaStoneWeiertass}

\begin{teorema}[Teorema de Stone-Weierstass]

    Sea $K$ un subconjunto compacto de $\R^p$ y sea $\mathcal{A}$ una colección de 
    funciones continuas de $K$ a $\R$ con las siguientes propiedades: 

    \begin{enumerate}
        \item La función constantemente uno, definida como $e(x)=1$, para cualquier $x\in K$ pertenece a $\mathcal{A}$.
        \item Cerrado para sumas y producto para escalares. Si $f,g$ pertenece a  $\mathcal{A}$, entonces $\alpha f + \beta g$ pertenece a $\mathcal{A}$ . 
        \item Cerrado para producto, $fg$ pertenece a $\mathcal{A}$. 
        \item Separación de $K$, es decir si $x \neq y$ pertenecientes a $K$, entonces existe una función $f$ en $\mathcal{A}$  de tal manera que $f(x) \neq f(y)$. 
    \end{enumerate}
    
    Se tiene que toda función continua de $K$ a $\R$ puede ser aproximada en $K$ por funciones de $\mathcal a$. 
    \begin{proof}
        
    \end{proof}
\end{teorema}

\endinput
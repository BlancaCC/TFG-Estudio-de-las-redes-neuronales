% !TeX root = ../../tfg.tex
% !TeX encoding = utf8
%
%*******************************************************
% Teorema de Stone Weiertrass 
%*******************************************************

\section{Teorema de Stone-Weierstrass }\label{ch:TeoremaStoneWeiertrass}

\begin{teorema}[Teorema de Stone-Weierstrass] 

    Sea $K$ un subconjunto compacto de $\R^p$ y sea $\mathcal{A}$ una colección de 
    funciones continuas de $K$ a $\R$ cumpliendo $\mathcal{A}$ y 
    que separa puntos en $K$, es decir cumpliendo las siguientes propiedades: 

    \begin{enumerate}
        \item La función constantemente uno, definida como $e(x)=1$, para cualquier $x\in K$ pertenece a $\mathcal{A}$.
        \item Cerrado para sumas y producto para escalares reales. Si $f,g$ pertenece a  $\mathcal{A}$, entonces $\alpha f + \beta g$ pertenece a $\mathcal{A}$ . 
        \item Cerrado para producto. Para $f,g \in \mathcal A$, se tiene que $fg$ pertenece a $\mathcal{A}$. 
        \item Separación de $K$, es decir si $x \neq y$ pertenecientes a $K$, entonces existe una función $f$ en $\mathcal{A}$  de tal manera que $f(x) \neq f(y)$. 
    \end{enumerate}
    
    Se tiene que toda función continua de $K$ a $\R$ puede ser aproximada en $K$ por funciones de $\mathcal A$. 

\end{teorema}  

La idea que subyace bajo la demostración del teorema es que a partir de las hipótesis de la estructura algebraica y 
separabilidad 
se prueba que es posible encontrar las funciones máximo y mínimo, 
construyéndolo a partir del valor absoluto.   

Con esta nueva propiedad y siendo cerrado para combinaciones lineales 
se puede aproximar uniformemente cualquier función continua definida en el compacto. 

\begin{proof}
    Sean $a, b \in \R$ y $x \neq y$ pertenecientes a $K$.  Por la hipótesis de separabilidad existirá 
    una función $f \in \mathcal{A}$ tal que $f(x) \neq f(y)$.  Además la existencia de un elemento neutro 
    $e \in \mathcal{A}$ en el álgebra nos permite encontrar reales $\alpha, \beta$ tales que 

    $$\alpha f(x) + \beta e(x) = a, \quad \alpha f(y) + \beta e(y) = b$$  

     Por el teorema de Heine, para $f \in \mathcal A$ está acotada por tomar imagen en un compacto, es decir $|f(x)| \leq M$ para $x \in K.$  

    Consideremos ahora la función valor absoluto, $\phi(t)=|t|$ definida en el dominio $I = [-M, M].$
    Por el teorema de aproximación de Weierstrass 
    \ref{teo:TeoremaAproximacionWeierstrass}
    
    para cualquier $\varepsilon > 0$ 
    existirá un polinomio $p$ cumpliendo que 
    $$||t|- p(t)| < \varepsilon, \quad \forall t \in I.$$

    Puesto que $t \in I$ no son más que las posibles imágenes que puede tomar $f$ en $K$ inferimos entonces que 

    $$||f(x)| - p \circ f(x)| < \varepsilon \quad \forall x \in K.$$

    Como $f \in \mathcal{A}$ y $p$ es un polinomio, es decir,  $p \circ f(x)$ son sumas de potencias multiplicadas por escalares de $f(x)$, luego por la hipótesis de ser cerrado a estas operaciones tenemos que la función 
    $|f|$ pertenece a $\mathcal{A}$ si $f \in \mathcal{A}.$  


    Tenemos con esto que $\mathcal{A}$ también es cerrada a supremo e ínfimo  gracias a que:   
    $$sup\{f,g\} = \frac{1}{2} \{f+g+ |f+g|\}$$
    $$inf\{f,g\} = \frac{1}{2} \{f+g -|f+g|\}$$
    
    Por lo que concluimos que cualquier función puede ser uniformemente aproximada por combinaciones lineales, supremos e ínfimo mediante funciones de $\mathcal{A}$.   Es decir, cualquier función continua en $K$
    puede ser uniformemente aproximado por funciones de $\mathcal{A}$. 
\end{proof}

Admite este teorema una reflexión sobre cuánto de restrictivas son las hipótesis
 o el conjunto de propiedades exigidas $\mathcal{A}$. 

Para ello tomemos un ejemplo concreto, sea $K =  \R^n$ se tiene que el conjunto 
$\mathcal{ A} = \{ \}$
Reflexión sobre el teorema, $K$ puede parecer todo lo abstracta posible, ¿Son por ende las 
hipótesis demasiado restrictivas?  Para responder a esta pregunta basta con pensar en un ejemplo concreto: 
Tomemos $K =  \R^n$  el cuerpo de los polinomios cumple todas las hipótesis, luego... 

Sin embargo aproximar uniformemente cualquier función continua es 
Lo restrictivo está en pensar ya que el conjunto de funciones continuas en un compacto es mucho  más pequeño que el de todas las funciones que podamos pensar.  

\endinput
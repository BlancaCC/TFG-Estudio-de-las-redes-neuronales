% !TeX root = ../../tfg.tex
% !TeX encoding = utf8
%
%*******************************************************
% Polinomois de Bernstein
%*******************************************************

\chapter{Polinomios de Bernstein}\label{ch:Bernstein}

\begin{definicion}[Polinomios de Bernstein] \label{def:Bernstein}
    Dada cierta función $f: [0,1] \rightarrow \R$, se define el n-ésimo polinomio de Bernstain para $f$ como 

    $$B_n(x) = B_n(x;f)=\sum_ {k=0}^{n} f \left( \frac{k}{n} \right) \left( \binom{n}{k} \right) x^k (1-x)^{n-k}.$$

\end{definicion}

La intuición que se esconde tras esta definición es la siguiente: 
Se pretende aproximar la función $f$ a través de los puntos $\frac{k}{n}$ con $n \in \N$ fijo
y $k \in \{0,...,n \}.$
De tal forma que si evaluamos $B_n(x)$ con $x$ lo suficientemente próximo a  $\frac{k}{n}$  
la imagen de $B_n(x)$ se acerque a $f \left( \frac{k}{n} \right).$

Para ello recordaremos el teorema del Binomio de Newton: 

\begin{teorema}[Binomio de Newton]
    Cualquier potencia de un binomio $x+y$ con $x,y \in R$,  puede ser expandido en una suma de la forma
    \[(x+y)^n = \sum_{k=0}^n \binom{n}{k} x^{n-k}y^k\]
\end{teorema}

Así pues en virtud de esta igualdad y puesto que nuestro dominio de definición de $f$ es $[0,1]$, para cualquier $x \in [0,1].$

Tenemos 

\begin{equation}
    1 = (x+ (1-x))^n = \sum_{k=0}^n \binom{n}{k} x^{k} (1-x)^{n-k}
\end{equation}

Multiplicamos ahora en ambos lados por $f(x)$

\begin{equation}
    f(x) = \sum_{k=0}^n f(x) \binom{n}{k} x^{k} (1-x)^{n-k}
\end{equation} 

Y tenemos que las diferencia entre $f(x)$ y $B_n(x)$ es

\begin{equation}
    f(x)-B_n(x) = \sum_{k=0}^n \left(f(x) - f \left( \frac{k}{n} \right)\right)
    \binom{n}{k} x^{k} (1-x)^{n-k}
\end{equation} 

Que en valores absolutos resulta 
\begin{equation} \label{eqn:berstein_difference}
    |f(x)-B_n(x)| = \sum_{k=0}^n \left|f(x) - f \left( \frac{k}{n} \right)\right|
    \binom{n}{k} x^{k} (1-x)^{n-k}
\end{equation} 

Observando esta ecuación \ref{eqn:berstein_difference} se desprende como es natural un teorema de convergencia. 

\begin{teorema}[Teorema de aproximación de Bernstein]\label{teo:aproximacion_bernstein}

    Sea $f$ una función continua en un intervalo $I$ con imágenes en los reales. 
    La secuencia de polinomio de Bernstein
    \ref{def:Bernstein} converge uniformemente a $f$ en $I.$
    
\end{teorema}
Recordaremos antes la definición de convergencia uniforme: 

\begin{definicion}[Convergencia uniforme para funciones reales]

    Dado $E$ un conjunto y $\{f_n\}_{n \in \N}$ una sucesión de funciones de $E$
     a los reales; se dice 
    que dicha sucesión converge uniformemente si para cualquier $\varepsilon > 0$ existe un número natural $m$ tal que 
    para todo $x   \in E$ y cualquier natural $n$ que cumpla $n \geq m$ se tiene que 

    \begin{equation*}
        |f_n(x) - f(x) | < \varepsilon
    \end{equation*}
    
\end{definicion}

Comencemos pues con a demostración \ref{teo:aproximacion_bernstein}
\begin{proof}
    Sin pérdida de generalidad supondremos que $I=[0,1]$, como veremos esto no es restrictivo ya que 
    si $I$ fuera un intervalo cerrado existiría un homeomorfismo $H$ tal que $H^*(I)=[0,1]$ y podríamos
    trabajar con $H \circ f$ la cual respetaría todas los argumentos utilizados en la demostración. 

    Si $I$ fuera un abierto consideraríamos su cierre y aplicaríamos el razonamiento anterior. 
    De esta manera los supremos e ínfimos se mantendrían, ahora como mínimos y máximos y no se alteraría
    de ninguna manera la continuidad. 

    Tras la aclaración anterior podemos comenzar.
    
    Sea $\varepsilon > 0$ queremos probar que existirá un $m_\varepsilon  \in \N$ tal que para 
    cualquier $x \in I$ e $n \geq m_\varepsilon$  se tenga que 
    $|f(x) - B_n(x)| < \varepsilon$.
    
     Para ello por estar $f$ definida en un intervalo, 
    se tienen dos consecuencias claves: 
    \begin{enumerate}
        \item Está acotada, supongamos por $M \in \R$, esto es $|f(x)| \leq M$.
        \item Es uniformemente continua, es decir: para cualquier $\varepsilon >0$ existirá un $\delta_\varepsilon$
        tal que para cualesquiera $x,y \in I$ que cumplan $|x-y| < \delta_\varepsilon$ entonces $|f(x)-f(y)| < \varepsilon$
    \end{enumerate}
    
    
    Dado $N \in  \N$ fijo pero arbitrario, para $\{ k | k \in \{1, ..., N\}$
    $\frac{k}{N} \in I$ y tomando $x \in I$ podemos acotar

    $$|f(x)- f\left( \frac{k}{N} \right) | \leq |f(x)| + |f(\frac{k}{N})|\leq 2M$$

    
    Fijado un $x$ del dominio, se tienen las siguientes particiones de índices: 
    
    $$\mathcal{A}_{x,N} = \{ k | k \in \{1, ..., N\} \text{ y }  |x- \frac{k}{N}| < \delta_\varepsilon \}$$

    Donde el $\delta_\varepsilon$ se ha obtenido de la observación 2 tomando como $\varepsilon$ el 
    buscado para la convergencia uniforme. 
 
    Sea $\varepsilon > 0$ y tomemos $\delta_ \varepsilon$ el delta de la definición de uniformemente continuo.
    
    Por otro lado se define
    $$\mathcal{B}_{x,N} = \{1, ..., N\} \\ \mathcal{A}_{x,N}$$


    Podemos elegir $m$ de manera conveniente de tal forma que 

    $$ n \geq sup \{ (\delta_\varepsilon) ^{-4}, \frac{M^2}{\varepsilon^2}\}$$


    y agruparemos los términos de la sumatoria en dos partes, para los que 
    $|x - \frac{k}{n}| < n^{ -\frac{1}{4}} \leq \delta_\varepsilon$. 

    Para éstos,  se tiene por Newton  

    \begin{equation*}
        \sum_{k=1}^n   \varepsilon \binom{n}{k} x^k (1-x)^{n-k} \leq \varepsilon \sum_{k=1}^n \binom{n}{k} x^k (1-x)^{n-k} =  \varepsilon
    \end{equation*}

    Para el resto de términos, aquellos cuya $k$ cumpla que $|x - \frac{x}{n}| \geq n ^\frac{-1}{4}$, se tiene que 
    $|x - \frac{x}{m}| \geq n ^\frac{-1}{2}$,

\end{proof}

 
 
\endinput 
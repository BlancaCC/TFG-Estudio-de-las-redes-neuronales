% !TeX root = ../../tfg.tex
% !TeX encoding = utf8
%
%*******************************************************
% Polinomois de Bernstein
%*******************************************************

\chapter{Polinomios de Bernstein}\label{ch:Bernstein}

\begin{definicion}[Polinomios de Bernstein] \label{def:Bernstein}
    Dada cierta función $f: [0,1] \rightarrow \R$, se define el n-ésimo polinomio de Bernstain para $f$ como 

    $$B_n(x) = B_n(x;f)=\sum_ {k=0}^{n} f \left( \frac{k}{n} \right) \left( \binom{n}{k} \right) x^k (1-x)^{n-k}.$$

\end{definicion}

La intuición que se esconde tras esta definición es la siguiente: 
Se pretende aproximar la función $f$ a través de los puntos $\frac{k}{n}$ con $n \in \N$ fijo
y $k \in \{0,...,n \}.$
De tal forma que si evaluamos $B_n(x)$ con $x$ lo suficientemente próximo a  $\frac{k}{n}$  
la imagen de $B_n(x)$ se acerque a $f \left( \frac{k}{n} \right).$

Para ello recordaremos el teorema del Binomio de Newton: 

\begin{teorema}[Binomio de Newton]
    Cualquier potencia de un binomio $x+y$ con $x,y \in R$,  puede ser expandido en una suma de la forma
    \[(x+y)^n = \sum_{k=0}^n \binom{n}{k} x^{n-k}y^k\]
\end{teorema}

Así pues en virtud de esta igualdad y puesto que nuestro dominio de definición de $f$ es $[0,1]$, para cualquier $x \in [0,1].$

Tenemos 

\begin{equation}
    1 = (x+ (1-x))^n = \sum_{k=0}^n \binom{n}{k} x^{k} (1-x)^{n-k}
\end{equation}

Multiplicamos ahora en ambos lados por $f(x)$

\begin{equation}
    f(x) = \sum_{k=0}^n f(x) \binom{n}{k} x^{k} (1-x)^{n-k}
\end{equation} 

Y tenemos que las diferencia entre $f(x)$ y $B_n(x)$ es

\begin{equation}
    f(x)-B_n(x) = \sum_{k=0}^n \left(f(x) - f \left( \frac{k}{n} \right)\right)
    \binom{n}{k} x^{k} (1-x)^{n-k}
\end{equation} 

Que en valores absolutos resulta 
\begin{equation} \label{eqn:berstein_difference}
    |f(x)-B_n(x)| = \sum_{k=0}^n \left|f(x) - f \left( \frac{k}{n} \right)\right|
    \binom{n}{k} x^{k} (1-x)^{n-k}
\end{equation} 

Observando esta ecuación \ref{eqn:berstein_difference} se desprende como es natural un teorema de convergencia. 

\begin{teorema}[Teorema de aproximación de Bernstein]\label{teo:aproximacion_bernstein}

    Sea $f$ una función continua en un intervalo $I$ con imágenes en los reales. 
    La secuencia de polinomio de Bernstein
    \ref{def:Bernstein} converge uniformemente a $f$ en $I.$
    
\end{teorema}
Recordaremos antes la definición de convergencia uniforme: 

\begin{definicion}[Convergencia uniforme para funciones reales]

    Dado $E$ un conjunto y $\{f_n\}_{n \in \N}$ una sucesión de funciones de $E$
     a los reales; se dice 
    que dicha sucesión converge uniformemente si para cualquier $\varepsilon > 0$ existe un número natural $m$ tal que 
    para todo $x   \in E$ y cualquier natural $n$ que cumpla $n \geq m$ se tiene que 

    \begin{equation*}
        |f_n(x) - f(x) | < \varepsilon
    \end{equation*}
    
\end{definicion}

Comencemos pues con a demostración \ref{teo:aproximacion_bernstein}
\begin{proof}
    Por estar $f$ definida en un intervalo cerrado, está acotada, supongamos por $M$. 
    Además es uniformemente continua. 
    
    TODO revisar continuidad.
    
    En el peor de los casos nuestra función será menor que $2M$. 


    Dividamos ahora en dos partes, los $x- frac{k}{n}$ para los cuales la diferencia es pequeña (TODO ¿cuánto de pequeña) y para los que es grande. 

    Sea $\varepsilon > 0$ y tomemos $\delta_ \varepsilon$ el delta de la definición de uniformemente continuo.  

    Podemos elegir $m$ de manera conveniente de tal forma que 

    $$ n \geq sup \{ (\delta_\varepsilon) ^{-4}, \frac{M^2}{\varepsilon^2}\}$$


    y agruparemos los términos de la sumatoria en dos partes, para los que 
    $|x - \frac{k}{n}| < n^{ -\frac{1}{4}} \leq \delta_\varepsilon$. 

    Para éstos,  se tiene por Newton  

    \begin{equation*}
        \sum_{k=1}^n   \varepsilon \binom{n}{k} x^k (1-x)^{n-k} \leq \varepsilon \sum_{k=1}^n \binom{n}{k} x^k (1-x)^{n-k} =  \varepsilon
    \end{equation*}

\end{proof}

 
 
\endinput 
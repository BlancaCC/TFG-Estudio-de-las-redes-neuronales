% !TeX root = ../../tfg.tex
% !TeX encoding = utf8
%
%*******************************************************
% Polinomois de Bernstein
%*******************************************************

\chapter{Polinomios de Bernstein}\label{ch:Bernstein}

\begin{definicion}[Polinomios de Bernstein] \label{def:Bernstein}
    Dada cierta función $f: [0,1] \rightarrow \R$, se define el n-ésimo polinomio de Bernstain para $f$ como 

    $$B_n(x) = B_n(x;f)=\sum_ {k=0}^{n} f \left( \frac{k}{n} \right) \left( \binom{n}{k} \right) x^k (1-x)^{n-k}.$$

\end{definicion}

La intuición que se esconde tras esta definición es la siguiente: 
Se pretende aproximar la función $f$ a través de los puntos $\frac{k}{n}$ con $n \in \N$ fijo
y $k \in \{0,...,n \}.$
De tal forma que si evaluamos $B_n(x)$ con $x$ lo suficientemente próximo a  $\frac{k}{n}$  
la imagen de $B_n(x)$ se acerque a $f \left( \frac{k}{n} \right).$

Para ello recordaremos el teorema del Binomio de Newton: 

\begin{teorema}[Binomio de Newton]
    Cualquier potencia de un binomio $x+y$ con $x,y \in R$,  puede ser expandido en una suma de la forma
    \[(x+y)^n = \sum_{k=0}^n \binom{n}{k} x^{n-k}y^k\]
\end{teorema}

Así pues en virtud de esta igualdad y puesto que nuestro dominio de definición de $f$ es $[0,1]$, para cualquier $x \in [0,1].$

Tenemos 

\begin{equation}\label{eq:uno_igual_binomio}
    1 = (x+ (1-x))^n = \sum_{k=0}^n \binom{n}{k} x^{k} (1-x)^{n-k}
\end{equation}

Multiplicamos ahora en ambos lados por $f(x)$

\begin{equation}
    f(x) = \sum_{k=0}^n f(x) \binom{n}{k} x^{k} (1-x)^{n-k}
\end{equation} 

Y tenemos que las diferencia entre $f(x)$ y $B_n(x)$ es

\begin{equation}
    f(x)-B_n(x) = \sum_{k=0}^n \left(f(x) - f \left( \frac{k}{n} \right)\right)
    \binom{n}{k} x^{k} (1-x)^{n-k}
\end{equation} 

Que en valores absolutos resulta 
\begin{equation} \label{eqn:berstein_difference}
    |f(x)-B_n(x)| = \sum_{k=0}^n \left|f(x) - f \left( \frac{k}{n} \right)\right|
    \binom{n}{k} x^{k} (1-x)^{n-k}
\end{equation} 

Observando esta ecuación \ref{eqn:berstein_difference} se desprende como es natural un teorema de convergencia. 

\begin{teorema}[Teorema de aproximación de Bernstein]\label{teo:aproximacion_bernstein}

    Sea $f$ una función continua en un intervalo $I$ con imágenes en los reales. 
    La secuencia de polinomio de Bernstein
    \ref{def:Bernstein} converge uniformemente a $f$ en $I.$
    
\end{teorema}
Recordaremos antes la definición de convergencia uniforme: 

\begin{definicion}[Convergencia uniforme para funciones reales]

    Dado $E$ un conjunto y $\{f_n\}_{n \in \N}$ una sucesión de funciones de $E$
     a los reales; se dice 
    que dicha sucesión converge uniformemente si para cualquier $\varepsilon > 0$ existe un número natural $m$ tal que 
    para todo $x   \in E$ y cualquier natural $n$ que cumpla $n \geq m$ se tiene que 

    \begin{equation*}
        |f_n(x) - f(x) | < \varepsilon
    \end{equation*}
    
\end{definicion}

Comencemos pues con a demostración \ref{teo:aproximacion_bernstein}
\begin{proof}
    Sin pérdida de generalidad supondremos que $I=[0,1]$, como veremos esto no es restrictivo ya que 
    si $I$ fuera un intervalo cerrado existiría un homeomorfismo $H$ tal que $H^*(I)=[0,1]$ y podríamos
    trabajar con $H \circ f$ la cual respetaría todas los argumentos utilizados en la demostración. 

    Si $I$ fuera un abierto consideraríamos su cierre y aplicaríamos el razonamiento anterior. 
    De esta manera los supremos e ínfimos se mantendrían, ahora como mínimos y máximos y no se alteraría
    de ninguna manera la continuidad. 

    Tras la aclaración anterior podemos comenzar.
    
    Sea $\varepsilon > 0$ queremos probar que existirá un $m_\varepsilon  \in \N$ tal que para 
    cualquier $x \in I$ e $n \geq m_\varepsilon$  se tenga que 
    $|f(x) - B_n(x)| < \varepsilon$.
    
     Para ello por estar $f$ definida en un intervalo, 
    se tienen dos consecuencias claves: 
    \begin{enumerate}
        \item Está acotada, supongamos por $M \in \R$, esto es $|f(x)| \leq M$.
        \item Es uniformemente continua, es decir: para cualquier $\varepsilon >0$ existirá un $\delta_\varepsilon$
        tal que para cualesquiera $x,y \in I$ que cumplan $|x-y| < \delta_\varepsilon$ entonces $|f(x)-f(y)| < \varepsilon$
    \end{enumerate}
    
    
    Dado $N \in  \N$ fijo pero arbitrario, para $\{ k | k \in \{1, ..., N\}$
    $\frac{k}{N} \in I$ y tomando $x \in I$ podemos acotar

    $$|f(x)- f\left( \frac{k}{N} \right) | \leq |f(x)| + |f(\frac{k}{N})|\leq 2M$$

    
    Fijado un $x$ del dominio, se tienen las siguientes particiones de índices: 
    
    $$\mathcal{A}_{x,N} = \{ k | k \in \{1, ..., N\} \text{ y }  |x- \frac{k}{N}| < \delta_\varepsilon \}$$

    Donde el $\delta_\varepsilon$ se ha obtenido de la observación 2 tomando como $\varepsilon$ el 
    buscado para la convergencia uniforme. 
 
    Sea $\varepsilon > 0$ y tomemos $\delta_ \varepsilon$ el delta de la definición de uniformemente continuo.
    
    Por otro lado se define
    $$\mathcal{B}_{x,N} = \{1, ..., N\} - \mathcal{A}_{x,N}$$


    Podemos elegir $m$ de manera conveniente de tal forma que 

    $$ n \geq sup \left\{ (\delta_\varepsilon) ^{-4}, \frac{M^2}{\varepsilon^2}\right\}$$


    y agruparemos los términos de la sumatoria en dos partes, para los que 
    $|x - \frac{k}{n}| < n^{ -\frac{1}{4}} \leq \delta_\varepsilon$. 

    Para éstos,  se tiene por Newton  

    \begin{equation*}
        \sum_{k=1}^n   \varepsilon \binom{n}{k} x^k (1-x)^{n-k} \leq \varepsilon \sum_{k=1}^n \binom{n}{k} x^k (1-x)^{n-k} =  \varepsilon
    \end{equation*}

    Para el resto de términos, aquellos cuya $k$ cumpla que $|x - \frac{x}{n}| \geq n ^\frac{-1}{4}$, se tiene que 
    $|x - \frac{x}{m}| \geq n ^\frac{-1}{2}$. 

    Queda el resto de subíndices resulta acotada por, deberemos de ver que para una determinada N, 
    esa desigualdad se queda en épsilon
    \begin{equation*}
        \sum_{k \in \mathcal{B}_{x,N}} 2M \binom{n}{k} x^k (1-x) ^{n-k} \\
    \end{equation*}

    Para acotar esto nos va a costar trabajar un poco más 
    \begin{equation*} 
        \begin{split}
        \label{eq:Bernstein_caso_a_acotar}
        & \sum_{k \in \mathcal{B}_{x,N}} 2M \binom{N}{k} x^k (1-x) ^{N-k} \\
        & = 2M  \sum_{k \in \mathcal{B}_{x,N}}  \frac{(x- \frac{k}{N})^2}{(x- \frac{k}{N})^2} \binom{N}{k} x^k (1-x) ^{N-k} \\
        & \leq 2M \sqrt{N} \sum_{k \in \mathcal{B}_{x,N}}  (x- \frac{k}{N})^2 \binom{n}{k} x^k (1-x) ^{N-k} \\
    \end{split}
    \end{equation*}


    Tengamos ahora presenten las siguientes igualdades 
    \begin{equation} \label{eq:binomio_menos_uno}
        \binom{n-1}{k-1} = \frac{(n-1)!}{(k-1)! (n-1-(k-1)!)} = \frac{k}{n} \binom{n}{k}
    \end{equation}
    \begin{equation} \label{eq:binomio_menos_dos}
        \binom{n-2}{k-2} = \frac{(n-2)!}{(k-2)! (n-2-(k-2)!)} = \frac{k(k-1)}{n(n-1)} \binom{n}{k}
    \end{equation}

    Partiendo de \ref{eq:uno_igual_binomio} se tiene que 
    \begin{equation}
        1 = (x+ (1-x))^n = \sum_{k=0}^n \binom{n}{k} x^{k} (1-x)^{n-k}
    \end{equation}

    Reemplazamos la $n$ por $n-1$ y la $k$ por $j$ y tenemos 
    \begin{equation}
        1 = \sum_{j=0}^{n-1} \binom{n-1}{j} x^{j} (1-x)^{(n-1)-j}
    \end{equation}
    Multiplicamos por $x$ y aplicamos la igualdad \ref{eq:binomio_menos_uno} resultando 

    \begin{equation}
        x = \sum_{j=0}^{n-1} \frac{j+1}{n} \binom{n}{j+1} x^{j+1} (1-x)^{(n-(j+1)}
    \end{equation}

    Renombramos $k= j+1$, por lo que resulta
    \begin{equation}
        x = \sum_{k=1}^{n} \frac{k}{n} \binom{n}{k} x^{k} (1-x)^{n-k}
    \end{equation}

    Como el término con $k=0$ es nulo podemos añadirlo a la sumatoria
    
    \begin{equation} \label{eq:desarrollo_binomio_uno}
        x = \sum_{k=0}^{n} \frac{k}{n} \binom{n}{k} x^{k} (1-x)^{n-k}
    \end{equation}

    %------------------ caso 2, no te confundas --------------------
    Haremos ahora un razonamiento similar sustituyendo $n$ por $n-2$

    Partiendo de \ref{eq:uno_igual_binomio} se tiene que 
    \begin{equation}
        1 = (x+ (1-x))^n = \sum_{k=0}^n \binom{n}{k} x^{k} (1-x)^{n-k}
    \end{equation}

    Reemplazamos la $n$ por $n-2$ y la $k$ por $j$ y tenemos 
    \begin{equation}
        1 = \sum_{j=0}^{n-2} \binom{n-2}{j} x^{j} (1-x)^{(n-2)-j}
    \end{equation}
    Multiplicamos por $x^2$ y aplicamos la igualdad \ref{eq:binomio_menos_dos} resultando 

    \begin{equation}
        x^2 = \sum_{j=0}^{n-2} \frac{(j+2)(j+1)}{n(n-1)} \binom{n}{j+2} x^{j+2} (1-x)^{(n-(j+2)}
    \end{equation}

    Renombramos $k= j+2$, por lo que resulta
    \begin{equation}
        x^2 = \sum_{k=2}^{n} \frac{k(k-1)}{n(n-1)} \binom{n}{k} x^{k} (1-x)^{n-k}
    \end{equation}

    Como con los términos $k=0$ y $k=1$ se anula, podemos añadir dichos índices sin modificar la suma 
    
    \begin{equation}
        x^2 = \sum_{k=0}^{n} \frac{k(k-1)}{n(n-1)} \binom{n}{k} x^{k} (1-x)^{n-k}
    \end{equation}

    Podemos reescribir la ecuación resultando: 

    \begin{equation} \label{eq:desarrollo_binomio_dos}
      (n^2 - n)  x^2 = \sum_{k=0}^{n} (k^2 - k) \binom{n}{k} x^{k} (1-x)^{n-k}
    \end{equation}
    
    
%--------------- fin de las igualdades del binomio de Newton 

Recordemos que nuestro objetivo era acotar \ref{eq:Bernstein_caso_a_acotar}

Para ello vamos a sumar las dos expresiones que hemos obtenido
 \ref{eq:desarrollo_binomio_uno} y \ref{eq:desarrollo_binomio_dos}

 resultando 
 TODO MIRAR CÓMO LO AJUSTA EN LOS APUNTES
 \begin{equation} 
    (n^2 - n)  x^2 + nx= \sum_{k=0}^{n} ((k^2 - k)+k) \binom{n}{k} x^{k} (1-x)^{n-k}
  \end{equation}

  \begin{equation} 
    (n^2 - n)  x^2 + nx= \sum_{k=0}^{n} ((k^2 - k)+k) \binom{n}{k} x^{k} (1-x)^{n-k}
  \end{equation}


\end{proof}

 
 
\endinput 
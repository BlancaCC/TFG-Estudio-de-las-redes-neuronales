% !TeX root = ../../tfg.tex
% !TeX encoding = utf8
%
%***************************************************************
% Contenido del artículo 4: Colorario 2.1
%***************************************************************

% Nociones topológicas previas 

Antes de proceder con los siguientes resultados será necesario introducir nociones
básicas de topología. 

\textcolor{red}{Javier, He querido introducir unas nociones básicas de topología por si 
a los de informática les suena a chino lo que iba tratar. Pero tampoco he querido profundizar
introduciendo conceptos como el de topología, norma... ¿Dejo las siguientes definiciones como están? 
¿Las borro? ¿Las amplío?
}

\begin{definicion}[Vecino]
    Sea $(X, \mathcal{T})$ un espacio. Se define por vecino $U(x)$  de $x$ al conjunto de abiertos 
    $O \in \mathcal{T}$ que contenga a $x$. (Definición de pag 63 \cite{james1966topology})
\end{definicion}

En el espacio particular que estamos trabajando el espacio será $(\R^r, \mathcal{T}_u)$ con 
$\mathcal{T}_u$ la topología usual en $\R^r$,  es decir todos los intervalos abiertos

\begin{equation}
    \begin{split}
            I = \{ (a_1, b_1) \times ... \times (a_r, b_r) : 
            a_i < b_i,
            \quad
            a_i, b_i \in \R \cup \{-\infty,  +\infty \},
            \quad 
            i \in \{1,..., r\}
        \} 
        \\
        \mathcal{T}_u = \text{ El conjunto vacío o uniones numerables de elementos de } I. 
    \end{split}
\end{equation}

\begin{definicion}[Hausdorff]
    Un espacio $X$ es Hausdorff o separable si para cada dos puntos distinto existen vecinos disjuntos, esto es 
    sea $p,q \in X$ que  si $p \neq q$ existirán dos abierto $O_p \in U(p)$, $O_q \in U(q)$ satisfaciendo que 
    \begin{equation}
        O_p \cap O_q = \emptyset.
    \end{equation}
    (Definición en pag 137-138 \cite{james1966topology})

    En nuestro caso $(\R^r, \mathcal{T}_u)$ es Hausdorff ya 
    que suponiendo cualesquiera 
    $(x_1,..x_r),(y_1, ..., y_r) \in R^r$ diferentes
    podemos considerar $d = \frac{1}{3}\min_{i\in \{1,...,r\}}|x_i-y_i|$ entonces está claro que 
    denotando a 
    \begin{equation}
        (x-d, x+d) = (x_1-d, x_1+d) \times ...  \times (x_r-d, x_r+d).
    \end{equation}
    Se cumple que 
    $(x-d, x+d) \in U(x)$,
    $(y-d, y+d) \in U(y)$  y 
    \begin{equation}
        (x-d, x+d) \cap (y-d, y+d) = \emptyset.
    \end{equation}

    En el caso de que los abiertos estén formados por uniones numerables de intervalos, 
    podemos aplicar la misma idea encontrando el $d$ mínimo que satisfaga la misma idea.
\end{definicion}

\begin{definicion}[Espacio Hausdorff normal]
    Se dice que un espacio es Hausdorff normal si para cada par de conjunto cerrados  
    estos tienen vecinos disjuntos.(Definición en pag 144 \cite{james1966topology})

    Se tiene que $(\R^r, \mathcal{T}_u)$ es un espacio Hausdorff normal, ya que los cerrados de $\R$
    son de la forma $C = \R^r - O$ con $O \in \mathcal{T}_u$ 
\end{definicion}

\begin{teorema}(Caracterización de normalidad de Tietze)
    Sea $X$ un espacio Hausdorff. Son equivalentes las siguientes afirmaciones: 
    \begin{enumerate}
        \item $X$ es normal.
        \item Para conjunto cerrado $A \subset X$ y para cualquier función continua 
        $f: A \longrightarrow \R$, $f$ admite una extensión continua $F:X \longrightarrow \R.$
        Además, si para todo $a \in A$ se cumple que $|f(a)| < c \in \R$, se puede elegir $F$
        de tal forma que satisfaga que $|F(x)| < c$ para todo $x\in X.$ 
    \end{enumerate}
    (Demostración en páginas 149-151 de \cite{james1966topology})
\end{teorema}

% Corolarios del artículo 

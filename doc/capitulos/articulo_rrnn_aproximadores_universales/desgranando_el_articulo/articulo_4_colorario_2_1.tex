% !TeX root = ../../tfg.tex
% !TeX encoding = utf8
%
%***************************************************************
% Contenido del artículo 4: Colorario 2.1
%***************************************************************

% Resultado de teoría de la medida 
Trataremos ahora de generalizar la tesis expuesta en 
 el teorema \ref{teo:2_4_rrnn_densas_M} sobre las funciones medibles. 
 Para ello recordaremos el teorema de Lusin.
% Teorema de Lusin 
\begin{teorema}[Teorema de Lusin] \label{teo:Lusin}
    Si $\mu$ es una medida regular de Borel, $E$ un conjunto de medida finita 
    y $f$ una función medible en $E$ entonces
    para cualquier $\epsilon > 0$ existirá un conjunto compacto 
    $K$ en $E$ tal que $\mu(E \setminus C) \leq \epsilon$ y donde $f$ es continua en $K$. 
\end{teorema}
\begin{proof}
    Demostración en páginas 242 y 243 de \cite{nla.cat-vn1819421}.
\end{proof}  

Notemos los puntos clave de este teorema, nos va a permitir \textit{trabajar} con una función medible como si fuera continua en un compacto
todo lo parecido a $\R^r$ como se quiera. 

% Nociones topológicas previas 
Antes de proceder con los siguientes resultados será necesario introducir nociones
básicas de topología. 

\textcolor{red}{Javier, He querido introducir unas nociones básicas de topología por si 
a los de informática les suena a chino lo que iba tratar. Pero tampoco he querido profundizar
introduciendo conceptos como el de topología, norma... ¿Dejo las siguientes definiciones como están? 
¿Las borro? ¿Las amplío?
}

\begin{definicion}[Vecino]
    Sea $(X, \mathcal{T})$ un espacio. Se define por vecino $U(x)$  de $x$ al conjunto de abiertos 
    $O \in \mathcal{T}$ que contenga a $x$. (Definición de pag 63 \cite{james1966topology})
\end{definicion}

En el espacio particular que estamos trabajando el espacio será $(\R^r, \mathcal{T}_u)$ con 
$\mathcal{T}_u$ la topología usual en $\R^r$,  es decir todos los intervalos abiertos

\begin{equation}
    \begin{split}
            I = \{ (a_1, b_1) \times ... \times (a_r, b_r) : 
            a_i < b_i,
            \quad
            a_i, b_i \in \R \cup \{-\infty,  +\infty \},
            \quad 
            i \in \{1,..., r\}
        \} 
        \\
        \mathcal{T}_u = \text{ El conjunto vacío o uniones numerables de elementos de } I. 
    \end{split}
\end{equation}

\begin{definicion}[Hausdorff]
    Un espacio $X$ es Hausdorff o separable si para cada dos puntos distinto existen vecinos disjuntos, esto es 
    sea $p,q \in X$ que  si $p \neq q$ existirán dos abierto $O_p \in U(p)$, $O_q \in U(q)$ satisfaciendo que 
    \begin{equation}
        O_p \cap O_q = \emptyset.
    \end{equation}
    (Definición en pag 137-138 \cite{james1966topology})

    En nuestro caso $(\R^r, \mathcal{T}_u)$ es Hausdorff ya 
    que suponiendo cualesquiera 
    $(x_1,..x_r),(y_1, ..., y_r) \in R^r$ diferentes
    podemos considerar $d = \frac{1}{3}\min_{i\in \{1,...,r\}}|x_i-y_i|$ entonces está claro que 
    denotando a 
    \begin{equation}
        (x-d, x+d) = (x_1-d, x_1+d) \times ...  \times (x_r-d, x_r+d).
    \end{equation}
    Se cumple que 
    $(x-d, x+d) \in U(x)$,
    $(y-d, y+d) \in U(y)$  y 
    \begin{equation}
        (x-d, x+d) \cap (y-d, y+d) = \emptyset.
    \end{equation}

    En el caso de que los abiertos estén formados por uniones numerables de intervalos, 
    podemos aplicar la misma idea encontrando el $d$ mínimo que satisfaga la misma idea.
\end{definicion}

\begin{definicion}[Espacio Hausdorff normal]
    Se dice que un espacio es Hausdorff normal si para cada par de conjunto cerrados  
    estos tienen vecinos disjuntos.(Definición en pag 144 \cite{james1966topology})

    Se tiene que $(\R^r, \mathcal{T}_u)$ es un espacio Hausdorff normal, ya que los cerrados de $\R$
    son de la forma $C = \R^r - O$ con $O \in \mathcal{T}_u$ 
\end{definicion}

\begin{teorema}(Caracterización de normalidad de Tietze)\label{teo:Tietze}
    Sea $X$ un espacio Hausdorff. Son equivalentes las siguientes afirmaciones: 
    \begin{enumerate}
        \item $X$ es normal.
        \item Para conjunto cerrado $A \subset X$ y para cualquier función continua 
        $f: A \longrightarrow \R$, $f$ admite una extensión continua $F:X \longrightarrow \R.$
        Además, si para todo $a \in A$ se cumple que $|f(a)| < c \in \R$, se puede elegir $F$
        de tal forma que satisfaga que $|F(x)| < c$ para todo $x\in X.$ 
    \end{enumerate}
    (Demostración en páginas 149-151 de \cite{james1966topology})
\end{teorema}

Como el ambiente actual en el que estamos trabajando 
es el espacio $(R^r, \mathcal{T})$ que sabemos que es normal y puesto que es habitual que tratemos compactos en $R^r$, conjuntos cerrados y acotados podremos extender esta función al dominio entero. 


% Corolarios del artículo 
% Corolario 2.1
\textcolor{red}{Javier, revisa cuidadosamente este teorema, ya que he cambiado el enunciado del corolario respecto al del paper, porque 
diría que no tiene mucho sentido la otra formulación y 
 que tampoco es coherente con las indicaciones de demostración
 que dan.
 Por otra parte he simplificado su demostración, quizás 
 hay por tanto un matiz que se me haya escapado. 
 }
\begin{corolario} \label{cor:2_1}
    Para cada función $g \in \fM$ existe un subconjunto compacto 
    $K$ de $\R^r$ y $f \in \rrnng$ tal que para cualquier 
    $\epsilon > 0$ se tiene que 
    \textcolor{red}{$\mu(K) > 1- \epsilon$} y para cada $x \in K$ tenemos que 
    \begin{equation}
        |f(x) - g(x) | < \epsilon,
    \end{equation}
    independientemente de la función de activación $\psi$, dimensión $r$ o medida $\mu$. 
\end{corolario}
\begin{proof}
    Sea $\epsilon > 0$ fijo pero arbitrario.  Gracias al teorema de Lusin \ref{teo:Lusin}
    existe un subconjunto compacto $K \subset \R^r$ de medida
    $\mu(K) > 1 - \epsilon$ donde la restricción  $g_{|K}$ es continua. 

    Por otra parte, en virtud de la caracterización de Tietze 
    \ref{teo:Tietze} 
    Por estar $g_{|K}$ definida en un compacto admite una 
    extensión continua $G:\R^r \longrightarrow \R$ tal que 
    \begin{equation}
        \begin{split}
            g_{|K} = G_{|K} .
    %        \text{ y } 
    %        \sup_{k \in K} |g(k)| 
    %        = 
    %        \sup_{x \in \R^r} |G(x)|
        \end{split}
    \end{equation}

    Por ser $G$ continua en un compacto, por la densidad de las redes neuronales en compactos de $\fC$(lema \ref{lema:A_5_uniformemente_denso_compactos} ) se tiene que existirá 
    una $f \in \rrnng$ tal que 
    \begin{equation}
        \sup_{x \in K} |G(x) - f(x)| < \epsilon.
    \end{equation}

    Por lo que podemos afirmar que para todo $x \in K$
    \begin{equation}
        |f(x) -g(x)| 
        \leq 
        | f(x) -G(x)| + |G(x) -g(x)|
        < \epsilon + 0 = \epsilon
    \end{equation}

    como queríamos probar.
\end{proof}

\subsubsection{Comentarios sobre el corolario \ref{cor:2_1}}  

Este resultado nos indica que existe una red neuronal con una 
capa oculta capaz de aproximar cualquier función medible con el grado 
de precisión que se desee dentro de un compacto.   

Notemos que la diferencia que presenta el corolario recién probado
\ref{cor:2_1} con respecto al teorema \ref{teo:2_4_rrnn_densas_M}
es que en el corolario se está fijando el compacto y esto nos 
permite tener convergencia uniforme en él, mientras que en el 
teorema \ref{teo:2_4_rrnn_densas_M} para funciones medible
tendremos asegurada tan solo la convergencia. 

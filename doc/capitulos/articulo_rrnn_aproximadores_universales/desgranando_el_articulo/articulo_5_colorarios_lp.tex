% !TeX root = ../../tfg.tex
% !TeX encoding = utf8
%
%***************************************************************
% Contenido del artículo 5: Colorarios LP
%***************************************************************
\begin{definicion}[Espacios Lp]
    Se llama espacio $L_p(\R^r, \mu)$ o simplemente $L_p$ al conjunto 
    de funciones $f \in \fM$ tales que 
    \begin{equation}
        \int |f(x)|^p d\mu < \infty. 
    \end{equation}

Se define la norma de $L_p$ como 
\begin{equation}
    \| f\|_p 
    =
    \left(\int |f(x)|^p d\mu \right)^\frac{1}{p}.
\end{equation}

La distancia asociada al espacio $L_p$ se define como 
\begin{equation}
    \rho_p(f,g) = \| f-g\|_p.
\end{equation}
\end{definicion}

% Corolario 2.2
\begin{corolario}
    Si existe un subconjunto compacto $K$ en $\R^r$ de medida
    $\mu(K) =1$ entonces $\rrnn$ es $\dlp$-denso en $L_p(\R^r, \mu)$
    para cualquier $p \in [1,\infty)$, independientemente de 
    $\psi$, $r$ o $\mu$.
\end{corolario}
\begin{proof}
    Se quiere probar que para cualquier $g \in L_p$ y 
    $\epsilon >0$ existe un $f \in \rrnn$ tales que 
    \begin{equation}
        \dlp(f,g) <\epsilon.
    \end{equation}   
    
    Por pertenecer $g$ a $L_p$ existe una constante $M$ real positiva
    los suficientemente grande 
    tal que si definimos la función $h =g 1_{|g|<M}$ esta satisface 
    que
    \begin{equation}\label{eq:corolario_2_2:h_compacto}
        \dlp(g,h) < \frac{\epsilon}{3}.
    \end{equation}
    
    Además como $h$ es una función acotada de $L_p$, podemos encontrar
    una función $s$ continua que es límite de una sucesión de
    funciones simples 
    ( pag 241-242,  teoremas 55C y 55D \cite{nla.cat-vn1819421})
    y la cual cumple que 

    \begin{equation}\label{eq:corolario_2_2:s_continua}
        \dlp(h,s) < \frac{\epsilon}{3}.
    \end{equation}

    Por el teorema \ref{teo:2_4_rrnn_densas_M}, al estar en un compacto $K$ y por ser $\rrnn$ uniformemente
    denso en compactos hay una $f \in \rrnn$ la cual cumple que
    \begin{equation}
        \sup_{x \in K} |f(x) -s(x)|^p 
        <
         \left( \frac{\epsilon}{3}\right) ^p.
    \end{equation}
    
    Y por hipótesis $\mu(K) =1$ y definición de la distancia $\dlp$ 
    se tiene la siguiente desigualdad: 

    \begin{equation} \label{eq:corolario_2_2:cota_rrnn}
        \dlp(f,s) = 
        \left(\int |f(x) - s(x)|^p d\mu \right)^\frac{1}{p}
        \leq 
        \left(\int  \left( \frac{\epsilon}{3}\right) ^p d\mu \right)^\frac{1}{p}
        = \left( \mu(K)  \left(\frac{\epsilon}{3} \right)^p\right) ^\frac{1}{p}
        = \frac{\epsilon}{3}.
    \end{equation}

    Gracias a la desigualdad triangular y las desigualdades
    (\refeq{eq:corolario_2_2:cota_rrnn})
    (\refeq{eq:corolario_2_2:h_compacto})
    (\refeq{eq:corolario_2_2:s_continua})

    \begin{equation}
        \dlp(f,g) 
        \leq
            \dlp(f,s)
            +\dlp(s,h)
            + \dlp(h,g)
        < 
        \frac{\epsilon}{3} + \frac{\epsilon}{3} + \frac{\epsilon}{3}
        = \epsilon.
    \end{equation}
Probando con ello lo buscado. 
\end{proof}  


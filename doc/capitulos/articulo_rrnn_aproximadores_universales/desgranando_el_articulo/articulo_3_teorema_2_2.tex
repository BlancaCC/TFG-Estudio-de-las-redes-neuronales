% !TeX root = ../../tfg.tex
% !TeX encoding = utf8
%
%***************************************************************
% Contenido del artículo 3: Avanzamos en la generalización
%***************************************************************

% Teorema 2.2 
\begin{teorema}
    Para cualquier función continua no constate $G$, $r \in \N$ y
    medida de probabilidad $\mu$ o $(\R^r, B^r)$, 
    se tiene que $\pmcg$ es $\dist$-denso en $\fM$. 
\end{teorema} 
\begin{proof}
    Debemos probar que para cualquier función $f \in \fM$ existe una 
    sucesión de funciones $\{h_n\}_{n\in \N}$ contenida en $\pmcg$ y 
    cumpliendo que $\dist(h_n, f) \longrightarrow 0.$
    
    Consideramos cualquier $f \in \fM$,
    por el lema \ref{lema:A_1_C_es_denso_en_M} sabemos que $\fC$ es $\dist$-denso en $\fM$; 
    es decir, existirá un sucesión $\{f_n\}_{n\in \N}$ de funciones de $\fC$ convergente a 
    $f$.  
    
    Por otra parte sabemos por el teorema \ref{teo:TeoremaConvergenciaRealEnCompactosDefinicionesEsenciales}, 
    que $\pmcg$ es uniformemente denso por compactos en $\fC$, luego en cualquier compacto 
    $K \subset \R^r$ existirá una sucesión (con $n$ fijo) $\{g(n)_m \} _{m \in \N}$ convergente 
    a $f_n$, el término n-ésimo de la sucesión convergente a $g$. 

    Así pues, denotando como $h_n$ al término $g(n)_n$, obtenemos una sucesión de funciones 
    en $\fM$ que converge uniformemente en compactos a $f$ y por el lema \ref{lema:2_2_convergencia_uniforme_en_compactos}
    tenemos que $\dist(h_n, f) \longrightarrow 0$ como queríamos probar. 

    
\end{proof}

% --- Faltan por demostrar -----
% Teorema 2.3
\begin{teorema}
    Para cualquier función de activación $\psi$, $r \in \N$ y
    medida de probabilidad $\mu$ o $(\R^r, B^r)$, 
    se tiene que $\rrnng$ es uniformemente denso en compactos
    en $\fC$ y denso en $\fM$ de acorde a la distancia $\dist$. 
\end{teorema}

% Teorema 2.4
\begin{teorema}
    Para cualquier función de activación $\psi$, $r \in \N$ y
    medida de probabilidad $\mu$ o $(\R^r, B^r)$, 
    se tiene que $\rrnn$ es uniformemente denso en compactos
    en $\fC$ y denso en $\fM$ de acorde a la distancia $\dist$. 
\end{teorema}
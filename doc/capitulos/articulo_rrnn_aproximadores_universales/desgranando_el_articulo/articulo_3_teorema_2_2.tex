% !TeX root = ../../tfg.tex
% !TeX encoding = utf8
%
%***************************************************************
% Contenido del artículo 3: Avanzamos en la generalización
%***************************************************************

% Teorema 2.2 
\begin{teorema}\label{teo:2_2_denso_funcion_continua}
    Para cualquier función continua no constate $G$, $r \in \N$ y
    medida de probabilidad $\mu$ o $(\R^r, B^r)$, 
    se tiene que $\pmcg$ es $\dist$-denso en $\fM$. 
\end{teorema} 
\begin{proof}
    Debemos probar que para cualquier función $f \in \fM$ existe una 
    sucesión de funciones $\{h_n\}_{n\in \N}$ contenida en $\pmcg$ y 
    cumpliendo que $\dist(h_n, f) \longrightarrow 0.$

    Consideramos cualquier $f \in \fM$,
    por el lema \ref{lema:A_1_C_es_denso_en_M} sabemos que $\fC$ es $\dist$-denso en $\fM$; 
    es decir, existirá un sucesión $\{f_n\}_{n\in \N}$ de funciones de $\fC$ convergente a 
    $f$.  
    
    Por otra parte sabemos por el teorema \ref{teo:TeoremaConvergenciaRealEnCompactosDefinicionesEsenciales}, 
    que $\pmcg$ es uniformemente denso por compactos en $\fC$, luego en cualquier compacto 
    $K \subset \R^r$ existirá una sucesión (con $n$ fijo) $\{g(n)_m \} _{m \in \N}$ convergente 
    a $f_n$, el término n-ésimo de la sucesión convergente a $g$. 

    Así pues, denotando como $h_n$ al término $g(n)_n$, obtenemos una sucesión de funciones 
    en $\fM$ que converge uniformemente en compactos a $f$ y por el lema \ref{lema:2_2_convergencia_uniforme_en_compactos}
    tenemos que $\dist(h_n, f) \longrightarrow 0$ como queríamos probar.     
\end{proof}

% --- Faltan por demostrar -----
% Lema A.2 
\begin{lema}\label{lema:a_2_paso_previo_denso}
    Sea F una función de activación continua y $\psi$ una función de activación arbitraria. 
    Para cualquier $\epsilon > 0$ existe un elemento $H_{\epsilon}$ de $\sum^1(\psi)$ cumpliendo que
    \begin{equation}
        \sup_{\lambda \in \R} | F(\lambda) - H_{\epsilon}(\lambda) | < \epsilon.
    \end{equation}
\end{lema} 
\begin{proof}
    TODO
\end{proof}      

% Teorema 2.3
\begin{teorema}
    Para cualquier función de activación $\psi$, $r \in \N$ y
    medida de probabilidad $\mu$ o $(\R^r, B^r)$, 
    se tiene que $\rrnng$ es uniformemente denso en compactos
    en $\fC$ y denso en $\fM$ de acorde a la distancia $\dist$. 
\end{teorema}
\begin{proof}
    En virtud del lema \ref{lema:2_2_convergencia_uniforme_en_compactos} y del 
    teorema \ref{teo:2_2_denso_funcion_continua} basta con probar que 
    $\rrnng$ es uniformemente denso en compactos de $\sum \prod^r(F)$, 
    donde $F$ es una función de activación continua 
    \footnote{el razonamiento 
    por el que con esta hipótesis es suficiente es idéntico al realizado para la 
    demostración del teorema \ref{teo:2_2_denso_funcion_continua}}.

    Para ello basta ver que cualquier función de la forma $\prod_{k=1}^l F(A_k(\cdot))$
    puede ser uniformemente aproximada por una una sucesión de funciones de $\rrnng$.

    Fijamos un $\epsilon > 0$  de manera arbitraria. 
    Gracias a la continuidad de la norma y de la operación multiplicación, existirá un $\delta >0$
    tal que para cualesquiera números reales $0 \leq a_k, b_k \leq 1,$ con $k \in \{1,...,l\}$ 
    se satisfagan que $|a_k -b_k| < \delta$ se cumpla que 
    \begin{equation} \label{eq:teorema_2_3__1}
        \left| 
            \prod^l_{k=1} a_k - \prod^l_{k=1} b_k 
        \right| 
        < 
        \epsilon.
    \end{equation}

    Por el lema \ref{lema:a_2_paso_previo_denso} existe una función 
    $ H_{\delta}(\cdot) = \sum_{t=1}^T \beta_t \psi(A_t(\cdot))$
    cumpliendo que 

    \begin{equation}
        \sup_{\lambda \in \R} |F(\lambda) - H_{\delta}(\lambda) | < \delta.
    \end{equation}

    que satisface las hipótesis de la desigualdad \refeq{eq:teorema_2_3__1} por lo que 
    resulta 
    \begin{equation}
        \sup_{x \in \R^r} 
        \left| 
            \prod ^l_{k=1} F(A_k(x))
            -
            \prod ^l_{k=1} H_\delta(A_k(x))
        \right| 
        < 
        \epsilon.
    \end{equation} 
    
    TODO : Falta por acabar la demostración.
\end{proof} 

%% Falta por probar 
% Teorema 2.4
\begin{teorema}
    Para cualquier función de activación $\psi$, $r \in \N$ y
    medida de probabilidad $\mu$ o $(\R^r, B^r)$, 
    se tiene que $\rrnn$ es uniformemente denso en compactos
    en $\fC$ y denso en $\fM$ de acorde a la distancia $\dist$. 
\end{teorema}
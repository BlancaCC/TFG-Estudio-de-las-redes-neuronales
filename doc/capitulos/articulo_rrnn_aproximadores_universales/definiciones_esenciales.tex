% !TeX root = ../../tfg.tex
% !TeX encoding = utf8
%
%*******************************************************
% Introducción artículo MFNAUA
%*******************************************************


\section{Definiciones esenciales}  

Se pretende con las siguientes definiciones una concepción y trabajo preciso de la clase de redes neuronales multicapa feedforward. 

\begin{definicion} Función afín

    Para cualquier natural mayor que cero $r$  denotaremos por $A^r$ como al conjunto de todas 
    las funciones afines de $\R^r$ a $\R$. Es decir el conjunto de funciones de la forma 
    $A(x) = w \cdot x + b$ donde $x$ y $w$ son vectores de $\R^r$ y $\cdot$ representa el producto 
    usual de escalares y $b \in \R$ es un escalar.  
    
\end{definicion}  

En este contexto, $x$ corresponde al vector entrada de la red neuronal, $w$ los pesos de la red
que se multiplicarán con $x$ en la capa intermedia y $b$ el sesgo. 
Nótese  además que a falta de una función de activación, se está describiendo a un perceptrón.

\begin{definicion} Función de activación \\
    Una función  $\psi: \R \longrightarrow [0,1]$ es una \textbf{ función de activación} si  cumple las siguientes propiedad
    \begin{enumerate}[label=(\roman*)]
        \item Es no decreciente
        \item $\lim _{x \rightarrow \infty} \psi(x) = 1
        $.
        \item $\lim _{x \rightarrow -\infty} \psi(x) = 0$.
    \end{enumerate}  

    Las funciones de activación son medibles ya que tienen como mucho un número contable de discontinuidades.
   
    Ejemplos comunes son
    \begin{itemize}
        \item Funciones umbral. 
        (Vale cero hasta que se cumple cierta condición de las entradas, a partir de ahí empezaría a valer uno).

        \item Funciones indicadoras: $\psi(\lambda) = 1_{\{\lambda > 0\}}$. 
        \item Ramp function: $\psi(\lambda)  = \lambda 1_{\{0 \leq \lambda \leq  1\}} + 1_{\{\lambda > 1\}}$
    
        \item La función \textit{cosine squasher} de Gallant and White (1988)
        \begin{equation*}
    \psi(lambda )= (1 + cos(\lambda + 3 \frac{\pi}{2}) \frac{1}{2}) 
     1_{\{\frac{-\pi}{2} \leq \lambda \leq  \frac{\pi}{2}\}}
     1_{\{ \frac{\pi}{2} < \lambda \}}
    \end{equation*}
    \end{itemize}
    
\end{definicion}
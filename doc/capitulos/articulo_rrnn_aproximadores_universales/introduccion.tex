% !TeX root = ../../tfg.tex
% !TeX encoding = utf8
%
%*******************************************************
% Introducción artículo MFNAUA
%*******************************************************

\chapter{Las redes neuronales multicapa son aproximadores universales}  

\section{Introducción}  

Tras las definiciones y construcciones expuestas durante el capítulo de presentación 
y construcción de las redes neuronales  \ref{ch:Aprendizaje}
es pertinente la pregunta de si todas las estructuras y técnicas presentadas solventan nuestro 
objetivo principal: Aproximar
con éxito una función genérica desconocida.   

Aunque las redes neuronales multicapa ya se venían aplicando con anterioridad, 
véase por ejemplo los usos expuestos durante la primera conferencia
internacional de redes neuronales de \cite{4307059} de 1987, 
no fue hasta 1989 que se descubrió formalmente su alcance.
 Tal delimitación se propuso en el artículo 
\textbf{Multilayer Feedforward Networks are Universal Approximators} \cite{HORNIK1989359}
 escrito por Kurt Hornik, Maxwell Stinchcombe y Halber White enunciando 

\begin{teorema}\textbf{Las redes \textit{feedforward} multicapa son una clase de aproximadores universales } \label{teo:MFNAUA}
    \\
    Una red neuronal \textit{feedforward} multicapa estándar con tan solo una capa oculta y con una función de activación cualquiera es capaz de aproximar cualquier 
    función Borel medible  con dominios y codominios de dimensión finita (no necesariamente iguales) y con el nivel de precisión que se desee siempre y cuando 
    se utilicen suficientes neuronas. En este sentido las redes \textit{feedforward} multicapa son una clase de aproximadores universales.

\end{teorema}

En las secciones siguientes, con el fin de alcanzar una comprensión profunda de las redes neuronales,
trataremos de desgranar y profundizar en el artículo y su demostración. 


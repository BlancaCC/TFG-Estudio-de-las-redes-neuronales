% !TeX root = ../../tfg.tex
% !TeX encoding = utf8
%
%*******************************************************
% Observaciones artículo MFNAUA
%*******************************************************

\section{Conclusiones y observaciones} 

% Son solo reflexiones que no quiero que se muestren 
\iffalse 
A falta de completar el estudio procedente del teorema \ref{teo:MFNAUA}, 
dejo reflejadas algunas observaciones. 

\subsection{Sobre la dimensión de dominio y codominio} 

Consecuencias sobre cuando en \ref{teo:MFNAUA} se establece que el dominio y codominio son finitos. 

La bondad de que el codominio sea finito depende del objetivo de función que queramos aproximar: 
\begin{itemize}
    \item Si la función pretende ser entendida o manejable es necesario y \textit{natural} su finitud. 
    \item Si por el contrario pretende abarcar alguna 
construcción matemática más abstracta le sería imposible ¿Tienen una aplicación práctica tales definiciones?
\item Para la definición del dominio ocurre la misma reflexión, la disyuntiva entre tratabilidad humana y la máxima generalización de un concepto. 
¿Se podría intentar analizar si los espacios se puede ampliar?  
\end{itemize}
\fi

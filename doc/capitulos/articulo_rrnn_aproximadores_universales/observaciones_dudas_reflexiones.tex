% !TeX root = ../../tfg.tex
% !TeX encoding = utf8
%
%*******************************************************
% Observaciones artículo MFNAUA
%*******************************************************

\section{Observaciones, reflexiones y dudas} 

Observaciones, reflexiones y dudas procedentes del teorema \ref{teo:MFNAUA}. 

\subsection{Sobre la dimensión de dominio y codominio} 

Consecuencias sobre cuando en \ref{teo:MFNAUA} se estable que el dominio y codiminio son finitos. 

La bondad de que el codiminio sea finito depende del objetivo de función que queramos aproximar: 
Si la función pretende ser entendida o manejable es necesario y \textit{natural} su finitud. 
Si por el contrario pretende abarcar alguna 
construcción matemática más abstracta le sería imposible ¿Tienen una aplicación práctica tales definiciones?

Para la definición del dominio ocurre la misma reflexión, la disyuntiva entre tratabilidad humana y la máxima generalización de un concepto. 
¿Se podría intentar analizar si los espacios se puede ampliar?  

\subsection{Número de capas ocultas}

Como ya he expuesto varias veces, me gustaría analizar las relaciones entre el número de capas y algunos de estos conceptos: 
\begin{itemize}
    \item   \textit{Velocidad} de convergencia. 
    \item Relación con una capa oculta. Por ejemplo si se podría establecer una relación de que tantas neuronas en una capa equivalen a tantas otras en dos capas con cada capa tantas. 
    \item Consecuencia en sobreentrenamiento. 
\end{itemize}

\subsection{Borel medible}  
¿Qué otras nociones de medir existen? ¿Qué ocurre en estas otras? 

\subsection{Número de neuronas y precisión} 

¿Se podría establecer una cotar de error al aumentar el número de neuronas? 
(Esto creo que se podría hacer con la cota que ofrecen los polinomios de Bernstein)
Análisis de la velocidad de esa cota. 
¿Cómo varía al tener varias capas ocultas? 



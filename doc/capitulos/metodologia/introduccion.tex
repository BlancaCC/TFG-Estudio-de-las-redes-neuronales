% !TeX root = ../../tfg.tex
% !TeX encoding = utf8
%
%*******************************************************
% Metodología
%*******************************************************

\chapter{Metodología}

La planificación y organización es un componente vital a la hora del desarrollo de software 
y la ciencia de datos. Por lo que no es de extrañar que sus beneficios sean extrapolables
a otras áreas de la ciencia; 
hemos aplicado por ende la 
metodología expuesta en el artículo \cite{DBLP:journals/corr/abs-2104-12545}, realizado 
el trabajo siguiendo una metodología ágil, de acorde a las siguientes hipótesis. 

\begin{enumerate}
    \item Se premia la \textbf{reproducibilidad}, \textit{la ciencia no puede ser ciencia si no es reproducible}. Es por 
    ello que se publica todo el material usado. 
    \item \textbf{Comprobaciones} en todo los niveles, \textit{el código que no se ha probado no funciona}. 
    Explicitamos unos requisitos y criterios claros de validación. 
    \item Conocimiento libre, tanto la memoria como el código se encuentra publicado y con licencia libre en nuestro repositorio 
    de github \cite{TFG-Estudio-de-las-redes-neuronales}.

    \textcolor{red}{ JJ, el cuarto punto de tu \cite{DBLP:journals/corr/abs-2104-12545} artículo: "Stakeholder collaboration over vertical chains-of-command" 
    no lo he puesto porque no creo que encaje con la situación, el rol profesor-alumno necesariamente tiene cierto 
    enfoque vertical, no por la autoridad; sino por la desigualdad de conocimientos, conduciendo a esa relación "desigual".
     }
\end{enumerate}  

\section{Descripción de la metodología }  

Concretamente la metodología seguida basada en \cite{que-es-un-trabajo-fin-de-x} ha sido la siguiente: 

El objetivo principal es resolver un problema usando técnicas tanto informáticas como matemáticas. Para ello 
se han definido unos objetivos claros a través de una \textit{metodología de personas}  \cite{personas-why-and-how-you-should-use-them}
y las respectivas historias de usuario derivadas.   

\subsection{Personas definidas}  

Las personas que hemos definido son:  

\begin{itemize}
    \item \textbf{Perfil de usuario Tribunal}

Lector experto  de la memoria y trabajo. Valorará que el trabajo presentado  cumpla ciertos requisitos inamovibles.
Puesto que para el doble grado no he encontrado nada específico procedo a combinar los requisitos encontrado para los grados:

\begin{enumerate}
    \item Declaración explícita firmada en la que se asume la originalidad del trabajo, entendida en el sentido de que no ha utilizado fuentes sin citarlas debidamente. Esta declaración se puede descargar en la web del Grado (matemáticas).
    \item Un índice detallado de capítulos y secciones.
    \item Un resumen amplio en inglés del trabajo realizado (se recomienda entre 800 y 1500 palabras).
    \item Una introducción en la que se describan claramente los objetivos previstos inicialmente en la propuesta de TFG, indicando si han sido o no alcanzados, los antecedentes importantes para el desarrollo, los resultados obtenidos, en su caso, y las principales fuentes consultadas.
    \item Una descripción de la metodología.
    \item Un estado del arte y diseño.
    \item Estimación de los costes.
    \item Análisis de las prestaciones.
    \item Una bibliografía final que incluya todas las referencias utilizadas.
    \item Para realizar la entrega cada estudiante: Como tarea de la asignatura Trabajo Fin de Grado, subirá a PRADO su memoria de TFG en un fichero pdf (Apellidos.pdf) y lo someterá a la herramienta de control de plagio Turnitin.

\end{enumerate}


\item \textbf{Perfil usuario: Benefactor del problema}

Busca una mejora de las redes neuronales en cualquier aspecto:

\begin{itemize}
    \item Necesidad de arquitecturas menos potentes.
    \item Mayor precisión para misma arquitectura.
    \item Resolver nuevos problemas.
\end{itemize}

\item \textbf{Perfil usuario: Matemático}
Busca una memoria y un estudio formal, exacto  y exhaustivo, que involucra aspectos como: 
\begin{itemize}
    \item Definir de manera rigurosa los conceptos.
    \item Abstracción de las técnicas informáticas.
    \item Planteará cuestiones que puedan tener relevancia en cuanto a optimizar RRNN.
    \item Comprobará la bondad de sus especulaciones realizando benchmark.
\end{itemize}
\end{itemize}

\subsection{Historias de usuario}  

A partir de los usuarios se han definido las historias de usuario, que han quedado registradas 
como \textit{issue} etiquetado con \textit{user story} y cabecera de formato
\texttt{[HUxx] título de la historia de usuario} en nuestro 
repositorio de Github \cite{TFG-Estudio-de-las-redes-neuronales},
 donde \texttt{HUxx} representa
la historia de usuario número $xx$.   

\subsubsection{Historias de usuario definidas}

\subsubsection*{[HU01] Optimización redes neuronales.}
    Como usuario benefactor me gustaría que se encontrara alguna optimización en las redes neuronales, ya sea:
\begin{itemize}
    \item  Necesidad de arquitecturas menos potentes.
    \item  Mayor precisión para misma arquitectura.
    \item  Resolver nuevos problemas.
\end{itemize}

\subsubsection*{[HU02] Requisitos tribunal}
Como tribunal de TFG la memoria debe de cumplir una serie de requisitos imprescindibles:

\begin{enumerate}
    \item Declaración explícita firmada en la que se asume la originalidad del trabajo, entendida en el sentido de que no ha utilizado fuentes sin citarlas debidamente. Esta declaración se puede descargar en la web del Grado (matemáticas).
    \item Un índice detallado de capítulos y secciones.
    \item Un resumen amplio en inglés del trabajo realizado (se recomienda entre 800 y 1500 palabras).
    \item Una introducción en la que se describan claramente los objetivos previstos inicialmente en la propuesta de TFG, indicando si han sido o no alcanzados, los antecedentes importantes para el desarrollo, los resultados obtenidos, en su caso, y las principales fuentes consultadas.
    \item Una descripción de la metodología.
    \item Un estado del arte y diseño.
    \item Estimación de los costes.
    \item Análisis de las prestaciones.
    \item Una bibliografía final que incluya todas las referencias utilizadas.
\end{enumerate}


\subsubsection*{ [HU03] Calidad de la memoria}
Como matemático me gustaría que la memoria fuera un estudio de calidad de las redes neuronales, 
respondiendo de manera autocontenida, clara, rigurosa y precisa a preguntas como:
\begin{itemize}
    \item ¿Qué son las redes neuronales?
    \item ¿Cómo se construyen?
    \item ¿Para qué sirven?
    \item ¿Por qué funcionan?
\end{itemize}

\subsubsection*{ [HU04] Indagar sobre redes neuronales}

Como matemático me gustaría plantear nuevas hipótesis para conocer en mayor profundidad, 
aclarar limitaciones, particularizar o abstraer en cualquier aspecto de las redes neuronales actuales.


%%%% fin historias de usuario %%%%

Estas historias de usuario dan lugar a los \textit{milestones}. 

\subsection{Milestones}  

Los milestones describen productos mínimos viables, nuestro caso hemos definido los siguientes: 

\begin{enumerate}
    \item Recopilación de bibliografía. 
    \item Metodología. Apartado que consiste en la planificación del proyecto. 
    \item Estudio de las redes neuronales. Definición y teorema de convergencia. 
    \item Fase de planteamiento de hipótesis y experimentación. 
\end{enumerate}  



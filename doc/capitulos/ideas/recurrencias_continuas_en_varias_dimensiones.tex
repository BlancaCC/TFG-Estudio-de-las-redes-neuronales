% !TeX root = ../../tfg.tex
% !TeX encoding = utf8
%
\section{Recurrencias continuas en varias variables}\label{ch:ideas_recurrencias_continuas_en_varias_variables}

\subsection{Idea}

Como hemos visto en la construcción de los polinomios de 
Bernstein \ref{ch:Bernstein}, para construir aproximaciones se necesitan puntos 
y cada punto nuevo a añadir añade complejidad al polinomio construido.

¿Es posible construir una función convergente de complejidad constante?

Combinando esta pregunta con el concepto de recurrencia discreta se me ocurrió definir 
la siguiente estructura. 

La dejo escrita simplemente porque me parece curiosa y quizás nos pueda ayudar en algún momento, 
no he buscado si alguien ha definido ya algo parecido con anterioridad, probablemente si es de utilidad sí. 

\subsection{Explicación}  

Comenzaré presentando la estructura con ejemplo:  

Supongamos que en el plano euclídeo se tiene definida una curva $\sigma_0$, 
que sin pérdida de generalidad restringe su dominio del intervalo $[0,1]$ al plano. 
Se define pues a la función recurrente continua de la curva $\sigma_0$ $\mathcal S_{\sigma_0}$
como la suma infinita numerable de
 las curvas $\sigma_n :[0,1] \longrightarrow \R^2$ con $n$ natural y construido de manera recursiva como
 $$\sigma_n(t) = \sigma_{n-1}(1) + \sigma_0(t) - \sigma_0(0)$$.  

 es decir $\mathcal S_{\sigma_0}(n + t) = \sigma_n(t)$  con $n$ natural y $t \in [0,1]$.

 Un ejemplo sencillo de esto sería el mostrado en la figura \ref{img:idea_recurrencia_ejemplo_sencillo}.  
 \begin{figure}[h!]
 \includegraphics[width=\textwidth]{ideas/recurencias_continuas/ejemplo_sencillo.jpg}
 \caption{Ejemplo sencillo de como quería los tres primeros elementos de la sucesión de sigmas.}
 \label{img:idea_recurrencia_ejemplo_sencillo}
\end{figure}

Por cómo se ha definido es fácil generalizar la construcción para que la curva se encuentre en 
espacios de mayor dimensión. 

Quizás a priori pueda no parecer interesante esta estructura, pero introduzcamos 
de manera informal nuevas restricciones a nuestro problema: 

Dado un plano se quiere construir una función que pase por los "puntos rojos" evitando los negros 
\ref{img:idea_recurrencia_plano}: 

\begin{figure}[h!]
    \includegraphics[width=\textwidth]{ideas/recurencias_continuas/ejemplo_de_plano.jpg}
    \caption{Plano del que evitar los puntos negros y tomar los rojos.}
    \label{img:idea_recurrencia_plano}
\end{figure}

Se podría resolver el problema valiéndose del polinomio de Bernstein o splines o
cualquier otro método, llegando a una función , como la dibujada en verde en la
figura \ref{img:idea_recurrencia_funcion_verde}

\begin{figure}[h!]
    \includegraphics[width=\textwidth]{ideas/recurencias_continuas/funcion_continua.jpg}
    \caption{En verde ejemplo de función solución.}
    \label{img:idea_recurrencia_funcion_verde}
\end{figure} 

Sin embargo si hacemos uso de la estructura recién descrita, bastaría con 
calcular solo $\sigma_0$.
 Ver \ref{img:idea_recurrencia_sigma_cero}.  

\begin{figure}[h!]
    \includegraphics[width=\textwidth]{ideas/recurencias_continuas/recurrente.jpg}
    \caption{En verde ejemplo de función solución.}
    \label{img:idea_recurrencia_sigma_cero}
\end{figure} 


\subsection{ Preguntas que introduce esta estructura}

1. Nótese que ha sido necesaria cierta simetría o patrón en la estructura del ejemplo
¿Se podría generalizar? 
La intuición me dice que sí.
Una forma de generalizarlo que se me ocurre y aún no he formalizado
 sería "añadir puntos negros y rojos auxiliares que den lugar al patrón buscado"
 como si calculáramos un mínimo común múltiplo geométrico.  





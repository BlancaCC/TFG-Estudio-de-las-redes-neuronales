% !TeX root = ../../tfg.tex
% !TeX encoding = utf8
%
%*******************************************************
% Construcción redes neuronales  
%*******************************************************
\section{Construcción de las redes neuronales \textit{Feedforward Networks}}

A lo largo de esta sección  explicaremos qué es una red neuronal, cómo está construida y en qué consiste el \textit{aprendizaje} de la misma, concretamente
construiremos el tipo particular \textit{Feedforward Networks}, al cual nos referiremos de ahora
en adelante como red neuronal.

\subsection{Modelo básico de una red neuronal} \label{rrnn:modelo_simple_rrnn}  

Con el fin de entender mejor cómo se definen y construyen las redes neuronales y porque en nuestro trabajo nos hemos centrado en estudiar
esta clase concreta de redes neuronales,
comenzaremos 
introduciendo las de una sola capa oculta. 
La información que se va a desarrollar a lo largo de esta 
sección proviene principalmente del capítulo cinco, páginas 227-256 del libro \cite{BishopPaterRecognition} y las notas online sobre redes neuronales de \cite{MostafaLearningFromData}.


\subsubsection*{Construcción de la primera capa}
La primera capa está compuesta por el conjunto de $M$ combinaciones
lineales del vector de entrada $(x_1, \ldots, x_d)$
a las cuales denominaremos \textit{activaciones}

\begin{equation}
    a_j = \sum_{i=1}^D w_{ji}^{(1)} x_i + w_{j0}^{(1)}
    \text{ con } j \in \{1, \ldots, M \}.
\end{equation}
El superíndice (1) indica que los parámetro $w$ correspondientes pertenecen a la primera capa. 
Nos referiremos a los  parámetros $w_{ji}^{(1)}$ como 
\textit{pesos} y al parámetro $w_{j0}^{(1)}$ como 
\textit{sesgo}.  

\subsubsection*{Unidades ocultas}
Cada una de esas \textit{activaciones} será transformada
utilizando una \textit{función de activación} $\sigma_j$ 
diferenciable y no linear

\begin{equation}
    z_j = \sigma_j(a_j).
\end{equation}
En el contexto de las redes neuronales a $z_j$ se le conoce como \textit{unidad oculta}. Ésta  podría ser de 
nuevo  transformada por una combinación lineal 
\begin{equation}
    a_k = \sum_{i=1}^M w_{ji}^{(2)} z_i + w_{k0}^{(2)}
    \text{ con } j \in \{1, \ldots, K \}.
\end{equation}
Nótese que ahora el tamaño de variables de entrada es $M$
y hay un total de $K$ unidades de activación, tanto $M$ como $K$ son
valores fijados por el diseñador de la red. 
Finalmente se define como \textbf{red neuronal con una capa oculta} $h_w \in \mathcal{H}_{D \times M \times K}$, con $h=(y_1, \ldots, y_K)$ a la combinación de las expresiones anteriores, es decir a: 
\begin{equation}
    y_k(x,w) = \theta_k 
    \biggl( 
        \sum^M_{j=1} w_{ji}^{(2)}
        \sigma_j 
        \biggl(
            \sum_{i=1}^D w_{ji}^{(1)} x_i + w_{j0}^{(1)}
        \biggr)
        + w_{k0}^{(2)}
    \biggr) 
    \text{ para cada  } k \in \{1, \ldots, K \}..
\end{equation}
Donde todos los pesos y sesgos han sido agrupados en el vector $w$. 

Si al vector de entrada se le añade una variable $x_0 = 1$, puede reescribirse cada expresión eliminando los sesgos y como producto vectorial

\begin{equation}
    y_k(x,w) = \theta_k 
    \bigl(
         w^{(2)} \cdot
        \sigma    
        \bigl(
             w^{(1)} \cdot x 
        \bigr)
    \bigr)
\end{equation}  

La función de activación $\theta_k$ será escogida de acorde a la
naturaleza del problema, es decir \textit{cómo se desee codificar la salida}, por ejemplo si se trata de un problema de regresión, de clasificación, de probabilidad. 
 
De ahora en adelante trabajaremos con esta notación. 

Es posible visualizar las relaciones de las entradas y los distintos nodos de la 
red neuronal como un grafo dirigido acíclico como se muestra en la figura \ref{img:ejemplo topología red neuronal}

\begin{figure}[h!] 
    \centering
    \includegraphics[width=0.65\textwidth]{introduccion_redes_neuronales/construccion_redes_neuronales/red-neuronal-simple-introducción.drawio.png}
    \caption{Ejemplo de red neuronal con una capa oculta}
    \label{img:ejemplo topología red neuronal}
\end{figure} 
 

En este ejemplo poseemos una capa oculta, 
puede definirse siguiendo esta misma idea
una red neuronal de múltiples capas ocultas. 

% Generalización de modelo 
\subsection{Construcción red neuronal de varias capas ocultas} \label{rrnn:construcción_generalizada}

Etiquetaremos a cada capa con $l \in \{0, \ldots, L \}$, donde $L+1$ es el número total de capas.  Donde 

\begin{itemize}
    \item La capa de entrada será la etiquetada con $l = 0$.
    \item La capa de salida que determina el valor de la red neuronal es la $l=L$.
    \item Las capas ocultas serán aquellas etiquetadas como $0 < l <L.$
\end{itemize}

Se usará un superíndice para hacer referencia a la capa. 
Cada capa posee una dimensión $d^{(l)}$, es decir que posee
$d^{(l)} + 1$ unidades o nodos. El nodo $d_0^{(l)}$ se trata del sesgo y siempre será uno. 

El modelo de red neuronal $\mathcal{H}_{n n}$ viene determinado una vez que se fija la arquitectura de la misma, es decir sus dimensiones $d$. 
\begin{equation}
    d = (d^{(0)}, d^{(1)}, \ldots, d^{(L)})
\end{equation}
y se tiene que cada red neuronal $h \in \mathcal{H}_{n n}$
viene determinada por sus pesos. 

\subsubsection*{Cálculo de una capa oculta}  
Cada nodo recibe una señal de entrada $s$ y determina una salida $x$. 
  
La relación que existe entre dos nodos de capas contiguas es la siguiente: si $x_i^{(l-1)}$ es la salida de la unidad $i$ de la capa $l-1$, 
entonces se calcula la entrada de la unidad $j$ de la capa $l$ como 
\begin{align}\label{eq:construcción_red_neuronas:calculo_una_capa_oculta}
    s_j^{(l)} &= w_{i j}^{(l)} \cdot x_i^{(l-1)}  \\
    x_j^{(l)} &= \theta(s_j^{(l)})
\end{align}

Es decir, que en cada capa $l$ intervienen los siguientes elementos:  
\begin{table}[h]
    \begin{center}
    \begin{tabular}{| l | l | l |}
    \hline
    Elementos & Notación & Representación 
    \\ \hline
    Vector de entrada & $s^{(l)}$ &  Vector de dimensión $d^{(l)}$ \\
    Vector de salida & $x^{(l)}$ &  Vector de dimensión $d^{(l)}+ 1$ \\
    Pesos entrada & $W^{(l)}$ & Matriz de dimensiones $(d^{(l-1)}+1) \times d^{(l)}$ \\
    Pesos salida & $W^{(l+1)}$ 
    & Matriz de dimensiones $(d^{(l)}+1) \times d^{(l+1)}$ \\
    \hline
    \end{tabular}
    \caption{Elementos capa oculta $l$}
    \label{tab:rrnn_elementos_capa_oculta}
    \end{center}
\end{table}

\subsection{ \textit{Forward propagación}}

Explicaremos en esta sección cómo calcular para una entrada una determinada salida, es decir
dada una red neuronal $h \in \mathcal{H}_{d^{(0)} \times \cdots \times d^{(L)}}$ y un vector de entrada $x \in \mathcal{X}$ calcularemos  $h(x)$, esto se hará gracias al algoritmo conocido como \textit{forward propagation}.

Teniendo presente la relación  explicada en (\refeq{eq:construcción_red_neuronas:calculo_una_capa_oculta}) se puede escribir de forma vectorial la siguiente relación: 
\begin{equation}
    x^{(l)} = 
    \left[ \begin{array}{c}
        1 \\
       \theta(s^{(l)})
        \end{array}
\right] .
\end{equation}
Donde $\theta(s^{(l)})$ es un vector de componentes $\theta(s^{(l)}_j)$. 
Para calcular el vector de entrada de la capa $l$, para cada nodo se hará
\begin{equation}
    s_j^{(l)} = \sum_{i=0}^{d^{(l-1)}} w_{i j}^{(l)}x_i^{(l-1)}.
\end{equation}
Que se formula de forma vectorial para toda la capa como 
\begin{equation}
    s^{(l)} = W^{(l)} x^{(l-1)}.
\end{equation}

Si el vector de entrada es $x \in \mathcal{X} \subseteq \R^d$, 
se inicializa  $x^{(0)} = (1,x_1, \ldots, x_d)^T$ y por tanto $d^{(0)} = d+1.$


La implementación del algoritmo sería:

\begin{algorithm}[H]
    \caption{Algoritmo \textit{Forward propagation} para evaluación de una red neuronal $h_w(x)$.}
    \begin{algorithmic}[1]
        \STATE $x^{(0)} \leftarrow x$ 
        \COMMENT{Inicialización}

        \STATE \COMMENT{\textit{Forward Propagation}}
        \For{$l = 1, \ldots , L$}{
            % Calcula s
            \STATE 
            \begin{equation}
                s^{(l)}
                    \leftarrow
                    \left(W^{(l)}\right)^T
                    x^{(l-1)}      
            \end{equation}

            %Calcula x
            \STATE
            \begin{equation}
                x^{(l)}
                    \leftarrow
                    \left[ 
                        \begin{array}{c}
                            1 \\ 
                            \theta \left( s^{(l)}\right)
                        \end{array}
                    \right]
            \end{equation}
        }
        \STATE $h_w(x) = x^{(L)}$ 
        \COMMENT{Salida}  
\end{algorithmic}
\end{algorithm}

% Imagen red neuronal con pesos concretos
Veamos un ejemplo concreto para la siguiente red neuronal de la imagen \ref{img:construccion_rrnn:rrnn-2-3-2-1}
\begin{figure}[h!]
    \includegraphics[width=\textwidth]{introduccion_redes_neuronales/construccion_redes_neuronales/rrnn-2-3-2-1-completa.png}
    \caption{Ejemplo de red neuronal con dos capas ocultas y pesos con valores concretos}
    \label{img:construccion_rrnn:rrnn-2-3-2-1}
\end{figure} 

La red de la imagen \ref{img:construccion_rrnn:rrnn-2-3-2-1} está 
compuesta por dos capas ocultas, acepta vectores de entrada de dimensión dos, 
la primera capa oculta está compuesta por tres neuronas, 
la segunda por dos y la salida por una. 
La notación del dibujo usada es la siguiente, entre paréntesis se 
especifica la capa, en el caso de la salida $x(l)i$ hace referencia a la salida $i$-ésima de la capa $l$. Las \textit{flechas} que conectan los nodos $w(l)ij$ hace referencia al peso que se le da a $x(l-1)i$ con respecto a la entrada $s(l)j$.

Representando los pesos de manera matricial
\begin{align}
    W^{(l)} = 
    \begin{bmatrix}
        w^{(l)}_{01} & w^{(l)}_{11} & \cdots & w^{(l)}_{d^{(l-1)} 1}\\
        w^{(l)}_{02} & w^{(l)}_{12} & \cdots & w^{(l)}_{d^{(l-1)} 2}\\
        \cdots & \cdots & \cdots & \cdots \\
        w^{(l)}_{0d} & w^{(l)}_{1d} & \cdots & w^{(l)}_{d^{(l-1)} d}\
    \end{bmatrix} 
\end{align}
las matrices de pesos de nuestra imagen son :
\begin{align}
    W^{(1)} = 
    \begin{bmatrix}
        0.4 & -0.3 & 0.5\\
        -0.2 & -0.2 & 0.2\\
        0.1 & 0 & -0.3
    \end{bmatrix} ,
    W^{(2)} = 
    \begin{bmatrix}
        1 & -1 & 0.3 & 0\\
        0.3& 0 & 0.3 & -0.6 
    \end{bmatrix} ,
    W^{(3)} = 
    \begin{bmatrix}
        0.33 & 0.33 & -0.6 \
    \end{bmatrix} .
\end{align}
Si inicializamos $x= (1,0)$ y tomamos como función de activación
a la tangente hiperbólica, la ejecución del algoritmo queda reflejada en la tabla \ref{tab:construcción_rnnn:ejemplo_forward_propagation} resultando que 
$h((1,0)) = 0.439$.
\begin{table}[H]
    \begin{center}
\begin{tabular}{| c | c | c | c| }
    \hline
    Capa $l$-ésima &  $W^{(l)}$ & $\bigl(s^{(l)}\bigr)^T $ & $\bigl(x^{(l)}\bigr)^T$ \\ \hline
    0 & & & $(1,1,0)$ 
    \\ \hline
    1 & 
    $\begin{bmatrix}
        0.4 & -0.3 & 0.5\\
        -0.2 & -0.2 & 0.2\\
        0.1 & 0 & -0.3
    \end{bmatrix}$ 
    & $(0.1, -0.4, 0.1)$ & $(1, 0.1, -0.38, 0.1)$
     \\ \hline
    2 & $\begin{bmatrix}
        1 & -1 & 0.3 & 0\\
        0.3& 0 & 0.3 & -0.6 
    \end{bmatrix}$
    & $(0.786, 0.126)$
    & $(1,0.656, 0.126)$
    \\ \hline
    3 & $\begin{bmatrix}
        0.33 & 0.33 & -0.6 
    \end{bmatrix}$ 
    & $(0.471)$ 
    & $(1,0.439)$
    \\ \hline
\end{tabular}
\caption{Ejemplo de ejecución del algoritmo de \textit{forward propagation}}
\label{tab:construcción_rnnn:ejemplo_forward_propagation}
\end{center}
\end{table}

\subsection{\textit{Backpropagation}}

Los parámetros que determinan una red neuronal son sus pesos, y esto es lo que \textit{aprende},
para actualizarlos utilizaremos la técnica de gradiente descendente ya explicado en la sección \ref{sec:gradiente-descendente}. 

\begin{equation}
    W(t+1) = w(t) - \eta \nabla E_{in}(w(t)). 
\end{equation}

Además, con el fin de reducir el coste del cálculo del gradiente, 
se utiliza el algoritmo conocido como \textit{backpropagation} que fue publicado en 
1989 en el artículo \cite{backpropagation-Hinton}. 

Sea $E_{in}(w)$ la función de error, la cual tomaremos como el error dentro de conjunto de entrenamiento, esto es,  si el conjunto 
de entrenamiento está constituido por $N$ datos de la forma $(x_n, y_n)$ con $x_n$ el vector de entrada y $y_n$ el estado o valor deseado para cualquier $n\in \{1, \ldots, N\}.$
\begin{equation}
    E_{in}(w) = \frac{1}{N} \sum^N_{n=1} (h_w(x)- y_n)^2. 
\end{equation}
Denotaremos como $e_n$ a 
\begin{equation}
    e_n = (h_w(x)- y_n)^2 
\end{equation}
que es una métrica para medir error entre, en nuestro caso  
la red neuronal $h_w$ y los valores deseados, con $w$ el vector que contiene las respectivas matrices de pesos de cada capa 
$W^{(l)} l \in \{1, \ldots, L\}.$  

%% Ejemplo 
Mostraremos un ejemplo primero antes de presentar el método general para facilitar la comprensión del algoritmo. 
% Imagen red neuronal simple
\begin{figure}[h!]
    \includegraphics[width=\textwidth]{introduccion_redes_neuronales/construccion_redes_neuronales/rrnn-1-2-1.drawio.png}
    \caption{Ejemplo de red neuronal con tres capas ocultas}
    \label{img:construccion_rrnn:rrnn-1-2-1}
\end{figure} 
Queremos actualizar los pesos $w$ de la red neuronal 
$f_w : \R \longrightarrow \R$ presentada en \ref{img:construccion_rrnn:rrnn-1-2-1}.
$f_w$ está compuesta de dos capas ocultas. Supongamos que nos basaremos en un dato 
$(x, y)$ así pues podemos suponer que 
\begin{equation}
    E_{in}(w) = \frac{1}{N}e(f_w(x), y) = \frac{1}{N} (f_w(x)- y)^2.
\end{equation}
Como queremos actualizar los pesos utilizando el método de gradiente descendente necesitamos calcular el gradiente $\nabla E_{in}(w)$, en nuestro caso tenemos $w=\{W^{(1)}, W^{(2)}\}$ con 
\begin{align}
    W^{(1)} = 
    \begin{bmatrix}
        w^{(1)}_{01} & w^{(1)}_{11} \\
        w^{(1)}_{02} & w^{(1)}_{12} \\
    \end{bmatrix} 
    \text{ y }
    W^{(2)} = 
    \begin{bmatrix}
        w^{(2)}_{01} & w^{(2)}_{11} & w^{(2)}_{21}\\
    \end{bmatrix}, 
\end{align}
luego 
\begin{equation}
    \nabla E_{in}(w) = 
    \left(
        % primera capa 
        \frac{\partial e}{\partial w^{(1)}_{01}},
        \frac{\partial e}{\partial w^{(1)}_{11}},
        \frac{\partial e}{\partial w^{(1)}_{02}},
        \frac{\partial e}{\partial w^{(1)}_{12}},
        % segunda capa
        \frac{\partial e}{\partial w^{(2)}_{01}},
        \frac{\partial e}{\partial w^{(2)}_{11}},
        \frac{\partial e}{\partial w^{(2)}_{21}}
    \right).
\end{equation} 
Cada parcial se calcula, utilizando la regla de la cadena como
\begin{align}
    \frac{\partial e}{\partial w^{(1)}_{01}} 
    &=
    \frac{\partial e}{\partial s_1^{2}}
    \frac{\partial s_1^{2}}{\partial w^{(1)}_{01}} 
    \\
    &= 
    \frac{\partial }{\partial w^{(1)}_{01}}
         \tanh \left(s^{(2)}_{1}\right)
    \\
    &= 
    \left(1- \tanh^2 \left(s^{(2)}_{1}\right)\right) 
    \frac{\partial s^{(1)}_{1}}{\partial w^{(1)}_{01}}
    \\
    &= 
    \left(1- \tanh^2 \left(s^{(2)}_{1}\right)\right) 
    \frac{\partial }{\partial w^{(1)}_{01}}
    \left(w^{(2)}x^{(1)}\right)
    \\
    &= 
    \left(1- \tanh^2 \left(s^{(2)}_{1}\right)\right) 
    \frac{\partial }{\partial w^{(1)}_{01}}
    \left(
        \sum^2_{i=0}
        w^{(2)}_{i1}x^{(1)}_i
    \right)
    \\
    &= 
    \left(1- \tanh^2 \left(s^{(2)}_{1}\right)\right) 
    \left(
        \sum^2_{i=0}
        w^{(2)}_{i1}\frac{\partial x^{(1)}_i }{\partial w^{(1)}_{01}}
    \right)
    \\
    &= 
    \left(1- \tanh^2 \left(s^{(2)}_{1}\right)\right) 
    \left(
        \sum^2_{i=1}
        w^{(2)}_{i1}\frac{\partial }{\partial w^{(1)}_{01}}
        \left(
            \tanh \left(s^{(1)}_{i}\right)
        \right)
    \right)
    \\
    &= 
    \left(1- \tanh^2 \left(s^{(2)}_{1}\right)\right) 
    \left(
        \sum^2_{i=1}
        w^{(2)}_{i1}
        \left(
            \left(1- \tanh^2 \left(s^{(1)}_{i}\right)\right)
            \frac{\partial  }{\partial w^{(1)}_{01}}
            \left(
                \sum^1_{j=0}\sum^2_{k=1}
                w^{(1)}_{j k}x^{(0)}_j
            \right)
        \right)
    \right)
    \\
    &= 
    \left(1- \tanh^2 \left(s^{(2)}_{1}\right)\right) 
    \left(
        \sum^2_{i=1}
        w^{(2)}_{i1}
        \left(
            \left(1- \tanh^2 \left(s^{(1)}_{i}\right)\right)
            x^{(0)}_0
        \right)
    \right).
\end{align}
Notemos que no se han evaluado las apariciones de $s_i^{(j)}$.
Otro ejemplo sería
\begin{align}
    \frac{\partial e}{\partial w^{(2)}_{21}} 
    &=
    \frac{\partial }{\partial w^{(2)}_{21}}
         \tanh \left(s^{(2)}_{1}\right)
    \\
    &= 
    \frac{\partial }{\partial w^{(2)}_{21}}
         \tanh \left(w^{(2)}x^{(1)}\right)
    \\
    &= \left(
    1- \tanh^2 \left(s^{(2)}_{1}\right) \right)x^{(1)}_2.
\end{align}

Notemos que no se han desarrolla los términos de la forma $s^{(i)}_j$. Además si existen $Q$ pesos la complejidad del cálculo será $\mathcal{O}(Q^2)$, sin embargo, como hemos visto existen términos que se repiten en ambas ecuaciones, por lo que utilizar una técnica de programación dinámica, concretamente la conocida como 
\textit{backpropagation} (\cite{backpropagation-Hinton}) nos permite reducir el coste a una complejidad de  $\mathcal{O}(Q).$

%% Fin del ejemplo 
%% COMIENZA EL ARTÍCULO 

%Formalizaremos primero qué parámetros debemos estimar del gradiente.

Sea $\theta$ una función de activación derivable, 
 $L$ el número de capas ocultas, $N$ el tamaño del conjunto de entrenamiento $x^{(0)} = (1, x_1, \ldots, x_{d^{(0)}})^T$ 
siendo $x = (x_1, \ldots, x_{d^{(0)}})^T$ la entrada de la red neuronal, $x^{(L)}$ la salida de la red neuronal y 
$d^{(l)}$ la dimensión, número de nodos en la capa $l$-ésima. 
Recordemos que 
para cualquier $l \in \{1, \ldots, L\}$ con
\begin{equation}
    x^{(l)}
     = 
     \theta \left( s^{(l)}\right) 
     = 
     \theta \left( W^{(l)} x^{(l-1)}\right),
\end{equation}
y
\begin{equation}
    E(w) = \frac{1}{N} 
    \sum_{n = 1}^{N}
    \sum_{i = 1}^{d^{(L)}}
    \left({x_n}^{(L)}_i-y_{n_i} \right)^2 
    = 
    \frac{1}{N}
    \sum^{N}_{n=1} e_n.
\end{equation}

%% Gradiente de la salida: 
Para simplificar la notación nos referiremos como $e$ a $e_n.$
Vamos a proceder a calcular primero los gradientes de la última capa, 
sea $w^{(L)}_{i j}$  con $j \in \{1, \ldots , d^{(L)}\}$, 
$i \in \{1, \ldots , d^{(L-1)}\}$  el peso que relaciona la salida 
$x_i ^{(L-1)}$ del  
nodo $i$ de la capa anterior con la entrada $s_j ^{(L)}$ del nodo $j$ de la última capa. 
Vamos a calcular $\frac{\partial e}{ w^{(L)}_{i j}}$ utilizando la regla de la cadena. 
\begin{equation}
    \frac{\partial{e}}{\partial w^{(L)}_{i j}}
     = 
     \frac{\partial{e}}{\partial x^{(L)}_j} 
     \frac{\partial x^{(L)}_j}{\partial s^{(L)}_j} 
     \frac{\partial s^{(L)}_j}{\partial w^{(L)}_{i j}}.
\end{equation}
Donde es fácil ver que para el tercer término
\begin{equation}\label{eq:backpropagation_s_última_capa_derivada}
    \frac{\partial s^{(L)}_{j}}{\partial w^{(L)}_{i j}}
    = 
    \frac{\partial }{\partial w^{(L)}_{i j}}
    \left(
        w^{(L)}_{\ast j } \cdot x^{(L-1)}
    \right)
    = 
    x^{(L-1)}_j
\end{equation}
Donde $w^{(L)}_{\ast j} = \left(w^{(L)}_{0 j}, w^{(L)}_{2 j}, \ldots, w^{(L)}_{d^{(L-1)} j}\right)$ representa los pesos correspondientes al nodo $j$,
 en forma de  vector fila y $x^{(L-1)} = \left(1, x ^{(L-1)}_1, \ldots, x ^{(L-1)}_{d^{(l-1)}}\right)^T$ el valor de la salida de la capa $L-1$
 en forma de vector columna.
 
 Por otro lado 
 \begin{equation}\label{eq:backpropagation_E_última_capa_derivada}
    \frac{\partial{e}}{\partial x^{(L)}_j} =
    2 
    \left(
    x^{(L)}_j - y_j
    \right)
 \end{equation}
 donde conocemos $x^{(L)}_j$ gracias al algoritmo de \textit{forward propagation}
 y $y_j$ la componente $j$-ésima del vector deseado en el entrenamiento.
y finalmente
\begin{equation}\label{eq:backpropagation_x_última_capa_derivada}
    \frac{\partial x^{(L)}_j}{\partial s^{(L)}_j} 
    = 
    \frac{d}{d s^{(L)}_j} 
        \theta \left( 
            s^{(L)}_j
        \right)
\end{equation}
que sabemos que se puede calcular por ser $\theta$ derivable y 
$s^{(L)}_j$ un valor conocido que ya ha sido calculado por el algoritmo de 
\textit{forward propagation.}

Por lo tanto, concluimos por 
(\refeq{eq:backpropagation_E_última_capa_derivada}),
(\refeq{eq:backpropagation_x_última_capa_derivada})
y  
(\refeq{eq:backpropagation_s_última_capa_derivada})
\begin{equation}
    \frac{\partial{e}}{\partial w^{(L)}_{i j}}
     = 
     \frac{\partial{e}}{\partial x^{(L)}_j} 
     \frac{\partial x^{(L)}_j}{\partial s^{(L)}_j} 
     \frac{\partial s^{(L)}_j}{\partial w^{(L)}_{i j}} 
    =
    2\left( x^{(L)}_j - y_j \right) 
    \theta' \left( s^{(L)}_j\right)
    x^{(L)}_j.
\end{equation}
%%%% Gradiente interior 
Denotaremos por \textit{sensibilidad} a 
\begin{equation}
    \delta^{(l)} = \frac{\partial e}{ \partial s^{(l)}}.
\end{equation}

Para calcular la derivada de pesos de capas interiores 
$l \in \{1 \ldots L-1\}$
procederemos de la siguiente manera, para 
$j \in \{1, \ldots , d^{(l)}\}$ y 
$i \in \{1, \ldots , d^{(l-1)}\}$ 
\begin{equation}
    \frac{\partial{e}}{\partial w^{(l)}_{i j}}
     = 
     \frac{\partial e}{\partial s^{(l)}_j} 
     \frac{\partial s^{(l)}_j}{\partial w^{(l)}_{i j}}
    = 
    \delta^{(l)}
    \frac{\partial}{\partial w^{(l)}_{i j}}
    w^{(l) \cdot x^{(l-1)}}
    = 
    \delta^{(l)} x^{(l-1)}_i,
\end{equation}
donde  $x^{(l-1)}_i$ es conocida por el algoritmo de 
$\textit{forward propagation}$. 

Por otra parte 
$\delta^{(l)}_j$ 
cumple que 
\begin{align}
    \delta^{(l)}_j 
    &= 
    \frac{\partial e}{\partial s^{(l)}_j}
    \\
    &= 
        \frac{\partial e}{\partial s^{(l+1)}}
        \frac{\partial s^{(l+1)}}{\partial s^{(l)}_j}
    \\
    &= 
    \delta^{(l+1)} 
    \frac{\partial}{\partial s^{(l)}_j}
        \left( w^{(l)} \cdot \theta(s^{(l)})\right)
    \\
    &= 
    \delta^{(l+1)}_j 
    w^{(l)}_j \cdot  \theta'(s^{(l)}_j). 
\end{align}
Por lo que de forma matricial tenemos que $\delta^{(l)}$ 
cumple que 
\begin{align}
    \delta^{(l)} 
    &= 
    \frac{\partial e}{\partial s^{(l)}}
    \\
    &= 
        \frac{\partial e}{\partial s^{(l+1)}}
        \frac{\partial s^{(l+1)}}{\partial s^{(l)}}
    \\
    &= 
    \delta^{(l+1)} 
    \otimes 
    \frac{\partial}{\partial s^{(l)}}
        \left( w^{(l)} \cdot \theta(s^{(l)})\right)
    \\
    &= 
    \delta^{(l+1)} 
    \otimes 
    \left(
    w^{(l)} \cdot \theta'(s^{(l)})
    \right). 
\end{align}

De esta manera a partir de las \textit{sensibilidades} de las 
capas posteriores es posible calcular las $l$-ésimas y puesto que 
la sensibilidad $\delta_{(L)}$ es conocida acabamos de determinar 
cómo calcular la derivada de todos los pesos  de manera constructiva. 
Procedemos a explicitar los cálculos. 

\subsubsection{Algoritmos para la actualización de los pesos}  

El razonamiento expuesto conduce al siguiente proceso algorítmico 
para el cálculo de los gradientes. 

% pseudo código cálculo de sensibilidades 
\begin{algorithm}[H]
    \caption{Algoritmo \textit{backpropagation} para calcular
    las sensibilidades $\delta^{(l)}$}
    \hspace*{\algorithmicindent} \textbf{Input}: un par de $(x,y)$ del conjunto de entrenamiento.  \\
    \hspace*{\algorithmicindent} \textbf{Output} 
    \begin{algorithmic}[1]
        % Forward propagation
        \STATE Se ejecuta el algoritmo de \textit{forward propagation} 
        
        con $x$ como entrada para calcular y guardar : 
        \begin{align}
            s^{(l)} \quad &\text{for } l = 1, \ldots, L;
            \\
            x^{(l)} \quad &\text{for } 0 = 1, \ldots, L;
        \end{align}
        % Inicializamos
        \STATE \COMMENT{Inicializamos sensibilidades últimas capas}
        \begin{equation}
            \delta^{(L)} \longleftarrow 2
            \left( 
                x^{(L)} - y
            \right)
            \theta' \left( s^{(L)} \right)
        \end{equation}
        \STATE 
        \COMMENT{ \textit{Backpropagation}}
        
        \For{$l = L-1$ to $1$}
        {
            \begin{equation}
                \delta^{(l)} 
                    \leftarrow
                \theta' 
                \left(
                    s^{(l)}
                \right)
                \otimes
                \left[
                    W^{(l+1)}
                    \delta^{(l+1)}
                \right]^{d^{(l)}}_1
            \end{equation}
        }
\end{algorithmic}
\end{algorithm}

% pseudo código cálculo cálculo de gradiente  
\begin{algorithm}[H]
    \caption{Algoritmo para el cálculo $E_{in}(w)$
    y $g = \nabla E_{in}(w).$}
    \hspace*{\algorithmicindent} \textbf{Input}:
    $w = \left\{ W^{(1)}, \ldots, W^{(L)}\right\};$
    $\mathcal{D} = (x_1, y_1), \ldots, (x_n, y_n).$\\
    \hspace*{\algorithmicindent} \textbf{Output:} error $E_{in}(w)$ y gradiente
    $g = \{G^{(1)}, \ldots, L\}.$  
    \begin{algorithmic}[1]
        \STATE Inicializamos $E_{in}(w)=0$ y $G^{(l)} = 0$ 
        for $l = 1, \ldots, L.$
        \STATE \For{ cada punto $(x_n, y_n),$
        n = 1, \ldots, N}{
            % cálculos de las x^l y sensibilidades
            \STATE Calcula $x^{(l)} \quad \text{for } l = 1, \ldots, L;$
            \COMMENT{ \textit{forward propagation}}
            \STATE Calcula $\delta^{(l)} \quad \text{for } l = L, \ldots, 1;$
            \COMMENT{ \textit{backpropagation}}
            % vamos sumando el erro en cada punto 
            \STATE
            \begin{equation}
                E_{in} 
                \leftarrow
                E_{in} + 
                \frac{1}{N}(x^{L}-y_n)^2 
            \end{equation}
            \For{l=1, \ldots, L}{
                \STATE
                \COMMENT{Calculamos gradiente en un punto}
                \begin{equation}
                    G^{(l)}\left( x_n\right) = 
                    \left[
                        x^{(l-1)}
                        \left(
                            \delta^{(l)}
                        \right)^T
                    \right]
                \end{equation}
                \STATE 
                \begin{equation}
                    G^{(l)}
                    \leftarrow
                    G^{(l)}
                    + 
                    \frac{1}{N}
                    G^{(l)}\left( x_n\right)
                \end{equation}
            }
        }
\end{algorithmic}
\end{algorithm}

Finalmente para conocer el valor actualizado de los pesos, lo actualizaremos usando la técnica de gradiente descendente. 

\begin{algorithm}[H]
    \caption{Algoritmo gradiente descendente.}
    \hspace*{\algorithmicindent} \textbf{Input}: Pesos $w$ y gradiente $g$. \\
    \hspace*{\algorithmicindent} \textbf{Output:} $w$ actualizado
    \begin{algorithmic}[1]
        \STATE
        \For{$l = 1, \ldots, L.$}{
            \begin{equation}
                W^{(l)}
                \leftarrow
                W^{(l)} 
                - 
                \eta G^{(l)}                    
            \end{equation}
        }
\end{algorithmic}
\end{algorithm}

% !TeX root = ../../tfg.tex
% !TeX encoding = utf8
%
%*******************************************************
% Polinomois de Bernstein
%*******************************************************

\chapter{Polinomios de Bernstein}\label{ch:Bernstein}

\begin{definition}[Polinomios de Bernstein] \label{def:Bernstein}
    Dada cierta función $f: [0,1] \rightarrow \R$, se define el n-ésimo polinomio de Bernstain para $f$ como 

    $$B_n(x) = B_n(x;f)=\sum_ {k=0}^{n} f \left( \frac{k}{n} \right) \left( \binom{n}{k} \right) x^k (1-x)^{n-k}.$$

\end{definition}

La intuición que se esconde tras esta definición es la siguiente: 
Se pretende aproximar la función $f$ a través de los puntos $\frac{k}{n}$ con $n \N$ fijo
y $k \in \{0,...,n \}.$
Se prentende que si evaluamos $B_n(x)$ con $x$ próximo a un valor de la forma $\frac{k}{n}$  
su valor se aproxima al de $f \left( \frac{k}{n} \right).$

Para ello recordaremos el teorema del Binomio de Newton: 

\begin{theorem}[Binomio de Newton]
    Cualquier potencia de un binomio $x+y$ con $x,y \in R$,  puede ser expandido en una suma de la forma
    \[(x+y)^n = \sum_{k=0}^n \binom{n}{k} x^{n-k}y^k\]
\end{theorem}

Así pues en virtud de esta igualdad y puesto que nuestro dominio de definición de $f$ es $[0,1]$, para cualquier $x \in [0,1].$

Tenemos 

\begin{equation}
    1 = (x+ (1-x))^n = \sum_{k=0}^n \binom{n}{k} x^{k} (1-x)^{n-k}
\end{equation}

Multiplicamos ahora en ambos lados por $f(x)$

\begin{equation}
    f(x) = \sum_{k=0}^n f(x) \binom{n}{k} x^{k} (1-x)^{n-k}
\end{equation} 

Y tenemos que las diferencia entre $f(x)$ y $B_n(x)$ es

\begin{equation}
    f(x)-B_n(x) = \sum_{k=0}^n \left(f(x) - f \left( \frac{k}{n} \right)\right)
    \binom{n}{k} x^{k} (1-x)^{n-k}
\end{equation} 

Que en valores absolutos resulta 
\begin{equation} \label{eqn:berstein_difference}
    |f(x)-B_n(x)| = \sum_{k=0}^n \left|f(x) - f \left( \frac{k}{n} \right)\right|
    \binom{n}{k} x^{k} (1-x)^{n-k}
\end{equation} 

Observando esta ecuación \ref{eqn:berstein_difference} se desprende como es natural un teorema de convergencia. 

\begin{theorem}[Teorema de aproximación d3e Bernstein]

    Sea $f$ una función continua en un intervalo $I$ con imágen en los reales. 
    La secuencia de polinomio de Berstein
    \ref{def:Bernstein} converge uniformemente a $f$ en $I.$
    
\end{theorem}

Por estar $f$ definida en un intervalo cerrado, está acotada y además es uniformemente continua. 
\endinput 
\section*{Introduction} % The \section*{} command stops section numbering
\addcontentsline{toc}{section}{\protect\numberline{}Introduction}

We have tried to apply agile principles to the whole development process, from
the more mathematical to the minimally viable product that was the developed
code and this paper \cite{DBLP:journals/corr/abs-2104-12545}.

Since in 1988, Hornik, Stinchombe and White established
that Multilayer Feedforward Networks are Universal Approximators \cite{HORNIK1989359} this tools have revolutionized artificial intelligence. 

Backpropagation \cite{backpropagation-Hinton} and its 
variants are relevant methods to update neural 
network's weights. However one of its weak spots comes from being iterative 
algorithms, since an initial point is needed (ie a initial setting of the weights of the neuronal network). In order to determine this point 
there are some heuristics such as random initialization
closed to null values or transference (\textcolor{red}{QUESTION: This should go in the art state? Anyway, some cites are need}). 

In order to determine that initial state we propose a initialization method based on the training data. 

To introduce our method we are going to use One Layer
Feedforward Networks \cite{HORNIK1989359} , from now on we will refer to them as Neural Networks. Their mathematical modeling is the following functional space: 

    For $X \subseteq \R^d, Y \subseteq \R^s$. 

    \begin{align} \label{nn_functional_space}
        \mathcal{H}(X,Y) 
        =
        \{
            h : X \longrightarrow Y 
            /& \quad 
            h_k(x) = 
            \sum_{i=1}^{n} \beta_{i k} \gamma( A_{i}(x))          
        \}.
    \end{align}
    Where $h_k$, $k \in \{1, \ldots, s\}$ is the k-projection of $h$, 
    $n \in \N,\gamma$ an activation function 
    \footnote{The definition of activation function will came in next sections }, $\beta_{i k} \in \R$ and $A_{i}$ is an affine function from $\R^d$ to $\R$. 



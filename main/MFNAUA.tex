\documentclass{book}
\usepackage[utf8]{inputenc}
\usepackage[english]{babel}
\usepackage{amsfonts}
\usepackage{blindtext}
\usepackage[T1]{fontenc}

\usepackage{mathtools}

\usepackage{amsthm}
\theoremstyle{definition}
\newtheorem{definition}{Definition}[section]

%Commands
\newcommand{\R}{\mathbb{R}} %Real numbers 
\newcommand{\N}{\mathbb{N}-\{0\}} % Natural number bigger than 0
\newcommand{\X}{\mathcal{X}} % Input space  
\newcommand{\x}{\textbf{x}} % Column vector 
\newcommand{\G}{\mathcal{G}} 
\newcommand{\C}{\sum ^r (\G)}

 

\begin{document}
    
\section{Multilayer Feedforward Networks are Universal Approximators}

In 1988 Maxwell Stinchcombe and Halber White show that Multilayer
Feedforward Networks with one hidden layer are Universal Approximators.  

The proof is based on that arbitrary squashing function are capable of approximating
any Borel measurable function from one finite dimensional space to another to any desired 
degree of accuracy, provided sufficiently hidden units are available. In this sense 
Feedforward networks are a class of Universal approximators. 
(REWRITE THREE LAST SENTENCES)

Firstly we are going to start with some basic definitions. 
Particularly we are going to introduce familiar of function from more particular to more general 
that will represent 

\begin{definition}Affine functions  

    Affine functions are defined as all the 
    function of the form $A(x) = w \cdot x + b$ 
    from $\R ^d$ to $\R$
    where $d \in \mathbb{N} - \{ 0 \}$, $w$ and $x$  are vector in $\R^d$,  
    $\cdot$ 
    denotes the usual dot product of vector, and $b \in \R$ is a scalar.  
    
\end{definition}

\begin{definition} Squashing function

    A function $\psi: \R \longrightarrow [0,1]$ is a squashing function if
    \begin{itemize}
        \item It is non-decreasing.
        \item $\lim _{x \rightarrow \infty} \psi(x) = 
        $.
        \item $\lim _{x \rightarrow -\infty} \psi(x) = 0$.
    \end{itemize}  

    Squashing functions are measurable due to the fact that have at most 
    countably discontinuities.   

    Most common examples are: 
    \begin{itemize}
        \item Threshold functions.\footnote{
         A threshold function is a function that takes the value 1 
         is a specified function of the arguments exceeds a given threshold and 
         0 otherwise. \cite{threshold-definition} 
        }

        \item Indicator functions: $\psi(\lambda) = 1_{\{\lambda > 0\}}$. 
        \item The ramp function: $\psi(\lambda)  = \lambda 1_{\{0 \leq \lambda \leq  1\}} + 1_{\{\lambda > 1\}}$
    
        \item The cosine squasher of Gallant and White (1988)
        \begin{equation*}
    \psi(lambda )= (1 + cos(\lambda + 3 \frac{\pi}{2}) \frac{1}{2}) 
     1_{\{\frac{-\pi}{2} \leq \lambda \leq  \frac{\pi}{2}\}}
     1_{\{ \frac{\pi}{2} < \lambda \}}
    \end{equation*}
    \end{itemize}
\end{definition}  

\begin{definition} General family for NN

    In order to introduce the familiar class of function for single hidden layer feedforward networks. 
    Now we are going to define the familiar class of function of output functions 
    


    For any Borel measurable function $\G(\cdot)$ mapping $\R$ to $\R$ and 
    $r \in  \N$, let define the class of function 

    \begin{equation*}
        \begin{split}
        \C = \{ 
        f: \R^r \longrightarrow \R = \\
        f(x)=\sum_{j = 1} ^q (
        \beta_j \G(A_{j}(x)), 
        x  \in \R^r, \beta_j \in \R, A_{j}\in A^r, l_j,q \in \N
        )
        \}
    \end{split}
    \end{equation*}

    When $\G$ is a squashing function, $\C$ is the familiar class of 
 we were searching for, the familiar class of output function for single hidden layer feedforward networks at the
 output layer. 
 
 The scalars $\beta_j$ correspond to network weight from hidden to output layers.
    
\end{definition}

\begin{definition} More general family definition 

    For any measurable function $\G(\cdot)$ mapping $\R$ to $\R$ and $r\N$, let 
    

    
    \begin{equation*}
        \begin{split}
        \sum \prod^r(\G) = \{ 
        f: \R^r \longrightarrow \R | \\
        f(x) = \sum_{j = 1} ^q , \beta_j \cdot \prod_{k=1}^{l_j}
        \G(A_{jk}(x)), 
        x  \in \R^r, \beta_j \in \R, A_{jk}\in A^r, l_j,q \in \N
        )
        \}
    \end{split}
    \end{equation*}

\end{definition}

\begin{definition} Multivariable continuos functions and Borel $\sigma$-field 

    Let $\mathcal C^r$ be the set of continuos functions from $\R^r$ to $\R$, and let $\mathcal M^r$ 
    be the set of all Borel measurable functions from $\R^r$ to $\R$. We denote the Borel 
    $\sigma$-field of $\R^r$ as $\mathcal B^r$
    
\end{definition}



In order to write a self-contained book 
we are going to define some basic analysis and algebra concepts.

\begin{definition} Ring \cite{rudin-lebesgue-theory}
    A family $\mathcal R$ of sets is called a ring if $A \in \mathcal{R}$ 
    and $B \in \mathcal{R}$ implies

    \begin{enumerate}
        \item $A \cup B \in \mathcal R.$
        \item $A \cap B \in \mathcal R.$ TODO ADD SINCE 301
    \end{enumerate}
\end{definition}


\begin{definition} $\sigma -$ring \cite{rudin-lebesgue-theory}
 
    A ring $\mathcal{R}$ is called a $\sigma -$ ring if 

    \begin{enumerate}

        \item $\cup^\infty _{n=1} A_n \in \mathcal{R}$ whenever 
        $A_n \mathcal{R} for n \in \N$.
    \end{enumerate}

 TODO ADD SINCE PAG 301 

\end{definition}

TODO 
\begin{definition} Additive functions \cite{rudin-lebesgue-theory}
    We define a additive function $\phi$, as a set function defined on $\mathcal{R}$
    if $\phi$ assigns to every $A \in \mathcal{R}$ a number $\phi(A) \in \R$ and 
    if for every $A,B  \in mathcal{R}$ that verified $A \cap B = \emptyset$, 
    this implies

    \begin{equation*}
        \phi (A \cup B) = \phi(A) + \phi(B),  
    \end{equation*}

    In addition, $\phi$ is defined as \textit{ countably additive} if 
    $A_i \cap A_j = \emptyset (i \neq j)$ implies 
    \begin{equation*}
        \phi( \cup_{n = 1} ^ \infty A_n) = \sum_{n=1} ^ \infty \phi(A_n)
    \end{equation*}
    

    We assume that the range of $\phi$ does not contain both $+\infty$ and $-\infty$.
    Also we exclude set functions whose only value is $+\infty$ and $-\infty$. 

    When $\phi$ is countably additive and $\phi(A) \geq 0$ for every $A  \in \mathcal{R}$, 
    It is called a \textit{non negative countably additive function.}
\end{definition}
 
\begin{definition} Measure spaces \cite{rudin-lebesgue-theory}

     
\end{definition}


The notation that we are going to use is

\begin{definition}
    \begin{itemize}
        \item Let $\mathcal C^r$ be the set of continuous functions from $\R^r$ to $\R$.
        \item Let $\mathcal{M}^r$ be the set of all Borel measurable functions from $\R^r$ to $\R$.
        \item The Borel $\sigma -$ field of $\R^r$ is going to be denoted as $\mathcal B^r$
    \end{itemize}
\end{definition}  

The classes $\C$ and $\sum \prod ^r (G)$ belong to $\mathcal{M}^r$ 
for any Borel measurable $\G$. When $\G$ is continuos, 

TODO pag 3-8  

\medskip

\printbibliography

\end{document}
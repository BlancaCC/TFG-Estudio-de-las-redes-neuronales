\documentclass{book}
\usepackage[utf8]{inputenc}
\usepackage[english]{babel}
\usepackage{amsfonts}
\usepackage{blindtext}
\usepackage[T1]{fontenc}

\usepackage{mathtools}

\usepackage{amsthm}
\theoremstyle{definition}
\newtheorem{definition}{Definition}[section]

%Commands
\newcommand{\R}{\mathbb{R}} %Real numbers 
\newcommand{\N}{\mathbb{N}-\{0\}} % Natural number bigger than 0
\newcommand{\X}{\mathcal{X}} % Input space  
\newcommand{\x}{\textbf{x}} % Column vector 
\newcommand{\G}{\mathcal{G}} 
\newcommand{\C}{\sum ^r (\G)}


\newcommand{Cr}{\mathcal{\C}^r}


\begin{definition} Dense subset  
    Let $(X,\rho)$ be a metric space and $S \subseteq X$ is $p-$\textit{dense} in a subset T
    if for every $\epsilon \in S$ and for every $t \in T$ 
    there is an $s \in S$ such that $\rho (s, t) < \epsilon$. 
\end{definition}  

(The same that in the paper)
The concept under this definition is that an element of S can approximate a element of T to any desired degree of 
accuracy. In our theorems below, T and X correspond to $\mathcal{C}^r$ or $\mathcal{M}^r$, $S$ corresponds to 
$\C$ or $\Cp$ fo specific choices of $\mathcal G$ and $\rho$ is chosen appropriately.



\begin{definition} Uniformly dense on compacta in $\mathcal{C} ^r$
    A subset $S$ of $\Cr$ is said to be \textit{uniformly dense on compacta in } $\Cr$ if for every compact subset
    $K \subset \R^r$, $S$ is a $\rho_k$-dense in $\Cr$, where for $f,g \in Cr$ 
    $\rho_k(f,g) \equiv sup_{x\in k} |f(x) - g(x)|$. 
\end{definition}

\begin{definition} Functions' convergence uniformly on compacta
    A sequence of functions  $\{f_n\}$ \mathtit{convergence to a function f uniformly on compacta} if for all
    compact $K \subset \R ^r$  $\{ \rho_k(f_n), f) \} \rightarrow 0$ as $\{n \} \rightarrow \infty$
\end{definition}



We are ready to proof our first theorem, Its idea 
is that $\sum \prod$ feedforward networks are capable of arbitrarily accurate approximation to 
any real-valued continuos function over a compact set. 

This individually is interesting since if a phenomenon could be represented by the lasted conditions 
it could be learnt. (Another question maybe for physician or philosophers would be
that what kind of real phenomenons have those properties, but any way, 
we are going to achieve a more general theorem :D) 

We can argue that be in a compact function is not so restrictive. 

\begin{theorem} $\Cp$ i uniformly dense on compacta in $\Cr$
 Let G be any continuous non constant function from $\R \rightarrow \R.$ 
 Then $\Cp$ is uniformly dense on compacta in $\Cr .$   
\end{theorem}
